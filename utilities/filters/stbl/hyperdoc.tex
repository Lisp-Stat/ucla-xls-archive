\documentstyle{article}
\def\href#1#2{\special{html:<a href="#1">}{#2}\special{html:</a>}}
\begin{document}
\section{Density Functions}
Consider a data set of $1000$ numbers. We already know to pictorially
summarize the distribution of these numbers by means of histograms.
Here, for \href{file:hist.lsp}{example}, is a histogram of $1000$
numbers. We can \href{file:smhist.lsp}{approximate} the histogram by a
smooth curve that displays the shape of the distribution after ironing
out some of the raggedness. Such smoothing loses some details in the
histogram and therefore can be thought of as an idealized form of the
histogram.  We can also say that the unevenness in the histogram is a
consequence of the classes we have chosen and so the curve is really a
better description of the data.  In a relative frequency histogram,
the areas of the bars are proportional to the relative frequency of
the classes. And if we add together all the relative frequencies, we
get $1$. Therefore, it is natural to ask that our idealization of the
histogram, the smooth curve have total area $1$ underneath it. The area
under the curve between any two values on the $x$-axis is then equal
to the proportion of observations falling between the two values. This
curve is called the density curve of the distribution of the data.

There are many density curves. Let us study an important one which is
called the normal density.

\subsection{The Normal Density}
Let us first take a look at the normal density. Please click
\href{file:stbl.lsp}{here} and continue reading this document for
further instructions. 

Now, if all went well, you should have a menu with the word {\bf
  Tables} on it. Press your mouse on the word {\bf Tables} and drag it
on to the {\bf Normal Distribution} menu item and release the mouse
button.

Do you see the density? It should have a bell-shape with a single
peak. Notice how the density is {\em symmetric\/} about $0$; that is,
the shape to the left of $0$ is the same as the shape to the right of
$0$. The exact density of a normal curve is described by giving
information about two quantities, the mean $\mu$, where the peak
occurs and the standard deviation $\sigma$, which specifies how widely
spread the curve is. The curve that you see now has $\mu=0$ and
$\sigma=1$ and is called the standard normal distribution. The exact
formula for a normal density curve with mean $\mu$ and standard
deviation $\sigma$ is given by 
\begin{equation}
\phi(x) = \frac{1}{\sigma\sqrt{2\pi}}
\exp{-\frac{1}{2}\biggl(\frac{x-\mu}{\sigma}\biggr)^2} 
\label{eq:normal-density}
\end{equation}

The normal density curve has the following property, which is often
referred to as the empirical rule. 

\begin{center}
\fbox{
  \parbox[b]{4in}{
    \paragraph{The $68$-$95$-$99$ Rule}
    \begin{itemize}
    \item \href{file:68.lsp}{$68$\%} of the observations fall within
      $\sigma$ of the mean $\mu$.
    \item \href{file:95.lsp}{$95$\%} of the observations fall within
      $2\sigma$ of the mean $\mu$.
    \item \href{file:99.lsp}{$99.7$\%} of the observations fall within
      $3\sigma$ of the mean $\mu$.
    \end{itemize}
    }
  }
\end{center}
\end{document}
