The program {\tt dmc.lsp} is designed to graphically illustrate and
run a discrete finite state Markov Chain process. The user provides a
transition matrix and an initial state. The program creates a
graphical window with buttons to run the Markov Chain.  There are two
kind of buttons: oval or circular buttons representing states and
rectangular buttons that perform some actions. Several additional
mouse controls are available.  Figure~\ref{fig:gambler} shows the
the appearance of such windows. Many routine
statistics are available, including sample path plots, number of
visits to each state, etc. 

We now describe the complete design of our objects and
algorithms. The task of describing our code would be much easier and
neater if we had a literate programming environment like Knuth's {\em
WEB\/}\cite{wayne} for {\tt lisp} (although there is John~Ramsdell's
SchemeWEB for scheme\cite{lispfaq}).  An alternative is to use Dorai~Sitaram's
S\LaTeX{}\cite{dorai} package which is geared towards both Scheme
and Common~Lisp; however, we were unable to get it going using the
subset of Common 
Lisp that {\tt Lisp-Stat} uses.  Therefore, we have  
chosen to break up the code description into fragments that perform a
particular function.  The indentation of the code is preserved to help
the user follow the scope of variables; this is necessary in
understanding the logic of a large chunk of code. Where necessary, an
indentation level number is provided at the beginning of each such
fragment for the benefit of the reader; if no indentation level number
is provided, then the level is zero.  










