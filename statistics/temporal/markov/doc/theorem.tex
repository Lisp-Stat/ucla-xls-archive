Consider the following transition matrix:
\begin{center}
\[ {\bf P} = \left[ \begin{array}{ccc}
                        .5 & .3 & .2 \\
                        .1 & .8 & .1 \\
                        .7 & .2 & .1 \\
               		\end{array} \right] \]
\end{center}
If this matrix is raised to successive powers, an interesting thing happens:
\begin{center}
\[ {\bf P}^{(5)} = \left[ \begin{array}{ccc}
                        .313 & .554 & .133 \\
                        .283 & .590 & .127 \\
                        .318 & .549 & .134 \\
                        \end{array} \right] \]
\[{\bf P}^{(10)} = \left[ \begin{array}{ccc}
                        .297 & .573 & .130 \\
                        .296 & .575 & .130 \\
                        .297 & .573 & .130 \\
                        \end{array} \right] \]
\end{center}
It seems that the entries in each column are approaching the same value.    

To better explain what this result means, a few more terms must be
explained. A state has {\em period d} if $P_{ii}^{n}=0$ if $n$ is not
divisible by $d$. For example, if the process starts in state $i$ and
it is possible for the  process to enter that state only at times 3,
6, 9,\ldots, then state $i$ has period 3.  Any state with period 1 is
{\em aperiodic}.  A state $i$ is  {\em positive recurrent} if,
starting in $i$, the process is expected to return to $i$ in a finite
amount of time.  A positive recurrent, aperiodic state is  called an
{\em ergodic} state.  A Markov chain is said to be {\em irreducible}
if all its states communicate, or, given any state $i$, it is possible
to reach state $j$ within a finite  time, and vice versa. 

We can now state what is known as The Ergodic Theorem:

For an irreducible ergodic Markov chain \( \lim_{n \rightarrow \infty}
P_{ij}^{n} \) exists and is independent of $i$, and is denoted by 
\begin{center}
\[ \pi_{j}=\lim_{n \rightarrow \infty} P_{ij}^{n}
 \]
\end{center}

Furthermore, $\pi_{j}$ is the unique nonnegative solution to
\[ \pi_{j} = \sum_{i=0}^{\infty} \pi_{i}P_{ij},  j \geq 0, 
\mbox{ with } \sum_{j=0}^{\infty} \pi_{j} = 1 \]

The value $\pi_{j}$ is called the {\em limiting probability} that the
process will be in state $j$ at time $n$.  It can also be shown that
this value is  also the proportion of time that the process will be in
state $j$ in the long run. 

Because {\tt dmc.lsp} is designed to calculate the proportion of time that 
the Markov chain is in each state for the time that the process has
run, we  can graphically illustrate the validity of this theorem.

The file {\tt thm.lsp} contains the following code:
\begin{verbatim}
(def n 15)
(def m (random-stochastic-matrix n))
(def initial-state 7)
(setf z (send dmc-proto :new m initial-state))
\end{verbatim}
In a sample run, the following transition matrix was created for a
5-state Markov chain by the {\tt random-stochastic-matrix} function:
\[ \left[ \begin{array}{ccccc}
	0.209948  & 0.197941  & 0.0526538 & 0.465235 & 0.0742217 \\
	0.217497  & 0.234188  & 0.311191  & 0.222549 & 0.0145759 \\
	0.029961  & 0.0702461 & 0.0834497 & 0.440806 & 0.375538  \\
	0.206709  & 0.172418  & 0.258068  & 0.169594 & 0.193211  \\
	0.0142226 & 0.178653  & 0.117397  & 0.366932 & 0.322796  
	  \end{array} \right] \]
Using {\tt dmc.lsp}, the Markov chain was run for one million steps.
The following results were obtained:
\begin{verbatim}
> (send z :run-dmc-silently 1000000)
NIL
> (send z :describe)
-------------------------------------------

State    No of visits  Proportion
-------------------------------------------
0           137858       0.13786
1           169561       0.16956
2           178713       0.17871
3           308272       0.30827
4           205597       0.20560
-------------------------------------------
\end{verbatim}
A calculation of the limiting probabilities for the transition matrix
above yielded:
\[ \begin{array}{c}
	\pi_{0}=0.13780 \\ 
   	\pi_{1}=0.16943 \\
   	\pi_{2}=0.17862 \\
   	\pi_{3}=0.30836 \\
   	\pi_{4}=0.20578
   \end{array} \]
The proportion of time that the Markov chain spent in each state is
remarkably close to the limiting probability for each state, as
predicted by the theorem.









