\documentstyle{article}
\setlength{\textwidth}{6in}          % 5 lines set page size
\setlength{\textheight}{8.75in}
\setlength{\topmargin}{-0.25in}
\setlength{\oddsidemargin}{0.25in}

\newcommand{\inv}{^{-1}}
\newcommand{\xtxinv}{\mbox{$(X^{T}X)\inv$}}
\newcommand{\betahat} {\mbox{$\hat{\beta}$}}
\newcommand{\gammahat}{\mbox{$\hat{\gamma}$}}
\newcommand{\sigmahat}{\mbox{$\hat{\sigma}$}}
\newcommand{\yhat}{\mbox{$\hat{y}$}}
\newcommand{\var}{\mbox{var}}
\newenvironment{references}{%
  \section{References}%
    \begin{list}%
            {}%
            {\setlength{\leftmargin}{.25in}\setlength{\itemindent}{-.25in}}}%
   {\end{list}}
\title{Dynamic Graphics and Regression Diagnostics using XLISP-STAT}
\author{
R. D. Cook  and  S. Weisberg \\
Department of Applied Statistics \\
University of Minnesota \\
St. Paul, Minnesota 55108\thanks{ This work was supported by National
Science Foundation Grant number 9001298.  We would like to thank Luke Tierney,
Jim Harner and Nate Wetzel for their help with this project, and Julian
Faraway for useful comments.}}
\date{Technical Report No. 565\\October 7, 1991}
\begin{document}
\maketitle

\begin{abstract}
A set of additions to the standard {\tt regression-model-prototype} in
LISP-STAT (Tierney, 1990) are described.  These additions do all of the
graphics described in Cook and Weisberg (1989a), as well as a number of
other graphics (e.g., Cook and Weisberg, 1983, 1989b, 1991).  
A set of other useful functions is also included.  The code 
for these additions is available in the XLISP-STAT library in {\tt statlib}.
\end{abstract}

\section{Introduction}
The LISP-STAT environment designed by Tierney (1990) provides an excellent
platform for dynamic graphics in statistics.  We have used this program
as a basis to implement the ideas for regression diagnostics that appear in
Cook and Weisberg (1989a), hereafter called ``CW89''.  
This document describes our code, gives an
indication of how the code works, summarizes what the code does, and provides
help on getting started.	To use the code, you will need to have
a copy of XLISP-STAT.  In this document, we assume that this program is
available to you.  We have tested the code on Unix workstations and on
Macintosh II computers; we have no reason to believe that it will not work
with the Windows version of XLISP-STAT.

We are distributing this code for several reasons.  First, we believe that the
methods provided here can be very helpful in regression
modeling.  The methods are difficult or impossible using standard programs.
Second, we show how XLISP-STAT can be used to implement new statistical
ideas.  Third, we provide code that may be helpful to others who are
interested in developing similar methodology.

These enhancements to the {\tt regression-model-proto} are available without
charge, and may be freely distributed or modified.  However, they are not to
be sold for a profit, or offered as an inducement to purchase a commercial
product.  In general, this code may be distributed using the same rules for
distributing XLISP-STAT.  Although we have attempted to produce bug-free code,
errors surely remain.  We would be glad to hear about bugs:  send e-mail to
{\sl sandy@umnstat.stat.umn.edu}.

\subsection{Using statlib To Get the Code}
The code
is available 
%will eventually be available 
in the XLISP-STAT library on {\tt statlib}.  
The get the code over the internet, send the message:
\begin{verbatim}
mail statlib@lib.stat.cmu.edu
Subject:
send dyndiag from xlispstat
\end{verbatim}
Save the information that is sent back to you in a file, and follow the
instructions included for decoding.  When unpacked, 
you will have a \LaTeX\ copy of this document called 
{\tt doc.tex}, and several files whose
names all end in {\tt .lsp}.
Put all of the .lsp files into the directory (or, on the Macintosh, folder)
from which you launch XLISP-STAT.  

\subsection{Distribution by {\sl ftp}}
The code may be obtained via anonymous {\sl ftp} to the machine {\sl
umnstat.stat.umn.edu}.  Here is an abridged transcript of a session that will
get the code:
\begin{verbatim}
your-machine% ftp umnstat.stat.umn.edu
Connected to umnstat.stat.umn.edu.
Name (umnstat.stat.umn.edu:): anonymous  
Password (umnstat.stat.umn.edu:anonymous): (use your e-mail address)
ftp> cd pub
250 CWD command successful.
ftp> get dyndiag
200 PORT command successful.
150 Opening data connection for dyndiag (ascii mode) (201290 bytes).
226 Transfer complete.
local: dyndiag remote: dyndiag
205792 bytes received in 11 seconds (18 Kbytes/s)
ftp> quit
\end{verbatim}
The file {\tt dyndiag} is a Unix shar file, identical to the one that could be
obtained via {\sl statlib}.

\subsection{Distribution by Disk}
If you have the code on a floppy disk, copy every file on the disk that ends
with {\tt .lsp} into the directory (folder on the Macintosh) from which you
launch XLISP-STAT.  The remaining file, {\tt doc.tex}
is a \LaTeX\ copy of this document.

\subsection{Getting Started}
When XLISP-STAT starts, it looks for a file called {\tt statinit.lsp} in the
current directory or folder.  If such a file exists, it is loaded by the
system.  To load the dynamic diagnostics code automatically, 
modify your {\tt statinit.lsp} file, or
create a new one, by adding the following line:
\begin{verbatim}
(load "cwinit")
\end{verbatim}
To load the code manually, enter the above command in the listener window
(which is the name of the window for typing commands or getting output), or,
on the Macintosh, select ``Load'' from the ``File'' menu, and, in the dialog
box that will appear, select ``cwinit.lsp.''
Included are most of the data files
used in CW89 (cars.lsp, cloud.lsp, logistic.lsp, rat.lsp, lathe.lsp and
trees.lsp), and these can be used to reproduce the graphs 
discussed in the paper.

To get a quick overview of the enhancements, enter the following command:
\begin{verbatim}
(load "trees")
\end{verbatim}
This will create a regression object called {\tt tree-reg}, and this object 
will be sent the 
{\tt :graphics-menu "Tree-data"} message.  
A menu called {\tt Tree-data} will appear in
your menu bar.  You can try out the items in this menu as a test of the
features available.

\section{Regression Methods}

	{\tt Regression-model-proto} is the base prototype for fitting 
regression models using LISP-STAT.  Included here are many additions to
the original version written by Luke Tierney, 
including most of the dynamic diagnostics proposed by CW89 
as well as a few other graphs and some new ``standard" features 
such as easy case 
deletion. The methods described below are not in the standard
distribution of regression-model-proto.  For a description of the standard
methods, see Tierney (1990. Sec. 6.5.2).

\subsection {Calling sequence}
The model is initialized with a function call as
described in Tierney (1990, p. 201).  Here is an example of the function call,
assuming the data is in a plain text file called ``tree.dat" with 3
columns in the current directory\footnote{On Unix systems, the current
directory is the same as the directory from which XLISP-STAT is started.  On
the Macintosh, the current directory or folder is the last folder accessed.
For example, if a file in the folder ``Datasets'' is opened for editing 
using the built-in editor, then the current folder is ``Datasets.''}:
\begin{verbatim}
(def tree (read-data-columns "tree.dat" 3))
(def tree-reg (regression-model (select tree '(1 2 )) 
                           (select tree 0) 
                           :predictor-names '("Diameter" "Height")
                           :response-name "Volume"))
\end{verbatim}
All of the methods described here are then accessed by sending messages to the
object {\tt tree-reg}.  For example, to get the graphics menu to appear, 
send the message:
\begin{verbatim}
(send tree-reg :graphics-menu "Tree-data")
\end{verbatim}
To get an ARES plot, enter:
\begin{verbatim}
(send reg :ares)
\end{verbatim}
or select the ``ARES plot'' item from the {\tt Tree-data} menu.

\subsection{Routines used}
This code uses standard XLISP-STAT routines, 
plus bits and pieces of ALL the code provided.  Thus, you must load EVERYTHING
for the code to work properly.  Besides additions to regression-model-proto,
there are additions to graph-proto and its descendants, a new prototype called
{\tt project-proto} that computes a QR factorization, and allows accessing
both the factorization and various related quantities, a large number of
plot overlays (see Tierney, 1990, p. 285), and several functions collected in
{\tt fun1.lsp}.  Some of these functions may be useful in other
circumstances, and they may be freely used elsewhere.

\subsection{Computational comments}
Regression-model-proto uses the sweep algorithm 
is its basic computational tool.  For some methods, such as ARES plots, 
sweep is not a convenient paradigm for computations, and the QR factorization 
provides a friendlier environment.  As a result, ARES and a few other 
methods use the QR factorization via the project prototype.  Rewriting all of
regression-model-proto to be QR based would not be very hard.

\subsection{Plotting methods}
Most of the plots created by sending a message to a regression model proto can
later be accessed and will be updated whenever the regression model is
updated.  In particular, when a case is deleted or restored, the regression
model is recomputed, and most plots are redrawn.  Most plots have two
additional menu items added to their standard menu.  These menu items, which
appear at the end of the menu, can be used to delete or restore cases to a
model from the plot.  Points not used in computations will generally appear as
selected points on a plot, but of course they can be marked as desired in the
usual ways.  Where there are points not used in calculations, one of the
additional menu items is ``Select Deletions'', so these points can always be
identified.

In addition, all plots include a number of plot
controls, buttons and sliders that can do useful things to the plot.  Almost
all of these are self explanatory, but see also the section on Plot
Controls.

\begin{itemize}
\item {\tt :plot (\&optional x-axis y-axis z-axis \&key (plot-controls t))}
makes plots.  If no specification is given, the default is a scatterplot 
of 'fit-values 
{\sl versus} 'studentized-residuals.  If only x-axis is specified, a histogram 
is drawn.  If two axes are specified, a scatterplot is drawn.  
If three axes are specified, a spin-plot is drawn.  
If :plot-controls is t, the default, then many appropriate control buttons, 
sliders, and so on are added to the plot. The quantities to be plotted can be 
either a list or vector of numbers, or else they can be a function 
that when funcalled will produce a list of numbers to be plotted.  If
the latter procedure is used, then the plot will be automatically updated
whenever the underlying regression model is changed, for example by changing
the 'included slot, thereby deleting or restoring cases,
or by transforming a predictor or the response.
If one of the names supplied below (they all begin with a 
single quote {\tt '}) is used for an axis specification, then a function is 
created that when funcalled will compute the values specified. 
	\begin{itemize}
   \item 'response (response variable)
	\item 'yvar (response for glims or regression models)
	\item 'y (response for regression models; working variate for glims)
	\item 'raw-residuals ($e_i$)
	\item 'residuals ($\sqrt{w}e_i$)
	\item 'studentized-residuals ($e_i/\sigmahat (1-h_{ii})$)
	\item 'externally-studentized-residuals ($e_i/\sigmahat _{(i)}(1-h_{ii})$)
	\item 'jackknife-residuals (same as externally studentized residuals)
	\item 'residuals-squared (see discussion under Non-graphical methods below)
	\item 'leverages ($h_{ii}$)
	\item 'cooks-distances 
	\item 'distances (same as Cook's distance)
	\item 'sqrt-cooks-distances
	\item 'fit-values 
	\item 'case-numbers 
	\item 'signed-local-inf (direction of maximum curvature; see discussion
under Non-graphical methods below)
	\item 'local-inf (absolute values of direction cosines)
	\item 'normal-scores (calls the quantile method above with no arguments)
	\end{itemize}
If you want to plot against a column of X, you may specify a column number, 
or its quoted predictor-name (remember that Lisp starts counting lists with
element zero, not element one).  To plot some other list of data, you can 
just give the data or its name.  This method automatically determines 
axis-labels and a plot title, but these can be modified in the usual way by
sending the graph the appropriate messages.  The apostrophe 
before the names shown above is required.  The method {\tt :return-function}
takes the keywords given above and turns them into function.  
For example, if you give the specification {\tt 'residuals}, then the 
following function is created:  {\tt \#'(lambda () (send self :residuals))}.  
Through this device, if the specification of a model is changed 
(for example, if a point is deleted), then a plot can be redrawn by 
executing the function that specifies the data.  You can supply any 
other function.  

Typical calls to plot might be:
\begin{verbatim}
(send reg :plot 'case-numbers 'local-influence) 
(send reg :plot 0 'y 1) 
(send reg :plot #'(lambda () (* (1- (send reg :leverages)) 
                              (send reg :fit-values))) 
                #'(lambda () (^ (send reg :raw-residuals) 2)))
\end{verbatim}
The first plot is of \{case numbers, direction of maximum local curvature for
case weight perturbations\}.  The second plot is a 3-D plot of \{first
predictor, response, second predictor\}.  The third plot is of
\{$(1-h_{ii})\yhat_i$, $e_i^2$\}, a plot for heteroscedasticity.  All three
plots will be correctly updated if cases are deleted because the axes are
specified by functions that create the data, not the data themselves.  

\item {\tt :scatterplot-matrix (arguments)}
This message draws a scatterplot-matrix, using the same arguments as the {\tt
:plot} command discussed above.  The default plot is a scatterplot matrix for
all the predictors and the response.
\item {\tt :qq-plot (\&keywords)} produces a quantile-quantile plot of
studentized residuals.  To change the quantity plotted, set the keyword
{\tt :x} by specifying a function or one of the keywords listed under the
:plot method above.  For example {\tt :x 'y} will give a normal probability 
plot of the response.  To change to a different distribution than 
the normal, set the keyword
{\tt :quantile-function} to the name of the
function that will produce the desired
quantiles.  The function passed to qq-plot must be of a single argument,
a number between 0 and 1, as is the case for the normal-quant
function.  LISP-STAT has a number of other built-in quantile functions, 
including
cauchy-quant, f-quant, poisson-quant, chisq-quant, beta-quant, and t-quant, but
most of these require two or more arguments.  It is therefore necessary to
define your own quantile function that fixes the values of the other
arguments.
For example, to get a $\chi^2_5$ probability plot of the response in a
regression object reg, you can type the following expression:
\begin{verbatim}
(send reg :qq-plot :x 'y 
            :quantile-function #'(lambda (p) (chisq-quant p 5)))
\end{verbatim}
An additional function called hnorm-quant in fun1.lsp computes 
quantiles of the standard half
normal distribution.  Thus, to get a half normal probability plot of the
square root of Cook's distances, as suggested by Atkinson (1985), type
\begin{verbatim}
(send reg :qq-plot :x 'sqrt-cooks-distances
            :quantile-function #'hnorm-quant)
\end{verbatim}
The final keyword is {\tt :plot-controls} which, if set to its default
value of t, adds control buttons to the plot.  Setting {\tt :plot-controls nil}
leaves the control buttons off (this would usually only be desirable if the
plot is to be saved for later printing and publication).  
If any cases in the data are not used in 
the calculations (that is, the corresponding element 
of {\tt (slot-value 'included)} is nil), then they are not used to compute 
quantiles.  These deleted points will be plotted just to the left of the 
included data.  This is non-standard, and some users may wish to change this
to something else.

\item {\tt :ares ()}  produces an {\sl A}dding {\sl RE}gressors 
{\sl S}moothly or {\sl ARES} 
plot.  The user will be given dialog boxes to choose the base 
model and the model to be added, and the method for adding 
(sequential or as a group).  See CW89 for details of what
this plot does, and how the calculations are done.  The dialog boxes use the
{\tt select-list-dialog} function included in fun1.lsp.

For example, to reproduce Figures~4 to 7 of CW89, p. 281, one
first needs to load the lathe data:
\begin{verbatim}
(load "lathe")
\end{verbatim}
This will create a regression object {\tt lathe-reg}, and send it the
:graphics-menu message.  One can then either type the command 
\begin{verbatim}
(send lathe-reg :ares) 
\end{verbatim}
or select ARES from the menu.  In either case a dialog box
appears.  In the left scrolling window will appear the names of all the
predictors in the model.  Double clicking on a name moves it to the right
list.  To reproduce Figure~4, double click on ``Speed'', and then click on
``OK''.  A second dialog box appears.  Double click on ``Feed'', and then
click on ``OK''.  The plot will now appear, and the animation is carried out
be clicking on the slider on the plot.

To get Figures 5 to 7, start a new ARES plot by typing its message or
selecting it from the ``Lathe-data'' menu.  In the first dialog box, 
select both
``Speed'' and ``Feed'', and then ``OK''.  In the second dialog box, select
``Speed$^2$'', ``Feed$^2$'', and then ``Speed*Feed'', in this order, and then
``OK''.  A third dialog box appears, giving a choice between adding the
variables as a group (all at once), or sequentially; select sequentially, and
then ``OK''.  The plot will appear with a slide bar.  The index on the slide
bar tells you which predictor is currently being added.  For example, if the
value on the slide bar is 1.5, then predictor number one in the list
``Speed$^2$'', ``Feed$^2$'', ``Speed*Feed'', or ``Feed$^2$'' since {\sl lisp} 
starts
counting at zero, is being added, and the current value of $\lambda$ is .5.
Figure 5 corresponds to $0 \leq \lambda \leq 1$; Figure 6 to $1 \leq \lambda
\leq 2$, and Figure 7 to $\lambda \geq 2$.

\item {\tt :avp ()} produces an added variable plot.  Variable 
selection is done via dialog boxes.  Buttons appear on the plot to show
a contour of constant local curvature using two perturbation schemes, case
weight perturbation and predictor perturbation.  The details are given by Cook
(1986) and Cook and Weisberg (1991).  For both perturbation schemes, 
the contour is drawn at a
value of  $\sqrt{.33} = .58$ (the total length of the perturbation
direction vector is 1, so a point that lies on or beyond the displayed contour
accounts for at least 33\% of the total length). 
Detrended avps, obtained via a button on
the plot, are discussed in CW89 (Sec. 4.3).

\item {\tt :spin-residuals ()} produces a 3-D residual plot 
\{$P_1Y, (I-P)Y, P_{2.1}Y$\} 
where $P$ is the projection on the full model, $P_1$ is the projection on the 
base model, and $P_{2.1}$ is the projection on the added variables after 
adjusting for the base model.  Dialog boxes as in :ares are used to get the
specification.  Details of the plot are given in CW89 (Sec.  5).

To reproduce Figure 14 of CW89 enter the command:
\begin{verbatim}
(load "cloud")
\end{verbatim}
and then select the ``Spin Residuals, 3-D'' item from the ``Cloud-data'' 
menu (or
enter the command from the keyboard).  The first dialog box asks for the base
model, which consists of all the predictors except ``Action''.  A second
dialog box asks for the added predictors, consisting here of just ``Action''.
This will produce the figure in the paper.

\item {\tt :avsp}  Produces an added variables plot (CW89, Sec. 4.4).
The model is specified via dialog boxes.  For example, Figure 12 in
CW89 can be reproduced by selecting the ``AVsP -- 3-D'' item from the 
``Cars-data'' menu after typing
\begin{verbatim}
(load "cars")
\end{verbatim}
The first dialog box allows choice of a ``Base model'', corresponding to $X_1$
in the paper.  Select ``Weight'' 
and then ``OK''.  A second dialog box appears to choose $X_2$.
Select ``Horsepower'', and then ``OK''.  A final dialog box is for $X_3$.
Select the remaining predictor ``Displacement'', and then ``OK''.  Figure 12
is three static views of this resulting plot.

Work reported in Cook and Weisberg (1989b) suggests that in this, 
and in all higher
dimensional residual plots, the quantities plotted on the axes should be
orthogonal to maximize resolution in the plot.  An efficient way to do this is
the change the quantity on the out-of-page axis to be the part of the second
added variable that is orthogonal to the base model and the first added
variable.  This can be done in this plot by selecting the
``Orthogonalize Axes'' button on the plot.

\item {\tt :dynamic-rankits (arguments)}
This does several regressions of
$Y^{(\lambda)}$ on the predictors, for $\lambda$ in the range $(0, 1)$ for
the Box-Cox modified power family of transformations.  
A display is then 
given of a confidence curve (Cook and Weisberg, 1990) for $\lambda$ 
based on the profile log-likelihood.  
In addition, a window is opened with a rankit (normal probability)
plot of the studentized 
residuals.   The window is controlled by a slide bar, which controls the 
value of $\lambda$.  We call this a dynamic rankit plot (CW89, Section 3).

To get the information contained in Figure 9 of CW89, first load the tree data,
and then call the method:
\begin{verbatim}
(load "trees")
\end{verbatim}
If you select ``Dynamic Rankits" from the Graphics menu, the keywords will be
set to their default values, which are different from the values used in the
paper.  The graph in the paper can be duplicated by typing the
following command:
\begin{verbatim}
(send tree-reg :dynamic-rankits :lambda '(-2 -1.5 -1 -.5 0 .33 .5 1 1.5 2))
\end{verbatim}
This method precomputes
all the regressions, and may therefore take some time, especially for large
data sets.  

The method as several optional keywords. The {\tt :residuals} keyword can be
used to change the specification of what to plot; the default is
{\tt :residuals 'studentized-residuals}; any other specification will use
ordinary residuals in place of studentized residuals.
The y-values can be shifted to plot $(Y + \mu)^{(\lambda)}$ 
by setting the  {\tt :shift} keyword to the number $\mu$.  
The values of $\lambda$ are
set by the {\tt :lambda} keyword, with default value '(-1.5 -1 -.5 -.33 -.25 0
.25 .33 .5 1 1.5).   The family of transformations is set by
the  {\tt :y-transform} keyword.  The default is:
\begin{verbatim}
:y-transform #'(lambda (c lam) 
                  (let ((ans (send self :bcp c lam)))
      					(if ans ans (error "Negative y"))))
\end{verbatim}
The :bcp method computes the normalized Box-Cox power family transformation
with power lam and shift c.  Finally the quantile
function, can be changed from {\tt \#'normal-quant} 
to some other distribution using the {\tt  :quantile-function} keyword.  An
example of a specification of a quantile function is given in the discussion
of the qq-plot above.

This plot calls the method {\tt :dynamic-plot} which can be used as the basis
of other animated plotting procedures.  For example, one could easily modify
the code provided to give a dynamic plot of residuals versus fitted values as
a power transformation of the response or a predictor is changed.

\item {\tt :hetero-score ()} Plot and statistics for score test for 
heteroscedasticity.  The menu item in the Graphics menu for this plot is
``Non-constant Var Score''.  See Cook and Weisberg (1983).

The plot is of $e_i^2/\sum (e_j^2)$ on the vertical axis versus an estimated
direction on the horizontal axis.  The model fit is of the form $y_i =
x_i^T\beta + \epsilon_i$ with $\var(\epsilon_i) = \sigma^2\exp (z_i^T\gamma)$.
The estimated direction is given by $z_i^T\gammahat$ where $\gammahat$ is an
estimate of $\gamma$.  $z_i$ is chosen using the ``Change Model'' button on
the plot.  Another button allows toggling between squared residuals and
absolute residuals.  At the top of the plot, three numbers are printed.  These
are, respectively, the score test for nonconstant variance (essentially
testing $\gamma = 0$), its asymptotic degrees of freedom, and its asymptotic
p-value from a Chi-squared approximation.


\item {\tt :graphics-menu (\&optional (title "Graphics"))}  
Adds a menu to the menu bar with the title ``Graphics'', or whatever
alternative title you specify.  This menu
allows accessing all the above graphs without
typing.  Here is a brief description of the items in this menu.  Items not
described were described above.
\begin{itemize}
	\item ``Stud. res. vs yhat" gives a 2-D plot of \{$r_i$ = studentized
residuals, $\yhat$ = fitted values\}.
	\item ``ARES plot" gives a 2-D ARES plot.
	\item ``AVP -- 2-D" gives a 2-D added variable plot.
	\item ``AVsP -- 3-D" gives a 3-D added variables plot.
   \item ``Spin residuals, 3-D" gives a 3-D spinning residual plot.  This plot
has an additional plot control called ``Rotations menu,'' described in
Section 2.5.4, that can be used to get special rotations in this plot.
	\item ``Local influence--absolute" gives a 2-D plot of \{case numbers,
$\mid$direction cosines for maximum local influence for case weight
perturbation$\mid$\}.  See Cook (1986) or Lawrance (1991) for details.
	\item ``Local influence--signed" gives a 2-D plot of \{case numbers,
direction cosines for maximum local influence for case weight
perturbation\}.
	\item ``Normal-plot" gives a normal probability plot of studentized
residuals.  To get a normal plot of any other quantity, the user can use the
{\tt :qq-plot} method described above.
	\item ``Non-constant Var. Score" sends the :hetero-score method, giving a
2-D plot.
	\item ``Dynamic-rankits" gives a dynamic rankit plot.
	\item ``Scatterplot Matrix" gives a scatterplot matrix of the predictors and
the response.
   \item ``Other plot" allows the user to draw an arbitrary plot.  When this
item is selected, a dialog appears in which the users can select one or more
quantities to be plotted.  The choices include all the predictors and the
quantity names given under the description of the plot command.  If one item
is chosen, a histogram is drawn; two items for a scatterplot; 3 items for a
spinning plot, and more than 3 items for a scatterplot matrix.  The first item
chosen is put on axis 0 (horizontal or $x$-axis), the second item on axis 1
(vertical or $y$-axis), and so on.
	\item ``Display fit'' prints the coefficient estimates and standard errors
in the listener window by sending the :display message.
	\item ``Remove graphics menu'' removes the Graphics menu from the 
menu bar on the Macintosh.
	\end{itemize}
\item {\tt :graphs} returns, as a list, all the graphs associated with the
regression model.
\item {\tt :last-graph} returns the last graph created.  
\end{itemize}

\subsection{Plot Controls}
All of the plots produced by selecting items in the graphics menu will produce
a selection of plot controls appropriate for that plot.  If you type the plot
message yourself, you can add a keyword, {\tt :plot-controls nil} to remove
the controls.  This is likely to be desirable only for plots for publication.
Many of the controls described here are available for any graph, whether
created by the :plot method or not, by sending the graph the message 
{\tt :plot-controls}.  All plots except histograms include a symbol palette
and, if the screen has color, a color palette.  To color a point, select the
point and then click on the color you want in the palette.  This can also be
done using standard XLISP-STAT menu items.
\subsubsection{Histograms}
Histograms have two or three plot controls, depending on the values in the
plot.  All controls are slide bars.  
\begin{itemize}
\item ``NumBins'' is
used to vary the number of equal sized bins in the plot.  
\item ``Gauss Ker Dens'' is used to fit a Gaussian kernel density estimate 
to the data; the
slide bar controls the bandwidth.  The slide bar varies from 0.01 to 0.50; to
get the bandwidth used, multiply the value in the slider by the range of the
visible data on the horizontal axis, so the bandwidth is determined as if the
data were in the range $(-1, 1)$.
\item Transformations.  If the data are strictly
positive, a slide bar appears controlling a power transformation for the data.
The data actually displayed in the plot is given by $(y^\lambda
-1)/\lambda$, where $\lambda$ is the value chosen by the slide bar (with
$\lambda = 0$ corresponding to $\ln (y)$) and if $\lambda = 1$ the original
data is displayed.  The
transformation is {\sl not} passed back to the underlying regression model,
but is used only in the display.  These controls can be added to any plot by
sending the :install-transform-control message.  Unlike other plot controls,
the transformation controls will not interact properly with other plot
controls that modify the data, such as removing linear trends, or changing the
underlying model by deleting a case.
\end{itemize}
\subsubsection{Scatterplot Matrices}
The plot controls here are slide bars for each strictly positive variable
in the plot.  As with histograms, these control power transformations applied
to the data in the plot, but these do not update the regression model.  
\subsubsection{Scatterplots}
A number of slide bars are given on scatter plots.  
\begin{itemize}
\item ``Rem. Lin. Trend'' removes/restores a linear trend from the plot.
This is done using OLS, based
on all visible points.  
\item ``OLS fit'' draws the OLS line, again based on all
visible points.  
\item ``Zero line'' draws a horizontal line at zero.
\item ``Join Points'' draws a line between the points, as might be 
appropriate if the values on the horizontal axis are time or case numbers.
\item ``M-est'' is a slide bar for fitting an M-estimate
to the plot, using the visible data (this is fit to the plot, not to the
underlying regression model), with Huber's $\psi$-function, and with tuning
constant selected in the slide bar.  The usual choice of tuning constant 
is about 1.25 to 1.5.  A
value of ``NIL'' on the slide bar indicates no line is fit.  
\item ``Gauss Smooth'' fits a kernel regression estimate to the visible 
data in the plot, with
bandwidth chosen in the slide bar, and with a default Gaussian kernel.  
\item ``Lowess" fits a lowess smoother with no robust steps, and the fraction
set to the value in the slide bar.  A value of ``NIL'' means the lowess curve
is not fit.
\end{itemize}
\subsubsection{Spinning Plots}
Standard 3-D plots have a number of button controls. 
\begin{itemize}
\item ``Rem. Lin. Trend'' removes/restores
 a linear trend in the plot by OLS using the visible points in the plot.
\item ``Orth. Axes'' replaces the out-of-page axis by the making it orthogonal
to the quantity on the horizontal axis.  The subspace spanned by these two
vectors is unchanged, but Cook and Weisberg (1989b) show that orthogonalizing
will maximize the resolution in the plot.
\item ``Fixed scaling'' toggles the scaling in the plot.  When not selected
the plot is in {\sl abc} scaling, with all axes scaled to maximize resolution
in the plot; when selected, the plot is in {\sl aaa}
scaling with all three axes scaled the same, which is appropriate if relative
magnitudes on the three axes are of interest.  
\item ``Extract Horiz'' will create a list of data with name chosen via a
dialog box 
of the data currently plotted on the horizontal axis on the screen.  For the
scaling of this list, see the next item.
\item ``Extract 2-D'' creates a 2-D plot of the current horizontal and
vertical axes.  The plot controls for 2-D plots can then be used on this 2-D
plot.  The plot so created will not be updated when the model is changed, but
it will be linked and will inherit colors, symbols, and point states.  Since
this plot is extracted from a 3-D plot, the coordinates on each axis are
contained in the range [-1, 1].
\item ``Rock Plot'' rocks the plot.
\item ``Rotations Menu''  This appears only on a spinning residual plot.  When
selected, a menu pops up giving a number of options.  ``ARES rotation'' gives
the rotation that is equivalent to the ARES plot.  ``Rotate to yhat'' puts the
linear combination of the horizontal and out of page axes that is equivalent
to $\yhat$ on the horizontal axis.  ``Rotate to e1'' puts the submodel
residuals on the vertical axis and submodel fitted values on the horizontal
axis.  ``Print Screen Coordinates'' prints the
current value of the transformation matirx.  The first column of this matrix,
for example, gives the linear combination of the data that currently appears
on the horizontal axis.  The next three items can be used to move specific
axes to specific places. 
\end{itemize}

\subsection{Non-graphical Methods}
	Here is a list of some new methods that the user may
want to use.  Several other internal methods that the developer might find
useful are given in the code.

\subsubsection {Accessor Functions that allow updating}  
Each of the following return 
slot values, or reset them if an argument is given:
\begin{itemize}
\item {\tt :pweights (\&optional new)} In regression-model-proto, 
this is the same 
as :weights.  In glim-proto, it sets/returns the prior weights.  It has been 
added to regression-model-proto for consistency with glim-proto.
\item {\tt :deviance (\&optional new)} Returns the residual sum of 
squares.  This has also
been added for consistency with glim-proto.
\item {\tt :yvar (\&optional new)} Returns the response values.  
For consistency with glim-proto.
\end{itemize}

\subsubsection {Accessor Functions without updating}.  Each of these methods returns a
slot value or does a computation, but accepts no arguments.
\begin{itemize}
\item {\tt :residuals-squared ()} Returns a list of $e_i^2/\sum e_j^2$.  For use
in heteroscedasticity plots and tests.
\item {\tt :local-inf ()} Computes the direction of maximum curvature
in the likelihood displacement for the coefficient vector when case weights
are perturbed and the statistic $C_{max}$; see Cook (1986).
Returns a list whose first element is $C_{max}$
and whose second element is the direction cosine vector.  
\end{itemize}

\subsubsection {Other functions.}
\begin{itemize}
\item {\tt :lin-combination (x)} returns a list of 2 elements, $x^T\betahat$
and its standard error, $\sigmahat \sqrt{x^T\xtxinv x}$.
\item {\tt :toggle-cases (\&optional (idnum nil))}  changes the 
'included status of the cases in the idnum list.  
idnum may be a number or a list.  If the list is left off or set to nil, all
cases are restored to the data.
\item {\tt :quantiles (keywords)}  Returns the normal quantiles of the response.
There are two keyword arguments.  The quantity for which quantiles are
computed is set with the :x keyword.  The function used to compute the
quantiles is set with the :quantile-function keyword. 
Cases for which the 'included slot is nil are not used in computing
quantiles.  This is generally used internally by the probability plotting
methods.
\item {\tt :bcp (c p)} computes the normalized Box-Cox power 
transformation of the
response, adding constant $c$, with power $p$.
\end{itemize}

\section{How it Works}
The main features of these enhancements are (1) the close linking of the
regression model and the graphs; (2) the plot controls  
and (3) the ability to delete cases from a fit using
the plot menu.  Here is an outline of how it all works.  
All of the graphical methods described above call
the  {\tt :make-plot} message.  On the first call to :make-plot, the 
{\tt :make-first-plot} 
message is sent.  This method defines a few slot values, and
creates overrides for the standard {\tt :compute} method.  In particular, a slot
value called 'graphs is created, and every new graph that is a plot of the
data is then listed in
this slot.  Whenever the new :compute method is called, first the fit is
recomputed, and then, if the slot 'graphs is not nil, all the graphs in this
list are sent the :update message.  Each plot has slots containing functions
that recompute the values on its axes, and these are funcalled to get the new
values for the plot.  The actual updating of the plot is done via a call to
the graph-proto method {\tt :draw-next-frame}.  

There are a variety of buttons and sliders that appear on various plots.
These are all descendants of the graph-control-proto of Tierney (1990), and
are organized in three files, overlay1.lsp, overlay2.lsp,
and overlay3.lsp.  The actions
associated with the buttons also get properly updated when a plot is changed
through the :draw-next-frame method.  Each time this method is called, it first
funcalls all the functions stored in a slot called  'start-next-frame,
then updates the plot, then funcalls the functions in 
'finish-next-frame.  A typical function in 'start-next-frame might be

\begin{verbatim}
#'(lambda () (send plot :clear-lines))
\end{verbatim}
which will remove any smoothers, or other lines on the plot.  The plot is then
updated, and a typical function in 'finish-next-frame will redraw any
smoothers or other lines if appropriate.  The order in which the functions in
these lists is executed can be important.  For example, if a plot is to be
detrended, the trend should be removed before any smoothers are added to a
plot.

The most common change to the model will be deletion/restoration of cases.
This is done from the graph using two menu items.  For example, if the user
selects a few points from the graph, and then selects the ``Delete
Selection'' item from the menu, the underlying regression model is sent the
{\tt :toggle-cases} message with the indices of the selected cases.  The new
{\tt :compute} method is then called, and the model and plots are automatically
updated.

Adding other plots and/or controls to plots should be relatively
straightforward by modifying the methods provided.  However, plots that are
not just of data, for example, plots of log-likelihood contours, may require
somewhat different treatment, including a more complex linking strategy than
that one used in this code.

The code is organized into several files, so that each file is less than 32K
in length (the maximum size for use with the built-in XLISP-STAT editor on the
Macintosh).  The regression proto methods are in reg1.lsp and reg2.lsp.  The
overlays (plot controls) are in overlay1.lsp, overlay2.lsp, and overlay3.lsp.  

\section{Graph-proto methods}
A few additional graph-proto methods are included with this distribution, and
these are in rgraph1.lsp.  Most of the added graph proto methods will
never be accessed by the end user directly, but may be of interest in
developing code.  However, the plot controls available in the
regression plots are in fact available to ANY plot by sending the graph named,
for example, {\tt plot}, the message
\begin{verbatim}
(send plot :plot-controls)
\end{verbatim}
In addition, the message:
\begin{verbatim}
(send plot :install-transform-control)
\end{verbatim}
will install sliders to transform the data on each axis in a plot via a power
transformation in the range (-1, 2) whenever the
data is strictly positive.  These controls act only on the graph, and do not
cause an underlying regression model to be updated.  Of course, they could be
modified by a change in their :do-action method, to update a regression
model as well.

\section{Project-proto}
A prototype called {\tt project-proto} 
is also provided with this distribution.  The code is in file fun1.lsp.
Project-proto is useful in dynamic graphics where projections provide a
useful paradigm for updating plots.  The prototype is very similar to
regression model proto, and its use should be clear from the code.

\section{Functions}
Here is a list of functions provided in fun1.lsp.  Arguments for
these functions are described with the code.
\begin{itemize}
\item {\tt sweep} Equivalent to the sweep function in {\sl S}.
\item {\tt rmel} removes an element from a list.
\item {\tt substr} Returns part of a string.  Useful in labeling spinning
plots or plot controls, where long labels can be a problem.
\item {\tt mytoggle-item-proto}  An alternative strategy for specifying menu
items with differing text depending on a test.
\item {\tt select-list-dialog}  A dialog for selecting an ordered list of
items.
\item {\tt range} Returns the range of a list.
\item {\tt midrange} Returns the midrange of a list.
\item {\tt geometric-mean} Returns the geometric-mean of a list.
\item {\tt cosangle} computes the cosine of the angle between a vector and a
subspace.
\item {\tt cosangle1} computes the cosine of the angle between two vectors.
\item {\tt angle} computes the angle between a vector and a subspace.
\item {\tt hnorm-quant} computes quantiles of the standard half-normal
distribution.
\item {\tt boxplot}  Overrides for all the standard boxplot methods so that
the boxplots produced are equivalent to the ``old'' boxplot command in 
{\sl S}.  In particular, cases with values outside the ``outer fences'' are
shown explicitly in the graph.  The points in the plot are numbered as if all
the data in the plot formed a single list.  
\item {\tt contour-plot}  Calls the standard contour-function routine, but
adds some useful mouse modes.
\end{itemize}
\begin{references}{}
\item Atkinson, A. C. (1985).  {\sl Plots, Transformations and Regression}.
Oxford:  Oxford University Press.
\item Cook, R. D. (1986).  Assessment of local influence (with discussion).  
{\sl Journal of the Royal Statistical Society, Ser. B},  133-155.
\item Cook, R. D. and Weisberg, S. (1983). Diagnostics for heteroscedasticity
in regression.  {\sl Biometrika}, 70, 1-10.
\item Cook, R. D. and Weisberg, S. (1989a). Regression diagnostics with dynamic
graphics (with discussion).  {\sl Technometrics}, 31,277-309.
\item Cook, R. D. and Weisberg, S. (1989b). Three dimensional residual plots.
In Berk, K. and Malone, L., eds., {\sl Computing Science and Statistics:
Interface '89}, Washington, D. C.:  American Statistical Association,
162-166. 
\item Cook, R. D. and Weisberg, S. (1990).  Confidence curves for nonlinear
regression.  {\sl Journal of the American Statistical Association}, 32, 544-551.
\item Cook, R. D. and Weisberg, S. (1991).  Added variable plots in linear
regression.  In Stahel, W. and Weisberg, S. (eds), {\sl Directions in
Robust Statistics and Diagnostics:  Part I}.  New York:  Springer, p. 47-60.
%\item H\"ardle, W. (1990). {\sl Applied Nonparametric Regression},
%Cambridge:  Cambridge University Press,.
\item Lawrance, A. J. (1991).  Local and deletion influence.   In Stahel, W.
and Weisberg, S. (eds), {\sl Directions in
Robust Statistics and Diagnsotics:  Part I}.  New York:  Springer, p.
141-157.
\item Tierney, L. (1990).  {\sl Lisp-Stat}.  New York:  Wiley.
%\item Thisted, R. (1988).  {\sl Elements of Statistical Computing}.  New York:
%Chapman \& Hall.
\end{references}
\end{document}
