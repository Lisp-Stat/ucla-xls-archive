The file {\tt dmc-opts.lsp} contains code that selects options for use
by {\tt dmc.lsp}. 
\begin{verbatim}
(case (screen-has-color)
      (nil
       (defvar *normal-state-color* 'white)
       (defvar *current-state-color* 'black)
       (defvar *normal-state-text-color* 'black)
       (defvar *current-state-text-color* 'white))
      (t 
       (defvar *normal-state-color* 'blue)
       (defvar *current-state-color* 'red)
       (defvar *normal-state-text-color* 'white)
       (defvar *current-state-text-color* 'black)))
(defvar *history-file* "history.mc")
(defvar *time-window* 50)
(defvar *run-step-delay* 30)
\end{verbatim}
Depending on whether the user has a color monitor or not, some global
variables are given appropriate values. The variable {\tt
  *normal-state-color*} refers to the background color of a normal
state, {\tt *current-state-color*} refers to the background color of
the current state, {\tt normal-state-text-color*} refers to the color
used for text in a normal state, and {\tt *current-state-text-color*}
refers to the color used for text in the current state.  The user can
change these to reflect his or her preference.  

For history recording, a default file name of ``history.mc'' is used.
Note that this does not imply history will be recorded; the user has
to enable history recording from the menu, and when that is done, the
user is given a choice to change the file name. The variable {\tt
  *time-window*} refers to the window width used for
sample path display---the sample path for only the last {\tt
  *time-window*} time ticks is displayed in the sample paths window.
If the user sets this variable to zero, then the user gets a
scroll-bar under the window which can be scrolled for the complete
sample path plot.  It must be noted that even when the value of {\tt
  *time-window*} is non-zero, the compplete history can be displayed
in the plot by rescaling the plot. 

Finally, since the transition takes place extremely quickly, at least
on a workstation, a delay is employed to slow things down.  A value of
$30$ means a delay of half a second. 

