The file {\tt stochast.lsp} consists of a few functions that work on
stochastic matrices.  Some of the functions are just conveniences and
others determine whether a matrix is stochastic or not. 

\subsubsection{The {\tt :random-stochastic-matrix} function}
\label{subsubsec:random-stochastic-matrix}
\begin{verbatim}
(defun random-stochastic-matrix (n)
  (let* ((tmp (uniform-rand (* n n)))
         (m (row-list (matrix (list n n) tmp))))
    (dotimes (j n)
             (setf (select m j)
                   (/ (select m j) 
                      (sum (select m j))))
             (setf (select (select m j) (- n 1)) 
                   (- 1 (sum 
                         (select (select m j)
                                 (iseq 0 (- n 2)))))))
    (matrix (list n n) (combine m))))
\end{verbatim}
This function will return a random stochastic matrix of dimension $n$. 
First, $n$ rows of $n$ uniform random variables are generated. Then
each row is normalized so that the entries are between $0$ and $1$ and
add up to 1.  We used this function mainly to debug some routines and
therefore, it probably serves not much of a purpose.

\subsubsection{The {\tt square-matrixp} function}
\label{subsubsec:square-matrixp}
\begin{verbatim}
(defun square-matrixp (m)
  (unless (matrixp m) 
          (error "not a matrix - ~a" m))
  (let ((dim (array-dimensions m)))
    (eql (select dim 0) (select dim 1))))
\end{verbatim}
This functions returns {\tt t} if the matrix {\tt m} is a square
matrix; otherwise, it returns {\tt nil}.  It just checks if the
dimensions of the matrix are equal.

\subsubsection{The {\tt stochastic-matrixp} function}
\label{subsubsec:stochastic-matrixp}
\begin{verbatim}
(defun stochastic-matrixp (m)
  (unless (square-matrixp m)
          (return))
  (dolist (row (row-list m) t)
          (unless (< (abs (- (sum row) 1)) 
                     machine-epsilon)
                  (return))))
\end{verbatim}
This function returns {\tt t} if {\tt m} is a stochastic matrix;
otherwise, it returns {\tt nil}.  Note the use of the global variable
{\tt machine-epsilon} to check whether each row sums to $1$. This is
necessary since we are dealing with machine representation of
probabilities. 


\subsubsection{The {\tt doubly-stochastic-matrixp} function}
\label{subsubsec:doubly-stochastic-matrixp}
\begin{verbatim}
(defun doubly-stochastic-matrixp (m)
  (and (stochastic-matrixp m)
       (stochastic-matrixp (transpose m))))
\end{verbatim}
This method just returns {\tt t} if the matrix is doubly stochastic,
{\tt nil} otherwise.

