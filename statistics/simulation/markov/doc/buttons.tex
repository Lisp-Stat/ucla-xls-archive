The file {\tt buttons.lsp} contains the code for the design of the
control buttons.  These buttons are rectangular and sit in the
topmost row of the window.  The user may click on any of these buttons
to invoke actions associated with the buttons.  It
is also possible for a button to have no action to perform, as is the
case with the {\tt Time} button, which merely indicates the time.

The {\tt button-obj-proto} is a prototype is defined as follows.
\begin{verbatim}
(defproto button-obj-proto '(boss title loc-and-size 
                                  click-action)) 
\end{verbatim}
The slot {\tt boss} is the identity of the window in which the button
will sit.  The {\tt title} slot holds the title for the button.  The
{\tt loc-and-size} slot holds a list of four values: the $x$, $y$
coordinates of the top left corner of the button and the width and
height of the button in the {\em boss\/} window. The {\tt
  click-action} slot holds an action to be performed when the mouse
is clicked on the button.  This value of this slot is actually {\tt
  funcall}ed to perform the action. 

\subsubsection{The {\tt :isnew} method}
\label{subsubsec:buttons-isnew}
\begin{verbatim}
(defmeth button-obj-proto :isnew (title loc-and-size action)
  (setf (slot-value 'title) title)
  (setf (slot-value 'loc-and-size) loc-and-size)
  (setf (slot-value 'click-action) action))
\end{verbatim}
The {\tt :isnew} method for {\tt button-obj-proto} requires three
arguments, a title string, a list of four values containing the $x$
and $y$ coordinates of the top left corner of the button, the width of the
button and the height of the button in the window where it is to be
installed, and a function that will perform some action when the mouse
is clicked on the button.  These arguments are stored in the
appropriate slots for future use. 

\subsubsection{The {\tt :title} method}
\label{subsubsec:buttons-title}
\begin{verbatim}
(defmeth button-obj-proto :title (&optional title)
  (if title
      (setf (slot-value 'title) title)
    (slot-value 'title)))
\end{verbatim}
This method sets and retrieves the title for the button. The title is
set if it is supplied.

\subsubsection{The {\tt :do-click} method}
\label{subsubsec:buttons-do-click}
\begin{verbatim}
(defmeth button-obj-proto :do-click ()
  (funcall (slot-value 'click-action)))
\end{verbatim}
This method performs the action associated with the button.  It is
called when the user clicks the mouse on the button.  Since the action
to be performed is stored in the {\tt click-action} slot, this method
just {\tt funcall}s the value in that slot.

\subsubsection{The {\tt :loc-and-size} method}
\label{subsubsec:buttons-loc-and-size}
\begin{verbatim}
(defmeth button-obj-proto :loc-and-size (&optional loc-size)
  (if loc-size
      (setf (slot-value 'loc-and-size) loc-size)
    (slot-value 'loc-and-size)))
\end{verbatim}
This method sets and retrieves the value of the slot {\tt loc-size}.
Recall that {\tt loc-size} should be a list of four values.

\subsubsection{The {\tt :redraw} method}
\label{subsubsec:buttons-redraw}
\begin{verbatim}
(defmeth button-obj-proto :redraw (window)
  (let* ((dc (send window :draw-color))
         (ls (send self :loc-and-size))
         (x (first ls))
         (y (second ls))
         (w (third ls))
         (h (fourth ls))
         (cx (+ x (round (* .5 w))))
         (cy (+ y (round (* .5 h)))))

    (send window :draw-color *button-back-color*)
    (send window :paint-rect x y w h)
    (send window :draw-color *button-text-color*)
    (send window :draw-text
          (slot-value 'title)
          cx (+ cy (round (* .5 
                             (send window :text-ascent)))) 1 0)
    (send window :draw-color dc)
    (send window :frame-rect x y w h)
    nil))
\end{verbatim}
This method is in charge of redrawing the button in the window where
it is installed as appropriate.  The colors to be used are available
in the global variables {\tt *button-back-color} and {\tt
  *button-text-color} denoting the background color and the text color
respectively. The window is sent the message to draw a framed
rectangle whose background color is {\tt *button-back-color*} and then
the text is drawn centered horizontally in the button using {\tt
  *button-text-color*}. The default drawing color is restored at the
end. 

\subsubsection{The {\tt :my-click} method}
\label{subsubsec:buttons-my-click}
\begin{verbatim}
(defmeth button-obj-proto :my-click (x y m1 m2)
  (let* ((ls (send self :loc-and-size))
         (left-x (first ls))
         (bot-y (second ls))
         (rt-x (+ left-x (third ls)))
         (top-y (+ bot-y (fourth ls))))
    (if (and (< left-x x rt-x) (< bot-y y top-y))
        t
      nil)))
\end{verbatim}
This method is used to detect whether the user clicked the mouse in
the button. All it does is check whether the coordinates where the
button was clicked in the window are inside the rectangle encompassing
the button.  If so, it returns {\tt t}, otherwise, it returns {\tt
  nil}.
