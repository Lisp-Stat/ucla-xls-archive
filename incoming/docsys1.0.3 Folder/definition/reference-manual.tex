\input /home/comet/rga6711/xlisp/definition/lispdef.tex
\hfuzz=5pt


\beginDefinition


\DefNameBox{df:*definition-hfuzz*}{Parameter}
\beginDocumentation
Amount of slop, or Hfuzz (in points), in TeX line breaking
  algorithm.  Increasting this value will encourage TeX to stop
  complaining about overfull hboxes.\endDocumentation
\Usage{0}{\cd{df:*definition-hfuzz*}}\endUsage

\Source{ /home/comet/rga6711/xlisp/definition/tex-print.lsp}
\endDefinition


\beginDefinition


\DefNameBox{:affiliation -\> (df:author-doc-proto)}{Message}
\beginDocumentation
Author's Institution.\endDocumentation
\Usage{14}{\cd{(send <obj> :affiliation &optional new-value)}}\endUsage
\beginArguments
\argument{\cd{new-value}}\typeArg{\cd{String}}\endArg
\endArguments

\Source{ /home/comet/rga6711/xlisp/definition/project-doc.lsp}
\endDefinition


\beginDefinition


\DefNameBox{df:author-doc-proto}{Class}
\beginDocumentation
This object holds information about an author or reviewer of the code.\endDocumentation
\Usage{35}{\cd{(send df:author-doc-proto :new  &key name affiliation email home-page)}}\endUsage
\beginSlots
\slot{\cd{name}}\initargsSlot\cd{(:Name)}\typeSlot{\cd{String}}\docSlot{Author's name.}\endSlot
\slot{\cd{affiliation}}\initargsSlot\cd{(:Affiliation)}\typeSlot{\cd{String}}\docSlot{Author's Institution.}\endSlot
\slot{\cd{email}}\initargsSlot\cd{(:Email)}\typeSlot{\cd{String}}\docSlot{Author's email address.}\endSlot
\slot{\cd{home-page}}\initargsSlot\cd{(:Home-Page)}\typeSlot{\cd{String}}\docSlot{Author's home-page URL.}\endSlot
\endSlots
\beginMethods{\nextInList
 :Home-Page\nextInList
 :Email\nextInList
 :Affiliation\nextInList
 :Name\nextInList
 :Isnew\nextInList
 Xlos:Shared-Initialize}\endMethods

\Source{ /home/comet/rga6711/xlisp/definition/project-doc.lsp}
\endDefinition


\beginDefinition


\DefNameBox{:authors -\> (df:project-doc-proto)}{Message}
\beginDocumentation
A list of author-doc-proto describing the author(s) of the code.\endDocumentation
\Usage{10}{\cd{(send <obj> :authors &optional new-value)}}\endUsage
\beginArguments
\argument{\cd{new-value}}\typeArg{\cd{List}}\endArg
\endArguments

\Source{ /home/comet/rga6711/xlisp/definition/project-doc.lsp}
\endDefinition


\beginDefinition


\DefNameBox{df:build-project-definitions}{Generic Function}
\beginDocumentation
This automatically extracts the exported object definitions from
the source files for a project and fills in the appropriate
documenation fields\endDocumentation
\Usage{30}{\cd{(df:build-project-definitions doc)}}\endUsage
\beginArguments
\argument{\cd{doc}}\typeArg{\cd{Definitions:Project-Doc-Proto}}\endArg
\endArguments
\beginReturn
\itemReturn \simpleReturn{\cd{doc}}\endsReturn
\itemReturn \simpleReturn{\cd{defs}}\endsReturn
\endReturn

\Source{ /home/comet/rga6711/xlisp/definition/project-doc.lsp}
\endDefinition


\beginDefinition


\DefNameBox{df:build-project-definitions Df:Project-Doc-Proto}{Primary Method}
\beginDocumentation
This automatically extracts the exported object definitions from
the source files for a project and fills in the appropriate
documenation fields\endDocumentation
\Usage{30}{\cd{(df:build-project-definitions doc)}}\endUsage
\beginArguments
\argument{\cd{doc}}\typeArg{\cd{Definitions:Project-Doc-Proto}}\endArg
\endArguments
\beginReturn
\itemReturn \simpleReturn{\cd{doc}}\endsReturn
\itemReturn \simpleReturn{\cd{defs}}\endsReturn
\endReturn

\Source{ /home/comet/rga6711/xlisp/definition/project-doc.lsp}
\endDefinition


\beginDefinition


\DefNameBox{:change-history -\> (df:project-doc-proto)}{Message}
\beginDocumentation
A list of strings, each
describing the changes since the last version.\endDocumentation
\Usage{17}{\cd{(send <obj> :change-history &optional new-value)}}\endUsage
\beginArguments
\argument{\cd{new-value}}\typeArg{\cd{List}}\endArg
\endArguments

\Source{ /home/comet/rga6711/xlisp/definition/project-doc.lsp}
\endDefinition


\beginDefinition


\DefNameBox{df:class-definition}{Class}
\beginDocumentation
A definition class for \<defclass\>.\endDocumentation
\Usage{35}{\cd{(send df:class-definition :new )}}\endUsage

\Parents{\nextInList Df:User-Type-Definition}\endParent
\Children{\nextInList
 Df:Class-Definition\nextInList
 Df:Condition-Definition}\endChildren
\beginMethods{\nextInList
 Df::Print-Html-Definition-Slot-Types\nextInList
 Df::Print-Tex-Definition-Slot-Types\nextInList
 Df:Make-Subdefinitions\nextInList
 Df:Make-Slot-Accessor-Definitions\nextInList
 Df:Definition-Usage\nextInList
 Df:Definition-Children\nextInList
 Df:Definition-Parents\nextInList
 Df::Definition-Methods\nextInList
 Df:Definition-Slots\nextInList
 Df:Definition-Documentation\nextInList
 Df:Definition-Definee\nextInList
 Df:Definition-Class-Nice-Name\nextInList
 :Isnew\nextInList
 Xlos:Shared-Initialize}\endMethods

\Source{ /home/comet/rga6711/xlisp/definition/type.lsp}
\endDefinition


\beginDefinition


\DefNameBox{df:condition-definition}{Class}
\beginDocumentation
A definition class for \<define-condition\>.\endDocumentation
\Usage{39}{\cd{(send df:condition-definition :new )}}\endUsage

\Parents{\nextInList Df:Class-Definition}\endParent
\beginMethods{\nextInList
 Df:Definition-Usage\nextInList
 Df:Definition-Class-Nice-Name\nextInList
 Df:Definition-Children\nextInList
 Df:Definition-Parents\nextInList
 Df:Definition-Slots\nextInList
 Df:Definition-Documentation\nextInList
 Df:Definition-Definee\nextInList
 :Isnew\nextInList
 Xlos:Shared-Initialize}\endMethods

\Source{ /home/comet/rga6711/xlisp/definition/condition.lsp}
\endDefinition


\beginDefinition


\DefNameBox{df:constant-definition}{Class}
\beginDocumentation
A definition class for \<defconstant\>.\endDocumentation
\Usage{38}{\cd{(send df:constant-definition :new )}}\endUsage

\Parents{\nextInList Df:Global-Variable-Definition}\endParent
\beginMethods{\nextInList
 Df:Definition-Class-Nice-Name\nextInList
 :Isnew\nextInList
 Xlos:Shared-Initialize}\endMethods

\Source{ /home/comet/rga6711/xlisp/definition/global.lsp}
\endDefinition


\beginDefinition


\DefNameBox{:copyright -\> (df:project-doc-proto)}{Message}
\beginDocumentation
String describing
copyright information and licence to include in library including any
restrictions on use (i.e., educational and research only, or no
redistribution without permission.\endDocumentation
\Usage{12}{\cd{(send <obj> :copyright &optional new-value)}}\endUsage
\beginArguments
\argument{\cd{new-value}}\typeArg{\cd{String}}\endArg
\endArguments

\Source{ /home/comet/rga6711/xlisp/definition/project-doc.lsp}
\endDefinition


\beginDefinition


\DefNameBox{:date -\> (df:version-doc-proto)}{Message}
\beginDocumentation
The date of the release.\endDocumentation
\Usage{7}{\cd{(send <obj> :date &optional new-value)}}\endUsage
\beginArguments
\argument{\cd{new-value}}\typeArg{\cd{String}}\endArg
\endArguments

\Source{ /home/comet/rga6711/xlisp/definition/project-doc.lsp}
\endDefinition


\beginDefinition


\DefNameBox{df:declare-check}{Macro}
\beginDocumentation
This macro generates type checking forms and has a syntax like \<declare\>.
Unfortunately, we can't easily have it also generate the declarations.\endDocumentation
\Usage{18}{\cd{(df:declare-check
 &rest decls)}}\endUsage

\Source{ /home/comet/rga6711/xlisp/definition/defs.lsp}
\endDefinition


\beginDefinition


\DefNameBox{setf df:definer-class}{Setf}
\beginDocumentation
Assign a definer class name to a definer symbol.\endDocumentation
\Usage{24}{\cd{(setf (df:definer-class definer) class-name)}}\endUsage

\Source{ /home/comet/rga6711/xlisp/definition/build.lsp}
\endDefinition


\beginDefinition


\DefNameBox{df:definer-class}{Function}
\beginDocumentation
Returns the name of the definition class for the \<definer\> symbol.

The predefined definer classes are: 
Class-Definition for defclass,
Constant-Definition for defconstant,
Function-Definition for defun,
Generic-Function-Definition for defgeneric,
Macro-Definition for defmacro,
Method-Definition for defmethod,
Package-Definition for defpackage,
Parameter-Definition for defparameter,
Setf-Definition for defsetf,
Structure-Definition for defstruct,
Type-Definition for deftype,
and
Variable-Definition for defvar.\endDocumentation
\Usage{18}{\cd{(df:definer-class definer)}}\endUsage
\beginArguments
\argument{\cd{definer}}\typeArg{\cd{Symbol}}\endArg
\endArguments
\beginReturn
\singleReturn \complexReturn{\cd{class-name}}\midReturn{\cd{Symbol}}\endcReturn
\endReturn

\Source{ /home/comet/rga6711/xlisp/definition/build.lsp}
\endDefinition


\beginDefinition


\DefNameBox{df:definition}{Class}
\beginDocumentation
Abstract root class for definition objects.\endDocumentation
\Usage{29}{\cd{(send df:definition :new  &key definition-form definition-path)}}\endUsage
\beginSlots
\slot{\cd{definition-form}}\initargsSlot\cd{(:Definition-Form)}\typeSlot{\cd{List}}\docSlot{The Lisp form that results from reading the definition.}\endSlot
\slot{\cd{definition-path}}\initargsSlot\cd{(:Definition-Path)}\typeSlot{\cd{(Or Pathname String)}}\docSlot{Pathanme of the file from which the definition was read.}\endSlot
\endSlots
\Children{\nextInList
 Df:Class-Definition\nextInList
 Df:Generic-Function-Definition\nextInList
 Df:Global-Variable-Definition\nextInList
 Df:Method-Definition\nextInList
 Df:User-Type-Definition\nextInList
 Df:Type-Definition\nextInList
 Df:Prototype-Definition\nextInList
 Df:Setf-Definition\nextInList
 Df:Condition-Definition\nextInList
 Df:Structure-Definition\nextInList
 Df:Class-Definition\nextInList
 Df:Message-Definition\nextInList
 Df:Macro-Definition\nextInList
 Df:Parameter-Definition\nextInList
 Df:Package-Definition\nextInList
 Df:Variable-Definition\nextInList
 Df:Constant-Definition\nextInList
 Df:Function-Definition\nextInList
 Df:Lambda-List-Definition}\endChildren
\beginMethods{\nextInList
 Df:Print-Html-Definition\nextInList
 Df::Print-Html-Definition-Signals\nextInList
 Df::Print-Html-Definition-Methods\nextInList
 Df::Print-Html-Definition-Source-Path\nextInList
 Df::Print-Html-Definition-Documentation\nextInList
 Df::Print-Html-Definition-Children\nextInList
 Df::Print-Html-Definition-Parents\nextInList
 Df::Print-Html-Definition-Returns\nextInList
 Df::Print-Html-Definition-Slot-Types\nextInList
 Df::Print-Html-Definition-Arg-Types\nextInList
 Df::Print-Html-Definition-Usage\nextInList
 Df::Print-Html-Definition-Headline\nextInList
 Df:Print-Tex-Definition\nextInList
 Df::Print-Tex-Definition-Signals\nextInList
 Df::Print-Tex-Definition-Methods\nextInList
 Df::Print-Tex-Definition-Source-Path\nextInList
 Df::Print-Tex-Definition-Documentation\nextInList
 Df::Print-Tex-Definition-Children\nextInList
 Df::Print-Tex-Definition-Parents\nextInList
 Df::Print-Tex-Definition-Returns\nextInList
 Df::Print-Tex-Definition-Slot-Types\nextInList
 Df::Print-Tex-Definition-Arg-Types\nextInList
 Df::Print-Tex-Definition-Usage\nextInList
 Df::Print-Tex-Definition-Headline\nextInList
 :Print\nextInList
 Df:Definition\<\nextInList
 Df:Make-Subdefinitions\nextInList
 Df:Definition-Class-Nice-Name\nextInList
 Df::Definition-Methods\nextInList
 Df:Definition-Children\nextInList
 Df:Definition-Parents\nextInList
 Df:Definition-Slots\nextInList
 Df:Definition-Declarations\nextInList
 Df:Definition-Documentation\nextInList
 Df:Definition-Arg-Types\nextInList
 Df:Definition-Lambda-List\nextInList
 Df:Definition-Usage\nextInList
 Df:Definition-Symbol\nextInList
 Df::Definition-Name-\>Tag\nextInList
 Df:Definition-Name-\>String\nextInList
 Df:Definition-Name\nextInList
 Df:Definition-Definee\nextInList
 Df::Definition-Path\nextInList
 Df:Definition-Form\nextInList
 :Isnew\nextInList
 Xlos:Shared-Initialize}\endMethods

\Source{ /home/comet/rga6711/xlisp/definition/definition.lsp}
\endDefinition


\beginDefinition


\DefNameBox{df:definition}{Primary Method}
\beginDocumentation
The Lisp form that results from reading the definition.\endDocumentation
\Usage{15}{\cd{(df:definition)}}\endUsage
\beginReturn
\singleReturn \typeReturn{\cd{List}}\endtReturn
\endReturn

\Source{ /home/comet/rga6711/xlisp/definition/definition.lsp}
\endDefinition


\beginDefinition


\DefNameBox{df:definition}{Primary Method}
\beginDocumentation
Pathanme of the file from which the definition was read.\endDocumentation
\Usage{15}{\cd{(df:definition)}}\endUsage
\beginReturn
\singleReturn \typeReturn{\cd{(Or Pathname String)}}\endtReturn
\endReturn

\Source{ /home/comet/rga6711/xlisp/definition/definition.lsp}
\endDefinition


\beginDefinition


\DefNameBox{df:definition-alpha\<}{Function}
\beginDocumentation
Order definitions alphabetically by name.
If the names are the same, call \<definition\<\>
to resolve the ambiguity.\endDocumentation
\Usage{22}{\cd{(df:definition-alpha< def0 def1)}}\endUsage
\beginArguments
\argument{\cd{def0}}\typeArg{\cd{Definitions:Definition}}\endArg
\argument{\cd{def1}}\typeArg{\cd{Definitions:Definition}}\endArg
\endArguments
\beginReturn
\singleReturn \typeReturn{\cd{(Member T Nil)}}\endtReturn
\endReturn

\Source{ /home/comet/rga6711/xlisp/definition/sort.lsp}
\endDefinition


\beginDefinition


\DefNameBox{df:definition-arg-types}{Generic Function}
\beginDocumentation
Returns a list of arg --- type pairs for the definition.\endDocumentation
\Usage{25}{\cd{(df:definition-arg-types def)}}\endUsage
\beginArguments
\argument{\cd{def}}\typeArg{\cd{Definitions:Definition}}\endArg
\endArguments
\beginReturn
\singleReturn \typeReturn{\cd{List}}\endtReturn
\endReturn

\Source{ /home/comet/rga6711/xlisp/definition/definition.lsp}
\endDefinition


\beginDefinition


\DefNameBox{df:definition-arg-types Df:Definition}{Primary Method}
\beginDocumentation
The default method returns ().\endDocumentation
\Usage{25}{\cd{(df:definition-arg-types def)}}\endUsage
\beginArguments
\argument{\cd{def}}\typeArg{\cd{Definitions:Definition}}\endArg
\endArguments
\beginReturn
\singleReturn \simpleReturn{\cd{nil}}\endsReturn
\endReturn

\Source{ /home/comet/rga6711/xlisp/definition/definition.lsp}
\endDefinition


\beginDefinition


\DefNameBox{df:definition-arg-types Df:Lambda-List-Definition}{Primary Method}
\beginDocumentation
Extract a list of lists of length 2, where each sublist is a name
--- type pair. The types are gotten first from the arg specializers,
if there are any, and are overridden by any top level type
declarations.\endDocumentation
\Usage{25}{\cd{(df:definition-arg-types def)}}\endUsage
\beginArguments
\argument{\cd{def}}\typeArg{\cd{Definitions:Lambda-List-Definition}}\endArg
\endArguments
\beginReturn
\singleReturn \typeReturn{\cd{List}}\endtReturn
\endReturn

\Source{ /home/comet/rga6711/xlisp/definition/fun.lsp}
\endDefinition


\beginDefinition


\DefNameBox{df:definition-arg-types Df:Macro-Definition}{Primary Method}
\beginDocumentation
I haven't figured out a good way to get at the equivalent of arg
type declarations for macros, so this just returns nil.\endDocumentation
\Usage{25}{\cd{(df:definition-arg-types def)}}\endUsage
\beginArguments
\argument{\cd{def}}\typeArg{\cd{Definitions:Macro-Definition}}\endArg
\endArguments
\beginReturn
\singleReturn \typeReturn{\cd{String}}\endtReturn
\endReturn

\Source{ /home/comet/rga6711/xlisp/definition/fun.lsp}
\endDefinition


\beginDefinition


\DefNameBox{df:definition-arg-types Df:Setf-Definition}{Primary Method}
\beginDocumentation
I haven't figured out a good way to get at the equivalent of arg
type declarations for defsetf, so this just returns nil.\endDocumentation
\Usage{25}{\cd{(df:definition-arg-types def)}}\endUsage
\beginArguments
\argument{\cd{def}}\typeArg{\cd{Definitions:Setf-Definition}}\endArg
\endArguments
\beginReturn
\singleReturn \simpleReturn{\cd{nil}}\endsReturn
\endReturn

\Source{ /home/comet/rga6711/xlisp/definition/fun.lsp}
\endDefinition


\beginDefinition


\DefNameBox{df:definition-body}{Generic Function}
\beginDocumentation
This generic function extracts the body from a
  lambda list definition.  It is used by a number of other methods to
  provide for different \&body syntaxes.\endDocumentation
\Usage{20}{\cd{(df:definition-body def)}}\endUsage
\beginArguments
\argument{\cd{def}}\typeArg{\cd{Definitions:Lambda-List-Definition}}\endArg
\endArguments
\beginReturn
\singleReturn \complexReturn{\cd{form}}\midReturn{\cd{List}}\endcReturn
\endReturn

\Source{ /home/comet/rga6711/xlisp/definition/fun.lsp}
\endDefinition


\beginDefinition


\DefNameBox{df:definition-body Df:Lambda-List-Definition}{Primary Method}
\beginDocumentation
Default method extracts (nthcdr 3)\endDocumentation
\Usage{20}{\cd{(df:definition-body def)}}\endUsage
\beginArguments
\argument{\cd{def}}\typeArg{\cd{Definitions:Lambda-List-Definition}}\endArg
\endArguments

\Source{ /home/comet/rga6711/xlisp/definition/fun.lsp}
\endDefinition


\beginDefinition


\DefNameBox{df:definition-body Df:Method-Definition}{Primary Method}
\Usage{20}{\cd{(df:definition-body def)}}\endUsage
\beginArguments
\argument{\cd{def}}\typeArg{\cd{Definitions:Method-Definition}}\endArg
\endArguments

\Source{ /home/comet/rga6711/xlisp/definition/fun.lsp}
\endDefinition


\beginDefinition


\DefNameBox{df:definition-body Df:Message-Definition}{Primary Method}
\beginDocumentation
Default method extracts (nthcdr 3)\endDocumentation
\Usage{20}{\cd{(df:definition-body def)}}\endUsage
\beginArguments
\argument{\cd{def}}\typeArg{\cd{Definitions:Message-Definition}}\endArg
\endArguments

\Source{ /home/comet/rga6711/xlisp/definition/xls-object.lsp}
\endDefinition


\beginDefinition


\DefNameBox{df:definition-children}{Generic Function}
\beginDocumentation
Returns a list of names of children (eg. direct
subclasses) of the definition.  For new Definition classes, it may
return anything that is reasonably thought of as a name of a ``child''
of the definition.  
  If package is non-nil, will only search for definitions in a given
(exported) package.\endDocumentation
\Usage{24}{\cd{(df:definition-children def &optional package)}}\endUsage
\beginArguments
\argument{\cd{def}}\typeArg{\cd{Definitions:Definition}}\endArg
\argument{\cd{package}}\typeArg{\cd{(Or Symbol String Package)}}\endArg
\endArguments
\beginReturn
\singleReturn \typeReturn{\cd{List}}\endtReturn
\endReturn

\Source{ /home/comet/rga6711/xlisp/definition/definition.lsp}
\endDefinition


\beginDefinition


\DefNameBox{df:definition-children Df:Definition}{Primary Method}
\beginDocumentation
The default method returns ().\endDocumentation
\Usage{24}{\cd{(df:definition-children def &optional package)}}\endUsage
\beginArguments
\argument{\cd{def}}\typeArg{\cd{Definitions:Definition}}\endArg
\argument{\cd{package}}\typeArg{\cd{T}}\endArg
\endArguments
\beginReturn
\singleReturn \simpleReturn{\cd{nil}}\endsReturn
\endReturn

\Source{ /home/comet/rga6711/xlisp/definition/definition.lsp}
\endDefinition


\beginDefinition


\DefNameBox{df:definition-children Df:Class-Definition}{Primary Method}
\beginDocumentation
Returns the names of the direct subclasses. This method requires
the class definition to be loaded and returns the names of all direct
subclasses, not just those that have corresponding definition
objects.\endDocumentation
\Usage{24}{\cd{(df:definition-children def &optional package)}}\endUsage
\beginArguments
\argument{\cd{def}}\typeArg{\cd{Definitions:Class-Definition}}\endArg
\argument{\cd{package}}\typeArg{\cd{T}}\endArg
\endArguments
\beginReturn
\singleReturn \typeReturn{\cd{List}}\endtReturn
\endReturn

\Source{ /home/comet/rga6711/xlisp/definition/type.lsp}
\endDefinition


\beginDefinition


\DefNameBox{df:definition-children Df:Condition-Definition}{Primary Method}
\beginDocumentation
Returns the names of the direct subclasses. This method requires
the class definition to be loaded and returns the names of all direct
subclasses, not just those that have corresponding definition
objects.\endDocumentation
\Usage{24}{\cd{(df:definition-children def &optional package)}}\endUsage
\beginArguments
\argument{\cd{def}}\typeArg{\cd{Definitions:Condition-Definition}}\endArg
\argument{\cd{package}}\typeArg{\cd{T}}\endArg
\endArguments
\beginReturn
\singleReturn \typeReturn{\cd{List}}\endtReturn
\endReturn

\Source{ /home/comet/rga6711/xlisp/definition/condition.lsp}
\endDefinition


\beginDefinition


\DefNameBox{df:definition-children Df:Prototype-Definition}{Primary Method}
\beginDocumentation
Returns the names of the direct subclasses. This method requires
the class definition to be loaded and returns the names of all direct
subclasses, not just those that have corresponding definition
objects.\endDocumentation
\Usage{24}{\cd{(df:definition-children def &optional package)}}\endUsage
\beginArguments
\argument{\cd{def}}\typeArg{\cd{Definitions:Prototype-Definition}}\endArg
\argument{\cd{package}}\typeArg{\cd{T}}\endArg
\endArguments
\beginReturn
\singleReturn \typeReturn{\cd{List}}\endtReturn
\endReturn

\Source{ /home/comet/rga6711/xlisp/definition/xls-object.lsp}
\endDefinition


\beginDefinition


\DefNameBox{df:definition-class-nice-name}{Generic Function}
\beginDocumentation
Returns a short string for the type of the definition,
eg. ``Class'' for \<Class-Definition\>.\endDocumentation
\Usage{31}{\cd{(df:definition-class-nice-name def)}}\endUsage
\beginArguments
\argument{\cd{def}}\typeArg{\cd{Definitions:Definition}}\endArg
\endArguments
\beginReturn
\singleReturn \typeReturn{\cd{String}}\endtReturn
\endReturn

\Source{ /home/comet/rga6711/xlisp/definition/definition.lsp}
\endDefinition


\beginDefinition


\DefNameBox{df:definition-class-nice-name Df:Definition}{Primary Method}
\Usage{31}{\cd{(df:definition-class-nice-name def)}}\endUsage
\beginArguments
\argument{\cd{def}}\typeArg{\cd{Definitions:Definition}}\endArg
\endArguments
\beginReturn
\singleReturn \simpleReturn{\cd{Unknown-Definition-Type}}\endsReturn
\endReturn

\Source{ /home/comet/rga6711/xlisp/definition/definition.lsp}
\endDefinition


\beginDefinition


\DefNameBox{df:definition-class-nice-name Df:Class-Definition}{Primary Method}
\Usage{31}{\cd{(df:definition-class-nice-name def)}}\endUsage
\beginArguments
\argument{\cd{def}}\typeArg{\cd{Definitions:Class-Definition}}\endArg
\endArguments
\beginReturn
\singleReturn \simpleReturn{\cd{Class}}\endsReturn
\endReturn

\Source{ /home/comet/rga6711/xlisp/definition/type.lsp}
\endDefinition


\beginDefinition


\DefNameBox{df:definition-class-nice-name Df:Condition-Definition}{Primary Method}
\Usage{31}{\cd{(df:definition-class-nice-name def)}}\endUsage
\beginArguments
\argument{\cd{def}}\typeArg{\cd{Definitions:Condition-Definition}}\endArg
\endArguments
\beginReturn
\singleReturn \simpleReturn{\cd{KR:Schema}}\endsReturn
\endReturn

\Source{ /home/comet/rga6711/xlisp/definition/condition.lsp}
\endDefinition


\beginDefinition


\DefNameBox{df:definition-class-nice-name Df:Constant-Definition}{Primary Method}
\Usage{31}{\cd{(df:definition-class-nice-name def)}}\endUsage
\beginArguments
\argument{\cd{def}}\typeArg{\cd{Definitions:Constant-Definition}}\endArg
\endArguments
\beginReturn
\singleReturn \simpleReturn{\cd{Constant}}\endsReturn
\endReturn

\Source{ /home/comet/rga6711/xlisp/definition/global.lsp}
\endDefinition


\beginDefinition


\DefNameBox{df:definition-class-nice-name Df:Function-Definition}{Primary Method}
\Usage{31}{\cd{(df:definition-class-nice-name def)}}\endUsage
\beginArguments
\argument{\cd{def}}\typeArg{\cd{Definitions:Function-Definition}}\endArg
\endArguments
\beginReturn
\singleReturn \simpleReturn{\cd{Function}}\endsReturn
\endReturn

\Source{ /home/comet/rga6711/xlisp/definition/fun.lsp}
\endDefinition


\beginDefinition


\DefNameBox{df:definition-class-nice-name Df:Generic-Function-Definition}{Primary Method}
\Usage{31}{\cd{(df:definition-class-nice-name def)}}\endUsage
\beginArguments
\argument{\cd{def}}\typeArg{\cd{Definitions:Generic-Function-Definition}}\endArg
\endArguments
\beginReturn
\singleReturn \simpleReturn{\cd{nil}}\endsReturn
\endReturn

\Source{ /home/comet/rga6711/xlisp/definition/fun.lsp}
\endDefinition


\beginDefinition


\DefNameBox{df:definition-class-nice-name Df:Macro-Definition}{Primary Method}
\Usage{31}{\cd{(df:definition-class-nice-name def)}}\endUsage
\beginArguments
\argument{\cd{def}}\typeArg{\cd{Definitions:Macro-Definition}}\endArg
\endArguments
\beginReturn
\singleReturn \simpleReturn{\cd{Macro}}\endsReturn
\endReturn

\Source{ /home/comet/rga6711/xlisp/definition/fun.lsp}
\endDefinition


\beginDefinition


\DefNameBox{df:definition-class-nice-name Df:Method-Definition}{Primary Method}
\Usage{31}{\cd{(df:definition-class-nice-name def)}}\endUsage
\beginArguments
\argument{\cd{def}}\typeArg{\cd{Definitions:Method-Definition}}\endArg
\endArguments
\beginReturn
\singleReturn \typeReturn{\cd{String}}\endtReturn
\endReturn

\Source{ /home/comet/rga6711/xlisp/definition/fun.lsp}
\endDefinition


\beginDefinition


\DefNameBox{df:definition-class-nice-name Df:Message-Definition}{Primary Method}
\Usage{31}{\cd{(df:definition-class-nice-name def)}}\endUsage
\beginArguments
\argument{\cd{def}}\typeArg{\cd{Definitions:Message-Definition}}\endArg
\endArguments
\beginReturn
\singleReturn \simpleReturn{\cd{Message}}\endsReturn
\endReturn

\Source{ /home/comet/rga6711/xlisp/definition/xls-object.lsp}
\endDefinition


\beginDefinition


\DefNameBox{df:definition-class-nice-name Df:Package-Definition}{Primary Method}
\Usage{31}{\cd{(df:definition-class-nice-name def)}}\endUsage
\beginArguments
\argument{\cd{def}}\typeArg{\cd{Definitions:Package-Definition}}\endArg
\endArguments
\beginReturn
\singleReturn \simpleReturn{\cd{Package}}\endsReturn
\endReturn

\Source{ /home/comet/rga6711/xlisp/definition/pack.lsp}
\endDefinition


\beginDefinition


\DefNameBox{df:definition-class-nice-name Df:Parameter-Definition}{Primary Method}
\Usage{31}{\cd{(df:definition-class-nice-name def)}}\endUsage
\beginArguments
\argument{\cd{def}}\typeArg{\cd{Definitions:Parameter-Definition}}\endArg
\endArguments
\beginReturn
\singleReturn \simpleReturn{\cd{Parameter}}\endsReturn
\endReturn

\Source{ /home/comet/rga6711/xlisp/definition/global.lsp}
\endDefinition


\beginDefinition


\DefNameBox{df:definition-class-nice-name Df:Prototype-Definition}{Primary Method}
\Usage{31}{\cd{(df:definition-class-nice-name def)}}\endUsage
\beginArguments
\argument{\cd{def}}\typeArg{\cd{Definitions:Prototype-Definition}}\endArg
\endArguments
\beginReturn
\singleReturn \simpleReturn{\cd{Prototype}}\endsReturn
\endReturn

\Source{ /home/comet/rga6711/xlisp/definition/xls-object.lsp}
\endDefinition


\beginDefinition


\DefNameBox{df:definition-class-nice-name Df:Setf-Definition}{Primary Method}
\Usage{31}{\cd{(df:definition-class-nice-name def)}}\endUsage
\beginArguments
\argument{\cd{def}}\typeArg{\cd{Definitions:Setf-Definition}}\endArg
\endArguments
\beginReturn
\singleReturn \simpleReturn{\cd{Setf}}\endsReturn
\endReturn

\Source{ /home/comet/rga6711/xlisp/definition/fun.lsp}
\endDefinition


\beginDefinition


\DefNameBox{df:definition-class-nice-name Df:Structure-Definition}{Primary Method}
\Usage{31}{\cd{(df:definition-class-nice-name def)}}\endUsage
\beginArguments
\argument{\cd{def}}\typeArg{\cd{Definitions:Structure-Definition}}\endArg
\endArguments
\beginReturn
\singleReturn \simpleReturn{\cd{Structure}}\endsReturn
\endReturn

\Source{ /home/comet/rga6711/xlisp/definition/type.lsp}
\endDefinition


\beginDefinition


\DefNameBox{df:definition-class-nice-name Df:Type-Definition}{Primary Method}
\Usage{31}{\cd{(df:definition-class-nice-name def)}}\endUsage
\beginArguments
\argument{\cd{def}}\typeArg{\cd{Definitions:Type-Definition}}\endArg
\endArguments
\beginReturn
\singleReturn \simpleReturn{\cd{Type}}\endsReturn
\endReturn

\Source{ /home/comet/rga6711/xlisp/definition/type.lsp}
\endDefinition


\beginDefinition


\DefNameBox{df:definition-class-nice-name Df:Variable-Definition}{Primary Method}
\Usage{31}{\cd{(df:definition-class-nice-name def)}}\endUsage
\beginArguments
\argument{\cd{def}}\typeArg{\cd{Definitions:Variable-Definition}}\endArg
\endArguments
\beginReturn
\singleReturn \simpleReturn{\cd{Variable}}\endsReturn
\endReturn

\Source{ /home/comet/rga6711/xlisp/definition/global.lsp}
\endDefinition


\beginDefinition


\DefNameBox{df:definition-declarations}{Generic Function}
\beginDocumentation
Returns all the decl-specs associated with the entire definition.
A decl-spec is a list like (type Fixnum x).  By associated with the
entire definition, we mean, for example, the declarations with scope
over an entire function body, excluding those local to a particular
\<let\>. At present, global declarations (from \<proclaim\> or \<declaim\>)
are ignored.\endDocumentation
\Usage{28}{\cd{(df:definition-declarations def)}}\endUsage
\beginArguments
\argument{\cd{def}}\typeArg{\cd{Definitions:Definition}}\endArg
\endArguments
\beginReturn
\singleReturn \typeReturn{\cd{List}}\endtReturn
\endReturn

\Source{ /home/comet/rga6711/xlisp/definition/definition.lsp}
\endDefinition


\beginDefinition


\DefNameBox{df:definition-declarations Df:Definition}{Primary Method}
\beginDocumentation
The default method returns ().\endDocumentation
\Usage{28}{\cd{(df:definition-declarations def)}}\endUsage
\beginArguments
\argument{\cd{def}}\typeArg{\cd{Definitions:Definition}}\endArg
\endArguments
\beginReturn
\singleReturn \simpleReturn{\cd{nil}}\endsReturn
\endReturn

\Source{ /home/comet/rga6711/xlisp/definition/definition.lsp}
\endDefinition


\beginDefinition


\DefNameBox{df:definition-declarations Df:Lambda-List-Definition}{Primary Method}
\beginDocumentation
Returns all the decl-specs from \<declare\> forms that come before
the first non-string non-\<declare\> form in the function body.  Forms
that begin with \<declare-check\> are also treated as declarations.\endDocumentation
\Usage{28}{\cd{(df:definition-declarations def)}}\endUsage
\beginArguments
\argument{\cd{def}}\typeArg{\cd{Definitions:Lambda-List-Definition}}\endArg
\endArguments
\beginReturn
\singleReturn \typeReturn{\cd{List}}\endtReturn
\endReturn

\Source{ /home/comet/rga6711/xlisp/definition/fun.lsp}
\endDefinition


\beginDefinition


\DefNameBox{df:definition-declarations Df:Setf-Definition}{Primary Method}
\beginDocumentation
Declarations are only allowed in the long version of defsetf.\endDocumentation
\Usage{28}{\cd{(df:definition-declarations def)}}\endUsage
\beginArguments
\argument{\cd{def}}\typeArg{\cd{Definitions:Setf-Definition}}\endArg
\endArguments
\beginReturn
\singleReturn \typeReturn{\cd{List}}\endtReturn
\endReturn

\Source{ /home/comet/rga6711/xlisp/definition/fun.lsp}
\endDefinition


\beginDefinition


\DefNameBox{df:definition-definee}{Generic Function}
\beginDocumentation
Returns the lisp object that was created when the
definition was loaded (which is assumed to have happen before the
definition object was created), or nil if it is not possible to
retrieve a lisp object corresponding to the definition. For example,
one can get the appropriate class object by calling find-class on the
\<definition-symbol\>, but, there is in general no portable way a lisp
object associated with the result of evaluating a defstruct.\endDocumentation
\Usage{23}{\cd{(df:definition-definee def)}}\endUsage
\beginArguments
\argument{\cd{def}}\typeArg{\cd{Definitions:Definition}}\endArg
\endArguments
\beginReturn
\singleReturn \typeReturn{\cd{T}}\endtReturn
\endReturn

\Source{ /home/comet/rga6711/xlisp/definition/definition.lsp}
\endDefinition


\beginDefinition


\DefNameBox{df:definition-definee Df:Definition}{Primary Method}
\beginDocumentation
The default method returns nil.\endDocumentation
\Usage{23}{\cd{(df:definition-definee def)}}\endUsage
\beginArguments
\argument{\cd{def}}\typeArg{\cd{Definitions:Definition}}\endArg
\endArguments
\beginReturn
\singleReturn \simpleReturn{\cd{nil}}\endsReturn
\endReturn

\Source{ /home/comet/rga6711/xlisp/definition/definition.lsp}
\endDefinition


\beginDefinition


\DefNameBox{df:definition-definee Df:Class-Definition}{Primary Method}
\beginDocumentation
Get the corresponding class object.\endDocumentation
\Usage{23}{\cd{(df:definition-definee def)}}\endUsage
\beginArguments
\argument{\cd{def}}\typeArg{\cd{Definitions:Class-Definition}}\endArg
\endArguments
\beginReturn
\singleReturn \typeReturn{\cd{Definitions::Class}}\endtReturn
\endReturn

\Source{ /home/comet/rga6711/xlisp/definition/type.lsp}
\endDefinition


\beginDefinition


\DefNameBox{df:definition-definee Df:Condition-Definition}{Primary Method}
\beginDocumentation
Get the corresponding class object.\endDocumentation
\Usage{23}{\cd{(df:definition-definee def)}}\endUsage
\beginArguments
\argument{\cd{def}}\typeArg{\cd{Definitions:Condition-Definition}}\endArg
\endArguments
\beginReturn
\singleReturn \typeReturn{\cd{Definitions::Class}}\endtReturn
\endReturn

\Source{ /home/comet/rga6711/xlisp/definition/condition.lsp}
\endDefinition


\beginDefinition


\DefNameBox{df:definition-definee Df:Prototype-Definition}{Primary Method}
\beginDocumentation
Get the corresponding prototype object.\endDocumentation
\Usage{23}{\cd{(df:definition-definee def)}}\endUsage
\beginArguments
\argument{\cd{def}}\typeArg{\cd{Definitions:Prototype-Definition}}\endArg
\endArguments
\beginReturn
\singleReturn \typeReturn{\cd{Object}}\endtReturn
\endReturn

\Source{ /home/comet/rga6711/xlisp/definition/xls-object.lsp}
\endDefinition


\beginDefinition


\DefNameBox{df:definition-definer}{Function}
\beginDocumentation
Returns the definer symbol, eg. \<defun\> or \<defclass\>.\endDocumentation
\Usage{23}{\cd{(df:definition-definer def)}}\endUsage
\beginArguments
\argument{\cd{def}}\typeArg{\cd{Definitions:Definition}}\endArg
\endArguments
\beginReturn
\singleReturn \typeReturn{\cd{Symbol}}\endtReturn
\endReturn

\Source{ /home/comet/rga6711/xlisp/definition/definition.lsp}
\endDefinition


\beginDefinition


\DefNameBox{df:definition-documentation}{Generic Function}
\beginDocumentation
Return the documentation string associated with \<def\>.
Return a string of length zero if there is no documentation string.\endDocumentation
\Usage{29}{\cd{(df:definition-documentation def)}}\endUsage
\beginArguments
\argument{\cd{def}}\typeArg{\cd{Definitions:Definition}}\endArg
\endArguments
\beginReturn
\singleReturn \typeReturn{\cd{String}}\endtReturn
\endReturn

\Source{ /home/comet/rga6711/xlisp/definition/definition.lsp}
\endDefinition


\beginDefinition


\DefNameBox{df:definition-documentation Df:Definition}{Primary Method}
\beginDocumentation
The default is the 4th item in the definition form, if it's a
string, otherwise we return the empty string.\endDocumentation
\Usage{29}{\cd{(df:definition-documentation def)}}\endUsage
\beginArguments
\argument{\cd{def}}\typeArg{\cd{Definitions:Definition}}\endArg
\endArguments
\beginReturn
\singleReturn \typeReturn{\cd{String}}\endtReturn
\endReturn

\Source{ /home/comet/rga6711/xlisp/definition/definition.lsp}
\endDefinition


\beginDefinition


\DefNameBox{df:definition-documentation Df:Class-Definition}{Primary Method}
\beginDocumentation
Return the class documentation string or an empty string.\endDocumentation
\Usage{29}{\cd{(df:definition-documentation def)}}\endUsage
\beginArguments
\argument{\cd{def}}\typeArg{\cd{Definitions:Class-Definition}}\endArg
\endArguments
\beginReturn
\singleReturn \typeReturn{\cd{String}}\endtReturn
\endReturn

\Source{ /home/comet/rga6711/xlisp/definition/type.lsp}
\endDefinition


\beginDefinition


\DefNameBox{df:definition-documentation Df:Condition-Definition}{Primary Method}
\beginDocumentation
Return the class documentation string or an empty string.\endDocumentation
\Usage{29}{\cd{(df:definition-documentation def)}}\endUsage
\beginArguments
\argument{\cd{def}}\typeArg{\cd{Definitions:Condition-Definition}}\endArg
\endArguments
\beginReturn
\singleReturn \typeReturn{\cd{String}}\endtReturn
\endReturn

\Source{ /home/comet/rga6711/xlisp/definition/condition.lsp}
\endDefinition


\beginDefinition


\DefNameBox{df:definition-documentation Df:Lambda-List-Definition}{Primary Method}
\beginDocumentation
The documentation string is the first string that comes before the
first non-\<declare\> form at the top of the function body, unless it's
the returned value.\endDocumentation
\Usage{29}{\cd{(df:definition-documentation def)}}\endUsage
\beginArguments
\argument{\cd{def}}\typeArg{\cd{Definitions:Lambda-List-Definition}}\endArg
\endArguments
\beginReturn
\singleReturn \typeReturn{\cd{String}}\endtReturn
\endReturn

\Source{ /home/comet/rga6711/xlisp/definition/fun.lsp}
\endDefinition


\beginDefinition


\DefNameBox{df:definition-documentation Df:Generic-Function-Definition}{Primary Method}
\beginDocumentation
Return the empty string if no :documentation option is present.\endDocumentation
\Usage{29}{\cd{(df:definition-documentation def)}}\endUsage
\beginArguments
\argument{\cd{def}}\typeArg{\cd{Definitions:Generic-Function-Definition}}\endArg
\endArguments
\beginReturn
\singleReturn \typeReturn{\cd{String}}\endtReturn
\endReturn

\Source{ /home/comet/rga6711/xlisp/definition/fun.lsp}
\endDefinition


\beginDefinition


\DefNameBox{df:definition-documentation Df:Package-Definition}{Primary Method}
\beginDocumentation
Returns the value of the :documentation option or an empty string.\endDocumentation
\Usage{29}{\cd{(df:definition-documentation def)}}\endUsage
\beginArguments
\argument{\cd{def}}\typeArg{\cd{Definitions:Package-Definition}}\endArg
\endArguments
\beginReturn
\singleReturn \typeReturn{\cd{String}}\endtReturn
\endReturn

\Source{ /home/comet/rga6711/xlisp/definition/pack.lsp}
\endDefinition


\beginDefinition


\DefNameBox{df:definition-documentation Df:Prototype-Definition}{Primary Method}
\beginDocumentation
Return the class documentation string or an empty string.\endDocumentation
\Usage{29}{\cd{(df:definition-documentation def)}}\endUsage
\beginArguments
\argument{\cd{def}}\typeArg{\cd{Definitions:Prototype-Definition}}\endArg
\endArguments
\beginReturn
\singleReturn \typeReturn{\cd{String}}\endtReturn
\endReturn

\Source{ /home/comet/rga6711/xlisp/definition/xls-object.lsp}
\endDefinition


\beginDefinition


\DefNameBox{df:definition-documentation Df:Setf-Definition}{Primary Method}
\beginDocumentation
The doc string for defsetf is, in the long version of defsetf, the
5th item in the definition form (if it`s a string), and, in the short
version, the 4th item (again, if it's a string).\endDocumentation
\Usage{29}{\cd{(df:definition-documentation def)}}\endUsage
\beginArguments
\argument{\cd{def}}\typeArg{\cd{Definitions:Setf-Definition}}\endArg
\endArguments
\beginReturn
\singleReturn \typeReturn{\cd{String}}\endtReturn
\endReturn

\Source{ /home/comet/rga6711/xlisp/definition/fun.lsp}
\endDefinition


\beginDefinition


\DefNameBox{df:definition-documentation Df:Structure-Definition}{Primary Method}
\beginDocumentation
Return the defstruct's doc string, or an empty string.\endDocumentation
\Usage{29}{\cd{(df:definition-documentation def)}}\endUsage
\beginArguments
\argument{\cd{def}}\typeArg{\cd{Definitions:Structure-Definition}}\endArg
\endArguments
\beginReturn
\singleReturn \typeReturn{\cd{String}}\endtReturn
\endReturn

\Source{ /home/comet/rga6711/xlisp/definition/type.lsp}
\endDefinition


\beginDefinition


\DefNameBox{df:definition-documentation Df:Type-Definition}{Primary Method}
\beginDocumentation
Type name strings should be capitalized.\endDocumentation
\Usage{29}{\cd{(df:definition-documentation def)}}\endUsage
\beginArguments
\argument{\cd{def}}\typeArg{\cd{Definitions:Type-Definition}}\endArg
\endArguments
\beginReturn
\singleReturn \typeReturn{\cd{String}}\endtReturn
\endReturn

\Source{ /home/comet/rga6711/xlisp/definition/type.lsp}
\endDefinition


\beginDefinition


\DefNameBox{df:definition-form}{Function}
\beginDocumentation
CLOS style reader for definition-form\endDocumentation
\Usage{20}{\cd{(df:definition-form def)}}\endUsage
\beginArguments
\argument{\cd{def}}\typeArg{\cd{T}}\endArg
\endArguments

\Source{ /home/comet/rga6711/xlisp/definition/definition.lsp}
\endDefinition


\beginDefinition


\DefNameBox{df:definition-initial-value}{Function}
\beginDocumentation
The initial value supplied for a global variable (or constant) definition.\endDocumentation
\Usage{29}{\cd{(df:definition-initial-value def)}}\endUsage
\beginArguments
\argument{\cd{def}}\typeArg{\cd{Definitions:Definition}}\endArg
\endArguments
\beginReturn
\singleReturn \typeReturn{\cd{T}}\endtReturn
\endReturn

\Source{ /home/comet/rga6711/xlisp/definition/global.lsp}
\endDefinition


\beginDefinition


\DefNameBox{df:definition-lambda-list}{Generic Function}
\beginDocumentation
Returns an arglist for the definition, or nil.\endDocumentation
\Usage{27}{\cd{(df:definition-lambda-list def)}}\endUsage
\beginArguments
\argument{\cd{def}}\typeArg{\cd{Definitions:Definition}}\endArg
\endArguments
\beginReturn
\singleReturn \typeReturn{\cd{List}}\endtReturn
\endReturn

\Source{ /home/comet/rga6711/xlisp/definition/definition.lsp}
\endDefinition


\beginDefinition


\DefNameBox{df:definition-lambda-list Df:Definition}{Primary Method}
\beginDocumentation
The default method returns ().\endDocumentation
\Usage{27}{\cd{(df:definition-lambda-list def)}}\endUsage
\beginArguments
\argument{\cd{def}}\typeArg{\cd{Definitions:Definition}}\endArg
\endArguments
\beginReturn
\singleReturn \simpleReturn{\cd{nil}}\endsReturn
\endReturn

\Source{ /home/comet/rga6711/xlisp/definition/definition.lsp}
\endDefinition


\beginDefinition


\DefNameBox{df:definition-lambda-list Df:Lambda-List-Definition}{Primary Method}
\beginDocumentation
The lambda list is the third item in most definitions.\endDocumentation
\Usage{27}{\cd{(df:definition-lambda-list def)}}\endUsage
\beginArguments
\argument{\cd{def}}\typeArg{\cd{Definitions:Lambda-List-Definition}}\endArg
\endArguments
\beginReturn
\singleReturn \typeReturn{\cd{List}}\endtReturn
\endReturn

\Source{ /home/comet/rga6711/xlisp/definition/fun.lsp}
\endDefinition


\beginDefinition


\DefNameBox{df:definition-lambda-list Df:Method-Definition}{Primary Method}
\beginDocumentation
Finding a method's lambda list requires checking for qualifiers.\endDocumentation
\Usage{27}{\cd{(df:definition-lambda-list def)}}\endUsage
\beginArguments
\argument{\cd{def}}\typeArg{\cd{Definitions:Method-Definition}}\endArg
\endArguments
\beginReturn
\singleReturn \typeReturn{\cd{List}}\endtReturn
\endReturn

\Source{ /home/comet/rga6711/xlisp/definition/fun.lsp}
\endDefinition


\beginDefinition


\DefNameBox{df:definition-lambda-list Df:Message-Definition}{Primary Method}
\beginDocumentation
Finding a method's lambda list.\endDocumentation
\Usage{27}{\cd{(df:definition-lambda-list def)}}\endUsage
\beginArguments
\argument{\cd{def}}\typeArg{\cd{Definitions:Message-Definition}}\endArg
\endArguments
\beginReturn
\singleReturn \typeReturn{\cd{List}}\endtReturn
\endReturn

\Source{ /home/comet/rga6711/xlisp/definition/xls-object.lsp}
\endDefinition


\beginDefinition


\DefNameBox{df:definition-lambda-list Df:Setf-Definition}{Primary Method}
\beginDocumentation
Returns a lamdba list one would have for the equivalent setf
method, that is, new value first, followed by the lambda list for the
generalized variable.\endDocumentation
\Usage{27}{\cd{(df:definition-lambda-list def)}}\endUsage
\beginArguments
\argument{\cd{def}}\typeArg{\cd{Definitions:Setf-Definition}}\endArg
\endArguments
\beginReturn
\singleReturn \typeReturn{\cd{List}}\endtReturn
\endReturn

\Source{ /home/comet/rga6711/xlisp/definition/fun.lsp}
\endDefinition


\beginDefinition


\DefNameBox{df:definition-method-qualifier}{Function}
\beginDocumentation
The method qualifier, eg., \<after\>. Returns \<:primary\> if no
qualifier present.\endDocumentation
\Usage{32}{\cd{(df:definition-method-qualifier def)}}\endUsage
\beginArguments
\argument{\cd{def}}\typeArg{\cd{Definitions:Method-Definition}}\endArg
\endArguments
\beginReturn
\singleReturn \typeReturn{\cd{Symbol}}\endtReturn
\endReturn

\Source{ /home/comet/rga6711/xlisp/definition/fun.lsp}
\endDefinition


\beginDefinition


\DefNameBox{df:definition-name}{Generic Function}
\beginDocumentation
Returns the name of a definition object, which is
usually either a symbol, eg. \<foo\> from (defun foo ...), (defclass Foo
...), etc., or a list, eg. (setf foo) from (defmethod (setf foo) ...)
or (defsetf foo ...).\endDocumentation
\Usage{20}{\cd{(df:definition-name def)}}\endUsage
\beginArguments
\argument{\cd{def}}\typeArg{\cd{Definitions:Definition}}\endArg
\endArguments
\beginReturn
\singleReturn \typeReturn{\cd{(Or Symbol List)}}\endtReturn
\endReturn

\Source{ /home/comet/rga6711/xlisp/definition/definition.lsp}
\endDefinition


\beginDefinition


\DefNameBox{df:definition-name Df:Definition}{Primary Method}
\beginDocumentation
By default, the \<definition-name\> is the second item in the
definition-form.\endDocumentation
\Usage{20}{\cd{(df:definition-name def)}}\endUsage
\beginArguments
\argument{\cd{def}}\typeArg{\cd{Definitions:Definition}}\endArg
\endArguments
\beginReturn
\singleReturn \typeReturn{\cd{(Or Symbol List)}}\endtReturn
\endReturn

\Source{ /home/comet/rga6711/xlisp/definition/definition.lsp}
\endDefinition


\beginDefinition


\DefNameBox{df:definition-name Df:Method-Definition}{Primary Method}
\beginDocumentation
The \<definition-name\> of a method is a list whose first item is the
symbol :method, whose second item is the function name, and whose
remaining items are the specializers for the required arguments.\endDocumentation
\Usage{20}{\cd{(df:definition-name def)}}\endUsage
\beginArguments
\argument{\cd{def}}\typeArg{\cd{Definitions:Definition}}\endArg
\endArguments
\beginReturn
\singleReturn \typeReturn{\cd{List}}\endtReturn
\endReturn

\Source{ /home/comet/rga6711/xlisp/definition/fun.lsp}
\endDefinition


\beginDefinition


\DefNameBox{df:definition-name Df:Message-Definition}{Primary Method}
\beginDocumentation
The \<definition-name\> of a method is a list whose first item is the
symbol :message, whose second item is the function name, and whose
remaining items are the specializers for the required arguments.\endDocumentation
\Usage{20}{\cd{(df:definition-name def)}}\endUsage
\beginArguments
\argument{\cd{def}}\typeArg{\cd{Definitions:Definition}}\endArg
\endArguments
\beginReturn
\singleReturn \typeReturn{\cd{List}}\endtReturn
\endReturn

\Source{ /home/comet/rga6711/xlisp/definition/xls-object.lsp}
\endDefinition


\beginDefinition


\DefNameBox{df:definition-name Df:Setf-Definition}{Primary Method}
\beginDocumentation
The name of a defsetf definition is a list like (setf foo).\endDocumentation
\Usage{20}{\cd{(df:definition-name def)}}\endUsage
\beginArguments
\argument{\cd{def}}\typeArg{\cd{Definitions:Setf-Definition}}\endArg
\endArguments
\beginReturn
\singleReturn \typeReturn{\cd{List}}\endtReturn
\endReturn

\Source{ /home/comet/rga6711/xlisp/definition/fun.lsp}
\endDefinition


\beginDefinition


\DefNameBox{df:definition-name Df:Structure-Definition}{Primary Method}
\beginDocumentation
Getting the name of a defstruct requires a little analysis of the
second item in the definition form.\endDocumentation
\Usage{20}{\cd{(df:definition-name def)}}\endUsage
\beginArguments
\argument{\cd{def}}\typeArg{\cd{Definitions:Structure-Definition}}\endArg
\endArguments
\beginReturn
\singleReturn \typeReturn{\cd{Symbol}}\endtReturn
\endReturn

\Source{ /home/comet/rga6711/xlisp/definition/type.lsp}
\endDefinition


\beginDefinition


\DefNameBox{df:definition-name-\>string}{Generic Function}
\beginDocumentation
Returns a string containing the name of a definition
object, appropriately capitalized.\endDocumentation
\Usage{28}{\cd{(df:definition-name->string def)}}\endUsage
\beginArguments
\argument{\cd{def}}\typeArg{\cd{Definitions:Definition}}\endArg
\endArguments
\beginReturn
\singleReturn \typeReturn{\cd{String}}\endtReturn
\endReturn

\Source{ /home/comet/rga6711/xlisp/definition/definition.lsp}
\endDefinition


\beginDefinition


\DefNameBox{df:definition-name-\>string Df:Definition}{Primary Method}
\beginDocumentation
The default method simply calls format on the \<definition-name\>,
printing in lower case.\endDocumentation
\Usage{28}{\cd{(df:definition-name->string def)}}\endUsage
\beginArguments
\argument{\cd{def}}\typeArg{\cd{Definitions:Definition}}\endArg
\endArguments
\beginReturn
\singleReturn \typeReturn{\cd{String}}\endtReturn
\endReturn

\Source{ /home/comet/rga6711/xlisp/definition/definition.lsp}
\endDefinition


\beginDefinition


\DefNameBox{df:definition-name-\>string Df:Method-Definition}{Primary Method}
\beginDocumentation
The name string for methods includes the specializers, so the
different methods for a generic function can be distinguished.\endDocumentation
\Usage{28}{\cd{(df:definition-name->string def)}}\endUsage
\beginArguments
\argument{\cd{def}}\typeArg{\cd{Definitions:Method-Definition}}\endArg
\endArguments
\beginReturn
\singleReturn \typeReturn{\cd{String}}\endtReturn
\endReturn

\Source{ /home/comet/rga6711/xlisp/definition/fun.lsp}
\endDefinition


\beginDefinition


\DefNameBox{df:definition-name-\>string Df:Message-Definition}{Primary Method}
\beginDocumentation
The name string for methods includes the specializers, so the
different methods for a generic function can be distinguished.\endDocumentation
\Usage{28}{\cd{(df:definition-name->string def)}}\endUsage
\beginArguments
\argument{\cd{def}}\typeArg{\cd{Definitions:Message-Definition}}\endArg
\endArguments
\beginReturn
\singleReturn \typeReturn{\cd{String}}\endtReturn
\endReturn

\Source{ /home/comet/rga6711/xlisp/definition/xls-object.lsp}
\endDefinition


\beginDefinition


\DefNameBox{df:definition-name-\>string Df:Package-Definition}{Primary Method}
\beginDocumentation
Package name strings should be capitalized.\endDocumentation
\Usage{28}{\cd{(df:definition-name->string def)}}\endUsage
\beginArguments
\argument{\cd{def}}\typeArg{\cd{Definitions:Package-Definition}}\endArg
\endArguments
\beginReturn
\singleReturn \typeReturn{\cd{String}}\endtReturn
\endReturn

\Source{ /home/comet/rga6711/xlisp/definition/pack.lsp}
\endDefinition


\beginDefinition


\DefNameBox{df:definition-parents}{Generic Function}
\beginDocumentation
Returns a list of the names of the parents (eg. direct
superclasses) of the definition.  For new Definition classes, it may
return anything that is reasonably thought of as a name of a
``parent'' of the definition.\endDocumentation
\Usage{23}{\cd{(df:definition-parents def)}}\endUsage
\beginArguments
\argument{\cd{def}}\typeArg{\cd{Definitions:Definition}}\endArg
\endArguments
\beginReturn
\singleReturn \typeReturn{\cd{List}}\endtReturn
\endReturn

\Source{ /home/comet/rga6711/xlisp/definition/definition.lsp}
\endDefinition


\beginDefinition


\DefNameBox{df:definition-parents Df:Definition}{Primary Method}
\beginDocumentation
The default method returns ().\endDocumentation
\Usage{23}{\cd{(df:definition-parents def)}}\endUsage
\beginArguments
\argument{\cd{def}}\typeArg{\cd{Definitions:Definition}}\endArg
\endArguments
\beginReturn
\singleReturn \simpleReturn{\cd{nil}}\endsReturn
\endReturn

\Source{ /home/comet/rga6711/xlisp/definition/definition.lsp}
\endDefinition


\beginDefinition


\DefNameBox{df:definition-parents Df:Class-Definition}{Primary Method}
\beginDocumentation
Returns the names of the direct superclasses.\endDocumentation
\Usage{23}{\cd{(df:definition-parents def)}}\endUsage
\beginArguments
\argument{\cd{def}}\typeArg{\cd{Definitions:Class-Definition}}\endArg
\endArguments
\beginReturn
\singleReturn \typeReturn{\cd{List}}\endtReturn
\endReturn

\Source{ /home/comet/rga6711/xlisp/definition/type.lsp}
\endDefinition


\beginDefinition


\DefNameBox{df:definition-parents Df:Condition-Definition}{Primary Method}
\beginDocumentation
Returns the names of the direct superclasses.\endDocumentation
\Usage{23}{\cd{(df:definition-parents def)}}\endUsage
\beginArguments
\argument{\cd{def}}\typeArg{\cd{Definitions:Condition-Definition}}\endArg
\endArguments
\beginReturn
\singleReturn \typeReturn{\cd{List}}\endtReturn
\endReturn

\Source{ /home/comet/rga6711/xlisp/definition/condition.lsp}
\endDefinition


\beginDefinition


\DefNameBox{df:definition-parents Df:Prototype-Definition}{Primary Method}
\beginDocumentation
Returns the names of the direct superclasses.\endDocumentation
\Usage{23}{\cd{(df:definition-parents def)}}\endUsage
\beginArguments
\argument{\cd{def}}\typeArg{\cd{Definitions:Prototype-Definition}}\endArg
\endArguments
\beginReturn
\singleReturn \typeReturn{\cd{List}}\endtReturn
\endReturn

\Source{ /home/comet/rga6711/xlisp/definition/xls-object.lsp}
\endDefinition


\beginDefinition


\DefNameBox{df:definition-returns}{Function}
\beginDocumentation
Returns the :returns decl-spec or nil if there isn't one.\endDocumentation
\Usage{23}{\cd{(df:definition-returns def)}}\endUsage
\beginArguments
\argument{\cd{def}}\typeArg{\cd{Definitions:Definition}}\endArg
\endArguments
\beginReturn
\singleReturn \typeReturn{\cd{List}}\endtReturn
\endReturn

\Source{ /home/comet/rga6711/xlisp/definition/definition.lsp}
\endDefinition


\beginDefinition


\DefNameBox{df:definition-slots}{Generic Function}
\beginDocumentation
Returns a list of slot specs, which need to be
interpreted in a Definition class specific manner (structure slot
specs are different from class slot specs).\endDocumentation
\Usage{21}{\cd{(df:definition-slots def)}}\endUsage
\beginArguments
\argument{\cd{def}}\typeArg{\cd{Definitions:Definition}}\endArg
\endArguments
\beginReturn
\singleReturn \typeReturn{\cd{List}}\endtReturn
\endReturn

\Source{ /home/comet/rga6711/xlisp/definition/definition.lsp}
\endDefinition


\beginDefinition


\DefNameBox{df:definition-slots Df:Definition}{Primary Method}
\beginDocumentation
The default method returns ().\endDocumentation
\Usage{21}{\cd{(df:definition-slots def)}}\endUsage
\beginArguments
\argument{\cd{def}}\typeArg{\cd{Definitions:Definition}}\endArg
\endArguments
\beginReturn
\singleReturn \simpleReturn{\cd{nil}}\endsReturn
\endReturn

\Source{ /home/comet/rga6711/xlisp/definition/definition.lsp}
\endDefinition


\beginDefinition


\DefNameBox{df:definition-slots Df:Class-Definition}{Primary Method}
\beginDocumentation
Return the forms defining the slots of this class.\endDocumentation
\Usage{21}{\cd{(df:definition-slots def)}}\endUsage
\beginArguments
\argument{\cd{def}}\typeArg{\cd{Definitions:Class-Definition}}\endArg
\endArguments
\beginReturn
\singleReturn \typeReturn{\cd{List}}\endtReturn
\endReturn

\Source{ /home/comet/rga6711/xlisp/definition/type.lsp}
\endDefinition


\beginDefinition


\DefNameBox{df:definition-slots Df:Condition-Definition}{Primary Method}
\beginDocumentation
Return the forms defining the slots of this class.\endDocumentation
\Usage{21}{\cd{(df:definition-slots def)}}\endUsage
\beginArguments
\argument{\cd{def}}\typeArg{\cd{Definitions:Condition-Definition}}\endArg
\endArguments
\beginReturn
\singleReturn \typeReturn{\cd{List}}\endtReturn
\endReturn

\Source{ /home/comet/rga6711/xlisp/definition/condition.lsp}
\endDefinition


\beginDefinition


\DefNameBox{df:definition-slots Df:Prototype-Definition}{Primary Method}
\beginDocumentation
Return the forms defining the visible slots of this class.\endDocumentation
\Usage{21}{\cd{(df:definition-slots def)}}\endUsage
\beginArguments
\argument{\cd{def}}\typeArg{\cd{Definitions:Prototype-Definition}}\endArg
\endArguments
\beginReturn
\singleReturn \typeReturn{\cd{List}}\endtReturn
\endReturn

\Source{ /home/comet/rga6711/xlisp/definition/xls-object.lsp}
\endDefinition


\beginDefinition


\DefNameBox{df:definition-slots Df:Structure-Definition}{Primary Method}
\beginDocumentation
Return the forms defining the slots of this structure.\endDocumentation
\Usage{21}{\cd{(df:definition-slots def)}}\endUsage
\beginArguments
\argument{\cd{def}}\typeArg{\cd{Definitions:Structure-Definition}}\endArg
\endArguments
\beginReturn
\singleReturn \typeReturn{\cd{List}}\endtReturn
\endReturn

\Source{ /home/comet/rga6711/xlisp/definition/type.lsp}
\endDefinition


\beginDefinition


\DefNameBox{df:definition-specializers}{Generic Function}
\beginDocumentation
This function finds the list of specializers for a method
  function.\endDocumentation
\Usage{28}{\cd{(df:definition-specializers def)}}\endUsage
\beginArguments
\argument{\cd{def}}\typeArg{\cd{Definitions:Method-Definition}}\endArg
\endArguments
\beginReturn
\singleReturn \complexReturn{\cd{specializers}}\midReturn{\cd{List}}\endcReturn
\endReturn

\Source{ /home/comet/rga6711/xlisp/definition/fun.lsp}
\endDefinition


\beginDefinition


\DefNameBox{df:definition-specializers Df:Method-Definition}{Primary Method}
\beginDocumentation
Clos style definition, specialiers come from lambda-list\endDocumentation
\Usage{28}{\cd{(df:definition-specializers def)}}\endUsage
\beginArguments
\argument{\cd{def}}\typeArg{\cd{Definitions:Method-Definition}}\endArg
\endArguments

\Source{ /home/comet/rga6711/xlisp/definition/fun.lsp}
\endDefinition


\beginDefinition


\DefNameBox{df:definition-specializers Df:Message-Definition}{Primary Method}
\beginDocumentation
XLISP style definition, specialiers come from second arg of defmeth\endDocumentation
\Usage{28}{\cd{(df:definition-specializers def)}}\endUsage
\beginArguments
\argument{\cd{def}}\typeArg{\cd{Definitions:Message-Definition}}\endArg
\endArguments

\Source{ /home/comet/rga6711/xlisp/definition/xls-object.lsp}
\endDefinition


\beginDefinition


\DefNameBox{df:definition-symbol}{Generic Function}
\beginDocumentation
Returns a symbol naming the definition.  For definitions
whose \<definition-name\> is a symbol, \<definition-symbol\> is the same.
For definitions whose \<definition-name\> is a list like (setf foo),
\<definition-symbol\> is \<foo\>.\endDocumentation
\Usage{22}{\cd{(df:definition-symbol def)}}\endUsage
\beginArguments
\argument{\cd{def}}\typeArg{\cd{Definitions:Definition}}\endArg
\endArguments
\beginReturn
\singleReturn \typeReturn{\cd{Symbol}}\endtReturn
\endReturn

\Source{ /home/comet/rga6711/xlisp/definition/definition.lsp}
\endDefinition


\beginDefinition


\DefNameBox{df:definition-symbol Df:Definition}{Primary Method}
\beginDocumentation
Returns a symbol naming the definition.  For definitions whose
\<definition-name\> is a symbol, \<definition-symbol\> is the same.  For
definitions whose \<definition-name\> is a list like (setf foo),
\<definition-symbol\> is \<foo\>.\endDocumentation
\Usage{22}{\cd{(df:definition-symbol def)}}\endUsage
\beginArguments
\argument{\cd{def}}\typeArg{\cd{Definitions:Definition}}\endArg
\endArguments
\beginReturn
\singleReturn \typeReturn{\cd{Symbol}}\endtReturn
\endReturn

\Source{ /home/comet/rga6711/xlisp/definition/definition.lsp}
\endDefinition


\beginDefinition


\DefNameBox{df:definition-symbol Df:Package-Definition}{Primary Method}
\beginDocumentation
`name' could be a string, so need to intern it in keyword package
to make package symbol.\endDocumentation
\Usage{22}{\cd{(df:definition-symbol def)}}\endUsage
\beginArguments
\argument{\cd{def}}\typeArg{\cd{Definitions:Definition}}\endArg
\endArguments
\beginReturn
\singleReturn \typeReturn{\cd{Symbol}}\endtReturn
\endReturn

\Source{ /home/comet/rga6711/xlisp/definition/pack.lsp}
\endDefinition


\beginDefinition


\DefNameBox{df:definition-symbol-name}{Function}
\beginDocumentation
Returns the symbol-name of the \<definition-symbol\>.\endDocumentation
\Usage{27}{\cd{(df:definition-symbol-name def)}}\endUsage
\beginArguments
\argument{\cd{def}}\typeArg{\cd{Definitions:Definition}}\endArg
\endArguments
\beginReturn
\singleReturn \typeReturn{\cd{String}}\endtReturn
\endReturn

\Source{ /home/comet/rga6711/xlisp/definition/definition.lsp}
\endDefinition


\beginDefinition


\DefNameBox{df:definition-type-declarations}{Function}
\beginDocumentation
Returns a list of the type decl specs. At the moment, a type decl
spec must have the symbol \<type\> as it's first item.  In the future
this may be extended to cover decl specs whose first entry is, for
example, \<Fixnum\>.\endDocumentation
\Usage{33}{\cd{(df:definition-type-declarations def)}}\endUsage
\beginArguments
\argument{\cd{def}}\typeArg{\cd{Definitions:Definition}}\endArg
\endArguments
\beginReturn
\singleReturn \typeReturn{\cd{List}}\endtReturn
\endReturn

\Source{ /home/comet/rga6711/xlisp/definition/definition.lsp}
\endDefinition


\beginDefinition


\DefNameBox{df:definition-usage}{Generic Function}
\beginDocumentation
Returns a string showing how to ``call'' the definition.
              Second value is indentation to use if it goes to
              multiple lines.\endDocumentation
\Usage{21}{\cd{(df:definition-usage def)}}\endUsage
\beginArguments
\argument{\cd{def}}\typeArg{\cd{Definitions:Definition}}\endArg
\endArguments
\beginReturn
\itemReturn \complexReturn{\cd{usage}}\midReturn{\cd{String}}\endcReturn
\itemReturn \complexReturn{\cd{indent}}\midReturn{\cd{Fixnum}}\endcReturn
\endReturn

\Source{ /home/comet/rga6711/xlisp/definition/definition.lsp}
\endDefinition


\beginDefinition


\DefNameBox{df:definition-usage Df:Definition}{Primary Method}
\beginDocumentation
The default for usage is just the \<definition-name-string\>.\endDocumentation
\Usage{21}{\cd{(df:definition-usage def)}}\endUsage
\beginArguments
\argument{\cd{def}}\typeArg{\cd{Definitions:Definition}}\endArg
\endArguments
\beginReturn
\itemReturn \complexReturn{\cd{usage}}\midReturn{\cd{String}}\endcReturn
\itemReturn \complexReturn{\cd{indent}}\midReturn{\cd{Fixnum}}\endcReturn
\endReturn

\Source{ /home/comet/rga6711/xlisp/definition/definition.lsp}
\endDefinition


\beginDefinition


\DefNameBox{df:definition-usage Df:Class-Definition}{Primary Method}
\beginDocumentation
Construct a string reflecting a typical function call.\endDocumentation
\Usage{21}{\cd{(df:definition-usage def)}}\endUsage
\beginArguments
\argument{\cd{def}}\typeArg{\cd{Definitions:Definition}}\endArg
\endArguments
\beginReturn
\itemReturn \complexReturn{\cd{usage}}\midReturn{\cd{String}}\endcReturn
\itemReturn \complexReturn{\cd{indent}}\midReturn{\cd{Fixnum}}\endcReturn
\endReturn

\Source{ /home/comet/rga6711/xlisp/definition/type.lsp}
\endDefinition


\beginDefinition


\DefNameBox{df:definition-usage Df:Condition-Definition}{Primary Method}
\beginDocumentation
Construct a string reflecting a typical function call.\endDocumentation
\Usage{21}{\cd{(df:definition-usage def)}}\endUsage
\beginArguments
\argument{\cd{def}}\typeArg{\cd{Definitions:Definition}}\endArg
\endArguments
\beginReturn
\itemReturn \complexReturn{\cd{usage}}\midReturn{\cd{String}}\endcReturn
\itemReturn \complexReturn{\cd{indent}}\midReturn{\cd{Fixnum}}\endcReturn
\endReturn

\Source{ /home/comet/rga6711/xlisp/definition/condition.lsp}
\endDefinition


\beginDefinition


\DefNameBox{df:definition-usage Df:Lambda-List-Definition}{Primary Method}
\beginDocumentation
Construct a string reflecting a typical function call.\endDocumentation
\Usage{21}{\cd{(df:definition-usage def)}}\endUsage
\beginArguments
\argument{\cd{def}}\typeArg{\cd{Definitions:Definition}}\endArg
\endArguments
\beginReturn
\itemReturn \complexReturn{\cd{usage}}\midReturn{\cd{String}}\endcReturn
\itemReturn \complexReturn{\cd{indent}}\midReturn{\cd{Fixnum}}\endcReturn
\endReturn

\Source{ /home/comet/rga6711/xlisp/definition/fun.lsp}
\endDefinition


\beginDefinition


\DefNameBox{df:definition-usage Df:Macro-Definition}{Primary Method}
\beginDocumentation
Construct a string for a typical call to the macro.\endDocumentation
\Usage{21}{\cd{(df:definition-usage def)}}\endUsage
\beginArguments
\argument{\cd{def}}\typeArg{\cd{Definitions:Definition}}\endArg
\endArguments
\beginReturn
\itemReturn \complexReturn{\cd{usage}}\midReturn{\cd{String}}\endcReturn
\itemReturn \complexReturn{\cd{indent}}\midReturn{\cd{Fixnum}}\endcReturn
\endReturn

\Source{ /home/comet/rga6711/xlisp/definition/fun.lsp}
\endDefinition


\beginDefinition


\DefNameBox{df:definition-usage Df:Message-Definition}{Primary Method}
\beginDocumentation
Construct a string reflecting a typical send call.\endDocumentation
\Usage{21}{\cd{(df:definition-usage def)}}\endUsage
\beginArguments
\argument{\cd{def}}\typeArg{\cd{Definitions:Definition}}\endArg
\endArguments
\beginReturn
\itemReturn \complexReturn{\cd{usage}}\midReturn{\cd{String}}\endcReturn
\itemReturn \complexReturn{\cd{indent}}\midReturn{\cd{Fixnum}}\endcReturn
\endReturn

\Source{ /home/comet/rga6711/xlisp/definition/xls-object.lsp}
\endDefinition


\beginDefinition


\DefNameBox{df:definition-usage Df:Package-Definition}{Primary Method}
\beginDocumentation
The example of package use is a call to \<in-package\>.\endDocumentation
\Usage{21}{\cd{(df:definition-usage def)}}\endUsage
\beginArguments
\argument{\cd{def}}\typeArg{\cd{Definitions:Definition}}\endArg
\endArguments
\beginReturn
\itemReturn \complexReturn{\cd{usage}}\midReturn{\cd{String}}\endcReturn
\itemReturn \complexReturn{\cd{indent}}\midReturn{\cd{Fixnum}}\endcReturn
\endReturn

\Source{ /home/comet/rga6711/xlisp/definition/pack.lsp}
\endDefinition


\beginDefinition


\DefNameBox{df:definition-usage Df:User-Type-Definition}{Primary Method}
\beginDocumentation
The example of use of a type definition is a call to \<typep\>.\endDocumentation
\Usage{21}{\cd{(df:definition-usage def)}}\endUsage
\beginArguments
\argument{\cd{def}}\typeArg{\cd{Definitions:Definition}}\endArg
\endArguments
\beginReturn
\itemReturn \complexReturn{\cd{usage}}\midReturn{\cd{String}}\endcReturn
\itemReturn \complexReturn{\cd{indent}}\midReturn{\cd{Fixnum}}\endcReturn
\endReturn

\Source{ /home/comet/rga6711/xlisp/definition/type.lsp}
\endDefinition


\beginDefinition


\DefNameBox{df:definition-usage Df:Prototype-Definition}{Primary Method}
\Usage{21}{\cd{(df:definition-usage def)}}\endUsage
\beginArguments
\argument{\cd{def}}\typeArg{\cd{Definitions:Definition}}\endArg
\endArguments
\beginReturn
\itemReturn \complexReturn{\cd{usage}}\midReturn{\cd{String}}\endcReturn
\itemReturn \complexReturn{\cd{indent}}\midReturn{\cd{Fixnum}}\endcReturn
\endReturn

\Source{ /home/comet/rga6711/xlisp/definition/xls-object.lsp}
\endDefinition


\beginDefinition


\DefNameBox{df:definition\<}{Generic Function}
\beginDocumentation
Resolve the ambiguity in alphabetic ordering for
multiple definitions with the same name.\endDocumentation
\Usage{16}{\cd{(df:definition< def0 def1)}}\endUsage
\beginArguments
\argument{\cd{def0}}\typeArg{\cd{Definitions:Definition}}\endArg
\argument{\cd{def1}}\typeArg{\cd{Definitions:Definition}}\endArg
\endArguments
\beginReturn
\singleReturn \typeReturn{\cd{(Member T Nil)}}\endtReturn
\endReturn

\Source{ /home/comet/rga6711/xlisp/definition/sort.lsp}
\endDefinition


\beginDefinition


\DefNameBox{df:definition\< Df:Definition System:T}{Primary Method}
\beginDocumentation
The default method returns nil (not comparable).\endDocumentation
\Usage{16}{\cd{(df:definition< def0 def1)}}\endUsage
\beginArguments
\argument{\cd{def0}}\typeArg{\cd{Definitions:Definition}}\endArg
\argument{\cd{def1}}\typeArg{\cd{Definitions:Definition}}\endArg
\endArguments
\beginReturn
\singleReturn \simpleReturn{\cd{nil}}\endsReturn
\endReturn

\Source{ /home/comet/rga6711/xlisp/definition/sort.lsp}
\endDefinition


\beginDefinition


\DefNameBox{df:definition\< Df:Generic-Function-Definition System:T}{Primary Method}
\beginDocumentation
Generic Functions come before Methods.\endDocumentation
\Usage{16}{\cd{(df:definition< def0 def1)}}\endUsage
\beginArguments
\argument{\cd{def0}}\typeArg{\cd{Definitions:Generic-Function-Definition}}\endArg
\argument{\cd{def1}}\typeArg{\cd{Definitions:Method-Definition}}\endArg
\endArguments
\beginReturn
\singleReturn \simpleReturn{\cd{t}}\endsReturn
\endReturn

\Source{ /home/comet/rga6711/xlisp/definition/sort.lsp}
\endDefinition


\beginDefinition


\DefNameBox{df:definition\< Df:Method-Definition System:T}{Primary Method}
\beginDocumentation
Method ordering attempts to mimic calling order for combinable
methods and be alphabetic otherwise.\endDocumentation
\Usage{16}{\cd{(df:definition< def0 def1)}}\endUsage
\beginArguments
\argument{\cd{def0}}\typeArg{\cd{Definitions:Method-Definition}}\endArg
\argument{\cd{def1}}\typeArg{\cd{Definitions:Method-Definition}}\endArg
\endArguments
\beginReturn
\singleReturn \typeReturn{\cd{(Member T Nil)}}\endtReturn
\endReturn

\Source{ /home/comet/rga6711/xlisp/definition/sort.lsp}
\endDefinition


\beginDefinition


\DefNameBox{df:definition\< Df:User-Type-Definition System:T}{Primary Method}
\beginDocumentation
User defined types (deftype, defstruct, defclass) come before others.\endDocumentation
\Usage{16}{\cd{(df:definition< def0 def1)}}\endUsage
\beginArguments
\argument{\cd{def0}}\typeArg{\cd{Definitions:User-Type-Definition}}\endArg
\argument{\cd{def1}}\typeArg{\cd{Definitions:Definition}}\endArg
\endArguments
\beginReturn
\singleReturn \simpleReturn{\cd{t}}\endsReturn
\endReturn

\Source{ /home/comet/rga6711/xlisp/definition/sort.lsp}
\endDefinition


\beginDefinition


\DefNameBox{:Definitions}{Package}
\Usage{0}{\cd{(in-package :Definitions)}}\endUsage

\Source{ /home/comet/rga6711/xlisp/definition/package.lsp}
\endDefinition


\beginDefinition


\DefNameBox{:email -\> (df:author-doc-proto)}{Message}
\beginDocumentation
Author's email address.\endDocumentation
\Usage{8}{\cd{(send <obj> :email &optional new-value)}}\endUsage
\beginArguments
\argument{\cd{new-value}}\typeArg{\cd{String}}\endArg
\endArguments

\Source{ /home/comet/rga6711/xlisp/definition/project-doc.lsp}
\endDefinition


\beginDefinition


\DefNameBox{:examples -\> (df:project-doc-proto)}{Message}
\beginDocumentation
A list of strings containing examples of common uses for the package.\endDocumentation
\Usage{11}{\cd{(send <obj> :examples &optional new-value)}}\endUsage
\beginArguments
\argument{\cd{new-value}}\typeArg{\cd{List}}\endArg
\endArguments

\Source{ /home/comet/rga6711/xlisp/definition/project-doc.lsp}
\endDefinition


\beginDefinition


\DefNameBox{df:exported-definition?}{Function}
\beginDocumentation
Has the \<definition-symbol\> of \<def\> been exported?\endDocumentation
\Usage{25}{\cd{(df:exported-definition? def)}}\endUsage
\beginArguments
\argument{\cd{def}}\typeArg{\cd{Definitions:Definition}}\endArg
\endArguments
\beginReturn
\singleReturn \typeReturn{\cd{(Member T Nil)}}\endtReturn
\endReturn

\Source{ /home/comet/rga6711/xlisp/definition/filter.lsp}
\endDefinition


\beginDefinition


\DefNameBox{:exported-functions -\> (df:project-doc-proto)}{Message}
\beginDocumentation
A list of functions exported by the package.\endDocumentation
\Usage{21}{\cd{(send <obj> :exported-functions &optional new-value)}}\endUsage
\beginArguments
\argument{\cd{new-value}}\typeArg{\cd{List}}\endArg
\endArguments

\Source{ /home/comet/rga6711/xlisp/definition/project-doc.lsp}
\endDefinition


\beginDefinition


\DefNameBox{:exported-objects -\> (df:project-doc-proto)}{Message}
\beginDocumentation
A list of objects exported by the package.\endDocumentation
\Usage{19}{\cd{(send <obj> :exported-objects &optional new-value)}}\endUsage
\beginArguments
\argument{\cd{new-value}}\typeArg{\cd{List}}\endArg
\endArguments

\Source{ /home/comet/rga6711/xlisp/definition/project-doc.lsp}
\endDefinition


\beginDefinition


\DefNameBox{:files -\> (df:project-doc-proto)}{Message}
\beginDocumentation
A list of strings each with the name of a file 
	    	(relative to documentation file) contained in the project.  
	    	Autodocumentation feature uses these files for definitions.\endDocumentation
\Usage{8}{\cd{(send <obj> :files &optional new-value)}}\endUsage
\beginArguments
\argument{\cd{new-value}}\typeArg{\cd{List}}\endArg
\endArguments

\Source{ /home/comet/rga6711/xlisp/definition/project-doc.lsp}
\endDefinition


\beginDefinition


\DefNameBox{df:function-definition}{Class}
\beginDocumentation
A definition class for \<defun\>.\endDocumentation
\Usage{38}{\cd{(send df:function-definition :new )}}\endUsage

\Parents{\nextInList Df:Lambda-List-Definition}\endParent
\beginMethods{\nextInList
 Df:Definition-Class-Nice-Name\nextInList
 :Isnew\nextInList
 Xlos:Shared-Initialize}\endMethods

\Source{ /home/comet/rga6711/xlisp/definition/fun.lsp}
\endDefinition


\beginDefinition


\DefNameBox{:functional-description -\> (df:project-doc-proto)}{Message}
\beginDocumentation
What the code is supposed to accomplish.\endDocumentation
\Usage{25}{\cd{(send <obj> :functional-description &optional new-value)}}\endUsage
\beginArguments
\argument{\cd{new-value}}\typeArg{\cd{String}}\endArg
\endArguments

\Source{ /home/comet/rga6711/xlisp/definition/project-doc.lsp}
\endDefinition


\beginDefinition


\DefNameBox{df:generic-function-definition}{Class}
\beginDocumentation
A definition class for \<defgeneric\>.\endDocumentation
\Usage{46}{\cd{(send df:generic-function-definition :new )}}\endUsage

\Parents{\nextInList Df:Lambda-List-Definition}\endParent
\beginMethods{\nextInList
 Df:Definition\<\nextInList
 Df:Definition-Documentation\nextInList
 Df:Definition-Class-Nice-Name\nextInList
 :Isnew\nextInList
 Xlos:Shared-Initialize}\endMethods

\Source{ /home/comet/rga6711/xlisp/definition/fun.lsp}
\endDefinition


\beginDefinition


\DefNameBox{df:global-variable-definition}{Class}
\beginDocumentation
An abstract super class for global variable definitions.\endDocumentation
\Usage{45}{\cd{(send df:global-variable-definition :new )}}\endUsage

\Parents{\nextInList Df:Definition}\endParent
\Children{\nextInList
 Df:Parameter-Definition\nextInList
 Df:Variable-Definition\nextInList
 Df:Constant-Definition}\endChildren
\beginMethods{\nextInList
 Df::Definition-Name-\>Tag\nextInList
 :Isnew\nextInList
 Xlos:Shared-Initialize}\endMethods

\Source{ /home/comet/rga6711/xlisp/definition/global.lsp}
\endDefinition


\beginDefinition


\DefNameBox{:global-vars -\> (df:project-doc-proto)}{Message}
\beginDocumentation
A list of var-doc-proto that describe global variables used by the package.\endDocumentation
\Usage{14}{\cd{(send <obj> :global-vars &optional new-value)}}\endUsage
\beginArguments
\argument{\cd{new-value}}\typeArg{\cd{List}}\endArg
\endArguments

\Source{ /home/comet/rga6711/xlisp/definition/project-doc.lsp}
\endDefinition


\beginDefinition


\DefNameBox{:home-page -\> (df:author-doc-proto)}{Message}
\beginDocumentation
Author's home-page URL.\endDocumentation
\Usage{12}{\cd{(send <obj> :home-page &optional new-value)}}\endUsage
\beginArguments
\argument{\cd{new-value}}\typeArg{\cd{String}}\endArg
\endArguments

\Source{ /home/comet/rga6711/xlisp/definition/project-doc.lsp}
\endDefinition


\beginDefinition


\DefNameBox{:instructions -\> (df:project-doc-proto)}{Message}
\beginDocumentation
User instructions for the package.\endDocumentation
\Usage{15}{\cd{(send <obj> :instructions &optional new-value)}}\endUsage
\beginArguments
\argument{\cd{new-value}}\typeArg{\cd{String}}\endArg
\endArguments

\Source{ /home/comet/rga6711/xlisp/definition/project-doc.lsp}
\endDefinition


\beginDefinition


\DefNameBox{df:lambda-list-arg-names}{Function}
\beginDocumentation
A list of all the arg names.\endDocumentation
\Usage{26}{\cd{(df:lambda-list-arg-names lambda-list)}}\endUsage
\beginArguments
\argument{\cd{lambda-list}}\typeArg{\cd{List}}\endArg
\endArguments
\beginReturn
\singleReturn \typeReturn{\cd{List}}\endtReturn
\endReturn

\Source{ /home/comet/rga6711/xlisp/definition/fun.lsp}
\endDefinition


\beginDefinition


\DefNameBox{df:lambda-list-definition}{Class}
\beginDocumentation
An abstract super class for definitions that include lambda lists.\endDocumentation
\Usage{41}{\cd{(send df:lambda-list-definition :new )}}\endUsage

\Parents{\nextInList Df:Definition}\endParent
\Children{\nextInList
 Df:Generic-Function-Definition\nextInList
 Df:Method-Definition\nextInList
 Df:Type-Definition\nextInList
 Df:Setf-Definition\nextInList
 Df:Message-Definition\nextInList
 Df:Macro-Definition\nextInList
 Df:Function-Definition}\endChildren
\beginMethods{\nextInList
 Df:Definition-Usage\nextInList
 Df:Definition-Arg-Types\nextInList
 Df:Definition-Declarations\nextInList
 Df:Definition-Documentation\nextInList
 Df:Definition-Lambda-List\nextInList
 Df:Definition-Body\nextInList
 Df::Definition-Name-\>Tag\nextInList
 :Isnew\nextInList
 Xlos:Shared-Initialize}\endMethods

\Source{ /home/comet/rga6711/xlisp/definition/fun.lsp}
\endDefinition


\beginDefinition


\DefNameBox{df:lambda-list-keyword-arg-names}{Function}
\beginDocumentation
A list of the names only of the \&keyword args.\endDocumentation
\Usage{34}{\cd{(df:lambda-list-keyword-arg-names lambda-list)}}\endUsage
\beginArguments
\argument{\cd{lambda-list}}\typeArg{\cd{List}}\endArg
\endArguments
\beginReturn
\singleReturn \typeReturn{\cd{List}}\endtReturn
\endReturn

\Source{ /home/comet/rga6711/xlisp/definition/fun.lsp}
\endDefinition


\beginDefinition


\DefNameBox{df:lambda-list-keyword-args}{Function}
\beginDocumentation
A list of the \&keyword args, with default values, etc.\endDocumentation
\Usage{29}{\cd{(df:lambda-list-keyword-args lambda-list)}}\endUsage
\beginArguments
\argument{\cd{lambda-list}}\typeArg{\cd{List}}\endArg
\endArguments
\beginReturn
\singleReturn \typeReturn{\cd{List}}\endtReturn
\endReturn

\Source{ /home/comet/rga6711/xlisp/definition/fun.lsp}
\endDefinition


\beginDefinition


\DefNameBox{df:lambda-list-optional-arg-names}{Function}
\beginDocumentation
A list of the names only of the \&optional args.\endDocumentation
\Usage{35}{\cd{(df:lambda-list-optional-arg-names lambda-list)}}\endUsage
\beginArguments
\argument{\cd{lambda-list}}\typeArg{\cd{List}}\endArg
\endArguments
\beginReturn
\singleReturn \typeReturn{\cd{List}}\endtReturn
\endReturn

\Source{ /home/comet/rga6711/xlisp/definition/fun.lsp}
\endDefinition


\beginDefinition


\DefNameBox{df:lambda-list-optional-args}{Function}
\beginDocumentation
A list of the \&optional args, with default values, etc.\endDocumentation
\Usage{30}{\cd{(df:lambda-list-optional-args lambda-list)}}\endUsage
\beginArguments
\argument{\cd{lambda-list}}\typeArg{\cd{List}}\endArg
\endArguments
\beginReturn
\singleReturn \typeReturn{\cd{List}}\endtReturn
\endReturn

\Source{ /home/comet/rga6711/xlisp/definition/fun.lsp}
\endDefinition


\beginDefinition


\DefNameBox{df:lambda-list-required-arg-names}{Function}
\beginDocumentation
a list of the name oif the required args (specializers are stripped off.)\endDocumentation
\Usage{35}{\cd{(df:lambda-list-required-arg-names lambda-list)}}\endUsage
\beginArguments
\argument{\cd{lambda-list}}\typeArg{\cd{List}}\endArg
\endArguments
\beginReturn
\singleReturn \typeReturn{\cd{List}}\endtReturn
\endReturn

\Source{ /home/comet/rga6711/xlisp/definition/fun.lsp}
\endDefinition


\beginDefinition


\DefNameBox{df:lambda-list-required-args}{Function}
\beginDocumentation
A list of the required args (with specializers when given).\endDocumentation
\Usage{30}{\cd{(df:lambda-list-required-args lambda-list)}}\endUsage
\beginArguments
\argument{\cd{lambda-list}}\typeArg{\cd{List}}\endArg
\endArguments
\beginReturn
\singleReturn \typeReturn{\cd{List}}\endtReturn
\endReturn

\Source{ /home/comet/rga6711/xlisp/definition/fun.lsp}
\endDefinition


\beginDefinition


\DefNameBox{df:lambda-list-rest-arg-name}{Function}
\beginDocumentation
The name of the \&rest or \&body arg.\endDocumentation
\Usage{30}{\cd{(df:lambda-list-rest-arg-name lambda-list)}}\endUsage
\beginArguments
\argument{\cd{lambda-list}}\typeArg{\cd{List}}\endArg
\endArguments
\beginReturn
\singleReturn \typeReturn{\cd{Symbol}}\endtReturn
\endReturn

\Source{ /home/comet/rga6711/xlisp/definition/fun.lsp}
\endDefinition


\beginDefinition


\DefNameBox{df:lambda-list-specializers}{Function}
\beginDocumentation
A list of the specializers for the required args, with T given for any
unspecialized args.\endDocumentation
\Usage{29}{\cd{(df:lambda-list-specializers lambda-list)}}\endUsage
\beginArguments
\argument{\cd{lambda-list}}\typeArg{\cd{List}}\endArg
\endArguments
\beginReturn
\singleReturn \typeReturn{\cd{List}}\endtReturn
\endReturn

\Source{ /home/comet/rga6711/xlisp/definition/fun.lsp}
\endDefinition


\beginDefinition


\DefNameBox{df:lambda-list-whole-arg}{Function}
\beginDocumentation
The name of the \&whole arg in a (macro's) lambda list.\endDocumentation
\Usage{26}{\cd{(df:lambda-list-whole-arg lambda-list)}}\endUsage
\beginArguments
\argument{\cd{lambda-list}}\typeArg{\cd{List}}\endArg
\endArguments
\beginReturn
\singleReturn \typeReturn{\cd{Symbol}}\endtReturn
\endReturn

\Source{ /home/comet/rga6711/xlisp/definition/fun.lsp}
\endDefinition


\beginDefinition


\DefNameBox{df:macro-definition}{Class}
\beginDocumentation
A definition class for \<defmacro\>.\endDocumentation
\Usage{35}{\cd{(send df:macro-definition :new )}}\endUsage

\Parents{\nextInList Df:Lambda-List-Definition}\endParent
\beginMethods{\nextInList
 Df:Definition-Arg-Types\nextInList
 Df:Definition-Usage\nextInList
 Df:Definition-Class-Nice-Name\nextInList
 :Isnew\nextInList
 Xlos:Shared-Initialize}\endMethods

\Source{ /home/comet/rga6711/xlisp/definition/fun.lsp}
\endDefinition


\beginDefinition


\DefNameBox{:major -\> (df:version-doc-proto)}{Message}
\beginDocumentation
The 1 in version 1.0.2.\endDocumentation
\Usage{8}{\cd{(send <obj> :major &optional new-value)}}\endUsage
\beginArguments
\argument{\cd{new-value}}\typeArg{\cd{Integer}}\endArg
\endArguments

\Source{ /home/comet/rga6711/xlisp/definition/project-doc.lsp}
\endDefinition


\beginDefinition


\DefNameBox{df:make-definitions}{Function}
\beginDocumentation
Make and return a list of the definition objects corresponding to
the result of evaluating \<form\> (which was read from the file
corresponding to \<path\>.\endDocumentation
\Usage{21}{\cd{(df:make-definitions form path)}}\endUsage
\beginArguments
\argument{\cd{form}}\typeArg{\cd{List}}\endArg
\argument{\cd{path}}\typeArg{\cd{Pathname}}\endArg
\endArguments
\beginReturn
\singleReturn \complexReturn{\cd{definitions}}\midReturn{\cd{List}}\endcReturn
\endReturn

\Source{ /home/comet/rga6711/xlisp/definition/build.lsp}
\endDefinition


\beginDefinition


\DefNameBox{df:make-slot-accessor-definitions}{Generic Function}
\beginDocumentation
Make definition objects corresponding to automatically
generated accessor functions.\endDocumentation
\Usage{35}{\cd{(df:make-slot-accessor-definitions def slot-spec path)}}\endUsage
\beginArguments
\argument{\cd{def}}\typeArg{\cd{Definitions:User-Type-Definition}}\endArg
\argument{\cd{slot-spec}}\typeArg{\cd{List}}\endArg
\argument{\cd{path}}\typeArg{\cd{Pathname}}\endArg
\endArguments
\beginReturn
\singleReturn \complexReturn{\cd{definitions}}\midReturn{\cd{List}}\endcReturn
\endReturn

\Source{ /home/comet/rga6711/xlisp/definition/build.lsp}
\endDefinition


\beginDefinition


\DefNameBox{df:make-slot-accessor-definitions Df:Class-Definition System:T System:T}{Primary Method}
\beginDocumentation
Make definition objects corresponding to automatically generated
accessor functions for a class.\endDocumentation
\Usage{35}{\cd{(df:make-slot-accessor-definitions def slot-spec path)}}\endUsage
\beginArguments
\argument{\cd{def}}\typeArg{\cd{Definitions:Class-Definition}}\endArg
\argument{\cd{slot-spec}}\typeArg{\cd{List}}\endArg
\argument{\cd{path}}\typeArg{\cd{Pathname}}\endArg
\endArguments
\beginReturn
\singleReturn \complexReturn{\cd{definitions}}\midReturn{\cd{List}}\endcReturn
\endReturn

\Source{ /home/comet/rga6711/xlisp/definition/build.lsp}
\endDefinition


\beginDefinition


\DefNameBox{df:make-slot-accessor-definitions Df:Structure-Definition System:T System:T}{Primary Method}
\beginDocumentation
Make definition objects corresponding to automatically generated
accessor functions for a structure.\endDocumentation
\Usage{35}{\cd{(df:make-slot-accessor-definitions def slot-spec path)}}\endUsage
\beginArguments
\argument{\cd{def}}\typeArg{\cd{Definitions:Structure-Definition}}\endArg
\argument{\cd{slot-spec}}\typeArg{\cd{List}}\endArg
\argument{\cd{path}}\typeArg{\cd{Pathname}}\endArg
\endArguments
\beginReturn
\singleReturn \complexReturn{\cd{definitions}}\midReturn{\cd{List}}\endcReturn
\endReturn

\Source{ /home/comet/rga6711/xlisp/definition/build.lsp}
\endDefinition


\beginDefinition


\DefNameBox{df:make-subdefinitions}{Generic Function}
\beginDocumentation
Make definition objects for definitions automatically generated
by the evaluation of the form corresponding to \<def\>.
An example of a subdefinition is a slot accessor function
automatically generated by a class definition.\endDocumentation
\Usage{24}{\cd{(df:make-subdefinitions def path)}}\endUsage
\beginArguments
\argument{\cd{def}}\typeArg{\cd{Definitions:Definition}}\endArg
\argument{\cd{path}}\typeArg{\cd{Pathname}}\endArg
\endArguments
\beginReturn
\singleReturn \complexReturn{\cd{definitions}}\midReturn{\cd{List}}\endcReturn
\endReturn

\Source{ /home/comet/rga6711/xlisp/definition/build.lsp}
\endDefinition


\beginDefinition


\DefNameBox{df:make-subdefinitions Df:Definition System:T}{Primary Method}
\beginDocumentation
The default method for \<make-subdefinitions\> returns ().\endDocumentation
\Usage{24}{\cd{(df:make-subdefinitions def path)}}\endUsage
\beginArguments
\argument{\cd{def}}\typeArg{\cd{Definitions:Definition}}\endArg
\argument{\cd{path}}\typeArg{\cd{Pathname}}\endArg
\endArguments
\beginReturn
\singleReturn \simpleReturn{\cd{nil}}\endsReturn
\endReturn

\Source{ /home/comet/rga6711/xlisp/definition/build.lsp}
\endDefinition


\beginDefinition


\DefNameBox{df:make-subdefinitions Df:Class-Definition System:T}{Primary Method}
\beginDocumentation
The method for classes returns a list of definition objects for the
slot accessor functions.\endDocumentation
\Usage{24}{\cd{(df:make-subdefinitions def path)}}\endUsage
\beginArguments
\argument{\cd{def}}\typeArg{\cd{Definitions:Definition}}\endArg
\argument{\cd{path}}\typeArg{\cd{Pathname}}\endArg
\endArguments
\beginReturn
\singleReturn \complexReturn{\cd{definitions}}\midReturn{\cd{List}}\endcReturn
\endReturn

\Source{ /home/comet/rga6711/xlisp/definition/build.lsp}
\endDefinition


\beginDefinition


\DefNameBox{df:make-subdefinitions Df:Structure-Definition System:T}{Primary Method}
\beginDocumentation
The method for structures returns a list of definition objects for
the automatically generated constructor, copier, and predicate
functions, if they are generated, and the slot accessor functions.\endDocumentation
\Usage{24}{\cd{(df:make-subdefinitions def path)}}\endUsage
\beginArguments
\argument{\cd{def}}\typeArg{\cd{Definitions:Definition}}\endArg
\argument{\cd{path}}\typeArg{\cd{Pathname}}\endArg
\endArguments
\beginReturn
\singleReturn \complexReturn{\cd{definitions}}\midReturn{\cd{List}}\endcReturn
\endReturn

\Source{ /home/comet/rga6711/xlisp/definition/build.lsp}
\endDefinition


\beginDefinition


\DefNameBox{df:message-definition}{Class}
\beginDocumentation
A definition class for defmeth.  Note call Message to avoid name
  class with CLOS method.\endDocumentation
\Usage{37}{\cd{(send df:message-definition :new )}}\endUsage

\Parents{\nextInList Df:Method-Definition}\endParent
\beginMethods{\nextInList
 Df:Definition-Specializers\nextInList
 Df::Definition-Name-\>Tag\nextInList
 Df:Definition-Name-\>String\nextInList
 Df:Definition-Lambda-List\nextInList
 Df:Definition-Name\nextInList
 Df:Definition-Class-Nice-Name\nextInList
 Df:Definition-Usage\nextInList
 Df:Definition-Body\nextInList
 :Isnew\nextInList
 Xlos:Shared-Initialize}\endMethods

\Source{ /home/comet/rga6711/xlisp/definition/xls-object.lsp}
\endDefinition


\beginDefinition


\DefNameBox{df:method-definition}{Class}
\beginDocumentation
A definition class for \<defmethod\>.\endDocumentation
\Usage{36}{\cd{(send df:method-definition :new )}}\endUsage

\Parents{\nextInList Df:Lambda-List-Definition}\endParent
\Children{\nextInList
 Df:Message-Definition}\endChildren
\beginMethods{\nextInList
 Df:Definition\<\nextInList
 Df:Definition-Specializers\nextInList
 Df:Definition-Body\nextInList
 Df:Definition-Lambda-List\nextInList
 Df::Definition-Name-\>Tag\nextInList
 Df:Definition-Name-\>String\nextInList
 Df:Definition-Name\nextInList
 Df:Definition-Class-Nice-Name\nextInList
 :Isnew\nextInList
 Xlos:Shared-Initialize}\endMethods

\Source{ /home/comet/rga6711/xlisp/definition/fun.lsp}
\endDefinition


\beginDefinition


\DefNameBox{:minor -\> (df:version-doc-proto)}{Message}
\beginDocumentation
The 0 in version 1.0.2.\endDocumentation
\Usage{8}{\cd{(send <obj> :minor &optional new-value)}}\endUsage
\beginArguments
\argument{\cd{new-value}}\typeArg{\cd{Integer}}\endArg
\endArguments

\Source{ /home/comet/rga6711/xlisp/definition/project-doc.lsp}
\endDefinition


\beginDefinition


\DefNameBox{:motivation -\> (df:project-doc-proto)}{Message}
\beginDocumentation
Explanation why the code was written.\endDocumentation
\Usage{13}{\cd{(send <obj> :motivation &optional new-value)}}\endUsage
\beginArguments
\argument{\cd{new-value}}\typeArg{\cd{String}}\endArg
\endArguments

\Source{ /home/comet/rga6711/xlisp/definition/project-doc.lsp}
\endDefinition


\beginDefinition


\DefNameBox{:name -\> (df:author-doc-proto)}{Message}
\beginDocumentation
Author's name.\endDocumentation
\Usage{7}{\cd{(send <obj> :name &optional new-value)}}\endUsage
\beginArguments
\argument{\cd{new-value}}\typeArg{\cd{String}}\endArg
\endArguments

\Source{ /home/comet/rga6711/xlisp/definition/project-doc.lsp}
\endDefinition


\beginDefinition


\DefNameBox{:nicknames -\> (df:project-doc-proto)}{Message}
\beginDocumentation
A list of package nicknames.\endDocumentation
\Usage{12}{\cd{(send <obj> :nicknames &optional new-value)}}\endUsage
\beginArguments
\argument{\cd{new-value}}\typeArg{\cd{List}}\endArg
\endArguments

\Source{ /home/comet/rga6711/xlisp/definition/project-doc.lsp}
\endDefinition


\beginDefinition


\DefNameBox{:os-dependencies -\> (df:project-doc-proto)}{Message}
\beginDocumentation
Description of operating system dependencies.\endDocumentation
\Usage{18}{\cd{(send <obj> :os-dependencies &optional new-value)}}\endUsage
\beginArguments
\argument{\cd{new-value}}\typeArg{\cd{String}}\endArg
\endArguments

\Source{ /home/comet/rga6711/xlisp/definition/project-doc.lsp}
\endDefinition


\beginDefinition


\DefNameBox{:package -\> (df:project-doc-proto)}{Message}
\beginDocumentation
Name of the package these
functions live in.  Warning:  leaving your module in the user package
could result in name conflicts with other packages and will make it
more difficult for the automatic documentation process to work.\endDocumentation
\Usage{10}{\cd{(send <obj> :package &optional new-value)}}\endUsage
\beginArguments
\argument{\cd{new-value}}\typeArg{\cd{(Or String T)}}\endArg
\endArguments

\Source{ /home/comet/rga6711/xlisp/definition/project-doc.lsp}
\endDefinition


\beginDefinition


\DefNameBox{df:package-definition}{Class}
\beginDocumentation
A definition class for \<defpackage\>.\endDocumentation
\Usage{37}{\cd{(send df:package-definition :new )}}\endUsage

\Parents{\nextInList Df:Definition}\endParent
\beginMethods{\nextInList
 Df:Definition-Usage\nextInList
 Df:Definition-Symbol\nextInList
 Df::Definition-Name-\>Tag\nextInList
 Df:Definition-Name-\>String\nextInList
 Df:Definition-Documentation\nextInList
 Df:Definition-Class-Nice-Name\nextInList
 :Isnew\nextInList
 Xlos:Shared-Initialize}\endMethods

\Source{ /home/comet/rga6711/xlisp/definition/pack.lsp}
\endDefinition


\beginDefinition


\DefNameBox{:package-dependencies -\> (df:project-doc-proto)}{Message}
\beginDocumentation
List of package dependencies.\endDocumentation
\Usage{23}{\cd{(send <obj> :package-dependencies &optional new-value)}}\endUsage
\beginArguments
\argument{\cd{new-value}}\typeArg{\cd{List}}\endArg
\endArguments

\Source{ /home/comet/rga6711/xlisp/definition/project-doc.lsp}
\endDefinition


\beginDefinition


\DefNameBox{df:parameter-definition}{Class}
\beginDocumentation
A definition class for \<defparameter\>.\endDocumentation
\Usage{39}{\cd{(send df:parameter-definition :new )}}\endUsage

\Parents{\nextInList Df:Global-Variable-Definition}\endParent
\beginMethods{\nextInList
 Df:Definition-Class-Nice-Name\nextInList
 :Isnew\nextInList
 Xlos:Shared-Initialize}\endMethods

\Source{ /home/comet/rga6711/xlisp/definition/global.lsp}
\endDefinition


\beginDefinition


\DefNameBox{:print -\> (df:definition)}{Message}
\beginDocumentation
A generic method for printing Definition objects.\endDocumentation
\Usage{8}{\cd{(send <obj> :print &optional stream)}}\endUsage
\beginArguments
\argument{\cd{stream}}\typeArg{\cd{(Or Stream T Nil)}}\endArg
\endArguments
\beginReturn
\singleReturn \simpleReturn{\cd{def}}\endsReturn
\endReturn

\Source{ /home/comet/rga6711/xlisp/definition/tex-print.lsp}
\endDefinition


\beginDefinition


\DefNameBox{:print -\> (df:version-doc-proto)}{Message}
\beginDocumentation
A generic method for printing Definition objects.\endDocumentation
\Usage{8}{\cd{(send <obj> :print &optional stream)}}\endUsage
\beginArguments
\argument{\cd{stream}}\typeArg{\cd{(Or Stream T Nil)}}\endArg
\endArguments
\beginReturn
\singleReturn \simpleReturn{\cd{self}}\endsReturn
\endReturn

\Source{ /home/comet/rga6711/xlisp/definition/project-doc.lsp}
\endDefinition


\beginDefinition


\DefNameBox{df:print-html-definition}{Generic Function}
\beginDocumentation
Print-Html a definition object in html.\endDocumentation
\Usage{26}{\cd{(df:print-html-definition def &key stream)}}\endUsage
\beginArguments
\argument{\cd{def}}\typeArg{\cd{Definitions:Definition}}\endArg
\argument{\cd{stream}}\typeArg{\cd{Stream}}\endArg
\endArguments

\Source{ /home/comet/rga6711/xlisp/definition/html-print.lsp}
\endDefinition


\beginDefinition


\DefNameBox{df:print-html-definition Df:Definition}{Primary Method}
\beginDocumentation
Print-Htmls out the reference manual entry for a definition object in html.\endDocumentation
\Usage{26}{\cd{(df:print-html-definition def &key stream)}}\endUsage
\beginArguments
\argument{\cd{def}}\typeArg{\cd{Definitions:Definition}}\endArg
\argument{\cd{stream}}\typeArg{\cd{Stream}}\endArg
\endArguments
\beginReturn
\singleReturn \simpleReturn{\cd{def}}\endsReturn
\endReturn

\Source{ /home/comet/rga6711/xlisp/definition/html-print.lsp}
\endDefinition


\beginDefinition


\DefNameBox{df:print-html-definitions}{Function}
\beginDocumentation
Print-Html a html representation of the definition objects in \<defs\>
on the file corresponding to \<path\>. 
  Title is used as the title of the page.\endDocumentation
\Usage{27}{\cd{(df:print-html-definitions defs path &key title)}}\endUsage
\beginArguments
\argument{\cd{defs}}\typeArg{\cd{List}}\endArg
\argument{\cd{path}}\typeArg{\cd{(Or String Pathname)}}\endArg
\argument{\cd{title}}\typeArg{\cd{String}}\endArg
\endArguments
\beginReturn
\singleReturn \simpleReturn{\cd{defs}}\endsReturn
\endReturn

\Source{ /home/comet/rga6711/xlisp/definition/html-print.lsp}
\endDefinition


\beginDefinition


\DefNameBox{df:print-html-documentation}{Generic Function}
\beginDocumentation
Print an html representation of the project documentation \<do\>
on the file corresponding to \<path\>.\endDocumentation
\Usage{29}{\cd{(df:print-html-documentation def path &key brief include-user-manual)}}\endUsage
\beginArguments
\argument{\cd{def}}\typeArg{\cd{T}}\endArg
\argument{\cd{path}}\typeArg{\cd{(Or String Pathname)}}\endArg
\argument{\cd{brief}}\typeArg{\cd{(Or T Null)}}\endArg
\argument{\cd{include-user-manual}}\typeArg{\cd{(Or T Null)}}\endArg
\endArguments
\beginReturn
\singleReturn \simpleReturn{\cd{defs}}\endsReturn
\endReturn

\Source{ /home/comet/rga6711/xlisp/definition/html-print-doc.lsp}
\endDefinition


\beginDefinition


\DefNameBox{df:print-html-documentation Df:Project-Doc-Proto System:T}{Primary Method}
\beginDocumentation
Print a Html representation of the project documenation \<do\>
on the file corresponding to \<path\>.  

If \<Brief\> is non-nil, then only the functional docs will be printed.\endDocumentation
\Usage{29}{\cd{(df:print-html-documentation doc path &key brief include-user-manual)}}\endUsage
\beginArguments
\argument{\cd{doc}}\typeArg{\cd{Definitions:Project-Doc-Proto}}\endArg
\argument{\cd{path}}\typeArg{\cd{(Or String Pathname)}}\endArg
\argument{\cd{brief}}\typeArg{\cd{(Or T Null)}}\endArg
\argument{\cd{include-user-manual}}\typeArg{\cd{(Or T Null)}}\endArg
\endArguments
\beginReturn
\singleReturn \simpleReturn{\cd{defs}}\endsReturn
\endReturn

\Source{ /home/comet/rga6711/xlisp/definition/html-print-doc.lsp}
\endDefinition


\beginDefinition


\DefNameBox{df:print-tex-definition}{Generic Function}
\beginDocumentation
Print a definition object in TeX.\endDocumentation
\Usage{25}{\cd{(df:print-tex-definition def &key stream)}}\endUsage
\beginArguments
\argument{\cd{def}}\typeArg{\cd{Definitions:Definition}}\endArg
\argument{\cd{stream}}\typeArg{\cd{Stream}}\endArg
\endArguments

\Source{ /home/comet/rga6711/xlisp/definition/tex-print.lsp}
\endDefinition


\beginDefinition


\DefNameBox{df:print-tex-definition Df:Definition}{Primary Method}
\beginDocumentation
Prints out the reference manual entry for a definition object in TeX.\endDocumentation
\Usage{25}{\cd{(df:print-tex-definition def &key stream)}}\endUsage
\beginArguments
\argument{\cd{def}}\typeArg{\cd{Definitions:Definition}}\endArg
\argument{\cd{stream}}\typeArg{\cd{Stream}}\endArg
\endArguments
\beginReturn
\singleReturn \simpleReturn{\cd{def}}\endsReturn
\endReturn

\Source{ /home/comet/rga6711/xlisp/definition/tex-print.lsp}
\endDefinition


\beginDefinition


\DefNameBox{df:print-tex-definitions}{Function}
\beginDocumentation
Print a TeX representation of the definition objects in \<defs\>
on the file corresponding to \<path\>.  \<Mode\> should be either :tex for
Plain TeX or :latex for LaTeX.\endDocumentation
\Usage{26}{\cd{(df:print-tex-definitions defs path &key mode)}}\endUsage
\beginArguments
\argument{\cd{defs}}\typeArg{\cd{List}}\endArg
\argument{\cd{path}}\typeArg{\cd{(Or String Pathname)}}\endArg
\argument{\cd{mode}}\typeArg{\cd{(Member (Quote (:Tex :Latex)))}}\endArg
\endArguments
\beginReturn
\singleReturn \simpleReturn{\cd{defs}}\endsReturn
\endReturn

\Source{ /home/comet/rga6711/xlisp/definition/tex-print.lsp}
\endDefinition


\beginDefinition


\DefNameBox{df:print-tex-documentation}{Generic Function}
\beginDocumentation
Print a TeX representation of the project documenation \<do\>
on the file corresponding to \<path\>.  \<Mode\> should be either :tex for
Plain TeX or :latex for LaTeX.\endDocumentation
\Usage{28}{\cd{(df:print-tex-documentation def path &key mode brief include-user-manual)}}\endUsage
\beginArguments
\argument{\cd{def}}\typeArg{\cd{T}}\endArg
\argument{\cd{path}}\typeArg{\cd{(Or String Pathname)}}\endArg
\argument{\cd{mode}}\typeArg{\cd{(Member (Quote (:Tex :Latex)))}}\endArg
\argument{\cd{brief}}\typeArg{\cd{(Or T Null)}}\endArg
\argument{\cd{include-user-manual}}\typeArg{\cd{(Or T Null)}}\endArg
\endArguments
\beginReturn
\singleReturn \simpleReturn{\cd{defs}}\endsReturn
\endReturn

\Source{ /home/comet/rga6711/xlisp/definition/tex-print-doc.lsp}
\endDefinition


\beginDefinition


\DefNameBox{df:print-tex-documentation Df:Project-Doc-Proto System:T}{Primary Method}
\beginDocumentation
Print a TeX representation of the project documenation \<do\>
on the file corresponding to \<path\>.  

\<Mode\> should be either :tex for Plain TeX or :latex for LaTeX.
If \<Brief\> is non-nil, then only the functional docs will be printed.\endDocumentation
\Usage{28}{\cd{(df:print-tex-documentation doc path &key mode brief include-user-manual)}}\endUsage
\beginArguments
\argument{\cd{doc}}\typeArg{\cd{Definitions:Project-Doc-Proto}}\endArg
\argument{\cd{path}}\typeArg{\cd{(Or String Pathname)}}\endArg
\argument{\cd{mode}}\typeArg{\cd{(Member (Quote (:Tex :Latex)))}}\endArg
\argument{\cd{brief}}\typeArg{\cd{(Or T Null)}}\endArg
\argument{\cd{include-user-manual}}\typeArg{\cd{(Or T Null)}}\endArg
\endArguments
\beginReturn
\singleReturn \simpleReturn{\cd{defs}}\endsReturn
\endReturn

\Source{ /home/comet/rga6711/xlisp/definition/tex-print-doc.lsp}
\endDefinition


\beginDefinition


\DefNameBox{df:project-doc-proto}{Class}
\beginDocumentation
This is the basic object for documenting a project.  This object
can be used to produce printed documentation (via TeX) or on-line
documentation (via HTML).\endDocumentation
\Usage{36}{\cd{(send df:project-doc-proto :new  &key title authors version package nicknames package-dependencies motivation functional-description stat-description exported-objects exported-functions global-vars instructions examples xls-version os-dependencies warnings change-history see-also references copyright reviewed-by files source-home)}}\endUsage
\beginSlots
\slot{\cd{title}}\initargsSlot\cd{(:Title)}\typeSlot{\cd{String}}\docSlot{Title of the code.}\endSlot
\slot{\cd{authors}}\initargsSlot\cd{(:Authors)}\typeSlot{\cd{List}}\docSlot{A list of author-doc-proto describing the author(s) of the code.}\endSlot
\slot{\cd{version}}\initargsSlot\cd{(:Version)}\typeSlot{\cd{Definitions:Version-Doc-Proto}}\docSlot{Version of the code.}\endSlot
\slot{\cd{package}}\initargsSlot\cd{(:Package)}\typeSlot{\cd{(Or String T)}}\docSlot{Name of the package these
functions live in.  Warning:  leaving your module in the user package
could result in name conflicts with other packages and will make it
more difficult for the automatic documentation process to work.}\endSlot
\slot{\cd{nicknames}}\initargsSlot\cd{(:Nicknames)}\typeSlot{\cd{List}}\docSlot{A list of package nicknames.}\endSlot
\slot{\cd{package-dependencies}}\initargsSlot\cd{(:Package-Dependencies)}\typeSlot{\cd{List}}\docSlot{List of package dependencies.}\endSlot
\slot{\cd{motivation}}\initargsSlot\cd{(:Motivation)}\typeSlot{\cd{String}}\docSlot{Explanation why the code was written.}\endSlot
\slot{\cd{functional-description}}\initargsSlot\cd{(:Functional-Description)}\typeSlot{\cd{String}}\docSlot{What the code is supposed to accomplish.}\endSlot
\slot{\cd{stat-description}}\initargsSlot\cd{(:Stat-Description)}\typeSlot{\cd{String}}\docSlot{Description of the statistical idea behind the code}\endSlot
\slot{\cd{exported-objects}}\initargsSlot\cd{(:Exported-Objects)}\typeSlot{\cd{List}}\docSlot{A list of objects exported by the package.}\endSlot
\slot{\cd{exported-functions}}\initargsSlot\cd{(:Exported-Functions)}\typeSlot{\cd{List}}\docSlot{A list of functions exported by the package.}\endSlot
\slot{\cd{global-vars}}\initargsSlot\cd{(:Global-Vars)}\typeSlot{\cd{List}}\docSlot{A list of var-doc-proto that describe global variables used by the package.}\endSlot
\slot{\cd{instructions}}\initargsSlot\cd{(:Instructions)}\typeSlot{\cd{String}}\docSlot{User instructions for the package.}\endSlot
\slot{\cd{examples}}\initargsSlot\cd{(:Examples)}\typeSlot{\cd{List}}\docSlot{A list of strings containing examples of common uses for the package.}\endSlot
\slot{\cd{xls-version}}\initargsSlot\cd{(:Xls-Version)}\typeSlot{\cd{String}}\docSlot{A list of version-doc-proto containing information on versions of xls that are compatible and incompatible with the package.}\endSlot
\slot{\cd{os-dependencies}}\initargsSlot\cd{(:Os-Dependencies)}\typeSlot{\cd{String}}\docSlot{Description of operating system dependencies.}\endSlot
\slot{\cd{warnings}}\initargsSlot\cd{(:Warnings)}\typeSlot{\cd{List}}\docSlot{A list of strings, each with a caveat for the well-intentioned but possibly naive user.}\endSlot
\slot{\cd{change-history}}\initargsSlot\cd{(:Change-History)}\typeSlot{\cd{List}}\docSlot{A list of strings, each
describing the changes since the last version.}\endSlot
\slot{\cd{see-also}}\initargsSlot\cd{(:See-Also)}\typeSlot{\cd{List}}\docSlot{A list of strings, each containing a helpful alternative to the package.}\endSlot
\slot{\cd{references}}\initargsSlot\cd{(:References)}\typeSlot{\cd{List}}\docSlot{A list of strings, each with a reference to a book, article, paper, etc. of likely interest.}\endSlot
\slot{\cd{copyright}}\initargsSlot\cd{(:Copyright)}\typeSlot{\cd{String}}\docSlot{String describing
copyright information and licence to include in library including any
restrictions on use (i.e., educational and research only, or no
redistribution without permission.}\endSlot
\slot{\cd{reviewed-by}}\initargsSlot\cd{(:Reviewed-By)}\typeSlot{\cd{Definitions:Author-Doc-Proto}}\docSlot{Who reviewed the package for the archive.}\endSlot
\slot{\cd{files}}\initargsSlot\cd{(:Files)}\typeSlot{\cd{List}}\docSlot{A list of strings each with the name of a file 
	    	(relative to documentation file) contained in the project.  
	    	Autodocumentation feature uses these files for definitions.}\endSlot
\slot{\cd{source-home}}\initargsSlot\cd{(:Source-Home)}\typeSlot{\cd{Pathname}}\docSlot{This should automatically capture the location of a project so we 
	    	can do the appropriate source grabbing for the definitions.  Users 
	    	should not need to set this, it should automatically be set when the
	    	project-doc instance object is created.  This in rather assumes the the 
	    	project-doc is created while reading a file in the top level 
	    	directory of the project.}\endSlot
\endSlots
\Children{\nextInList
 Df:Project-Doc-Proto\nextInList
 Df:Project-Doc-Proto\nextInList
 Df:Project-Doc-Proto}\endChildren
\beginMethods{\nextInList
 Df::Print-Html-Documentation-Var-Docs\nextInList
 Df::Print-Html-Documentation-Vars\nextInList
 Df::Print-Html-Documentation-Fun-Docs\nextInList
 Df::Print-Html-Documentation-Funs\nextInList
 Df::Print-Html-Documentation-Object-Docs\nextInList
 Df::Print-Html-Documentation-Objects\nextInList
 Df::Print-Html-Documentation-Files\nextInList
 Df::Print-Html-Documentation-Package\nextInList
 Df::Print-Html-Documentation-Dependencies\nextInList
 Df::Print-Html-Documentation-Os\nextInList
 Df::Print-Html-Documentation-Xls\nextInList
 Df::Print-Html-Documentation-See\nextInList
 Df::Print-Html-Documentation-Changes\nextInList
 Df::Print-Html-Documentation-Warnings\nextInList
 Df::Print-Html-Documentation-Examples\nextInList
 Df::Print-Html-Documentation-Instructions\nextInList
 Df::Print-Html-Documentation-Refs\nextInList
 Df::Print-Html-Documentation-Stat\nextInList
 Df::Print-Html-Documentation-Motivation\nextInList
 Df::Print-Html-Documentation-Reviewers\nextInList
 Df::Print-Html-Documentation-Copyright\nextInList
 Df::Print-Html-Documentation-Functional\nextInList
 Df::Print-Html-Documentation-Version\nextInList
 Df::Print-Html-Documentation-Authors\nextInList
 Df::Print-Html-Documentation-Title\nextInList
 Df:Print-Html-Documentation\nextInList
 Df::Print-Tex-Documentation-Var-Docs\nextInList
 Df::Print-Tex-Documentation-Vars\nextInList
 Df::Print-Tex-Documentation-Fun-Docs\nextInList
 Df::Print-Tex-Documentation-Funs\nextInList
 Df::Print-Tex-Documentation-Object-Docs\nextInList
 Df::Print-Tex-Documentation-Objects\nextInList
 Df::Print-Tex-Documentation-Files\nextInList
 Df::Print-Tex-Documentation-Package\nextInList
 Df::Print-Tex-Documentation-Dependencies\nextInList
 Df::Print-Tex-Documentation-Os\nextInList
 Df::Print-Tex-Documentation-Xls\nextInList
 Df::Print-Tex-Documentation-See\nextInList
 Df::Print-Tex-Documentation-Changes\nextInList
 Df::Print-Tex-Documentation-Warnings\nextInList
 Df::Print-Tex-Documentation-Examples\nextInList
 Df::Print-Tex-Documentation-Instructions\nextInList
 Df::Print-Tex-Documentation-Refs\nextInList
 Df::Print-Tex-Documentation-Stat\nextInList
 Df::Print-Tex-Documentation-Motivation\nextInList
 Df::Print-Tex-Documentation-Reviewers\nextInList
 Df::Print-Tex-Documentation-Copyright\nextInList
 Df::Print-Tex-Documentation-Functional\nextInList
 Df::Print-Tex-Documentation-Version\nextInList
 Df::Print-Tex-Documentation-Authors\nextInList
 Df::Print-Tex-Documentation-Title\nextInList
 Df:Print-Tex-Documentation\nextInList
 Df:Build-Project-Definitions\nextInList
 :Source-Home\nextInList
 :Files\nextInList
 :Reviewed-By\nextInList
 :Copyright\nextInList
 :References\nextInList
 :See-Also\nextInList
 :Change-History\nextInList
 :Warnings\nextInList
 :Os-Dependencies\nextInList
 :Xls-Version\nextInList
 :Examples\nextInList
 :Instructions\nextInList
 :Global-Vars\nextInList
 :Exported-Functions\nextInList
 :Exported-Objects\nextInList
 :Stat-Description\nextInList
 :Functional-Description\nextInList
 :Motivation\nextInList
 :Package-Dependencies\nextInList
 :Nicknames\nextInList
 :Package\nextInList
 :Version\nextInList
 :Authors\nextInList
 :Title\nextInList
 :Isnew\nextInList
 Xlos:Shared-Initialize}\endMethods

\Source{ /home/comet/rga6711/xlisp/definition/project-doc.lsp}
\endDefinition


\beginDefinition


\DefNameBox{df:prototype-definition}{Class}
\beginDocumentation
A definition class for defproto\endDocumentation
\Usage{39}{\cd{(send df:prototype-definition :new )}}\endUsage

\Parents{\nextInList Df:User-Type-Definition}\endParent
\beginMethods{\nextInList
 Df::Print-Html-Definition-Slot-Types\nextInList
 Df::Print-Tex-Definition-Slot-Types\nextInList
 Df:Definition-Class-Nice-Name\nextInList
 Df:Definition-Usage\nextInList
 Df:Definition-Children\nextInList
 Df:Definition-Parents\nextInList
 Df::Definition-Methods\nextInList
 Df:Definition-Slots\nextInList
 Df:Definition-Documentation\nextInList
 Df:Definition-Definee\nextInList
 :Isnew\nextInList
 Xlos:Shared-Initialize}\endMethods

\Source{ /home/comet/rga6711/xlisp/definition/xls-object.lsp}
\endDefinition


\beginDefinition


\DefNameBox{df:read-definitions-from-file}{Function}
\beginDocumentation
Read all the forms in the file corresponding to \<path\>, and create
and return a list of definition objects corresponding to the result of
evaluating those forms (for which \<definer-class\> does not return
\<nil\>).\endDocumentation
\Usage{31}{\cd{(df:read-definitions-from-file path)}}\endUsage
\beginArguments
\argument{\cd{path}}\typeArg{\cd{(Or String Pathname)}}\endArg
\endArguments
\beginReturn
\singleReturn \typeReturn{\cd{List}}\endtReturn
\endReturn

\Source{ /home/comet/rga6711/xlisp/definition/read.lsp}
\endDefinition


\beginDefinition


\DefNameBox{df:read-definitions-from-files}{Function}
\beginDocumentation
Call \<read-definitions-from-file\> on each path in \<paths\>,
concatenating together the results.\endDocumentation
\Usage{32}{\cd{(df:read-definitions-from-files paths)}}\endUsage
\beginArguments
\argument{\cd{paths}}\typeArg{\cd{T}}\endArg
\endArguments
\beginReturn
\singleReturn \typeReturn{\cd{List}}\endtReturn
\endReturn

\Source{ /home/comet/rga6711/xlisp/definition/read.lsp}
\endDefinition


\beginDefinition


\DefNameBox{:references -\> (df:project-doc-proto)}{Message}
\beginDocumentation
A list of strings, each with a reference to a book, article, paper, etc. of likely interest.\endDocumentation
\Usage{13}{\cd{(send <obj> :references &optional new-value)}}\endUsage
\beginArguments
\argument{\cd{new-value}}\typeArg{\cd{List}}\endArg
\endArguments

\Source{ /home/comet/rga6711/xlisp/definition/project-doc.lsp}
\endDefinition


\beginDefinition


\DefNameBox{:release -\> (df:version-doc-proto)}{Message}
\beginDocumentation
a = alpha, b = beta, c = gamma for code 
version; perhaps some
other scheme for xls-version, say: x = code should work with this
xls version ; y = code tested and works with this xls version ; z =
code doesn't work with this xls version \endDocumentation
\Usage{10}{\cd{(send <obj> :release &optional new-value)}}\endUsage
\beginArguments
\argument{\cd{new-value}}\typeArg{\cd{String}}\endArg
\endArguments

\Source{ /home/comet/rga6711/xlisp/definition/project-doc.lsp}
\endDefinition


\beginDefinition


\DefNameBox{:rev-level -\> (df:version-doc-proto)}{Message}
\beginDocumentation
The 0 in version 1.0.2.\endDocumentation
\Usage{12}{\cd{(send <obj> :rev-level &optional new-value)}}\endUsage
\beginArguments
\argument{\cd{new-value}}\typeArg{\cd{Integer}}\endArg
\endArguments

\Source{ /home/comet/rga6711/xlisp/definition/project-doc.lsp}
\endDefinition


\beginDefinition


\DefNameBox{:reviewed-by -\> (df:project-doc-proto)}{Message}
\beginDocumentation
Who reviewed the package for the archive.\endDocumentation
\Usage{14}{\cd{(send <obj> :reviewed-by &optional new-value)}}\endUsage
\beginArguments
\argument{\cd{new-value}}\typeArg{\cd{Definitions:Author-Doc-Proto}}\endArg
\endArguments

\Source{ /home/comet/rga6711/xlisp/definition/project-doc.lsp}
\endDefinition


\beginDefinition


\DefNameBox{:see-also -\> (df:project-doc-proto)}{Message}
\beginDocumentation
A list of strings, each containing a helpful alternative to the package.\endDocumentation
\Usage{11}{\cd{(send <obj> :see-also &optional new-value)}}\endUsage
\beginArguments
\argument{\cd{new-value}}\typeArg{\cd{List}}\endArg
\endArguments

\Source{ /home/comet/rga6711/xlisp/definition/project-doc.lsp}
\endDefinition


\beginDefinition


\DefNameBox{df:setf-definition}{Class}
\beginDocumentation
A definition class for \<defsetf\>.\endDocumentation
\Usage{34}{\cd{(send df:setf-definition :new )}}\endUsage

\Parents{\nextInList Df:Lambda-List-Definition}\endParent
\beginMethods{\nextInList
 Df:Definition-Declarations\nextInList
 Df:Definition-Documentation\nextInList
 Df:Definition-Arg-Types\nextInList
 Df:Definition-Lambda-List\nextInList
 Df:Definition-Name\nextInList
 Df:Definition-Class-Nice-Name\nextInList
 Df::Definition-Name-\>Tag\nextInList
 :Isnew\nextInList
 Xlos:Shared-Initialize}\endMethods

\Source{ /home/comet/rga6711/xlisp/definition/fun.lsp}
\endDefinition


\beginDefinition


\DefNameBox{:source-home -\> (df:project-doc-proto)}{Message}
\beginDocumentation
This should automatically capture the location of a project so we 
	    	can do the appropriate source grabbing for the definitions.  Users 
	    	should not need to set this, it should automatically be set when the
	    	project-doc instance object is created.  This in rather assumes the the 
	    	project-doc is created while reading a file in the top level 
	    	directory of the project.\endDocumentation
\Usage{14}{\cd{(send <obj> :source-home &optional new-value)}}\endUsage
\beginArguments
\argument{\cd{new-value}}\typeArg{\cd{Pathname}}\endArg
\endArguments

\Source{ /home/comet/rga6711/xlisp/definition/project-doc.lsp}
\endDefinition


\beginDefinition


\DefNameBox{:stat-description -\> (df:project-doc-proto)}{Message}
\beginDocumentation
Description of the statistical idea behind the code\endDocumentation
\Usage{19}{\cd{(send <obj> :stat-description &optional new-value)}}\endUsage
\beginArguments
\argument{\cd{new-value}}\typeArg{\cd{String}}\endArg
\endArguments

\Source{ /home/comet/rga6711/xlisp/definition/project-doc.lsp}
\endDefinition


\beginDefinition


\DefNameBox{df:structure-definition}{Class}
\beginDocumentation
A definition class for \<defstruct\>.\endDocumentation
\Usage{39}{\cd{(send df:structure-definition :new )}}\endUsage

\Parents{\nextInList Df:User-Type-Definition}\endParent
\beginMethods{\nextInList
 Df:Make-Subdefinitions\nextInList
 Df:Make-Slot-Accessor-Definitions\nextInList
 Df:Definition-Slots\nextInList
 Df:Definition-Documentation\nextInList
 Df:Definition-Name\nextInList
 Df:Definition-Class-Nice-Name\nextInList
 :Isnew\nextInList
 Xlos:Shared-Initialize}\endMethods

\Source{ /home/comet/rga6711/xlisp/definition/type.lsp}
\endDefinition


\beginDefinition


\DefNameBox{:title -\> (df:project-doc-proto)}{Message}
\beginDocumentation
Title of the code.\endDocumentation
\Usage{8}{\cd{(send <obj> :title &optional new-value)}}\endUsage
\beginArguments
\argument{\cd{new-value}}\typeArg{\cd{String}}\endArg
\endArguments

\Source{ /home/comet/rga6711/xlisp/definition/project-doc.lsp}
\endDefinition


\beginDefinition


\DefNameBox{df:type-check}{Macro}
\Usage{15}{\cd{(df:type-check
 type &rest args)}}\endUsage

\Source{ /home/comet/rga6711/xlisp/definition/defs.lsp}
\endDefinition


\beginDefinition


\DefNameBox{df:type-definition}{Class}
\beginDocumentation
A definition class for \<deftype\>.\endDocumentation
\Usage{34}{\cd{(send df:type-definition :new )}}\endUsage

\Parents{\nextInList Df:User-Type-Definition\nextInList Df:Lambda-List-Definition}\endParent
\beginMethods{\nextInList
 Df:Definition-Documentation\nextInList
 Df:Definition-Class-Nice-Name\nextInList
 :Isnew\nextInList
 Xlos:Shared-Initialize}\endMethods

\Source{ /home/comet/rga6711/xlisp/definition/type.lsp}
\endDefinition


\beginDefinition


\DefNameBox{df:user-type-definition}{Class}
\beginDocumentation
An abstract super class for user defined types.\endDocumentation
\Usage{39}{\cd{(send df:user-type-definition :new )}}\endUsage

\Parents{\nextInList Df:Definition}\endParent
\Children{\nextInList
 Df:Class-Definition\nextInList
 Df:Type-Definition\nextInList
 Df:Prototype-Definition\nextInList
 Df:Condition-Definition\nextInList
 Df:Structure-Definition\nextInList
 Df:Class-Definition}\endChildren
\beginMethods{\nextInList
 Df:Definition\<\nextInList
 Df:Definition-Usage\nextInList
 Df::Definition-Name-\>Tag\nextInList
 :Isnew\nextInList
 Xlos:Shared-Initialize}\endMethods

\Source{ /home/comet/rga6711/xlisp/definition/type.lsp}
\endDefinition


\beginDefinition


\DefNameBox{df:variable-definition}{Class}
\beginDocumentation
A definition class for \<defvar\>.\endDocumentation
\Usage{38}{\cd{(send df:variable-definition :new )}}\endUsage

\Parents{\nextInList Df:Global-Variable-Definition}\endParent
\beginMethods{\nextInList
 Df:Definition-Class-Nice-Name\nextInList
 :Isnew\nextInList
 Xlos:Shared-Initialize}\endMethods

\Source{ /home/comet/rga6711/xlisp/definition/global.lsp}
\endDefinition


\beginDefinition


\DefNameBox{:version -\> (df:project-doc-proto)}{Message}
\beginDocumentation
Version of the code.\endDocumentation
\Usage{10}{\cd{(send <obj> :version &optional new-value)}}\endUsage
\beginArguments
\argument{\cd{new-value}}\typeArg{\cd{Definitions:Version-Doc-Proto}}\endArg
\endArguments

\Source{ /home/comet/rga6711/xlisp/definition/project-doc.lsp}
\endDefinition


\beginDefinition


\DefNameBox{df:version-doc-proto}{Class}
\beginDocumentation
This object holds information on
versions.  It is used to hold the version information for the code and
for xls.  The meaning of the release slot changes with the context.\endDocumentation
\Usage{36}{\cd{(send df:version-doc-proto :new  &key major minor rev-level date release)}}\endUsage
\beginSlots
\slot{\cd{major}}\initargsSlot\cd{(:Major)}\typeSlot{\cd{Integer}}\docSlot{The 1 in version 1.0.2.}\endSlot
\slot{\cd{minor}}\initargsSlot\cd{(:Minor)}\typeSlot{\cd{Integer}}\docSlot{The 0 in version 1.0.2.}\endSlot
\slot{\cd{rev-level}}\initargsSlot\cd{(:Rev-Level)}\typeSlot{\cd{Integer}}\docSlot{The 0 in version 1.0.2.}\endSlot
\slot{\cd{date}}\initargsSlot\cd{(:Date)}\typeSlot{\cd{String}}\docSlot{The date of the release.}\endSlot
\slot{\cd{release}}\initargsSlot\cd{(:Release)}\typeSlot{\cd{String}}\docSlot{a = alpha, b = beta, c = gamma for code 
version; perhaps some
other scheme for xls-version, say: x = code should work with this
xls version ; y = code tested and works with this xls version ; z =
code doesn't work with this xls version }\endSlot
\endSlots
\beginMethods{\nextInList
 Df:Version\>\nextInList
 :Print\nextInList
 :Release\nextInList
 :Date\nextInList
 :Rev-Level\nextInList
 :Minor\nextInList
 :Major\nextInList
 :Isnew\nextInList
 Xlos:Shared-Initialize}\endMethods

\Source{ /home/comet/rga6711/xlisp/definition/project-doc.lsp}
\endDefinition


\beginDefinition


\DefNameBox{df:version\>}{Generic Function}
\beginDocumentation
Returns T if \<ver1\> is more recent than \<ver2\>\endDocumentation
\Usage{13}{\cd{(df:version> ver1 ver2)}}\endUsage
\beginArguments
\argument{\cd{ver1}}\typeArg{\cd{Definitions:Version-Doc-Proto}}\endArg
\argument{\cd{ver2}}\typeArg{\cd{Definitions:Version-Doc-Proto}}\endArg
\endArguments

\Source{ /home/comet/rga6711/xlisp/definition/project-doc.lsp}
\endDefinition


\beginDefinition


\DefNameBox{df:version\> Df:Version-Doc-Proto System:T}{Primary Method}
\Usage{13}{\cd{(df:version> ver1 ver2)}}\endUsage
\beginArguments
\argument{\cd{ver1}}\typeArg{\cd{Definitions:Version-Doc-Proto}}\endArg
\argument{\cd{ver2}}\typeArg{\cd{T}}\endArg
\endArguments

\Source{ /home/comet/rga6711/xlisp/definition/project-doc.lsp}
\endDefinition


\beginDefinition


\DefNameBox{:warnings -\> (df:project-doc-proto)}{Message}
\beginDocumentation
A list of strings, each with a caveat for the well-intentioned but possibly naive user.\endDocumentation
\Usage{11}{\cd{(send <obj> :warnings &optional new-value)}}\endUsage
\beginArguments
\argument{\cd{new-value}}\typeArg{\cd{List}}\endArg
\endArguments

\Source{ /home/comet/rga6711/xlisp/definition/project-doc.lsp}
\endDefinition


\beginDefinition


\DefNameBox{:xls-version -\> (df:project-doc-proto)}{Message}
\beginDocumentation
A list of version-doc-proto containing information on versions of xls that are compatible and incompatible with the package.\endDocumentation
\Usage{14}{\cd{(send <obj> :xls-version &optional new-value)}}\endUsage
\beginArguments
\argument{\cd{new-value}}\typeArg{\cd{String}}\endArg
\endArguments

\Source{ /home/comet/rga6711/xlisp/definition/project-doc.lsp}
\endDefinition

