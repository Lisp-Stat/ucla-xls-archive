\input /home/comet/rga6711/xlisp/definition/lispdef.tex
\hfuzz=5pt


\beginDefinition


\DefNameBox{xlos:defclass}{Macro}
\beginDocumentation
This macro provides the defclass syntax for the XLISP-STAT object
system.  It creates an appropriate defproto form, along with
appropriate creation (:isnew) and accessor methods.  For more
information, see Steele [1990], p 822.

    Usage
    defclass class-name (\{superclass-name\}*)(\{slot-specifier\}*)[class-
      option]
    class-name:: symbol
    superclass-name:: symbol
    slot-specifier:: slot-name {\tt|} (slot-name [slot-option])
    slot-name:: symbol
    slot-option::
     \{:reader reader-function-name\}* {\tt|}
     \{:writer writer-function-name\}* {\tt|}
     \{:accessor reader-function-name\}* {\tt|}
     \{:allocation allocation-type\} {\tt|}
     \{:initarg initarg-name\}* {\tt|}
     \{:initform form\} {\tt|}
     \{:type type-specifier\} {\tt|}
     \{:documentation string\}
    reader-function-name:: symbol
    writer-function-name:: function-specifier
    function-specifier:: \{symbol \}
    initarg-name:: symbol
    allocation-type:: \{:instance {\tt|} :class {\tt|} :prototype\}
    class-option::
     (:default-initargs initarg-list) {\tt|}
     (:documentation string) 
    initarg-list:: \{initarg-name default-initial-value-form\}*
\endDocumentation
\Usage{15}{\cd{(xlos:defclass
 name prototypes slot-specifications &rest class-options)}}\endUsage

\Source{ /home/comet/rga6711/xlisp/xlos/defclass.lsp}
\endDefinition


\beginDefinition


\DefNameBox{xlos:defgeneric}{Macro}
\beginDocumentation
This macro creates a ``generic function'', a call to send with the
appropriate arguments.\endDocumentation
\Usage{17}{\cd{(xlos:defgeneric
 name arglist &body options)}}\endUsage

\Source{ /home/comet/rga6711/xlisp/xlos/defgeneric.lsp}
\endDefinition


\beginDefinition


\DefNameBox{xlos:defmethod}{Macro}
\beginDocumentation
This translates a defmethod form into the approprate defmeth form.\endDocumentation
\Usage{16}{\cd{(xlos:defmethod
 name ((obj-arg proto-name) . otherargs) &body body)}}\endUsage

\Source{ /home/comet/rga6711/xlisp/xlos/defgeneric.lsp}
\endDefinition


\beginDefinition


\DefNameBox{xlos:initialize-slot}{Function}
\beginDocumentation
This function does the work of initializing a slot.  It searches,
the list of args for the first one matching initargs.  If found, it
sets the slot to that value.  If not, it evaluates initform and sets
the slot to that value. 
  If the initial value does not match the type of the slot, it signals
an error.\endDocumentation
\Usage{22}{\cd{(xlos:initialize-slot obj slot-name initargs args initform)}}\endUsage
\beginArguments
\argument{\cd{obj}}\typeArg{\cd{T}}\endArg
\argument{\cd{slot-name}}\typeArg{\cd{T}}\endArg
\argument{\cd{initargs}}\typeArg{\cd{T}}\endArg
\argument{\cd{args}}\typeArg{\cd{T}}\endArg
\argument{\cd{initform}}\typeArg{\cd{T}}\endArg
\endArguments

\Source{ /home/comet/rga6711/xlisp/xlos/defclass.lsp}
\endDefinition


\beginDefinition


\DefNameBox{xlos:make-instance}{Function}
\beginDocumentation
This function is a quick and dirty implementation of make-instance.  It 
  expects either an XLS object or a symbol whose value is an XLS object.  
  It sends the object a :new method with args \<initargs\> and returns the 
  result.\endDocumentation
\Usage{20}{\cd{(xlos:make-instance obj &rest initargs)}}\endUsage
\beginArguments
\argument{\cd{obj}}\typeArg{\cd{(Or Symbol Object)}}\endArg
\argument{\cd{initargs}}\typeArg{\cd{T}}\endArg
\endArguments
\beginReturn
\singleReturn \complexReturn{\cd{new-obj}}\midReturn{\cd{Obj}}\endcReturn
\endReturn

\Source{ /home/comet/rga6711/xlisp/xlos/defclass.lsp}
\endDefinition


\beginDefinition


\DefNameBox{xlos:my-check-type}{Macro}
\beginDocumentation
This is a version of the check-type macro which will accept the
  name (symbol) of an object as a type name.\endDocumentation
\Usage{20}{\cd{(xlos:my-check-type
 place spec &optional string)}}\endUsage

\Source{ /home/comet/rga6711/xlisp/xlos/defclass.lsp}
\endDefinition


\beginDefinition


\DefNameBox{xlos:mytypep}{Function}
\beginDocumentation
This function performs the same as typep, only if the second
  argument is the name (symbol) of a prototype, then it check to see
  if val is an object of the appropriate kind.\endDocumentation
\Usage{14}{\cd{(xlos:mytypep val typ)}}\endUsage
\beginArguments
\argument{\cd{val}}\typeArg{\cd{T}}\endArg
\argument{\cd{typ}}\typeArg{\cd{T}}\endArg
\endArguments
\beginReturn
\singleReturn \typeReturn{\cd{(Member (Quote (T Nil)))}}\endtReturn
\endReturn

\Source{ /home/comet/rga6711/xlisp/xlos/defclass.lsp}
\endDefinition


\beginDefinition


\DefNameBox{xlos:objtypep}{Function}
\beginDocumentation
Tests to see if val is an XLS object which is an instance of \<obj-name\>.\endDocumentation
\Usage{15}{\cd{(xlos:objtypep val obj-name)}}\endUsage
\beginArguments
\argument{\cd{val}}\typeArg{\cd{T}}\endArg
\argument{\cd{obj-name}}\typeArg{\cd{Symbol}}\endArg
\endArguments
\beginReturn
\singleReturn \typeReturn{\cd{(Member (Quote (T Nil)))}}\endtReturn
\endReturn

\Source{ /home/comet/rga6711/xlisp/xlos/defclass.lsp}
\endDefinition


\beginDefinition


\DefNameBox{system:pathname}{Type}
\Usage{0}{\cd{(typep x 'system:pathname)}}\endUsage

\Source{ /home/comet/rga6711/xlisp/xlos/defclass.lsp}
\endDefinition


\beginDefinition


\DefNameBox{xlos:process-initargs}{Function}
\beginDocumentation
This function does the processing for the default initargs.  It
searches through the list of initargs and extracts the ones which are
not already on the list and appends them to the end.\endDocumentation
\Usage{23}{\cd{(xlos:process-initargs default-initargs actual-args)}}\endUsage
\beginArguments
\argument{\cd{default-initargs}}\typeArg{\cd{List}}\endArg
\argument{\cd{actual-args}}\typeArg{\cd{List}}\endArg
\endArguments
\beginReturn
\singleReturn \complexReturn{\cd{actual-args}}\midReturn{\cd{List}}\endcReturn
\endReturn

\Source{ /home/comet/rga6711/xlisp/xlos/defclass.lsp}
\endDefinition


\beginDefinition


\DefNameBox{xlos:shared-initialize}{Function}
\beginDocumentation
The shared-initialize generic function takes care of initializing
slot values.  It takes the keyword part of the initialization lambda
list and sets up the initial values for the slots to either (a) the
first appropriate keyword it finds or else (b) it executes the
initform for the slot.\endDocumentation
\Usage{24}{\cd{(xlos:shared-initialize obj &rest keyword-args)}}\endUsage
\beginArguments
\argument{\cd{obj}}\typeArg{\cd{Object}}\endArg
\argument{\cd{keyword-args}}\typeArg{\cd{T}}\endArg
\endArguments

\Source{ /home/comet/rga6711/xlisp/xlos/defclass.lsp}
\endDefinition


\beginDefinition


\DefNameBox{xlos:shared-initialize}{Constant}
\beginDocumentation
The name for the shared-initialize method.\endDocumentation
\Usage{0}{\cd{xlos:shared-initialize}}\endUsage

\Source{ /home/comet/rga6711/xlisp/xlos/defclass.lsp}
\endDefinition


\beginDefinition


\DefNameBox{:slot-type -\> (system:*object*)}{Message}
\beginDocumentation
Gets/Sets the type-specifier for the slot named \<slot-name\>.
Results are undefined in \<type-specifier\> is not a legal type
specifier.  This type specifier is only check when setting a slot
through a constructor made by the defclass system.\endDocumentation
\Usage{12}{\cd{(send <obj> :slot-type slot-name &optional type-specifier)}}\endUsage
\beginArguments
\argument{\cd{slot-name}}\typeArg{\cd{Symbol}}\endArg
\argument{\cd{type-specifier}}\typeArg{\cd{(Or T Null)}}\endArg
\endArguments
\beginReturn
\singleReturn \complexReturn{\cd{type-specifier}}\midReturn{\cd{T}}\endcReturn
\endReturn

\Source{ /home/comet/rga6711/xlisp/xlos/defclass.lsp}
\endDefinition


\beginDefinition


\DefNameBox{xlos:slot-types}{Constant}
\beginDocumentation
The name of a slot for storing type information.\endDocumentation
\Usage{0}{\cd{xlos:slot-types}}\endUsage

\Source{ /home/comet/rga6711/xlisp/xlos/defclass.lsp}
\endDefinition


\beginDefinition


\DefNameBox{"Xlos"}{Package}
\Usage{0}{\cd{(in-package "Xlos")}}\endUsage

\Source{ /home/comet/rga6711/xlisp/xlos/defclass.lsp}
\endDefinition


\beginDefinition


\DefNameBox{"Xlos"}{Package}
\Usage{0}{\cd{(in-package "Xlos")}}\endUsage

\Source{ /home/comet/rga6711/xlisp/xlos/defgeneric.lsp}
\endDefinition

