\input /home/comet/rga6711/xlisp/definition/lispdef.tex
\hfuzz=5pt


\beginProject{ XLOS:  CLOS Macros for XLISP objects}
\projectTitle{ XLOS:  CLOS Macros for XLISP objects}
\projectAuthor{ Russell Almond}{ Educational Testing Service}{ almond@acm.org}
\projectVersion{ Version 1.0.1 (Dec 18, 1998)}
\beginMotivationDesc
 (1) To provide a series of tools for easily
  translating code written in CLOS (Common Lisp Object System) to the
  XLISP object system.  (2) To provide some of the features of the
  Defclass macro, such as automatic creation of accessor functions and
  initialization functions and tightly coupled object defintition and
  documentation. 
\endMotivationDesc
\projectCopyright{ This work is public domain.  The authors make no
warantee as to its suitability for any particular purpose.}
\projectReviewer{ }{ }
\beginFunctionalDesc
 Provides a Defclass macro for
  defining prototypes and simplified versions of the defgeneric and
  defmethod macros.  (These are limited to one dispatching argument
  which must be a prototype.)  The macros work by translating the CLOS
  style definitions into corresponding defproto, defun and defmeth
  macros.  Thus XLOS only provides the syntatic sugar of CLOS while
  retaining the prototype-instance flavor of XLISP. 
\endFunctionalDesc
\beginStatDesc
 Providing a self-documenting defclass form
  (as opposed to the more limited defprotot) will encourage better
  documentation of XLS objects and will work well with the Definitions
  documentation package.  It also simpilifies translation of Common
  Lisp packages to XLS. 
\endStatDesc
\beginRefs
\projectRef{ Steele, G.L., Jr. [1990] ``Common LISP:  The
Language, Second Edition.'' Digital Press.}
\projectRef{ Kiczales, G., J. des Rivieres, and D.G. Bobrow [1991]``The Art of the Metaobject Protocol.''  MIT Press. }
\endRefs
\beginInstructions
 Including XLOS in the package use list for your
  package gives you access to the defclass, defgeneric and defmethod
  macros.  Syntax for these is described in Steele[1990] (or other
  Common Lisp references. 
\endInstructions
\beginExamples
\projectEx{ (xlos:defclass author-doc-proto ()
	       ((name :accessor :name :initarg :name :type string 
		      :initform ""
		      :documentation "Author's name.")
		(affiliation :accessor :affiliation 
			     :initarg :affiliation :type string 
			     :initform ""
		      :documentation "Author's Institution.")
		(email :accessor :email :initarg :email :type string 
		      :initform ""
		      :documentation "Author's email address.")
		(home-page :accessor :home-page :initarg :home-page :type string 
		      :initform ""
		      :documentation "Author's home-page URL.")
		)
	       (:documentation "This object holds information about an author or reviewer of the code.")
	       )}
\projectEx{ The forgoing example would automatically create a
  defproto statement, and defmeths for the :isnew, :name, etc
  methods.}
\projectEx{ (defgeneric print-tex-documentation-title (doc &key stream)
  (declare (type Project-Doc-Proto doc)
	   (type Stream  stream)
	   (:returns (type Project-Doc-Proto doc)))
  (:documentation "Prints (tex format) the title of this project description.")
  )
}
\projectEx{ The second example defines an alias function for (send doc
  print-tex-documenation :stream stream)}
\projectEx{ (defmethod print-tex-documentation-title ((doc Project-Doc-Proto) 
					&key (stream *standard-output*))
  "Prints (tex format) the title of this project description."
  (format stream "\projectTitle{ ~A}~%" (send doc :title))
  doc)}
\projectEx{ The last example is equivalent to (defmeth Project-Doc-Proto
  print-tex-documenation-title ...)}
\endExamples
\beginWarnings
\projectWarn{ Only syntatic sugar, still uses XLS object
  system.}
\projectWarn{ Only a subset of CLOS.  In particular, does
  not allow for dispatch on multiple objects, or non-object types.
  XLS type system and object system are not integerated.}
\projectWarn{ Only allows generic functions to dispatch on
  one argument.}
\projectWarn{ Adds slot XLOS:SLOT-TYPES and method
  :slot-type to *object*.  This may conflict with other packages.}
\endWarnings
\beginSeeAlsos
\projectSee{ Definitions package contains
  examples of usage.}
\projectSee{ Kiczales, des Riveres and Bobrow [1991]
  contains an alternative CLOSette implementation.}
\endSeeAlsos
\projectXLSVersion{ Tested with 3.52.5; unlikely to work with
  earlier versions.}
\projectOSVersion{ No known incompatabilities; tested on Unix
  (Sun Solaris) and Mac.}
\beginDepList \cd{ "Lisp" \hfudge } \endProjectList
\projectPackage{ "Xlos"}{}
\beginChanges
\projectChange{ Version 1.0.1 Added shared-initialize generic function.}
\endChanges
\beginFileList \cd{"/Home/Comet/Rga6711/Xlisp/Xlos/Defclass.Lsp" \hfudge "/Home/Comet/Rga6711/Xlisp/Xlos/Defgeneric.Lsp" \hfudge } \endProjectList
\beginObjectList \cd{Pathname \hfudge } \endProjectList
\beginFunList \cd{Defclass \hfudge Defgeneric \hfudge Defmethod \hfudge Initialize-Slot \hfudge Make-Instance \hfudge My-Check-Type \hfudge Mytypep \hfudge Objtypep \hfudge Pathname \hfudge Process-Initargs \hfudge Shared-Initialize \hfudge } \endProjectList
\beginVarList \cd{Shared-Initialize \hfudge Slot-Types \hfudge } \endProjectList
\beginObjects

\beginDefinition


\DefNameBox{system:pathname}{Type}
\Usage{0}{\cd{(typep x 'system:pathname)}}\endUsage

\Source{ /home/comet/rga6711/xlisp/xlos/defclass.lsp}
\endDefinition

\beginProjectFuns

\beginDefinition


\DefNameBox{xlos:defclass}{Macro}
\beginDocumentation
This macro provides the defclass syntax for the XLISP-STAT object
system.  It creates an appropriate defproto form, along with
appropriate creation (:isnew) and accessor methods.  For more
information, see Steele [1990], p 822.

    Usage
    defclass class-name (\{superclass-name\}*)(\{slot-specifier\}*)[class-
      option]
    class-name:: symbol
    superclass-name:: symbol
    slot-specifier:: slot-name {\tt|} (slot-name [slot-option])
    slot-name:: symbol
    slot-option::
     \{:reader reader-function-name\}* {\tt|}
     \{:writer writer-function-name\}* {\tt|}
     \{:accessor reader-function-name\}* {\tt|}
     \{:allocation allocation-type\} {\tt|}
     \{:initarg initarg-name\}* {\tt|}
     \{:initform form\} {\tt|}
     \{:type type-specifier\} {\tt|}
     \{:documentation string\}
    reader-function-name:: symbol
    writer-function-name:: function-specifier
    function-specifier:: \{symbol \}
    initarg-name:: symbol
    allocation-type:: \{:instance {\tt|} :class {\tt|} :prototype\}
    class-option::
     (:default-initargs initarg-list) {\tt|}
     (:documentation string) 
    initarg-list:: \{initarg-name default-initial-value-form\}*
\endDocumentation
\Usage{15}{\cd{(xlos:defclass
 name prototypes slot-specifications &rest class-options)}}\endUsage

\Source{ /home/comet/rga6711/xlisp/xlos/defclass.lsp}
\endDefinition


\beginDefinition


\DefNameBox{xlos:defgeneric}{Macro}
\beginDocumentation
This macro creates a ``generic function'', a call to send with the
appropriate arguments.\endDocumentation
\Usage{17}{\cd{(xlos:defgeneric
 name arglist &body options)}}\endUsage

\Source{ /home/comet/rga6711/xlisp/xlos/defgeneric.lsp}
\endDefinition


\beginDefinition


\DefNameBox{xlos:defmethod}{Macro}
\beginDocumentation
This translates a defmethod form into the approprate defmeth form.\endDocumentation
\Usage{16}{\cd{(xlos:defmethod
 name ((obj-arg proto-name) . otherargs) &body body)}}\endUsage

\Source{ /home/comet/rga6711/xlisp/xlos/defgeneric.lsp}
\endDefinition


\beginDefinition


\DefNameBox{xlos:initialize-slot}{Function}
\beginDocumentation
This function does the work of initializing a slot.  It searches,
the list of args for the first one matching initargs.  If found, it
sets the slot to that value.  If not, it evaluates initform and sets
the slot to that value. 
  If the initial value does not match the type of the slot, it signals
an error.\endDocumentation
\Usage{22}{\cd{(xlos:initialize-slot obj slot-name initargs args initform)}}\endUsage
\beginArguments
\argument{\cd{obj}}\typeArg{\cd{T}}\endArg
\argument{\cd{slot-name}}\typeArg{\cd{T}}\endArg
\argument{\cd{initargs}}\typeArg{\cd{T}}\endArg
\argument{\cd{args}}\typeArg{\cd{T}}\endArg
\argument{\cd{initform}}\typeArg{\cd{T}}\endArg
\endArguments

\Source{ /home/comet/rga6711/xlisp/xlos/defclass.lsp}
\endDefinition


\beginDefinition


\DefNameBox{xlos:make-instance}{Function}
\beginDocumentation
This function is a quick and dirty implementation of make-instance.  It 
  expects either an XLS object or a symbol whose value is an XLS object.  
  It sends the object a :new method with args \<initargs\> and returns the 
  result.\endDocumentation
\Usage{20}{\cd{(xlos:make-instance obj &rest initargs)}}\endUsage
\beginArguments
\argument{\cd{obj}}\typeArg{\cd{(Or Symbol Object)}}\endArg
\argument{\cd{initargs}}\typeArg{\cd{T}}\endArg
\endArguments
\beginReturn
\singleReturn \complexReturn{\cd{new-obj}}\midReturn{\cd{Obj}}\endcReturn
\endReturn

\Source{ /home/comet/rga6711/xlisp/xlos/defclass.lsp}
\endDefinition


\beginDefinition


\DefNameBox{xlos:my-check-type}{Macro}
\beginDocumentation
This is a version of the check-type macro which will accept the
  name (symbol) of an object as a type name.\endDocumentation
\Usage{20}{\cd{(xlos:my-check-type
 place spec &optional string)}}\endUsage

\Source{ /home/comet/rga6711/xlisp/xlos/defclass.lsp}
\endDefinition


\beginDefinition


\DefNameBox{xlos:mytypep}{Function}
\beginDocumentation
This function performs the same as typep, only if the second
  argument is the name (symbol) of a prototype, then it check to see
  if val is an object of the appropriate kind.\endDocumentation
\Usage{14}{\cd{(xlos:mytypep val typ)}}\endUsage
\beginArguments
\argument{\cd{val}}\typeArg{\cd{T}}\endArg
\argument{\cd{typ}}\typeArg{\cd{T}}\endArg
\endArguments
\beginReturn
\singleReturn \typeReturn{\cd{(Member (Quote (T Nil)))}}\endtReturn
\endReturn

\Source{ /home/comet/rga6711/xlisp/xlos/defclass.lsp}
\endDefinition


\beginDefinition


\DefNameBox{xlos:objtypep}{Function}
\beginDocumentation
Tests to see if val is an XLS object which is an instance of \<obj-name\>.\endDocumentation
\Usage{15}{\cd{(xlos:objtypep val obj-name)}}\endUsage
\beginArguments
\argument{\cd{val}}\typeArg{\cd{T}}\endArg
\argument{\cd{obj-name}}\typeArg{\cd{Symbol}}\endArg
\endArguments
\beginReturn
\singleReturn \typeReturn{\cd{(Member (Quote (T Nil)))}}\endtReturn
\endReturn

\Source{ /home/comet/rga6711/xlisp/xlos/defclass.lsp}
\endDefinition


\beginDefinition


\DefNameBox{system:pathname}{Type}
\Usage{0}{\cd{(typep x 'system:pathname)}}\endUsage

\Source{ /home/comet/rga6711/xlisp/xlos/defclass.lsp}
\endDefinition


\beginDefinition


\DefNameBox{xlos:process-initargs}{Function}
\beginDocumentation
This function does the processing for the default initargs.  It
searches through the list of initargs and extracts the ones which are
not already on the list and appends them to the end.\endDocumentation
\Usage{23}{\cd{(xlos:process-initargs default-initargs actual-args)}}\endUsage
\beginArguments
\argument{\cd{default-initargs}}\typeArg{\cd{List}}\endArg
\argument{\cd{actual-args}}\typeArg{\cd{List}}\endArg
\endArguments
\beginReturn
\singleReturn \complexReturn{\cd{actual-args}}\midReturn{\cd{List}}\endcReturn
\endReturn

\Source{ /home/comet/rga6711/xlisp/xlos/defclass.lsp}
\endDefinition


\beginDefinition


\DefNameBox{xlos:shared-initialize}{Function}
\beginDocumentation
The shared-initialize generic function takes care of initializing
slot values.  It takes the keyword part of the initialization lambda
list and sets up the initial values for the slots to either (a) the
first appropriate keyword it finds or else (b) it executes the
initform for the slot.\endDocumentation
\Usage{24}{\cd{(xlos:shared-initialize obj &rest keyword-args)}}\endUsage
\beginArguments
\argument{\cd{obj}}\typeArg{\cd{Object}}\endArg
\argument{\cd{keyword-args}}\typeArg{\cd{T}}\endArg
\endArguments

\Source{ /home/comet/rga6711/xlisp/xlos/defclass.lsp}
\endDefinition

\beginProjectVars

\beginDefinition


\DefNameBox{xlos:shared-initialize}{Constant}
\beginDocumentation
The name for the shared-initialize method.\endDocumentation
\Usage{0}{\cd{xlos:shared-initialize}}\endUsage

\Source{ /home/comet/rga6711/xlisp/xlos/defclass.lsp}
\endDefinition


\beginDefinition


\DefNameBox{xlos:slot-types}{Constant}
\beginDocumentation
The name of a slot for storing type information.\endDocumentation
\Usage{0}{\cd{xlos:slot-types}}\endUsage

\Source{ /home/comet/rga6711/xlisp/xlos/defclass.lsp}
\endDefinition

\endProject

