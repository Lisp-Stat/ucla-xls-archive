\begin{slide}{}
\chapter{Final Notes}
\end{slide}

\begin{slide}{}
\section{Where can you go from here?}
Get a copy of the software for your computer and try it on your problems:
\begin{itemize}
\item try the standard tools
\item try building tools of your own
\item try graphics tailored to your problems
\end{itemize}
If you develop new tools and ideas that might be useful to others,
please share them (e.g. by submitting them to \dcode{statlib} or the
\dcode{stat-lisp-news} mailing list).
\end{slide}

\begin{slide}{}
\section{Electronic sources of information}
There is a mailing list of users who help to answer questions, share
ideas, etc.

To join the mailing list, send mail to:
\begin{center}
\Large
\dcode{stat-lisp-news-request@umnstat.stat.umn.edu}
\end{center}
Once you are on the list, you can send mail to all members of the list
by sending a message to
\begin{center}
\dcode{stat-lisp-news@umnstat.stat.umn.edu}
\end{center}
The \dcode{statlib} archive contains contributed Lisp-Stat code, as
well as the source code for XLISP-STAT.

To find out what is available from \dcode{statlib}, send mail to
\begin{center}
\dcode{statlib@lib.stat.cmu.edu}
\end{center}
containing the single line
\begin{center}
\dcode{send index}
\end{center}
\dcode{statlib} can also be accessed by {\em ftp}.
\end{slide}

\begin{slide}{}
\section{Where Lisp-Stat is headed}
Lisp-Stat is an evolving system.

Plans for the near term are to
\begin{itemize}
\item improve and update current versions
\item develop a Common Lisp version
\end{itemize}
Longer-term plans include
\begin{itemize}
\item improving the support for compound data objects
\item some support for missing data conventions
\item increased graphics capabilities
\item support for constraints
\end{itemize}
\end{slide}

\begin{slide}{}
Further development depends on the users of Lisp-Stat:
\begin{itemize}
\item
If people find the system useful, new methods for handling a variety
of problems will become available (e.g. through \dcode{statlib}).
\item
As new ideas are implemented, the need for new tools and primitives,
or for modifications to existing ones, will become evident.
\end{itemize}
\end{slide}

\begin{slide}{}
Some books on Lisp and Lisp programming:

\refitem{{\sc Abelson, H. and Sussman, G. J.} (1985), {\em Structure and 
Interpretation of Computer Programs}, New York: McGraw-Hill.}

\refitem{{\sc Franz Inc.} (1988), {\em Common Lisp: The Reference},
Reading, MA: Addison-Wesley.}

\refitem{{\sc Steele, Guy L.} (1990), {\em Common Lisp: The Language}, 
second edition, Bedford, MA: Digital Press.}

\refitem{{\sc Winston, Patrick H. and Berthold K. P. Horn}, (1988),
{\em LISP}, 3rd Ed., New York: Addison-Wesley.}
\end{slide}
