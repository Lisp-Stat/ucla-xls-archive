%\includeonly{objects}
\documentstyle[handout,12pt]{report}
% Psfig/TeX Release 1.2
% dvips version
%
% All software, documentation, and related files in this distribution of
% psfig/tex are Copyright 1987, 1988 Trevor J. Darrell
%
% Permission is granted for use and non-profit distribution of psfig/tex 
% providing that this notice be clearly maintained, but the right to
% distribute any portion of psfig/tex for profit or as part of any commercial
% product is specifically reserved for the author.
%
% $Header: psfig.tex,v 1.9 88/01/08 17:42:01 trevor Exp $
% $Source: $
%
% Thanks to Greg Hager (GDH) and Ned Batchelder for their contributions
% to this project.
%
\catcode`\@=11\relax
\newwrite\@unused
\def\typeout#1{{\let\protect\string\immediate\write\@unused{#1}}}
\typeout{psfig/tex 1.2-dvips}


%% Here's how you define your figure path.  Should be set up with null
%% default and a user useable definition.

\def\figurepath{./}
\def\psfigurepath#1{\edef\figurepath{#1}}

%
% @psdo control structure -- similar to Latex @for.
% I redefined these with different names so that psfig can
% be used with TeX as well as LaTeX, and so that it will not 
% be vunerable to future changes in LaTeX's internal
% control structure,
%
\def\@nnil{\@nil}
\def\@empty{}
\def\@psdonoop#1\@@#2#3{}
\def\@psdo#1:=#2\do#3{\edef\@psdotmp{#2}\ifx\@psdotmp\@empty \else
    \expandafter\@psdoloop#2,\@nil,\@nil\@@#1{#3}\fi}
\def\@psdoloop#1,#2,#3\@@#4#5{\def#4{#1}\ifx #4\@nnil \else
       #5\def#4{#2}\ifx #4\@nnil \else#5\@ipsdoloop #3\@@#4{#5}\fi\fi}
\def\@ipsdoloop#1,#2\@@#3#4{\def#3{#1}\ifx #3\@nnil 
       \let\@nextwhile=\@psdonoop \else
      #4\relax\let\@nextwhile=\@ipsdoloop\fi\@nextwhile#2\@@#3{#4}}
\def\@tpsdo#1:=#2\do#3{\xdef\@psdotmp{#2}\ifx\@psdotmp\@empty \else
    \@tpsdoloop#2\@nil\@nil\@@#1{#3}\fi}
\def\@tpsdoloop#1#2\@@#3#4{\def#3{#1}\ifx #3\@nnil 
       \let\@nextwhile=\@psdonoop \else
      #4\relax\let\@nextwhile=\@tpsdoloop\fi\@nextwhile#2\@@#3{#4}}
% 
%
\def\psdraft{
	\def\@psdraft{0}
	%\typeout{draft level now is \@psdraft \space . }
}
\def\psfull{
	\def\@psdraft{100}
	%\typeout{draft level now is \@psdraft \space . }
}
\psfull
\newif\if@prologfile
\newif\if@postlogfile
\newif\if@noisy
\def\pssilent{
	\@noisyfalse
}
\def\psnoisy{
	\@noisytrue
}
\psnoisy
%%% These are for the option list.
%%% A specification of the form a = b maps to calling \@p@@sa{b}
\newif\if@bbllx
\newif\if@bblly
\newif\if@bburx
\newif\if@bbury
\newif\if@height
\newif\if@width
\newif\if@rheight
\newif\if@rwidth
\newif\if@clip
\newif\if@verbose
\def\@p@@sclip#1{\@cliptrue}

%%% GDH 7/26/87 -- changed so that it first looks in the local directory,
%%% then in a specified global directory for the ps file.

\def\@p@@sfile#1{\def\@p@sfile{null}%
	        \openin1=#1
		\ifeof1\closein1%
		       \openin1=\figurepath#1
			\ifeof1\typeout{Error, File #1 not found}
			\else\closein1
			    \edef\@p@sfile{\figurepath#1}%
                        \fi%
		 \else\closein1%
		       \def\@p@sfile{#1}%
		 \fi}
\def\@p@@sfigure#1{\def\@p@sfile{null}%
	        \openin1=#1
		\ifeof1\closein1%
		       \openin1=\figurepath#1
			\ifeof1\typeout{Error, File #1 not found}
			\else\closein1
			    \def\@p@sfile{\figurepath#1}%
                        \fi%
		 \else\closein1%
		       \def\@p@sfile{#1}%
		 \fi}

\def\@p@@sbbllx#1{
		%\typeout{bbllx is #1}
		\@bbllxtrue
		\dimen100=#1
		\edef\@p@sbbllx{\number\dimen100}
}
\def\@p@@sbblly#1{
		%\typeout{bblly is #1}
		\@bbllytrue
		\dimen100=#1
		\edef\@p@sbblly{\number\dimen100}
}
\def\@p@@sbburx#1{
		%\typeout{bburx is #1}
		\@bburxtrue
		\dimen100=#1
		\edef\@p@sbburx{\number\dimen100}
}
\def\@p@@sbbury#1{
		%\typeout{bbury is #1}
		\@bburytrue
		\dimen100=#1
		\edef\@p@sbbury{\number\dimen100}
}
\def\@p@@sheight#1{
		\@heighttrue
		\dimen100=#1
   		\edef\@p@sheight{\number\dimen100}
		%\typeout{Height is \@p@sheight}
}
\def\@p@@swidth#1{
		%\typeout{Width is #1}
		\@widthtrue
		\dimen100=#1
		\edef\@p@swidth{\number\dimen100}
}
\def\@p@@srheight#1{
		%\typeout{Reserved height is #1}
		\@rheighttrue
		\dimen100=#1
		\edef\@p@srheight{\number\dimen100}
}
\def\@p@@srwidth#1{
		%\typeout{Reserved width is #1}
		\@rwidthtrue
		\dimen100=#1
		\edef\@p@srwidth{\number\dimen100}
}
\def\@p@@ssilent#1{ 
		\@verbosefalse
}
\def\@p@@sprolog#1{\@prologfiletrue\def\@prologfileval{#1}}
\def\@p@@spostlog#1{\@postlogfiletrue\def\@postlogfileval{#1}}
\def\@cs@name#1{\csname #1\endcsname}
\def\@setparms#1=#2,{\@cs@name{@p@@s#1}{#2}}
%
% initialize the defaults (size the size of the figure)
%
\def\ps@init@parms{
		\@bbllxfalse \@bbllyfalse
		\@bburxfalse \@bburyfalse
		\@heightfalse \@widthfalse
		\@rheightfalse \@rwidthfalse
		\def\@p@sbbllx{}\def\@p@sbblly{}
		\def\@p@sbburx{}\def\@p@sbbury{}
		\def\@p@sheight{}\def\@p@swidth{}
		\def\@p@srheight{}\def\@p@srwidth{}
		\def\@p@sfile{}
		\def\@p@scost{10}
		\def\@sc{}
		\@prologfilefalse
		\@postlogfilefalse
		\@clipfalse
		\if@noisy
			\@verbosetrue
		\else
			\@verbosefalse
		\fi
}
%
% Go through the options setting things up.
%
\def\parse@ps@parms#1{
	 	\@psdo\@psfiga:=#1\do
		   {\expandafter\@setparms\@psfiga,}}
%
% Compute bb height and width
%
\newif\ifno@bb
\newif\ifnot@eof
\newread\ps@stream
\def\bb@missing{
	\if@verbose{
		\typeout{psfig: searching \@p@sfile \space  for bounding box}
	}\fi
	\openin\ps@stream=\@p@sfile
	\no@bbtrue
	\not@eoftrue
	\catcode`\%=12
	\loop
		\read\ps@stream to \line@in
		\global\toks200=\expandafter{\line@in}
		\ifeof\ps@stream \not@eoffalse \fi
		%\typeout{ looking at :: \the\toks200 }
		\@bbtest{\toks200}
		\if@bbmatch\not@eoffalse\expandafter\bb@cull\the\toks200\fi
	\ifnot@eof \repeat
	\catcode`\%=14
}	
\catcode`\%=12
\newif\if@bbmatch
\def\@bbtest#1{\expandafter\@a@\the#1%%BoundingBox:\@bbtest\@a@}
\long\def\@a@#1%%BoundingBox:#2#3\@a@{\ifx\@bbtest#2\@bbmatchfalse\else\@bbmatchtrue\fi}
\long\def\bb@cull#1 #2 #3 #4 #5 {
	\dimen100=#2 bp\edef\@p@sbbllx{\number\dimen100}
	\dimen100=#3 bp\edef\@p@sbblly{\number\dimen100}
	\dimen100=#4 bp\edef\@p@sbburx{\number\dimen100}
	\dimen100=#5 bp\edef\@p@sbbury{\number\dimen100}
	\no@bbfalse
}
\catcode`\%=14
%
\def\compute@bb{
		\no@bbfalse
		\if@bbllx \else \no@bbtrue \fi
		\if@bblly \else \no@bbtrue \fi
		\if@bburx \else \no@bbtrue \fi
		\if@bbury \else \no@bbtrue \fi
		\ifno@bb \bb@missing \fi
		\ifno@bb \typeout{FATAL ERROR: no bb supplied or found}
			\no-bb-error
		\fi
		%
		\count203=\@p@sbburx
		\count204=\@p@sbbury
		\advance\count203 by -\@p@sbbllx
		\advance\count204 by -\@p@sbblly
		\edef\@bbw{\number\count203}
		\edef\@bbh{\number\count204}
		%\typeout{ bbh = \@bbh, bbw = \@bbw }
}
%
% \in@hundreds performs #1 * (#2 / #3) correct to the hundreds,
%	then leaves the result in @result
%
\def\in@hundreds#1#2#3{\count240=#2 \count241=#3
		     \count100=\count240	% 100 is first digit #2/#3
		     \divide\count100 by \count241
		     \count101=\count100
		     \multiply\count101 by \count241
		     \advance\count240 by -\count101
		     \multiply\count240 by 10
		     \count101=\count240	%101 is second digit of #2/#3
		     \divide\count101 by \count241
		     \count102=\count101
		     \multiply\count102 by \count241
		     \advance\count240 by -\count102
		     \multiply\count240 by 10
		     \count102=\count240	% 102 is the third digit
		     \divide\count102 by \count241
		     \count200=#1\count205=0
		     \count201=\count200
			\multiply\count201 by \count100
		 	\advance\count205 by \count201
		     \count201=\count200
			\divide\count201 by 10
			\multiply\count201 by \count101
			\advance\count205 by \count201
			%
		     \count201=\count200
			\divide\count201 by 100
			\multiply\count201 by \count102
			\advance\count205 by \count201
			%
		     \edef\@result{\number\count205}
}
\def\compute@wfromh{
		% computing : width = height * (bbw / bbh)
		\in@hundreds{\@p@sheight}{\@bbw}{\@bbh}
		%\typeout{ \@p@sheight * \@bbw / \@bbh, = \@result }
		\edef\@p@swidth{\@result}
		%\typeout{w from h: width is \@p@swidth}
}
\def\compute@hfromw{
		% computing : height = width * (bbh / bbw)
		\in@hundreds{\@p@swidth}{\@bbh}{\@bbw}
		%\typeout{ \@p@swidth * \@bbh / \@bbw = \@result }
		\edef\@p@sheight{\@result}
		%\typeout{h from w : height is \@p@sheight}
}
\def\compute@handw{
		\if@height 
			\if@width
			\else
				\compute@wfromh
			\fi
		\else 
			\if@width
				\compute@hfromw
			\else
				\edef\@p@sheight{\@bbh}
				\edef\@p@swidth{\@bbw}
			\fi
		\fi
}
\def\compute@resv{
		\if@rheight \else \edef\@p@srheight{\@p@sheight} \fi
		\if@rwidth \else \edef\@p@srwidth{\@p@swidth} \fi
}
%		
% Compute any missing values
\def\compute@sizes{
	\compute@bb
	\compute@handw
	\compute@resv
}
%
% \psfig
% usage : \psfig{file=, height=, width=, bbllx=, bblly=, bburx=, bbury=,
%			rheight=, rwidth=, clip=}
%
% "clip=" is a switch and takes no value, but the `=' must be present.
\def\psfig#1{\vbox {
	% do a zero width hard space so that a single
	% \psfig in a centering enviornment will behave nicely
	%{\setbox0=\hbox{\ }\ \hskip-\wd0}
	%
	\ps@init@parms
	\parse@ps@parms{#1}
	\compute@sizes
	%
	\ifnum\@p@scost<\@psdraft{
		\if@verbose{
			\typeout{psfig: including \@p@sfile \space }
		}\fi
		%
		\special{ps::[begin] 	\@p@swidth \space \@p@sheight \space
				\@p@sbbllx \space \@p@sbblly \space
				\@p@sbburx \space \@p@sbbury \space
				startTexFig \space }
		\if@clip{
			\if@verbose{
				\typeout{(clip)}
			}\fi
			\special{ps:: doclip \space }
		}\fi
		\if@prologfile
		    \special{ps: plotfile \@prologfileval \space } \fi
		\special{ps: plotfile \@p@sfile \space }
		\if@postlogfile
		    \special{ps: plotfile \@postlogfileval \space } \fi
		\special{ps::[end] endTexFig \space }
		% Create the vbox to reserve the space for the figure
		\vbox to \@p@srheight true sp{
			\hbox to \@p@srwidth true sp{
				\hss
			}
		\vss
		}
	}\else{
		% draft figure, just reserve the space and print the
		% path name.
		\vbox to \@p@srheight true sp{
		\vss
			\hbox to \@p@srwidth true sp{
				\hss
				\if@verbose{
					\@p@sfile
				}\fi
				\hss
			}
		\vss
		}
	}\fi
}}
\def\psglobal{\typeout{psfig: PSGLOBAL is OBSOLETE; use psprint -m instead}}
\catcode`\@=12\relax



% This command enables hyphenation if \tt mode by changed \hyphencharacter
% in the 10 point typewriter font. To work in other point sizes it would
% have to be redefined. It may be Bator to just make the change globally 
% and have it apply to anything that is set in \tt mode
\newcommand{\dcode}[1]{{\tt \hyphenchar\tentt="2D #1\hyphenchar\tentt=-1}}

\newcommand{\param}[1]{$<$\paramref{#1}$>$}
\newcommand{\paramref}[1]{{\em #1}}

\newcommand{\lispcodesize}{\normalsize}

\newcommand{\refitem}[1]{%
  \begin{list}%
        {}%
        {\setlength{\leftmargin}{.25in}\setlength{\itemindent}{-.25in}}
  \item #1%
  \end{list}}

\newcommand{\protoimage}[1]{\begin{picture}(190,20)\put(0,0){\makebox(190,20){\tt #1}}\put(95,10){\oval(190,20)}\end{picture}}

\newcommand{\macbold}[1]{{\bf #1}}

\newenvironment{slide}[1]{\newpage\vspace*{\fill}\begin{flushleft}}{\par\end{flushleft}\vspace*{\fill}}

\renewcommand{\heading}[1]{\begin{center} \underline{\bf #1} \end{center}}
\renewcommand{\subheading}[1]{\begin{center} \underline{#1} \end{center}}
\renewcommand{\chapter}[1]{\begin{center}\Huge\bf #1\end{center}}
\renewcommand{\section}[1]{\heading{#1}}
\renewcommand{\subsection}[1]{\subheading{#1}}
\renewcommand{\subsubsection}[1]{\subheading{#1}}

\setlength{\parskip}{4mm}

\course{ENAR Spring Meeting, 1992}
\instructor{Tierney}

\title{Statistical Computing and Dynamic Graphics Using Lisp-Stat}
\author{Luke Tierney\\School of Statistics\\University of Minnesota}

\begin{document}

\maketitle

\begin{abstract}
Lisp-Stat is an environment for general statistical computing and for
using and developing dynamic graphical methods. The Lisp-Stat system
contains a set of functions for performing basic statistical
calculations and constructing dynamic statistical graphs such as
linked scatterplots and rotatable three-dimensional plots. In
addition, new numerical and graphical tools can be added to the system
using the Lisp programming language and a simple object-oriented
programming system.

The course will begin with an overview of the capabilities of the
system and then introduce basic elements of Lisp programming and
concepts of object-oriented programming. These ideas will then be used
on several examples to illustrate how to add new graphical and
numerical methods to the system. The examples may include a simple set
of tools for fitting generalized linear models and an implementation
of the Grand Tour for viewing data sets in four or more dimensions.

Several extended breaks will be scheduled and a small number of
microcomputers or workstations will be made available during these
breaks to allow participants to experiment with the ideas presented.
\end{abstract}

\LARGE

We begin with a brief overview of Markov chains. For an excellent
introduction, the reader is referred to Feller\cite{feller} or
Ross\cite{rosstext}. 

A {\em stochastic process\/} is a collection of random  variables,
$\{X_t,\ t\in T\}$.  The set $T$, the index set, is commonly
called {\em time\/}.  In this paper, we only  consider the situation
where $T$ is a countable set; specifically, we shall use $T={\bf N}$,
the set of natural numbers. Thus,  
$\{X_k,\ k=0,1,2,\ldots\}$ will denote the stochastic process. 
Such a process is called a {\em discrete time\/} process.
The set of all possible values of the random variables $X_k$ is called
the {\em state space\/}of the process.  We again restrict ourselves to
finite or countable state spaces and denote the states by integers. If
$X_{k}=i$, then we say the process is in state $i$ at time $k$.  

A {\em Markov chain\/} is a special kind of stochastic process.
It is a stochastic process such that:
\begin{center}
$P\{X_{k+1}=j|X_{k}=i, X_{k-1}=i_{k-1}, \ldots,X_{1}=i_{1},
X_{0}=i_{0}\}=P_{ij}$ 
\end{center}
for states $i_{0},i_{1}, \ldots,i_{k-1},i,j$ and $k \geq 0$.  This can
be interpreted as saying that the probability that the process will go
to state $j$ given its past history (i.e. the states it has been in
for all times before $k$) is independent of those states, and depends
only on the the state it is presently occupying.

Whenever the process is in state $i$, there is a probability $P_{ij}$
that the process will go from state $i$ to state $j$. These
probabilities can be specified by means of a {\em transition matrix\/}
${\bf P}=[P_{ij}]$. The $i$th element in the $j$th row of this matrix
{\bf P} is the value $P_{ij}$: 
$$
{\bf P} = \left[ \begin{array}{cccc}
			P_{00} & P_{01} & P_{02} & \ldots \\
			P_{10} & P_{11} & P_{12} & \ldots \\
			\vdots & \vdots & \vdots &        \\
			P_{i0} & P_{i1} & P_{i2} & \ldots \\
			\vdots & \vdots & \vdots &        
			\end{array} \right]
$$

Because the process must make some sort of transition (perhaps even to
the same state), the value $\sum_{j=0}^{\infty} P_{ij}$, which is the
sum of the probabilities of all possibilities when the process is
currently in state $i$, must equal 1, for every $i$.  

A {\em discrete time finite state Markov chain} is a markov chain with a
finite state space and discrete time. We shall denote the states by
integers $0,1,\ldots,n$, for some integer $n$. 


%% You may not have these on you system, if so, just comment them out.
\input/bass/users/almond/text/rgafig
%\input psfig
%\psfigtrue
\def\directory{/bass/users/almond/ElToY/doc}
%Use Standard Postscript Fonts
\input/bass/users/almond/text/psfonts

\def\eltoy/{{\tt ElToY}}
\def\disttoy/{{\tt Dist-toy}}
\def\clttoy/{{\tt CLTToy}}
\def\xlispstat/{{\tt XLISP-STAT}}
\def\lispstat/{{\tt LISP-STAT}}
\def\begincode{\begingroup\tt\obeylines\obeyspaces}
\def\endcode{\endgroup}
\def\:#1 {{\tt :#1\/} }
\def\meth#1(#2)---{\noindent{\tt #1 (#2)}---}
\hfuzz=14pt



\centerline{{\tt ElToY\/}:  A tool for Education and Elicitation\/}

\centerline{\it Russell Almond\/}

\beginsection{1.  What is \eltoy/?}

At the winter 1992 ASA meating, several colleagues and I were
discussing the apparent lack of Bayesian software.  It occurred to me
then that there are two principle problems with the construction of
Bayesian software:  the first problem is philosophical; most Bayesians
feel that each problem is unique and the model should be assessed anew
for each problem.
However, good Bayesian software could provide the user with template
analyses which can be modified to fit the needs of the assessed model.
The second problem is Elicitation---the process of encoding prior
information as a probability judgement.  
\eltoy/ ({\bf El}icitation {\bf To}ol for {\bf Y}ou)
is meant to address this issue.
({\bf Y}ou should be taken loosely in the spirit of de~Finetti's usage.)

The process of encoding prior information as a probability
distribution is not at all simple or obvious.  Although statisticians
are familiar with probability distributions, most are not familiar
with the consequence of choice of prior distribution in a Bayesian
model.  If this subject is taught, it is as an advanced study for
graduate students, and not as part of the undergraduate service
courses.

To further complicate this training need, the typical user of such
software will not be a statistician but a statistical consumer.
A physician or an engineer is a good example of such an individual.
Although highly knowledgeable in her field, she has little statistical
training (probably one course packed into a demanding curriculum), 
and that little may have gone rusty through disuse.  Thus if we expect
statistical consumers to routinely use prior information, we need to
train them.

Thus the need for \eltoy/.  \eltoy/ actually consists of three
separate toys for exploring statistics:  \disttoy/, which teaches
about the relationship between distributions and their parameters;
\clttoy/, which demonstrates the central limit theorem; and \eltoy/
which demonstrates the relationship between prior and posterior
distributions and sample data.

\disttoy/ is a subset of \eltoy/ which displays probability
distributions.  It was originally built to create distribution display
facilities for \eltoy/.  \disttoy/ (and \eltoy/) reguard probability
distributions as *{\it objects\/}* which can be acted upon.  \disttoy/
action is to display a graph of its p.d.f. (or p.m.f. if the
distribution is discrete) along with a control box of its parameters.
Changing the parameters changes the display.

\clttoy/ is an example of the advantages of object oriented
programming.  Once the distribution object had been defined in
\disttoy/ we could use it to watch convergence in the CLT for very
little additional effort.  \clttoy/ takes a probability distribution
and sample from it, displaying a histogram of the result with a
superimposed normal curve.  Pressing the ``more'' button turns it into
a sample of the sum of two draws.  Pressing the button repeatedly
makes it a picture of the sum of three, four, or more samples from the
original distribution.  

\eltoy/ displays two \disttoy/ displays:  a prior distribution graph
display and parameter control box, and a posterior distribution graph
display and control box.  A fifth display shows sample data.  All of
the displays are linked so that changes in one display are propagated
to all of the others.  Thus if you change the prior parameters, the
posterior changes; if you change the data the posterior changes; if
you change the posterior, the prior changes.

\eltoy/ acts on conjugate families of distributions in a say that is
analogous to the way \disttoy/ works on ordinary distribution
families.  Conjugate families are important for the simple reason that
the relationship between prior and posterior distributions can be
easily described.  Theoretically, any distributional type could be
used as a prior or likelihood, all we need is a rule for updating the
posterior when the {\it likelihood\/} prior or data change, and rules
for updating the prior when the posterior changes.  Thus in theory
methods like Gaussian quadriture (already available in \lispstat/).
Future versions of eltoy may support just that.

Good[1976] introduces the idea of Bayesian robustness through the concept
of a {\it black box}.  When the inputs
(the prior and data) are wiggled, the output wiggles as well.  Thus
You (the generalized user of \eltoy/) can see the consequences of
changes in the prior judgements and judge the sensitivity of Your
results to those changes.  Furthermore, You can see the sensitivity of
Your conclusions to questionable data values.  \eltoy/ achieves this
through dynamically linked multiple views.  These provide You with
immediate feedback as to the consequences of Your prior judgements.
This should provide You with the understanding necessary to assess
realistic priors or classes of priors for Your Bayesian analyses.

\beginsection{2. How do I use \eltoy/?}

\eltoy/ consists of three different toys:  \disttoy/, \clttoy/, and
\eltoy/.  The routine operation of these three toys are described
below.  There are two other things you need to know to operate
\eltoy/: how to get a copy and how to start it and stop it.  As
installation only needs to be done once, the installation instructions
will be put last.  Finally, I include some ``exercises'' for classroom
or self-taught demonstration.

\eltoy/ is written in \xlispstat/(Tierney[1990]).  If you already known
how to use \xlispstat/ you should have no problems with \eltoy/.  If
you don't, don't worry; \eltoy/ provides a menu interface so you never
need to learn LISP (of course, if you do learn LISP, you can extend
the system.)

\xlispstat/ is currently available for Apple MacIntosh${}^{\rm TM}$
Computers (with 24 Mbytes of memory) and Unix${}^{\rm TM}$
workstations running X Windows${}^{\rm TM}$.  As some instructions are
slightly different for the two systems, paragraphs that apply mainly
to MacIntosh users will start with \#+MacIntosh, and those that apply
mainly to Unix users will start with \#+Unix.

\beginsection{2.1 Getting started.}

\#+MacIntosh:  To start \eltoy/, first start \xlispstat/.  Do this in
the usual way, {\it i.e.,} find the \xlispstat/ icon and double click
it.  This will start \xlispstat/ which will open a LISP listener
window.  \xlispstat/ will load some files and then signal that it is
ready for a new command by prompting with a `$>$' in the listener
window.  Now select `Load' from the `File' menu
(`Open' opens a file for editing, `Load' loads it into LISP)
and look in the ElToY folder for a file marked ``load-eltoy.lsp.''
Selecting that file will start \eltoy/ loading.  Selecting
``loaddisttoy'' will load \disttoy/ and \clttoy/ but not \eltoy/
(this takes about half the memory).
There is a third load file, ``loadall-debug.lsp.''  This loads \eltoy/
but with the LISP debugger turned on.  This is very useful for
programmers wishing to debug problems with loading the files, but is
probably confusing to users unfamiliar with LISP.

\#+MacIntosh:  After you have selected `Load', the listener window will
show you the files it is loading.  This will take a little while.  You
will also receive a warning about Loading without documentation.  As
the documentation is intended mainly for programmers, this should not
present any difficulties.  Loading and other operations should be must
faster on a machine with more memory.  Every once in a while, LISP
will stop and try and recycle unused memory in an operation called
``garbage collection.''  You will notice garbage collection because
the cursor temporarily becomes a trash bag with the letters ``GC''
written on it.

\#+Unix:  Your system administrator should have set up the loading
files somewhere on your normal search path (usually in
`/usr/local/bin'). 
Two different loading commands are available:  `disttoy' which loads
only \disttoy/ and \clttoy/ and `eltoy' which loads \eltoy/ as well as
the other two.  Your window will display messages about the loading of
files as the system loads data for first \xlispstat/ then \disttoy/
and \eltoy/.  \eltoy/ should create a ``Menu Bar'' window
(a quick and dirty emulation of the MacIntosh menu bar) and your
xterm window will become the LISP listener window.  It will inform you
that it is ready to go by prompting with a `$>$' in the LISP listener
window.

You need never interact with the LISP listener window; all the commands
you need to operate \eltoy/ are available from the menu bar, including
the commands to exit.  You may exit at any time by selecting the
appropriate menu item.  You can also exit by typing 
`{\tt (exit)}'
in the LISP listener window
(the parentheses are required; they indicate to LISP that you want to
run a command, not examine a variable).
Exiting LISP will close all \eltoy/ windows.

\#+MacIntosh.  On the MacIntosh, the Quit item is located in its usual
place in the file menu, and can be accessed through the shortcut
frimsy-Q.  Unfortunately, there is no way to interrupt a LISP program
behaving strangely; you may need to reboot.  The usual interrupt
command (frimsy-{.}) is bound to something else
(return to top level from debug).

\#+Unix.  On the menu bar window, there is a button marked `menu'.
Press that button and a pop-up menu will appear with a single choice 
`{\tt Exit}'.  Chosing 
`Exit' will exit \xlispstat/.  Control-C in the listener window will
not exit \xlispstat/, but it will interrupt \xlispstat/ and give you a
LISP prompt from which you can type 
`{\tt (exit)}' to exit.  Typing a Control-D in the listener window will
cause \xlispstat/ to exit.  Typing Control-Z will suspend the LISP
job; you can resume a suspended job with the Unix 
`{\tt fg}' command
and terminate it with the Unix 
`{\tt kill}' command.  Exiting \xlispstat/ will remove all \eltoy/
windows.

\beginsection{2.2 Using \disttoy/}

To use \disttoy/ press the mouse button over the name \disttoy/ on the
menu bar.  A pull down menu should appear giving you a list of all
distributions currently known to \eltoy/.  Select any one of them
(note that selecting `undefined-distribution' will create an error).
\disttoy/ should then open two windows (one displaying a graph and
one displaying a set of parameter controls).  (It will usually display
them in an inconvenient location, {\it i.e.}, on top of the Menu Bar
window on an X windows display.  Move them to someplace more
convenient.)

In this version of \eltoy/ you do most of your interaction with the 
{\it parameter\/} control window.  There is one slider for each
parameter, and you should be able to manipulate them using the normal
ways of manipulating sliders on your system.  If you would rather
enter an exact value, you can edit the value shown above the slider.
You can also edit the upper and lower limits on the slicer value.  To
enter these changes press the `Update' button for the particular
parameter you wish to change.  \eltoy/ will not notice your changes
before you hit the Update button.

When you change the value of a parameter, the graph of the  p.d.f.
(continuous distribution) or of the p.m.f. (discrete distribution)
should change accordingly.  In many cases, this will cause the picture
to go partially or even completely off the screen.  This can be fixed
by selecting the `rescale' item from the graph menu.
(\#+Macintosh: Select the graph window and the graph menu should
appear on the menu bar.)
(\#+Unix:  The graph menu psps up when you press the button marked
`menu' on the graph window.)  

Some values of the parameters may not
produce valid distributions
({\it e.g.}, a negative variance or a non-integral number of trials).
In these cases \eltoy/ will signal an error and not change the
internal values.

To close a \disttoy/, select the same distribution again from the main
\disttoy/ menu.  It is possible to have more than one \disttoy/ open
at the same time, but if you keep opening them you will clutter your
screen and run out of memory.  Pressing the `close' button on a window
will remove the window, but will leave \disttoy/ in a peculiar internal
state.  It will then be difficult to open a new \disttoy/ on the same
distribution.  

\#+Unix Warning:  Pressing the close button on the Menu Bar window
will cause it to go away, leaving just the LISP Listener window.  You
will now need to type `(exit)' in the Listener (Xterm) window to exit
LISP.

\beginsection{2.3 Using \clttoy/}

To launch \clttoy/ press on the name `CLTTOY' on the Menu Bar.  This
will create a pull-down menu of distributions known to \clttoy/;
select one of them (again selecting `Unknown' will cause an error).
Two windows should appear on your screen, a histogram window and a
control and explanation dialog window.  (Again these will probably
appear in inconvenient locations and sizes.  Adjust them accordingly.)

The dialog window explains what the histogram window is showing.
When you first start a \clttoy/, it will show a sample of $500$ draws
$(X_1, \ldots\ X_{500})$ from the selected distribution.  Superimposed
on this histogram is a normal curve with the same mean and variance as
the sample.

Now press the button `more' on the dialog.
The computer will draw another sample of $500$ random values from the
selected distribution $(Y_1,\ldots, Y_{500})$.  The normal curve will
be set to match the mean and variance of the sample distribution of
totals.  Pressing the `more' button  again adds another sample of
five hundred to the data, making the displayed data a sample from the
distribution of the total of three draws.  And so forth.

As you watch, you should see the histogram come closer and closer to
the normal curve.  If you want a lot of `more's, selecting the same
distribution from the \clttoy/ menu performs the same action as
pressing the `more' button ten times.

Pressing the button marked `Add QQ-plot' or selecting the `Add
QQ-plot' item from the graph menu creates a Quantile Quantile plot.
(Inconveniently, it usually does so right over the histogram window.)
The Quantile Quantile plot plots quantiles of a standard normal
distribution ($x$-axis) vs the observed quantiles of the distribution
currently displayed in the histogram window.  When the distribution of
totals converges to normal, this should follow a straight line with a
roughly 45${}^\circ$ angle (in a square plotting region).  Try this
option, it's really fun!

There is no easy way to get rid of \clttoy/ windows (aside from
exiting LISP).  Pressing the `close' button will remove the window,
but it will then be impossible to restart a \clttoy/ with the same
distribution.

\beginsection{2.4 Using \eltoy/}

Bear in mind that in order to use \eltoy/ you must first have loaded
it.  See Section~2.1 on starting \eltoy/.  If you just loaded
\disttoy/, ElToY will not appear on the menu bar.  You can load
\eltoy/ from LISP at any time.  Just type {\tt (load-eltoy)\/} in the
Lisp listener windows (the parenthesis are important).

\#+Macintosh: You can load \eltoy/ now, the same way you would load it from
the beginning, {\it i.e.}, select load and the file Load-eltoy.  This
will only load the files it needs (\disttoy/ files will not be
reloaded).


When \eltoy/ is loaded, there will be a menu titled \eltoy/ on the
menu bar.  Pull that down to get a list of currently known conjugate
distributions.  Select any one of them (except the Unknown
distribution).  \eltoy/ will create 7 new windows (mostly on top of
each other).  Move them around until they are mostly visible.  The
windows are identified by the first part of their titles; the full 
title indicates the specific distribution.

\item{\eltoy/} This is a main control window.  There are two buttons:
``Spawn New Data Link'' and ``Return Values and Exit''.  The first
button will create a new data-link and posterior distribution objects.
The second button will create a pop-up window with the current values
of the prior parameters and an opportunity to approve or reject them.
If you approve them, it will return the values and destroy all
windows.  If you don't approve them, you will return to normal
operation.  This is the proper way to clean up all windows and exit.

%Programmers note:  You can also invoke an \ELToY/ directly with the
%Lisp expression {\tt (make-el-tool {\it conj-family\/})}.  This
%function will return an {\tt el-tool} object (the collection of
%windows on the screen).  If it is created with {\tt (make-el-tool {\it
%conj-family\/} :catch-return t)} then it with throw the values to any
%form catching {\tt eltoy-return}.  Thus:
%\begingroup \obeylines\obeyspaces\tt
%(catch 'eltoy-return
%\        (make-el-tool {\it conjugate-family})
%\        (break))
%\endgroup
%will return those values.

\item{Prior:}  There are actually two windows with this name, one is a
graph and one is a set of parameters.  This is a picture of the prior
distribution.  They behaves like the \disttoy/ windows except that the
results are also passed through a ``data-link'' to the posterior
distribution windows.

\item{Posterior:}  Again these are two windows which behave like
\disttoy/ windows accept that they are linked to the prior windows
through the ``data-link.''

\item{Data:}  This window records both data and nuisance parameters
associated with the data.  The bottom half of the window allows entry
of the data.  This is entered as an arbitrary LISP expression.  This
expression when evaluated, can either be a single value or a list of
values.  In particular, it could be the name of an \xlispstat/
variable.  The button ``Evaluate X'' causes \eltoy/ to read the
current expression in the editing window.  The button ``Reset X''
causes \eltoy/ to redisplay the current value of the parameters.

\item{}  The upper half of the Data/Parameters window reads in a nuisance
parameter associated with each data point.  In the Normal-normal
family, this is the standard deviation of the observations.  In the
beta-binomial family, it is the number of trial associated with each data
value.  In the gamma-Poisson family, it is the observation time period
for each observation.

\item{} The data and nuisance parameters are linked to the posterior
distribution through the data-link object.

\item{Data-Link} The ``data-link'' object links a posterior and prior
distribution through a collection of data.  In particular, it uses the
conjugate family to figure out the posterior parameters from the prior
parameters and the data.  \eltoy/ uses the following rules to update
in response to a change:

\itemitem{1.} If the prior parameters change, update the posterior
parameters.

\itemitem{2.} If the data change, update the posterior parameters.

\itemitem{3.} If the nuisance parameters change, update the posterior
parameters. 

\itemitem{4.} If the posterior parameters change, update the prior
parameters.  

The Data-Link window is a probe which monitors the internal data-link
object.  It displays the rule which was last used to update the
display.

\medskip

The best way to exit from an \eltoy/ is to press the ``Return Values
and Exit'' button on the \eltoy/ window.

\beginsection 2.5 Installing \eltoy/.


In order to run \eltoy/, you will first need a copy of \xlispstat/.  If
you don't have a copy of \xlispstat/, don't worry, its free.  You can
obtain a copy via electronic mail.  Send mail to {\tt
statlib@temper.stat.cmu.edu\/} continuing the single line {\tt send
index from xlispstat\/}.  This mail should contain no subject header
and no extraneous information because it will be processed by the
computer and not by a human.  Follow the instructions there to install
\xlispstat/ on your system.

After you have gotten \xlispstat/ installed, copy the ElToY files to
their home.  The only part that should require customization is the
``loadall'' files.  These are in the top level directory of the
distribution.   Each of them contains a line of the form:
$$\vbox{
\tt\obeyspaces
(def ElToY-directory "/home0/almond/ElToY/" )
}$$
Edit this to reflect the location of \eltoy/ on your system.

\#+MacIntosh:  MacIntosh pathnames are sort of like Unix pathnames,
except that they use ``{\tt :}'' instead of ``{\tt /}'' to separate
the directories.  Thus for your system it will look something like:
$$\vbox{
\tt\obeyspaces
(def ElToY-directory ":Hard Drive:Free Software:ElToY:" )
}$$
Once you've edited this line in all three ``loadall'' files, you
should be ready to run.

\#+Unix:  There are three more files you need to modify, the startup
files.  These are also in the top level directory and are named ``{\tt
eltoy\/}'', ``{\tt dist-toy\/}'' and ``{\tt debug-toy\/}.''  They look
something like:
\begincode
\#!/bin/csh
xlispstat \~almond/ElToY/loadall.lsp
\endcode
Edit them to reflect the placement of the \eltoy/ directory on your
system.  If you have system privileges, you will probably want to copy
the files {\tt eltoy\/} and {\tt dist-toy\/} to {\tt /usr/local/bin\/}
to that they can be accessed by everybody.  {\tt debug-toy\/} is meant
for serious programmers and should probably be left where it is.

You should now be ready to go.


{\it Debugging problems with loading of files.\/} The ``loadall'' file
{\tt loadall-debug.lsp\/} (and the command file {\tt debug-toy\/}
which loads it) is identical to {\tt loadeltoy\/}, except that the
LISP debugger is turned on before the first file is loaded.  If you
can program in LISP, this will help you tract down specific problems.
Hopefully, there won't be any problems unless you change something.

\beginsection 2.6 Exercises (and other fun things to try)

\item{1.} Change the parameters on the normal distribution and then
hit rescale plot.  Notice how the distribution when rescaled looks the
same as the standard normal.  The {\it mean\/} of a normal
distribution is called a {\it location parameter\/} and the {\it
standard deviation\/} is called a {\it scale parameter\/}.  Why?

\item{2.} Open up both a Binomial and a Hypergeometric Dist-toy.  Give
the $r$ and $b$ parameters of the hypergeometric (number of red balls
and black balls) large values and match the binomial parameter
$\theta=r/(r+b)$.  Let the $n$ parameter of the binomial and the $k$
of the hypergeometric be the same.  If $n$ and $k$ are small compared
to $r$ and $b$, the two distributions should be close.  When $n$ and $k$ are
large compared to one of $r$ or $b$, then the distributions will be
different.  Recall that the binomial distribution is like taking $n$
samples from an urn, replacing the ball on each sample.  The
hypergeometric is like sampling from the urn without replacement.
When is the binomial formula a good approximation for sampling without
replacement?

\item{3.} Play with the CLT Toys with several of the standard
distributions.  In particular, pick a symmetric distribution like the
binomial with $\theta=.5$, a skewed distribution like the exponential, a
flat distribution like the continuous uniform and a distribution with
heavy tails like the student-$t$.  Which distributional shapes converge
fastest to the normal?  How many samples does it take before the
sum/average is reasonably close to normal?  What does this tell you
about rates of convergence in the Central Limit Theorem?

\item{4.}  Try the following:
\itemitem{4a)} Open up a binomial ``Dist-Toy'' and let $n$ get large
for various values of $\theta$.
\itemitem{4b)} Open up a Poisson distribution and let $\lambda$ get
large.
\itemitem{4c)} Open up a gamma distribution and let the shape parameter
get large.
\itemitem{4d)} Open up a negative binomial distribution and let $r$ get
large.  
\item{}In each case, the distributional shape should begin to take on
a familiar form.  Interpret each random variable as the sum of other
random variables and use the central limit theorem to explain what you
see.

\item{5.} Open up a clt-toy on the ``Cauchy'' distribution.  Does it
converge to the normal?  If so, what happens as you add more samples?
The Cauchy distribution has such extreme tails that it doesn't have a
mean or a variance (the appropriate integrals don't converge).  Does
it satisfy the conditions of the central limit theorem?

\item{6.} (Probably want to try this on after you learn the $t$-test).
Call up the student-t distribution and fool around with the degrees of
freedom parameter.  Notice that as the degrees of freedom increase it
gets closer to the normal and as they decrease it gets closer to the
Cauchy.  What does this mean in terms of sample size?


\beginsection {3.}  How can I extend \eltoy/?

\eltoy/ is a library of \xlispstat/ objects and methods.  If you are
willing to program in \xlispstat/, you can easily extend the system.
The rest of this book assume you can program in \xlispstat/, if not I
recommend you read Tierney[1990] first.

Even though there is not much in the way of user's documentation
(except this file), there is a substantial amount of programmer's
documentation.  This is available on-line for Unix users by sending
any ElToY object the {\tt :documentation\/} and {\tt :doc-topic\/}
messages.  It is not loaded on the MacIntosh (the program assumes
limited memory), but they are present in the source code.  The file
``{\tt OBJECTS\/}'' lists all of the objects supported by \eltoy/ and
gives a list of methods which they specialize.


The most obvious way to extend \eltoy/ is to add new probability
distributions.  Therefore this section talks about the protocol for
probability distribution families and conjugate families.  The rest of
this section describes these implementations.  

One of the advantages of object oriented programming is that it encourages
code re-use.  In particular, the specification for probability
distributions described here should be reusable in other contexts.  A
good example of this is the CLT-Toy.  CLT-Toy was not part of the
original \eltoy/ project specification.  However, I wanted a central
limit theorem demonstration for a class and the distribution library
was already written.  It was relatively straightforward to build the
CLT-Toy on top of the existing \disttoy/ code.  

\xlispstat/ implements a prototype object system.  This means that
every object inherits default values and methods---(methods are
functions which manipulate the object) from the prototype object.
Prototypes behave much the same way that classes do in a class based
system (like CLOS).  Prototypes are also objects themselves, so they
can accept messages and behave properly.  

(Note: there is a relatively serious bug in the \xlispstat/ object
system: when the method of a prototype object is changed, the
unspecialized methods of its children are not changed.  This is very
annoying, but can be worked around by reloading all code which was
loaded after the prototype was changed.  The function {\tt
new-provide\/} provided in the \eltoy/ distribution helps to work
around that problem.  It behaves roughly the same way as {\tt
provide\/}, except that if the file is being reloaded, then all
modules which were reloaded after the current file are marked as unloaded
(removed from {\tt *modules*\/}).  A ``loadall'' file which has
require statements for all necessary files will then reload and hence
recreate any code which might be affected by the change.)


\beginsection 3.1 Distribution Families.

A good exercise in object oriented programming for statisticians is to
think about the probability distribution family as an object.  If we
think of a probability distribution, we can think of all sorts of
statistical operations we would like to perform on it, calculate the
mean and variance, draw random values, find quantiles, find the mass
or density, etc.  If we restrict our attention to univariate
distributions, we also can divide the class of univariate
distributions into two subclasses:  discrete and continuous
distributions.  (Version~1.0 of \eltoy/ does not support mixtures).

Before we can implement the exciting statistical properties of
distributions, we need to implement some lower level functions which
assign and return default values of parameters, constrain parameters
and so forth.  Fortunately we can benefit from object oriented
programming here; most of the methods for default parameter values and
related concepts can be inherited from the {\tt family-proto\/} object
which is the prototype for all families.  

Objects have internal ``slots'' for storing values.  In particular,
family objects have the following slots:
$$\vbox{
\halign{\tt # \hfil\ & \ \it #\hfil \ &\rm #\hfil\cr
{\rm Slot} & {\rm Type} & Description\cr
\noalign{\smallskip\hrule\smallskip}
rv-name & List (of symbols) & Name(s) of random variable(s)\cr
parameter-names & List (of symbols) & Name(s) of parameter(s)\cr
default-parameters & List (of numbers) & Default value(s) of parameter(s)\cr
parameter-limits & List (of number pairs) & Hard limits of parameter(s)\cr
parameter-range-default & List (of number pairs) & Elicitation range of parameter(s)\cr
parameter-granularity & List (of numbers) & Elicitation step size for parameter(s) \cr
parameter-integer? & List (of {\tt T} or {\tt Nil\/}) & Is parameter(s) integer(s)?\cr 
parameter-constraint-fun & Function (args parameters) & Constraint function\cr
nparams & Integer & Number of parameters \cr
}}$$
Except for the last two, all of the slots store lists.  In particular,
{\tt rv-name\/} is a list of names (anticipating multivariate distributions
which are not yet implemented) for the random variable and {\tt
parameter-names\/} is also a list of names.  For all of the
parameters, {\tt parameter-names\/}, {\tt default-parameters\/}, {\tt
parameter-limits\/}, {\tt parameter-range-default\/}, {\tt
parameter-limits\/}, {\tt parameter-granularity\/} and {\tt
parameter-integer?\/} are lists of the same length,  one value (or
pair of values) for each parameter.  Specifically, {\tt
parameter-names\/} stores the name (which should be a symbol), {\tt
default-parameters\/} stores the default values for each parameter,
{\tt parameter-limits\/} stores the upper and lower bounds for the
parameter value (as a list) and {\tt parameter-integer?\/} stores {\tt
t\/} if the parameter must be an integer and {\tt nil} otherwise.
{\tt parameter-range-default\/} and {\tt parameter-granularity\/} are
related to the slider objects which are used to input parameters; the
first slot stores the default upper and lower bound for the slider and
the second stores the default step size for the slider.  

The slot {\tt parameter-constraint-fun\/} stores a function which
should take as its sole argument the list of parameters and returns
{\tt t\/} if they are in a legal relationship to each other and {\tt
nil\/} otherwise.  For example, the parameters of the normal
distribution can be in any relationship to each other (the variance
will have a lower limit of zero, this will be handled automatically by
the {\tt parameter-limits\/} and doesn't need to be build into the
constraint function).  The parameters of the uniform
distribution---the upper and lower bound---must be in decreasing
order.  The file ``{\tt constraints.lsp\/}'' in the ``{\tt utils\/}''
directory provides several useful constraint functions.  In
particular, the {\tt \#'vacuous-constraint-fun\/} which puts no
constraints on the parameters, the {\tt
\#'increasing-constraint-fun\/} which forces the parameters to be in
increasing order and the {\tt \#'decreasing-constraint-fun\/} which
forces the parameters to be in decreasing order.

The {\tt family-proto\/} is a child of the {\tt named-object\/}
prototype, so the {\tt family-proto\/} also inherits the slot {\tt
name\/} which is the name of the distribution.  

\medskip

The mechanism for setting and accessing these slots is mostly
established.  You will, however, need to specify default values for
most of them.  This is usually done in the {\tt :isnew\/} method.
(The {\tt :isnew\/} method initializes the various variables.)  This is
usually done immediately after we define the family prototype.
(Distribution families are themselves prototypes, you can create a new
instance of that kind of distribution by sending it the {\tt :new\/}
message.) 

These definitions are usually placed in a file (which can then be
loaded when needed).  The following is code is from the file for the
normal distribution (``{\tt Dist-Lib/Normal.lsp\/}''.  It is
recommended that you try to edit a template from one of the existing
distributions, rather than start from scratch).

\begincode
;;; This defines normal distribution families.

(require :el-family (strcat ElToY-directory "fam" *d-sep*  "family.lsp"))
\endcode

The first expression makes sure that the \eltoy/ code defining the
family objects is loaded before the code for the normal family is
loaded.  

\begincode
;;; generic normal family  Note that a family is a proto-type which is
;;; sent a :new message to create a specific instance of that family

(defproto normal-family '() '() (list continuous-family-proto))
\endcode

The second expression defines the {\tt normal-family\/} prototype.  It
has no slots of its own (the first '()) and it has no shared slots
(the second '()).  It inherits from the {\tt
continuous-family-proto\/}.  There is also a {\tt
discrete-family-proto\/} from which discrete families should inherit.

\begincode
;;; :isnew --- mostly this sets up default args for the call to the
;;; family-proto :isnew
(defmeth normal-family :isnew
\  	(\&rest args
\	 \&key (name (gensym "NORMAL"))
\	      (rv-names '(X))
\	      (parameter-names '(Mean StDev))
\	      (default-parameters '(0 1))
\	      (parameter-range-default '((-10 10) (0 5)))
\	      (parameter-limits (list (list *-infty* *infty*)
\				      (list 0 *infty*)))
\	      (parameter-granularity (list .25 .1))
\	      (parameter-integer? '(nil nil))
\	      \&allow-other-keys)
\  (apply \#'call-next-method
\   :name name
\   :rv-names rv-names
\   :parameter-names parameter-names
\   :default-parameters default-parameters
\   :parameter-range-default parameter-range-default
\   :parameter-limits parameter-limits
\   :parameter-integer? parameter-integer?
\   :parameter-granularity parameter-granularity
\   args))
\endcode

This expression defines a method for the {\tt :isnew\/} method for the
normal, which mostly calls the {\tt :isnew\/} method for the {\tt
family-proto\/} with certain defaults set.  These defaults can be
overridded in the creation of a new {\tt normal-family\/} with the
appropriate keyword arguments.  First, the default name for the
distribution is a new symbol starting with ``NORMAL''  thus we will
get NORMAL-1, NORMAL-2, etc.  This can be directly specified with the
keyword {\tt :name\/}.  Next is the name of the random variable, in
this case, ``X''.  Then come the name of the parameters ``Mean'' and
``StDev'', their default values (0 and 1, given the standard-normal),
the default elicitation range (between -10 and 10 for the mean,
between 0 and 5 for the SD), the limits (note that {\tt *infty*\/} is
a big number and {\tt *-infty*\/} is {\tt (- 0 *infty*)}, the
granularity (.25 for the mean and .1 for the SD) and neither parameter
is constrained to be an integer.  These arguments are passed to the
function {\tt call-next-method\/} which passes them to the {\tt
:isnew\/} method for the parent of {\tt normal-family\/} which in this
case is {\tt family-proto\/} (it skips over {\tt
continuous-family-proto\/} which does not specialize that method).

The parent objects {\tt family-proto\/} and {\tt
continuous-family-proto\/} provide sensible defaults for many useful
methods (such as {\tt :print\/}).  However, in order to the {\tt
:print\/} method to work successfully with the prototype, the {\tt
:name\/} argument must be initialized.  Thus:

\begincode
;; sensible default for print
(send normal-family :name :normal0)
\endcode

All distributions must have the following methods:  {\tt
:quantiles\/}, {\tt :cdf\/}, {\tt :rv-limits\/}, {\tt :mean\/}, and
{\tt :variance\/}.

All continuous distributions must additionally have the following
method: {\tt :density\/}.

All discrete distributions must additionally have the following
methods:  {\tt rv-integer?\/}, {\tt :atoms\/}, and {\tt :mass\/}

Sensible default methods are provided for the following methods, but
they should probably be specialized for particular families:  {\tt
rv-default\/} (defaults to median), {\tt :rv-range-default\/}
(defaults to limits), {\tt :rand\/} (defaults to inverse cdf method).

All of these methods take the \:parameters as a keyword argument.  If
this argument is omitted, the default-parameters are used instead.
Some methods require other arguments (as specified below).


\noindent{\it Methods Required of All Distributions}

\meth :quantiles ({\it prob} \&key :parameters)---The argument {\it
prob\/} should be either a probability or list of probabilities
(values between 0 and 1).  The function returns a list of the
corresponding quantiles (value $x_p$, such that $F(x_p)=p$)

\begincode
;; :quantiles  -- takes parameters as optional second argument.  First
;; argument should be probabilities expressed as fraction
(defmeth normal-family :quantiles
\         (prob \&key (parameters (send self :default-parameters)))
\  (unless (listp prob) (setq prob (list prob)))
\  (let ((mean (car parameters))
\	(stdev (cadr parameters)))
\    (+ mean (* stdev (normal-quant prob)))))
\endcode
	
\meth :cdf ({\it quant\/} \&key :parameters)---This takes a list of
quantiles and returns a list of the cumulative density function
($F(x)=\Pr(X\leq x)$) for the given quantiles.

\begincode
;; :cdf  -- takes parameters as optional second argument.  First
;; argument should be quantiles
(defmeth normal-family :cdf
\         (quant \&key (parameters (send self :default-parameters)))
\   (let ((mean (car parameters))
\	 (stdev (cadr parameters)))
\     (normal-cdf (/ (- quant mean) stdev))))
\endcode

\meth :rv-limits (\&key :parameters)---Returns a list of limits (lower
and upper bound) for each random variable.

\begincode
;; :rv-limits
(defmeth normal-family :rv-limits
\         (\&key (parameters (send self :default-parameters)))
\   (list (list *-infty* *infty*)))
\endcode

\meth :mean (\&key :parameters)---Returns a list of means for all
random variables.

\begincode
(defmeth normal-family :mean (\&key
\			      (parameters (send self :default-parameters)))
\  (let ((mean (car parameters))
\	(stdev (cadr parameters)))
\    (list mean)))
\endcode

\meth :variance (\&key :parameters)---Returns a list of variances for
all random variables.

\begincode
(defmeth normal-family :variance (\&key
\			      (parameters (send self :default-parameters)))
\  (let ((mean (car parameters))
\	(stdev (cadr parameters)))
\    (list (* stdev stdev))))
\endcode

\noindent{\it Methods Required of Continuous Distributions}

\meth :density ({\it quant} \&key :parameters)---This method takes a
list of quantiles and returns a list of the corresponding probability
density function (p.d.f., $f(x)$) values. 

\begincode
;; :density  -- takes parameters as optional second argument.  First
;; argument should be quantiles
(defmeth normal-family :density
\         (quant \&key (parameters (send self :default-parameters)))
\   (let ((mean (car parameters))
\	 (stdev (cadr parameters)))
\     (/ (normal-dens (/ (- quant mean) stdev))
\	stdev)))
\endcode

\noindent{\it Methods Required of Discrete Distributions}

\meth :rv-integer? (\&key :parameters)---returns {\tt t} if random
variables must be integers, {\tt nil\/} otherwise.

\begincode
;; :rv-integer?   
(defmeth binomial-family :rv-integer? (\&key \&allow-other-keys)
\  t)
\endcode

\meth :atoms (\&key :parameters :min-mass)---Returns a list of atoms of the
distribution.  \:min-mass is a lower cut-off for mass, this prevents
this list from being infinite.

\begincode
;; :atoms  -- takes parameters as keyword argument.  
;; minimum mass as optional third
;; trick use normal approximation to find approximate range of values.
(defmeth binomial-family :atoms
\         (\&key (parameters (send self :default-parameters))
\	       (min-mass 0.001))
\   (let ((theta (car parameters))
\	 (n (cadr parameters)))
\     (if (eql 0 min-mass)
\	 (iseq 0 n)
\       (let ((mu (* n theta))
\	     (sigma (sqrt (* n theta (- 1 theta))))
\	     (z (abs (normal-quant min-mass))))
\	 (apply \#'iseq (force2-between
\			(list (floor (- mu (* z sigma)))
\			      (ceiling (+ mu (* z sigma))))
\			(list 0 n)))))))
\endcode

\meth :mass ({\it quant} \&key :parameters)---Returns a list of
probability mass function (p.m.f., $p(x) = \Pr(X=x)$) values for the
list of quantiles.  If the quantile is not an atom, it 
the mass is zero.

\begincode
;; :mass  -- takes parameters as optional second argument.  First
;; argument should be atoms
(defmeth binomial-family :mass
\         (quant \&key (parameters (send self :default-parameters)))
\   (let ((theta (car parameters))
\	 (n (cadr parameters)))
\     (binomial-pmf quant n theta)))
\endcode

\noindent{\it Methods Which are Not Required but have Sensible Defaults}

\meth :rv-default (\&key :parameters)---Default value for random
variable  (returns a list for each random variable).  Defaults to
median. 

\begincode
;; :rv-default  --- default value of random variable.

(defmeth normal-family :rv-default
\         (\&key (parameters (send self :default-parameters)))
\   (list (first parameters)))		;mean
\endcode
Exploits the fact that the mean (first parameter) is a sensible
default.  


\meth :rv-range-default (\&key :parameters)---Default elicitation
range for random variables (defaults to limits).  Returns a list of
ranges, one for each random variable.

\begincode
;; :rv-range-default
(defmeth normal-family :rv-range-default
\         (\&key (parameters (send self :default-parameters)))
\   (let ((mean (car parameters))
\	 (stdev (cadr parameters)))
\     (list (+ (* '(-3 3) stdev) mean))))
\endcode

\meth :rand ({\it num}\&key :parameters)---Random number generator.
Returns {\it num\/} random numbers.  Defaults to inverse c.d.f.
method.

\begincode
(defmeth normal-family :rand (num \&key
\			      (parameters (send self :default-parameters)))
\  (let ((mean (car parameters))
\	(stdev (cadr parameters)))
\    (+ (* (normal-rand num) stdev) mean)))
\endcode
We have an efficient normal random number generator, so use it.


\noindent{\it Other Methods for Which the defaults are always useful}

\meth{:continuous?} ()---Returns {\tt t\/} if the distribution is continuous
and {\tt nil\/} otherwise.  

\meth{:dist-graph-proto} ()---Returns {\it pmf-graph-proto\/} (displays
probability mass function) for discrete distributions and {\it
pdf-graph-proto\/} for continuous distributions.

Additionally, the following methods which must be defined for children
of {\tt discrete-family-proto\/} have sensible defaults for {\tt
continuous-family-proto\/}, namely \:rv-integer? which returns {\tt
nil}, \:atoms which returns {\tt nil} (empty list) and \:mass which
signals a continuable error (continuing returns the density).

\medskip

The file which defines the normal distribution ends with
several lines to establish the families place in the \eltoy/ system.

\begincode
(send normal-family :isnew :name :normal)
\endcode
This line causes the {\tt normal-family\/} to initialize itself, so it
can be used as specific normal distribution as well as a prototype.

\begincode
(push normal-family *Known-families*)
\endcode
The variable {\tt *Known-families*\/} keeps track of families known to
\eltoy/.  \eltoy/ uses this to build its menus.  This expression adds
{\tt normal-family\/} to the list.

\begincode
(new-provide :el-normal)
\endcode
This expression lets other programs know that the file containing {\tt
normal-family\/} definitions has been successfully loaded.

To provide for automatic loading of a distribution family when \eltoy/
is loaded, add a line to the file {\tt Dist-Lib/Dist-Lib.lsp\/} like
the following:

\begincode
(require :el-normal (strcat ElToY-directory "Dist-Lib" *d-sep*  "Normal.lsp"))
\endcode

\beginsection 3.2 Finite and Bootstrap families.

There is a representation which works for  all discrete distributions
over finite outcome spaces, namely a list of atoms and a list of mass
values.  A special case, the bootstrap family, is formed by giving
all of the atoms equal mass ($1/n$), although there can be duplicate
values among the atoms.  These two families are particularly powerful
as they can be used to create a number of special effects.  The {\tt
:wavy\/}, {\tt :holey\/} and {\tt :outliers\/} distributions included
in \clttoy/ are a special case of this effect.

The {\tt :new\/} method for the {\tt finite-family\/} object takes
keyword arguments {\tt :atoms\/} which should be a list of the atoms
and {\tt :mass\/} which should be a list of the mass values
corresponding to the atoms.  Atoms may be repeated in the list, the
appropriate mass values are summed to create the final mass list.

The {\tt :mass\/} and {\tt :atoms\/} lists of an instance of {\tt
finite-family\/} can be set with the {\tt :mass\/} and {\tt :atoms\/}
messages.  This is done by using the {\tt :set\/} keyword, whose
argument should be the new {\tt :atoms\/} or {\tt :mass\/} list.
Note:  The {\tt :atoms\/} list is stored sorted with duplicates
removed.  The {\tt :mass\/} list should correspond to the order of the
sorted, duplicates removed list.

The following code fragments implement the example distributions for
\clttoy/

\begincode
(defun normalize (vals)
\  "Normalize list of values by dividing by their sum. *Vectorized*"
\  (/ vals (sum vals)))
\endcode
This function is used to normalize distribution functions, so they can
be specified more flexibly.

\begincode
(def wavy-dist
\     (send finite-family :new :name :wavy
\	   :atoms (iseq 1 9)
\	   :mass (normalize '(2 4 2 1 3 5 6 5 3))))
\endcode
This specifies the distribution function.

\begincode
(push wavy-dist *CLT-Families*)
\endcode
The special variable {\tt *CLT-Families*\/} is a list of families to
be used with \clttoy/.  It is used in the automatic menu
building for \clttoy/.


The {\tt bootstrap-family\/} is even simpler.  It is specified with a
single list of ``data.''  This can be set with the {\tt :data\/}
keyword argument to the {\tt bootstrap-family :isnew\/} method.
The {\tt :data\/} message with no arguments returns the data.  If an
argument is given, it becomes the new data.

\begincode
(def holey-dist (send bootstrap-family :new :name :holey
\		      :data '( -4 -3 -3 -2 2 3 3 4)))

(push holey-dist *CLT-Families*)
\endcode


\beginsection 3.3 Conjugate Families.

The definition of a conjugate-family is very simple, all it requires
is a {\tt prior-family\/} and a {\tt likelihood-family\/}.  Most of the
method which work on a conjugate family reference the appropriate
value.  The only methods which currently must be specialized are the
:forward-link and :reverse-link functions which translate prior
hyperparameters into posterior hyperparameters.   Of course, we first
need to define the prior and likelihood families.

\medskip

The class of {\tt prior-families\/} is very similar to the class of
families defined above.  The only difference is an extra parent {\tt
prior-family-proto\/} is added to the list of parents.  Thus:

\begincode
(defproto normal-prior-family () () (list normal-family continuous-prior-proto)
  "Normal prior distribution")
\endcode

Sets up the new prototype, taking most of its method from the normal
family.  In this case, we want to change some of the default values,
so we write a new \:isnew method.  This is the only extra method
needed to implement the normal prior.  Occasionally extra code will be
needed to implement an alternate parameter set.

The completed prior is pushed onto the list of {\tt
*Know-prior-families*\/}. 

\medskip

The {\tt likelihood-family-proto\/} is not so simple.  In particular, it
is quite possible (and in fact usual) to have nuisance parameters
floating around.  For example, the {\tt normal-likelihood-family\/}
has the nuisance parameter ``sigma'' which is the variance of the
observations.  

In particular, the {\tt likelihood-family-proto\/} supports the
following methods for returning/setting properties of the likelihood
function:  {\tt :nui-parameter-names\/}, {\tt
:nui-default-parameters\/}, {\tt :nui-parameter-limits\/}, {\tt
:nui-parameter-range-default\/}, {\tt :nui-parameter-granularity\/},
{\tt :nui-parameter-integer?\/}, \hskip 0pt plus 1in \ \break{\tt :nui-parameter-constraint-fun\/}
and {\tt :data-constraint-fun\/}.  All of these can be (and should be)
set through the {\tt :isnew\/} method for {\tt
likelihood-family-proto\/} as well.   Finally, the {\tt :prior\/}
keyword argument to the creation ({\tt :new, :isnew\/}) method
specifies the prior distribution to be used with this likelihood.
This allows the likelihood function to access information about the
prior distribution if needed in the methods.

Most of these specialized methods need a little explanation and their
meaning and syntax is analogous to the parameter methods of {\tt
family-proto\/}.   The {\tt :data-constraint-fun\/} is a function of
two arguments the {\it data\/} and the {\it nui-parameters\/} is
and return true if the the data values are legal for the given set of
nuisance parameters.  Data constraint functions should take
keyword arguments.  In particular, it should take the {\tt :warn\/}
argument.  If {\it warn\/} is true, then a warning message should be
printed if the constraint is violated.  Thus, 

\begincode
\  :data-constraint-fun 
\   \#'(lambda (data nui-parameters \&key (warn t) \&allow-other-keys)
\       $\ldots$)
\endcode

\noindent will produce a legal data constraint function.  

Two data constraint functions are distributed with \eltoy/:  {\tt
vacuous-data-constraint\/} always returns true; and {\tt
max-data-constraint\/} returns true if the data values are less than
or equal to the nuisance parameter.  This is used for the binomial
likelihood, where the nuisance parameter is $n$, the number of draws.

\smallskip

Finally, the {\tt likelihood-family-proto\/} modifies the syntax for
some of the methods by adding a {\tt :nui-parameters\/} keyword to
specify the nuisance parameters.  The modified methods include:
{\tt :quantiles\/}, {\tt :atoms\/}, {\tt :mass\/}, {\tt :density\/},
{\tt :cdf\/}, {\tt :rv-default\/},  {\tt :rv-limits\/}, {\tt
:rv-range-default\/}, {\tt :rand\/}, {\tt :mean\/} and {\tt
:variance\/}.  

The variable {\tt *Known-likelihood-families*\/} records the known
likelihood families.

\bigskip

After you have specified the prior and likelihood families, you are
ready to specify the conjugate family.  The creation ({\tt :isnew})
method for conjugate families uses two keyword arguments. The {\tt
:prior-family\/} keyword argument specifies the prior family {\it
prototype\/} and {\tt :likelihood-family\/} keyword argument specifies
the likelihood family {\it prototype\/}.  Note that these (as are all
family object in \eltoy/) are prototypes, and they will be sent the
{\tt :new\/} message when the creation of a conjugate family occurs.

The {\tt conjugate-family-proto\/} prototype supports a large range of
messages for accessing information from the prior and likelihood
families.  The {\tt :prior-family\/} and {\tt :likelihood-family\/}
messages also support either pass messages on to the appropriate
object, or return the object itself.

There are four messages which the user must specify for each conjugate
family, they are {\tt :default-data\/}, {\tt :data-range-default\/},
{\tt :reverse-link\/} and {\tt :forward-link\/}.  The latter two are
particularly important as they define the protocol for Bayesian
updating.  The {\tt :forward-link\/} message turns prior
hyperparameters and data into posterior hyperparameters and the {\tt
:reverse-link\/} turns posterior hyperparameters and data into prior
hyperparameters.  These are at the heart of the idea of ``conjugate''
in \eltoy/ and any class of distribution which can support these
methods can be considered a conjugate family.

\medskip

Here is an annotated definition for the {\tt normal-normal-family\/}:

First, we need to define the family object.  We will also want to
define its initialization method.  These are done below:
\begincode
(defproto normal-normal-family '() '() (list conjugate-family-proto))

;; :isnew method sets up defaults then inherits from conjugate-family-proto
(defmeth normal-normal-family :isnew
\         (\&rest args
\	  \&key  (prior-family normal-prior-family)
\	        (likelihood-family normal-likelihood-family)
\	  \&allow-other-keys)
\ (apply \#'call-next-method
\	:prior-family prior-family
\	:likelihood-family likelihood-family 
\	args))

(send normal-normal-family :isnew)
\endcode
The last form initializes the normal-normal prototype for use as a
family object.

Next we define the method for the forward link.  In this case the
prior family has hyperparameters $\mu$ (mean) and $\tau$ (standard
deviation).  The data value is $X$ and the nuisance parameter (the
standard deviation of the data) is $\sigma$.  In general $X$ can be a
vector (data-structure) and $\sigma$ can either be a single number or
a vector of matching length.  We add (using \lispstat/ vectorization)
{\tt (- X X)\/} to {\tt sigma\/} to insure that it is a of the same
shape as {\tt X\/}.  As the normal is easier to parameterize
in terms of precisions, we use {\tt tau-2\/} for $1/\tau^2$ and {\tt
sigma-2\/} for $1/\sigma^2$.  The code computes the posterior mean and
precision: $(1/\tau_*^2) = (1/\tau^2) + \sum (1/\sigma_i^2)$ and 
$$ \mu_* = { (1/\tau^2) \mu + \sum (1/\sigma_i^2) X_i \over
(1/\tau_*^2) } $$  The posterior mean and variance are then returned.
\begincode
;;; required link methods for conjugate-families

;;; :forward-link --- returns posterior hyperparameters from prior
;;; hyperparameters 
(defmeth normal-normal-family :forward-link
\  	(\&rest args
\	 \&key  (parameters (send self :default-hyperparameters))
\	       (nui-parameters (send self :nui-default-parameters))
\	       (data (send self :default-data))
\	 \&allow-other-keys)
\  (let* ((mu (car parameters))
\	 (tau-2 (/ 1.0 (* (cadr parameters) (cadr parameters))))
\	 (X (car data))
\	 (sigma-2 (+ (/ 1.0 (* (car nui-parameters) (car nui-parameters)))
\		     (- X X)))
\					;makes the same shape as X
\	 (tau-2* (+ tau-2 (sum sigma-2)))
\	 (mu* (/ (+ (* tau-2 mu) (sum (* X sigma-2))) tau-2*)))
\    (list mu* (sqrt (/ 1.0 tau-2*)))))
\endcode

The reverse link just solves the formulas used above for the prior
parameters.  There is one complication, in that the posterior
precision can be underspecified.  This will result in a negative prior
variance.  The reverse link traps for this.

\begincode
;;; :reverse-link --- returns posterior hyperparameters from prior
;;; hyperparameters 
(defmeth normal-normal-family :reverse-link
\  	(\&rest args \&key
\	       parameters
\	       (nui-parameters (send self :nui-parameters-default))
\	       (data (send self :default-data))
\	       \&allow-other-keys)
\  (let* ((mu* (car parameters))
\	 (tau-2* (/ 1.0 (* (cadr parameters) (cadr parameters))))
\	 (X (car data))
\	 (sigma-2 (+ (/ 1.0 (* (car nui-parameters) (car nui-parameters)))
\		     (- X X)))
\	 (tau-2 (max *epsilon* (- tau-2* (sum sigma-2))))
\	 (mu (/ (- (* tau-2* mu*) (sum (* X sigma-2))) tau-2)))
\    (if (<= tau-2 0) (uerror "force to *epsilon*"
\			     "Negative variance mu*: \~S tau-2*:\~A X:\~S
sigma-2: \~S tau-2: \~S mu: \~S~\%" mu* tau-2* X sigma-2 tau-2 mu))
\    (list mu (sqrt (/ 1.0 tau-2)))))
\endcode

Finally, we push the appropriate values on the {\tt
*Known-conjugate-families*\/} list.
\begincode
(push normal-normal-family *Known-conjugate-families*)

(new-provide :el-norm-norm)
\endcode

\beginsection {4.}  How does \eltoy/{} work inside?

At any given time, \eltoy/ is showing a number of different views of
the same probability model.  The challenge is to maintain the
consistency of those views.  Thus when the user make a small change on
one of the interaction objects, the change must be propagated
globally.  This was the principle challenge of the \eltoy/ system.  It
is essentially just a giant constraint maintenance system.

%This code will probably burble on your system.  Try commenting out
%the line and printing the figure El-Toy.eps separately.  
\prologue{\adobe}
\rgafig1[El-Toy]{{\tt ElToY\/} system diagram}hsize=6.25truein
vsize=5.5truein/ 

Figure~1 presents an overview of the \eltoy/ system.  It is laid out
as a tree, the {\it tool-tree\/}.  At the root of the tool tree is the
{\tt el-tool\/} object which owns all of the others.  The leaves of the tree
are the various interaction objects.  The rest of the tree serves as a
framework from which to hang the display objects.  By passing messages
up through the branches to the root and back down to the leaves,
consistency is maintained.

\eltoy/ uses an {\tt :update\/} and {\tt :local\/} strategy to pass
messages and maintain consistency.  Each object in the tool tree
(except the root) has a distinct owner.  Although children of the
owner may inherit references to an object owned by their parent, they
do not own the object itself.  Thus they may access but not change its
value.  This is particularly important for the parameter objects.  In
particular, the parameter display object cannot directly change the
values of the parameter, it must first pass the change message up to
the owner of the parameter object, in this case the root El-Tool.

The updating strategy works as follows.  Whenever an {\tt el-tool\/} object
receives a request to change data that it is owned by its parent in
the tool tree, it passes the change request to the parent.  The top
level data object owning the value to be changed, then sends an
{\tt :update\/} message to itself with the message it received as the
arguments.

The {\tt :update\/} method for all {\tt el-tool\/} objects resends the change
message with the keyword flay {\tt :local\/} placed before the data.
This enables any local processing of data (for example, the owner will
change the value, a display object will change the display).  
It then sends an update message to each of its children.

\begingroup\narrower\sl
{\bf Example\/}: Suppose we change the posterior mean.  The following
illustrates the series of messages passed (in some cases detail is
suppressed where it is peripheral to the update strategy).

\item{1} The {\tt posterior parameter display\/} will generate a {\tt
:parameters\/}  message requesting a change to the posterior
hyperparameters.  

{\tt parameter display $\rightarrow$ :parameters mean 5}


\item{2} As the  {\tt display tool\/} does not own the posterior
hyperparameters, but rather inherits them from the {\tt data-link\/}
(through the distribution display), the unqualified
method for the {\tt :parameters\/} message sends {\tt
:posterior-parameters\/} message to the {\tt data-link\/}.  

{\tt distribution display $\rightarrow$ :parameters mean 5}\par
{\tt data-link $\rightarrow$ :posterior-parameters mean 5}\par

\item{3} The {\tt data-link\/} is the highest object in the tool tree
containing the posterior parameters.  Its method for making a change
not marked as {\tt :local\/}, however, is to change the prior
hyperparameters using the reverse link.  This in turn is done by
sending a {\tt :hyperparameters\/} message to the {\tt el-tool\/} to
change the hyperparameters.

{\tt el-tool $\rightarrow$ :hyperparameters mean (send conjugate-family
:reverse-link ...)}

\item{4} The {\tt el-tool\/} then sends itself an {\tt :update\/}
message, whose arguments are the request to change 
the hyperparameters.  (Assume that the new prior mean after the
reverse link is 7.5).

{\tt el-tool $\rightarrow$ :update :hyperparameters mean 7.5}


\item{5} It handles the update message by first sending
itself a {\tt :hyperparameters :local\/} message and then sending and
{\tt :update :hyperparameters\/} message to each of its children.  The
{\tt :hyperparameters :local\/} message actually changed the prior
hyperparameters.  

{\tt el-tool $\rightarrow$ :hyperparameters :local mean 7.5}\par
{\tt el-tool $\rightarrow$ :update :hyperparameters mean 7.5}\par
{\tt Prior distribution display $\rightarrow$ :update :hyperparameters mean 7.5}\par
{\tt data-link $\rightarrow$ :update :hyperparameters mean 7.5}\par

\item{6} Following the change down just the {\tt data-link} side of
the tree, the {\tt :update\/} message is
received by the data-link which sends itself a {\tt :hyperparameters
:local\/} message.  It also generates an {\tt :update
:posterior-parameters\/}.  Changing the posterior parameters is held
until now because some changes in the posterior may lead to invalid
priors.  

{\tt data-link $\rightarrow$ :posterior-parameters :local (send
conjugate-family :forward-link)}\par
{\tt data-link $\rightarrow$ :update :posterior-parameters mean 5}

\item{7} The {\tt :update\/} message is now passed down to the
children of the {\tt :data-link\/} object.  In particular, the {\tt
Distribution Display\/} and the {\tt Data Display\/}.  

{\tt Distribution Display $\rightarrow$ :update :posterior-parameters mean 5}\par
{\tt Data Display $\rightarrow$ :update :posterior-parameters mean 5}\par

\item{8}  The data display, having no message for the {\tt
:posterior-parameters\/} ignores the message.  The Distribution display
changes them into parameter messages which it sends to its children.

{\tt Parameter Display $\rightarrow$ :update :parameters mean 5}\par
{\tt Graph $\rightarrow$ :update :parameters mean 5}\par

\item{9} These become {\tt :parameter :local\/} messages which update
the displays.

{\tt Parameter Display $\rightarrow$ :parameters :local mean 5}\par
{\tt Graph $\rightarrow$ :parameters :local mean 5}\par 

Note:  To simplify the presentation, some of the arguments to the
messages have been simplified.

\endgroup

The example is the most complex of the message passing strategies.  In
particular, the {\tt data-link\/} needed to to both forward and
backwards conversion of messages for the strategy to succeed.  As the
role of the {\tt data-link\/} object is to convert messages, it is
attached to a display object which shows the current message
conversion strategy.  The role of the {\tt el-tool\/} object is simply
to reflect change request messages comming up the tree and turn them
into {\tt :update\/} messages going back down.

\disttoy/ is a simplified version of \eltoy/.  It simply chops the
tool tree off at the distribution display and adds a new {\tt
dist-tool\/} root node.  It was in fact developed to debug the lowest
levels of the \eltoy/ system and the design of the {\tt :update\/}
message passing strategy. 

There are other strategies for maintaining consistency among objects
which could have been used.  Another approach would have used would be
to use mediators (Sullivan and Notkin[1990]) to handle the changes.
The {\tt Garnet\/} user interface toolkit (Meyers {\it et al.\/}
[1990]) provides a constraint maintenance object system (Guise[1989,1990])
which will automatically keep track of changes and reciprocal changes.
The {\tt data-link\/} object provides a major challenge to all of the
constraint maintenance strategies and it must be able to transform the
sort of message it supports.

One of the advantages of the current system is its extensibility.
The constraint propagation strategy theoretically supports new ways of
propagating the message.  For example, one could add a ``hot point''
(say the mode) to the graph of the prior and posterior distributions.
Moving this point could produce a {\tt :parameters\/} message, which
would in turn use the \:update  strategy to propagate its effects.
Adding additional displays (such as a display of the predictive
distributions) can also be easily incorporated into the existing {\tt
:update\/} strategy.

\smallskip

The other advantage of this update strategy, it the way the framework
can be re-used.  In particular, changing the {\tt conjugate-family\/}
slot of the parent {\tt el-tool\/} specialized the display for a
particular distribution.  Certain pieces of the framework can be
broken off and replaced as well.  For example, the {\tt graph\/} slot
of the distribution display object can be replaced with a display of
the p.m.f. or p.d.f. depending on whether the distribution is discrete
or continuous.


\beginsection {5.}  What's in the future for \eltoy/?

There are several ways in which \eltoy/ needs to be extended to
fully support the elicitation process, in particular:

\item{1.} Providing support for other representations, in particular,
support for using the predictive distribution for elicitation (as in
Chaloner and Duncan[1983]).

\item{2.} Providing support for alternative parameterizations.  (For
example, mean and observational equivalence (number of observations)
for the binomial).

\item{3.} Providing support for bivariate and multivariate prior
distributions and data.  

\item{4.} Providing support for non-conjugate prior as posteriors
through numeric integration strategies.

However, in order to be truly functional as an elicitation system, it
needs a help system.  This is a fairly large challenge and has brought
my development efforts on \eltoy/ to a temporary hold.  As a first
pass, the help system should integrate into the \lispstat/ protocol for
help.  Although \lispstat/ provides a protocol for documenting objects
and methods (through the {\tt :documentation\/} message), this
provides help for the programmer and not the more casual user.  There
is as of yet no established \lispstat/ user help protocol, and this
needs to be addressed by the \lispstat/ community before work could
proceed. 

Ideally, \eltoy/ would have a hypertext help facility.  This would
include an online glossary for the definition of unfamiliar words, and
perhaps sample animations illustrating key concepts.  There should
also be synonyms available for many technical terms.  This will allow
the program to be automatically specialized to the needs of a
particular user community (for example, engineers or physicians).
Such a facility will take a major development effort.

\smallskip

Another part of my reluctance to expand on \eltoy/ is the nature of
the \lispstat/ programming environment.  \xlispstat/ as is currently
available has several difficulties.  In particular:

\item{1.} There is a lack of consistent documentation.  In particular,
there is no programmer's manual describing all available interaction
object and the messages they accept.  The principle documentation,
Tierney[1990], is a tutorial on too many levels.  It contains a brief
introduction to LISP, examples of how to use many of the \lispstat/
interaction objects and some examples of statistical programming.
Unfortunately, it does not have time to provide a thorough or systematic
description of all the functions, objects and methods.

\item{2.} Many of the interaction objects are difficult to extend or
specialize.  Some of them do not completely conform to the \lispstat/
documentation protocol.  In particular, it is difficult to figure out
how to add a slider to the bottom of a graph.  It was also difficult
to figure out how to make the endpoints of a slider user changeable
values.  (This eventually involved going back to the C code to find
out what the slider object did.)  The \eltoy/ distribution contains a
modified slider object.

\item{3.} The \lispstat/ object system is relatively immature.
\lispstat/ could greatly benefit from being ported to a more mature 
object system such as CLOS (now standard with Common LISP,
Steele[1990]) or KR (Guise [1991], distributed with {\tt Garnet\/},
Myers {\it et al.\/} [1990]).  These systems also provide additional
features and data and function abstractions.

\item{} There is one particularly annoying bug associated with the
\lispstat/ object system.  When a modified {\it defproto\/} statement
is  reloaded, a new prototype is created rather than the
old one being updated.  Furthermore, old references to the prototype
object use the old prototype and not the updated one.  This makes
debugging very difficult.  To work around that problem, \eltoy/
provides the {\it new-provide\/} and {\it new-require\/}.  New-provide
destructively modifies the {\tt *modules*\/} variable to mark any
modules read in after a given file as unread.  Thus a call for a
system load will load in all files which potentially could be effected
by the change.  This is a work around for an incomplete system.

\item{4.} The \lispstat/ system currently works on top of XLISP, which
is a Common LISP subset.  In particular, it has relatively crude
versions of the Common LISP development environment available in more
complete LISPs.  It is sometimes difficult to anticipate which
features will be available in XLISP.

As my understanding is that Luke Tierney plans to move \lispstat/ to a
more complete version of LISP (AKCL), some of these objections will be
removed.  For the present, the best development strategy for \eltoy/
is to wait and see what happens.

Despite my criticisms of \lispstat/, it does provide two critical
features which made the development of \eltoy/ possible:

\item{1.} \lispstat/ contains interaction objects for common types of
statistical graphs.  These are not commonly available with user
interface toolkits.  Although \lispstat/ interaction objects could be
more flexible and extensible, and should probably contain additional
control features, they are *there* which is unique to the \lispstat/
environment. 

\item{2.} The \lispstat/ mapping operations over complex data
structures is a nice generalization of a feature which has been
partially available in matrix based languages like {\tt S\/}.  Although
not used extensively in \eltoy/ it looks exceedingly useful for other
data analysis tasks.

Unfortunately, the programming style which \lispstat/ seems to support
best is to hack existing objects or examples into new code.  In
particular, the development time for the \clttoy/ was much shorter
than that for \disttoy/.   However, the structure of \disttoy/ and how
to extend it is relatively obvious from its code, while the structure
of \clttoy/ is convoluted and assumes that you intimately understand
how the  \lispstat/ graph objects work.  While this produces
rapid creation of prototype implementations, it does not provide a
solid basis for developing supporting more complete applications.


\medskip

Although future development on \eltoy/ is on the back burner, I am
interested to hear your comments and suggestions for improvement.  I
am also interested in any modifications you make to the code to
support a particular application.  Someday (when funding catches up
with this particular project) I hope to extend this facility.
Meanwhile, I hope that some use can be made of this project either as
tool for exploring conjugate distributions or an example of developing
a complex system in an object oriented language.

I can be reached at:
\begingroup\narrower\obeylines
Russell Almond			   
Statistical Sciences, Inc.	
1700 Westlake Ave., N Suite 500
Seattle, WA  98109			
(206) 283-8802				
{\tt almond@statsci.com\/}
\endgroup
0r
\begingroup\narrower\obeylines
U. Washington
Statistics, GN-22
Seattle, WA  98195
{\tt almond@stat.washington.edu\/}
\endgroup



\beginsection References

\def\refrence#1{\par\hang\hskip-\parindent {\bf #1\/}.}

\refrence {Chaloner, Kathryn M. and Duncan, George T.[1983]}
``Assessment of a Beta Prior Distribution: PM Elicitation.''  {\it The
Statistician.\/}  {\bf 32\/}, 174--180.

\refrence{Good, I. J. [1976]}, ``The Bayesian Influence, or How to Sweep
Subjectivism Under the Carpet,''  {\it Foundations of Probability
Theory, Statistical Inference, and Statistical Theories of Science,\/}
Proc. of a Conference in May, 1973, Hooker, C. A. and Harper, W.,
eds., Vol. 2 (Dordrecht, Holland, D. Reidel, 1976), pp 125-174. 
Reprinted in {\it Good Thinking\/}, University of Minnesota Press,
1983,   pp 22--55. 

\refrence{Guise, Dario A. [1990]} ``Efficient Knowledge Representation
Systems.'' {\it Knowledge Engineering Review.\/} {\bf 5}(1), pp 35--50.

\refrence{Guise, Dario A. [1991]} ``KR: Constraint-Based Knowledge
Representation.'' Carnegie Mellon University Computer Science,
Technical Report.

\refrence{Myers, Brad A., Guise, Dario A., Dannenberg, Roger B.,
Zanden, Brad Vander, Kosbie, David S., Pevin Ed, Mickish, Andrew, and
Marchal, Phillipe [1990]} ``Comprehensive Support for Graphical,
Highly-Interactive User Interfaces:  The Garnet User Interface
Development Environment.''  {\it IEEE Computer\/} {\bf 23\/}(11), pp
71-85.  


\refrence{Sullivan, Kevin J. and Notkin, David [1990]} ``Reconciling
environment integration and component independence.''  In {\it
SIGSOFT'90:  Fourth Symposium on Software Development Environments,\/}
Irvine, CA, pp 208--225.  

\refrence{Tierney, Luke [1990]} {\it LISP-STAT:  An Object Oriented
Environment for Statistical Computing and Dynamic Graphics\/}.  John
Wiley and Sons.

\bye

\begin{slide}{}
\chapter{A Tutorial Introduction}
\end{slide}

\begin{slide}{}
We'll start with an introduction to using Lisp-Stat as a statistical
calculator and plotter.

We will see how to
\begin{itemize}
\item interact with the interpreter
\item perform numerical operations
\item modify data
\item construct systematic and random data
\item use built-in dynamic plots
\item construct linear regression models
\item define simple functions
\item use these functions for
\begin{itemize}
\item a simple animation
\item fitting nonlinear regression models
\item maximum likelihood estimation
\item approximate posterior computations
\end{itemize}
\end{itemize}
\end{slide}

\begin{slide}{}
\section{The Interpreter}
Your interaction with Lisp-Stat is a conversation between you and the
interpreter.

When the interpreter is ready to start the conversation, it gives you
a prompt like
\begin{verbatim}
>
\end{verbatim}
When you type in an expression and hit {\em return}, the interpreter
evaluates the expression, prints the result, and gives a new prompt:
\begin{verbatim}
> 1
1
>
\end{verbatim}
\end{slide}

\begin{slide}{}
Operations on numbers are performed with {\em compound expressions}:
\begin{verbatim}
> (+ 1 2)
3
> (+ 1 2 3)
6
> (* (+ 2 3.7) 8.2)
46.74
\end{verbatim}

The basic rule: everything is evaluated.

Numbers evaluate to themselves:
\begin{verbatim}
> 416
416
> 3.14
3.14
> -1
-1
\end{verbatim}
\end{slide}

\begin{slide}{}
Logical values and strings also evaluate to themselves:
\begin{verbatim}
> t               ; true
T
> nil             ; false
NIL
> "This is a string 1 2 3 4"
"This is a string 1 2 3 4"
\end{verbatim}
The semicolon ``\dcode{;}'' is the Lisp comment character.

Symbols are evaluated by looking up their values, if they have one:
\begin{verbatim}
> pi
3.141593
> PI
3.141593
> x
error: unbound variable - X
\end{verbatim}
Symbol names are {\em not} case-sensitive.
\end{slide}

\begin{slide}{}
Compound expressions like \dcode{(+ 1 2)} are evaluated by
\begin{itemize}
\item looking up the function definition of the symbol \dcode{+}
\item evaluating the argument expressions
\item applying the function to the arguments
\end{itemize}
{[Exception: If the function definition is a {\em special form}]}

Compound expressions are evaluated recursively:
{\Large
\begin{verbatim}
> (+ (* 2 3) 4)
10
\end{verbatim}}
Operators like \dcode{+}, \dcode{*}, etc., are functions, like
\dcode{exp} and \dcode{sqrt}
{\Large
\begin{verbatim}
> (exp 1)
2.718282
> (sqrt 2)
1.414214
> (sqrt -1)
#C(0 1)
\end{verbatim}}
\end{slide}

\begin{slide}{}
Numbers, strings, and symbols are {\em simple data}.

Several forms of {\em compound data} are available.

The most basic form of compound data is the list:
\begin{verbatim}
> (list 1 2 3)
(1 2 3)
> (list 1 "a string" (list 2 3))
(1 "a string" (2 3))
\end{verbatim}
Other forms of compound data:
\begin{itemize}
\item vectors
\item multi-dimensional arrays
\end{itemize}
\end{slide}

\begin{slide}{}
Data sets can be named (symbols given values) using \dcode{def}:
\begin{verbatim}
> (def x (list 1 2 3))
X
> x
(1 2 3)
\end{verbatim}
\dcode{def} is a {\em special form}. 
\end{slide}

\begin{slide}{}
Quoting an expression tells the interpreter {\em not} to evaluate it.
\begin{verbatim}
> (quote (+ 1 2))
(+ 1 2)
\end{verbatim}
Lisp quotation is similar to English quotation:

Think about
\begin{quote}
Say your name!
\end{quote}
and
\begin{quote}
Say ``your name''!
\end{quote}
\dcode{quote} is a special form.
\end{slide}

\begin{slide}{}
Since \dcode{quote} is needed very frequently, there is a shorthand form:
\begin{verbatim}
> '(+ 1 2)
(+ 1 2)
> (list '+ 1 2)
(+ 1 2)
> '(1 2 3)
(1 2 3)
\end{verbatim}
Compound expressions are just lists.
\end{slide}

\begin{slide}{}
Complicated expressions are easier to read when they are properly
indented:
\begin{verbatim}
> (+ (- (* 3 5) (* (- 7 2) 6)) 3)
-12
> (+ (- (* 3 5) 
        (* (- 7 2) 
           6)) 
     3)
-12
\end{verbatim}
The interpreter should help you with indentation, and with matching
parentheses.

Exiting from Lisp-Stat:
\begin{itemize}
\item Choose \macbold{Quit} from the \macbold{File} menu
\item Type \dcode{(exit)}
\end{itemize}
\end{slide}

\begin{slide}{}
\section{Elementary Computations and Graphs}

\subsection{One-Dimensional Summaries and Plots}
A small data set:

Precipitation levels in inches recorded during the
month of March in the Minneapolis-St.~Paul area over a 30-year period:
\begin{center}
\begin{tabular}{rrrrrr}
0.77 & 1.74 & 0.81 & 1.20 & 1.95 & 1.20 \\
0.47 & 1.43 & 3.37 & 2.20 & 3.00 & 3.09 \\
1.51 & 2.10 & 0.52 & 1.62 & 1.31 & 0.32 \\
0.59 & 0.81 & 2.81 & 1.87 & 1.18 & 1.35 \\
4.75 & 2.48 & 0.96 & 1.89 & 0.90 & 2.05 
\end{tabular}
\end{center}
Use \dcode{def} and \dcode{list} to enter the data:
\begin{verbatim}
(def precipitation
     (list .77 1.74 .81 ...))
\end{verbatim}
or
\begin{verbatim}
(def precipitation '(.77 1.74 .81 ...))
\end{verbatim}
We will see later how to read the data from a file.
\end{slide}

\begin{slide}{}
Some summaries:
\begin{verbatim}
> (mean precipitation)
1.675
\end{verbatim}
\begin{verbatim}
> (median precipitation)
1.47
\end{verbatim}
\begin{verbatim}
> (standard-deviation precipitation)
1.0157
\end{verbatim}
\begin{verbatim}
> (interquartile-range precipitation)
1.145
\end{verbatim}

And some plots:
\begin{verbatim}
> (histogram precipitation)
#<Object: ...>
> (boxplot precipitation)
#<Object: ...>
\end{verbatim}
The results returned by these functions will be used later.
\end{slide}

\begin{slide}{}
In Lisp-Stat, arithmetic operations are applied elementwise to
compound data:
\begin{verbatim}
> (+ 1 precipitation)
(1.77 2.74 1.81 2.2 2.95 ...) 
> (log precipitation)
(-0.2613648 0.5538851 -0.210721 ...)
> (sqrt precipitation)
(0.877496 1.31909 0.9 1.09545 ...)
\end{verbatim}

The results can be passed to summary or plotting functions using
compound expressions like
\begin{verbatim}
(mean (sqrt precipitation))
\end{verbatim}
and
\begin{verbatim}
(histogram (sqrt precipitation))
\end{verbatim}

The \dcode{boxplot} function produces a parallel boxplot when given a
list of datasets:
\begin{verbatim}
(boxplot (list urban rural))
\end{verbatim}
\end{slide}

\begin{slide}{}
\subsection{Two-Dimensional Plots}
The \dcode{precipitation} data were collected over time.

It may be useful to look at a plot against time to see if there is any trend.

To construct a list of integers from 1 to 30 we use
\begin{verbatim}
> (iseq 1 30)
(1 2 ... 30)
\end{verbatim}
A scatterplot of \dcode{precipitation} against time is produced
by
\begin{verbatim}
(plot-points (iseq 1 30) precipitation)
\end{verbatim}
Sometimes it is easier to see temporal patterns in a plot if the
points are connected by lines:
\begin{verbatim}
(plot-lines (iseq 1 30) precipitation)
\end{verbatim}
\end{slide}

\begin{slide}{}
A connected lines plot is also useful for plotting functions.

As an example, let's plot the sine curve over the range $[-\pi,\pi]$.

A sequence of 50 equally spaced points between $-\pi$ and $\pi$ is
constructed by
\begin{verbatim}
(rseq (- pi) pi 50)
\end{verbatim}
An expression for plotting $\sin(x)$ is
\begin{verbatim}
(plot-lines (rseq (- pi) pi 50)
            (sin (rseq (- pi) pi 50)))
\end{verbatim}
You can simplify this expression by first defining a variable
\dcode{x} as
\begin{verbatim}
(def x (rseq (- pi) pi 50))
\end{verbatim}
and then constructing the plot with
\begin{verbatim}
(plot-lines x (sin x))
\end{verbatim}
\end{slide}

\begin{slide}{}
Scatterplots are of course particularly useful for
bivariate data.

As an example, \dcode{abrasion-loss} contains the amount of rubber
lost in an abrasion test for each of 30 specimens;
\dcode{tensile-strength} contains the tensile strengths of the
specimens.

A scatterplot of \dcode{abrasion-loss} against
\dcode{tensile-strength} is produced by
\begin{verbatim}
(plot-points tensile-strength
             abrasion-loss)
\end{verbatim}
\end{slide}

\begin{slide}{}
\subsection{Plotting Functions}
A simpler way to plot a function like $\sin(x)$ is to use
\dcode{plot-function}.

It expects a function, a lower limit, and an upper limit as arguments.

Unfortunately, just using
\begin{verbatim}
(plot-function sin (- pi) pi)
\end{verbatim}
will not work:
\begin{verbatim}
> (plot-function sin (- pi) pi)
error: unbound variable - SIN
\end{verbatim}
The reason is that symbols have separate values and function
definitions.

This can be a bit of a nuisance.

But it means that you can't accidentally destroy the \dcode{list}
function by defining a variable called \dcode{list}.
\end{slide}

\begin{slide}{}
To get the {\em function definition}\/ of \dcode{sin} we can use
\begin{verbatim}
(function sin)
\end{verbatim}
or the shorthand form
\begin{verbatim}
#'sin
\end{verbatim}
A plot of $\sin(x)$ between $-\pi$ and $\pi$ is then produced by
{\Large
\begin{verbatim}
(plot-function (function sin) (- pi) pi)
\end{verbatim}}
or
{\Large
\begin{verbatim}
(plot-function #'sin (- pi) pi)
\end{verbatim}}
\end{slide}

\begin{slide}{}
\section{More on the Interpreter}
\subsection{Saving Your Work}
It is possible to
\begin{itemize}
\item
save a transcript of your session with the \dcode{dribble} function\\
(available in one of the menus on a Macintosh or the MS Windows
version)
\item
save one or more defined variables to a file using the \dcode{savevar}
function.
\item
save a copy of a plot
\begin{itemize}
\item to the clipboard on a Macintosh or in Windows
\item to a postscript file in {\em X11}
\end{itemize}
\end{itemize}
\end{slide}

\begin{slide}{}
\subsection{A Command History Mechanism}
There is a simple command history mechanism:
\begin{center}
\begin{tabular}{rl}
{\tt -}   & the current input expression\\
{\tt +}   & the last expression read\\
{\tt ++}  & the previous value of \dcode{+}\\
{\tt +++} & the previous value of \dcode{++}\\
{\tt *}   & the result of the last evaluation\\
{\tt **}  & the previous value of \dcode{*}\\
{\tt ***} & the previous value of \dcode{**}
\end{tabular}
\end{center}
The variables {\tt *}, {\tt **}, and {\tt ***} are the most
useful ones.
\end{slide}

\begin{slide}{}
\subsection{Getting Help}
The \dcode{help} function provides a brief description of
a function:
{\Large
\begin{verbatim}
> (help 'median)
MEDIAN                           [function-doc]
Args: (x)
Returns the median of the elements of X.
\end{verbatim}}
\dcode{help} is an ordinary function -- the quote in front of
\dcode{median} is essential.
\end{slide}

\begin{slide}{}
\dcode{help*} gives help information about all functions with names
containing its argument:
{\Large
\begin{verbatim}
> (help* 'norm)
-----------------------------------------------
BIVNORM-CDF                      [function-doc]
Args: (x y r)
Returns the value of the standard bivariate
normal distribution function with correlation R
at (X, Y). Vectorized.
-----------------------------------------------
...
-----------------------------------------------
NORMAL-CDF                       [function-doc]
Args: (x)
Returns the value of the standard normal
distribution function at X. Vectorized.
-----------------------------------------------
...
\end{verbatim}}
\end{slide}

\begin{slide}{}
The function \dcode{apropos} gives a listing of the symbols that
contain its argument:
{\Large
\begin{verbatim}
> (apropos 'norm)
NORMAL-QUANT
NORMAL-RAND
NORMAL-CDF
NORMAL-DENS
NORMAL
BIVNORM-CDF
NORM
\end{verbatim}}
\end{slide}

\begin{slide}{}
\subsection{Listing and Undefining Variables}
\dcode{variables} gives a listing of variables created with \dcode{def}:
\begin{verbatim}
> (variables)
(PRECIPITATION RURAL URBAN ...)
\end{verbatim}
To free up space, you may want to get rid of some variables:
\begin{verbatim}
> (undef 'rural)
RURAL
> (variables)
(PRECIPITATION URBAN ...)
\end{verbatim}
\end{slide}

\begin{slide}{}
\subsection{Interrupting a Calculation}
Occasionally you may need to interrupt a calculation.

Each system has its own method for doing this:
\begin{itemize}
\item
On the Macintosh, {\em hold down} the \macbold{Command} key and the
\macbold{Period} key.
\item
In MS Windows, hold down CONTROL-C
\item
On UNIX systems, use the standard interrupt, usually CONTROL-C.
\end{itemize}
\end{slide}

\begin{slide}{}
\section{Some Data-Handling Functions}
\subsection{Generating Systematic Data}
We have already seen two functions,
\begin{itemize}
\item \dcode{iseq} for generating a sequence of consecutive integers
\item \dcode{rseq} for generating equally spaced real values.
\end{itemize}
\dcode{iseq} can also be used with a single integer argument $n$;
this produces a list of integers $0, 1, \ldots, n - 1$:
\begin{verbatim}
> (iseq 10)
(0 1 2 3 4 5 6 7 8 9)
\end{verbatim}
\end{slide}

\begin{slide}{}
\dcode{repeat} is useful for generating sequences with a particular pattern:
\begin{verbatim}
> (repeat 2 3)
(2 2 2)

> (repeat (list 1 2 3) 2)
(1 2 3 1 2 3)

> (repeat (list 1 2 3) (list 3 2 1))
(1 1 1 2 2 3)
\end{verbatim}
\end{slide}

\begin{slide}{}
For example, if the data in the table
{\Large
\begin{center}
\begin{tabular}{|c|c|c|c|}
\hline
& \multicolumn{3}{c|}{Variety}\\
\cline{2-4}
Density & 1 & 2 & 3 \\
\hline
1 & ~9.2 12.4 ~5.0 & ~8.9 ~9.2 ~6.0 & 16.3 15.2 ~9.4 \\
\hline
2 & 12.4 14.5 ~8.6 & 12.7 14.0 12.3 & 18.2 18.0 16.9 \\
\hline
3 & 12.9 16.4 12.1 & 14.6 16.0 14.7 & 20.8 20.6 18.7 \\
\hline
4 & 10.9 14.3 ~9.2 & 12.6 13.0 13.0 & 18.3 16.0 13.0 \\
\hline
\end{tabular}
\end{center}}
are entered by rows
\begin{verbatim}
(def yield (list  9.2 12.4  5.0 ...))
\end{verbatim}
then the density and variety levels are generated by
\begin{verbatim}
(def variety 
     (repeat (repeat (list 1 2 3) 
                     (list 3 3 3))
             4))
\end{verbatim}
and
\begin{verbatim}
(def density (repeat (list 1 2 3 4)
                     (list 9 9 9 9)))
\end{verbatim}
\end{slide}

\begin{slide}{}
\subsection{Generating Random Data}
50 uniform variates are generated by
\begin{verbatim}
(uniform-rand 50)
\end{verbatim}
\dcode{normal-rand} and \dcode{cauchy-rand} are similar.

50 gamma variates with unit scale and exponent 4 are generated by
\begin{verbatim}
(gamma-rand 50 4)
\end{verbatim}
\dcode{t-rand} and \dcode{chisq-rand} are similar.

50 beta variates with $\alpha=3.5$ and $\beta=7.2$ are generated by
\begin{verbatim}
(beta-rand 50 3.5 7.2)
\end{verbatim}
\dcode{f-rand} is similar.

Binomial and Poisson variates are generated by
\begin{verbatim}
(binomial-rand 50 5 .3)
(poisson-rand 50 4.3)
\end{verbatim}
\end{slide}

\begin{slide}{}
The sample size arguments may also be lists of integers.

The result will then be a list of samples.

The function \dcode{sample} selects a random sample without
replacement from a list:
\begin{verbatim}
> (sample (iseq 1 20) 5)
(20 11 3 14 10)
\end{verbatim}
If \dcode{t} is given as an optional third argument, the sample is
drawn with replacement:
\begin{verbatim}
> (sample (iseq 1 20) 5 t)
(2 14 14 11 18)
\end{verbatim}
Any value other than \dcode{nil} can be given as the third argument.

{[Logical operations usually interpret any value other than
\dcode{nil} as true.]}

Giving \dcode{nil} as the third argument is equivalent to omitting
the argument.
\end{slide}

\begin{slide}{}
\subsection{Forming Subsets and Deleting Cases}
The \dcode{select} function lets you to select one or more elements
from a list:
\begin{verbatim}
(def x (list 3 7 5 9 12 3 14 2))
> (select x 0)
3
> (select x 2)
5
\end{verbatim}
Lisp, like C but in contrast to FORTRAN, numbers
elements of lists starting at zero.

To get a group of elements at once, use a list of indices:
\begin{verbatim}
> (select x (list 0 2))
(3 5)
\end{verbatim}
\end{slide}

\begin{slide}{}
To select all elements of \dcode{x} except the element
with index 2, you can use
\begin{verbatim}
> (select x (remove 2 (iseq 8)))
(3 7 9 12 3 14 2)
\end{verbatim}
or
\begin{verbatim}
> (select x (which (/= 2 (iseq 8))))
(3 7 9 12 3 14 2)
\end{verbatim}
The \dcode{/=} function produces
\begin{verbatim}
> (/= 2 (iseq 8))
(T T NIL T T T T T)
\end{verbatim}
and \dcode{which} turns this into indices of the \dcode{t} elements:
\begin{verbatim}
> (which (/= 2 (iseq 8)))
(0 1 3 4 5 6 7)
\end{verbatim}
\end{slide}

\begin{slide}{}
\subsection{Combining Several Lists}
\dcode{append} and \dcode{combine} allow you to merge several lists:
{\Large
\begin{verbatim}
> (append (list 1 2 3) (list 4) (list 5 6 7 8))
(1 2 3 4 5 6 7 8)
> (combine (list 1 2 3)
           (list 4)
           (list 5 6 7 8))
(1 2 3 4 5 6 7 8)
\end{verbatim}}
\dcode{append} requires its arguments to be lists and only appends at
one level.

\dcode{combine} allows simple data arguments, and works recursively
through nested lists:
{\Large
\begin{verbatim}
> (combine 1 2 (list 3 4))
(1 2 3 4)
> (append (list (list 1 2) 3) (list 4 5))
((1 2) 3 4 5)
> (combine (list (list 1 2) 3) (list 4 5))
(1 2 3 4 5)
\end{verbatim}}
\end{slide}

\begin{slide}{}
\subsection{Modifying Data}
\dcode{setf} can be used to modify individual elements or sets of elements:
{\Large
\begin{verbatim}
(def x (list 3 7 5 9 12 3 14 2))
X
> x
(3 7 5 9 12 3 14 2)
> (setf (select x 4) 11)
11
> x
(3 7 5 9 11 3 14 2)
> (setf (select x (list 0 2)) (list 15 16))
(15 16)
> x
(15 7 16 9 11 3 14 2)
\end{verbatim}}
\end{slide}

\begin{slide}{}
\dcode{setf} {\em destructively modifies}\/ a list, it does not copy it:
\begin{verbatim}
> (def x (list 1 2 3 4))
X
> (def y x)
Y
> (setf (select y 0) 'a)
A
> y
(A 2 3 4)
> x
(A 2 3 4)
\end{verbatim}
The symbols \dcode{x} and \dcode{y} are two different names for the
same list, and \dcode{setf} has changed that list.
\end{slide}

\begin{slide}{}
To protect a list, you can copy it before making modifications:
\begin{verbatim}
> (def y (copy-list x))
Y
> (setf (select y 1) 'b)
B
> y
(A B 3 4)
> x
(A 2 3 4)
\end{verbatim}
\end{slide}

\begin{slide}{}
The general form of a \dcode{setf} call is
\begin{flushleft}\tt
(setf \param{form} \param{value})
\end{flushleft}
\dcode{setf} can be used to modify other compound data, such as
vectors and arrays.

Like \dcode{def}, \dcode{setf} can also be used to assign a value to a
symbol:
\begin{verbatim}
> (setf z 3)
3
> z
3
\end{verbatim}
There are three differences between \dcode{def} and \dcode{setf}:
\begin{itemize}
\item
\dcode{def} returns the symbol, \dcode{setf} returns the value.
\item
\dcode{def} keeps track of the variables it creates;
\dcode{variables} returns a list of them.
\item
\dcode{def} can only change global variable values, not local ones.
\end{itemize}
\end{slide}

\begin{slide}{}
\subsection{Reading Data Files}
Two functions let you read data from a standard text file:
{\Large
\begin{flushleft}\tt
(read-data-columns [\param{file} [\param{columns}]])\\
(read-data-file [\param{file}])
\end{flushleft}}
The items in the file must be separated by white space (any number or
combination of spaces, tabs or returns), not commas or other
delimiters.

The arguments are optional:
\begin{itemize}
\item
If \param{file} is omitted, a dialog is presented to let you
select the file to read from.
\item
If \dcode{columns} is omitted, the number is determined from the first
line of the file.
\end{itemize}
\dcode{read-data-columns} returns a list of lists representing the
columns in the file.

\dcode{read-data-file} returns a list of the items in the file read
in a row at a time.
\end{slide}

\begin{slide}{}
Suppose you have a file \dcode{abrasion.dat} of the form
\begin{flushleft}\tt
372~~~~~162~~~~~45\\
206~~~~~233~~~~~55\\
175~~~~~232~~~~~61\\
...     ...     ...
\end{flushleft}
Then
\begin{verbatim}
> (read-data-columns "abrasion.dat" 3)
((372 206 175 ...)
 (162 233 232 ...)
 (45 55 61 ...))
> (read-data-file "abrasion.dat")
(372 162 45 206 233 55 ...)
\end{verbatim}
The items in the file can be any items you would type into the
interpreter (numbers, strings, symbols, etc.).
\end{slide}

\begin{slide}{}
You can use these functions together with \dcode{select}:
{\Large
\begin{verbatim}
> (def mydata
       (read-data-columns "abrasion.dat"))
> (def abr (select mydata 0))
ABR
> (def tens (select mydata 1))
TENS
> (def hard (select mydata 2))
HARD
> abr
(372 206 175 154 ...)
\end{verbatim}}
\end{slide}

\begin{slide}{}
These functions should be adequate for most purposes.

If you need to read a file that is not of this form, you can use
low-level file handling functions available in Lisp.

You can also use the \dcode{load} function or the \macbold{Load} menu
command to load a file of Lisp expressions.

\dcode{load} requires the file name to have a \dcode{.lsp} extension.

\end{slide}

\begin{slide}{}
\section{Dynamic Graphs}
The abrasion loss data set used earlier includes a second covariate,
the hardness values of the rubber samples.

Two-dimensional static graphs alone are not adequate for examining the
relationship among three variables.

Instead, we can use some dynamic graphs.

Dynamic graphs use motion and interaction to allow us to explore
higher-dimensional aspects of a data set.

We will look at
\begin{itemize}
\item spinning plots
\item scatterplot matrices
\item brushing and selecting
\item linking plots
\end{itemize}
\end{slide}

\begin{slide}{}
\subsection{Spinning Plots}
Three variables can be displayed in a three-dimensional rotating
scatterplot.

Rotation, together with depth cuing, allow your mind to form a
three-dimensional image of the data.

A rotating plot of the abrasion loss data is constructed by
\begin{verbatim}
(spin-plot (list hardness
                 tensile-strength
                 abrasion-loss))
\end{verbatim}
You can rotate the plot by placing the mouse cursor in one of the
\dcode{Pitch}, \dcode{Roll}, or \dcode{Yaw} squares and pressing the
mouse button.

If you press the mouse button using the {\em extend modifier}, then
the plot continues to rotate after you release the mouse; it stops the
next time you click in the button bar.
\end{slide}

\begin{slide}{}
Every plot window provides a menu for
communicating with the plot.

The menu on a rotating plot allows you to change the speed of
rotation, or to choose whether to use depth cuing\index{depth cuing}
or whether to show the coordinate axes.

By default, the plot uses depth cuing: points closer to the viewer are
drawn larger than points farther away.

The \macbold{Options} item in the menu lets you switch the background
color from black to white; it also lets you change the type of scaling
used.
\end{slide}

\begin{slide}{}
\dcode{spin-plot} accepts several {\em keyword arguments}, including
\begin{description}
\item[]
\dcode{:title} -- a title string for the plot
\item[]
\dcode{:variable-labels} -- a list of strings to use as axis labels
\item[]
\dcode{:point-labels} -- a list of strings to use as point labels
\item[]
\dcode{:scale} -- one of the symbols \dcode{variable}, \dcode{fixed},
or \dcode{nil}. (\dcode{nil} evaluates to itself, but the others need
to be quoted).
\end{description}
A {\em keyword symbol}\/ is a symbol starting with a colon.

Keyword symbols evaluate to themselves, so they do not need to be
quoted.

Keyword arguments to a function must be given after all required (and
optional) arguments.

Keyword arguments can be given in any order.
\end{slide}

\begin{slide}{}
For example,
{\Large
\begin{verbatim}
(spin-plot (list hardness
                 tensile-strength
                 abrasion-loss)
           :title "Abrasion Loss Data"
           :variable-labels
           (list "Hardness"
                 "Tensile Strength"
                 "Abrasion Loss"))
\end{verbatim}}
and
{\Large
\begin{verbatim}
(spin-plot (list hardness
                 tensile-strength
                 abrasion-loss)
           :variable-labels
           (list "Hardness"
                 "Tensile Strength"
                 "Abrasion Loss")
           :title "Abrasion Loss Data")
\end{verbatim}}
are equivalent.
\end{slide}

\begin{slide}{}
The center of rotation is the midrange of the data.

There are three scaling options:
\begin{description}
\item[]
\dcode{variable} (the default) -- each variable is scaled by a
different amount to make their ranges equal to $[-1,1]$.
\item[]
\dcode{fixed} -- a common scale factor is applied to make the largest
range equal to $[-1,1]$
\item[]
\dcode{nil} -- no scaling is used; the variables are assumed to have
been scaled to $[-1,1]$.
\end{description}
You can center the variables at their means and scale them by a common
factor with
{\Large
\begin{verbatim}
(spin-plot
  (list (/ (- hardness (mean hardness))
           140)
        (/ (- tensile-strength 
              (mean tensile-strength))
           140)
        (/ (- abrasion-loss
              (mean abrasion-loss))
           140))
  :scale nil)
\end{verbatim}}
\end{slide}

\begin{slide}{}
\subsection{Scatterplot Matrices}
Another way to look at three (or more) variables is
to use a scatterplot matrix of all pairwise scatterplots:
{\Large
\begin{verbatim}
(scatterplot-matrix (list hardness
                          tensile-strength
                          abrasion-loss)
                    :variable-labels
                    (list "Hardness"
                          "Tensile Strength"
                          "Abrasion Loss"))
\end{verbatim}}
One way to use this plot is to approximately condition on one or more
variables using {\em selecting}\/ and {\em brushing}.
\end{slide}

\begin{slide}{}
Two mouse modes are available:
\begin{description}
\item[]
{\em Selecting}. A plot is in selecting mode when the cursor is an
arrow; this is the default.

In this mode you can 
\begin{itemize}
\item select a point by clicking on it
\item select a group of points by dragging a selection rectangle
\item add to a selection by clicking or dragging with the extend modifier
\end{itemize}
\item[]
{\em Brushing}. In this mode the cursor looks like a paint brush, and
a dashed rectangle, the {\em brush}, is attached to it.

In this mode
\begin{itemize}
\item
as you move the brush, points in it are highlighted
\item
this highlighting is transient; it is turned off when points move out
of the brush
\item you can select points by clicking and dragging
\end{itemize}
\end{description}
\end{slide}

\begin{slide}{}
To change the mouse mode, choose \macbold{Mouse Mode} from the plot
menu, and then select the mode you want from the dialog that is
presented.

You can change the shape of the brush by selecting \macbold{Resize
Brush} from the plot menu.

Brushing and selecting are available in all plots.

These operations can be useful in spinning plots to highlight
features, for example by looking at the shape of a slice of the data.

It is possible to change the way these two standard modes behave and to
define new mouse modes; we will see examples of this later on.
\end{slide}

\begin{slide}{}
\subsection{Interacting with Individual Plots}
Plot menus contain several items that can be used together with
selecting and brushing:
\begin{itemize}
\item
If the \macbold{Labels} item is selected, point labels are shown
next to selected or highlighted points.
\item
The selected points can be removed by choosing \macbold{Remove Selection}
\item
Unselected points can be removed by choosing \macbold{Focus on Selection}
\item
The plot can be rescaled after adding or removing points.
\item
You can change the color or the symbol used to draw a point.
\item
You can specify a set of indices to select or save the currently
selected indices to a variable with the \macbold{Selection} item.
\end{itemize}
Depth cuing in spinning plots works by changing symbols, so it has to
be turned off if you want to use different symbols.
\end{slide}

\begin{slide}{}
\subsection{Linked Plots}
A scatterplot matrix links points in separate scatterplots.

You can link any plots by choosing \macbold{Link View}
from the menus of the plots you want to link.

For example, you can link a histogram of \dcode{hardness} to a
scatterplot of \dcode{abrasion-loss} against \dcode{tensile-strength}.

If you want to be able to select points with a particular label, you
can use \dcode{name-list}.
\begin{verbatim}
(name-list 30)
\end{verbatim}
You can also give \dcode{name-list} a list of strings.

Linking is based on the point index, so you have to use the same
observation order in each plot.
\end{slide}

\begin{slide}{}
\subsection{Modifying a Scatterplot}
After producing a plot, you can add more points or lines to it.

To do this, you need to save the {\em plot object}\/ returned by
plotting functions in a variable:
{\Large
\begin{verbatim}
> (setf p (plot-points tensile-strength
                       abrasion-loss))
#<Object: 302592132, prototype = SCATTER...>
\end{verbatim}}
To use this object, you can send it {\em messages}.

This is done with expressions like
\begin{flushleft}\Large\tt
(send~\param{object}\\
~~~~~~\param{message selector}\\
~~~~~~\param{argument 1} ...)
\end{flushleft}
For example,
\begin{verbatim}
(send p :abline -2.18 0.66)
\end{verbatim}
adds a regression line to the plot.
\end{slide}

\begin{slide}{}
Plot objects understand a number of other messages.

The help message provides a (partial) listing:
{\Large
\begin{verbatim}
> (send p :help)
SCATTERPLOT-PROTO
Scatterplot.
Help is available on the following:

:ABLINE :ACTIVATE :ADD-FUNCTION :ADD-LINES
...
\end{verbatim}}
The list of topics is the same for all scatterplots, but is somewhat
different for rotating plots, scatterplot matrices, and histograms.
\end{slide}

\begin{slide}{}
The \dcode{:clear} message clears the plot.
{\Large
\begin{verbatim}
> (send p :help :clear)
:CLEAR
Message args: (&key (draw t))
Clears the plot data. If DRAW is true the plot
is redrawn; otherwise its current screen image
remains unchanged.
\end{verbatim}}
\dcode{:add-lines} and \dcode{:add-points} add new data:
{\Large
\begin{verbatim}
> (send p :help :add-points)
:ADD-POINTS
Method args:
        (points &key point-labels (draw t))
or:     (x y  &key point-labels (draw t))
Adds points to plot. POINTS is a list of
sequences, POINT-LABELS a list of strings. If
DRAW is true the new points are added to the
screen. For a 2D plot POINTS can be replaced
by two sequences X and Y.
\end{verbatim}}
The term {\em sequence}\/\index{sequences} means a list or a vector.
\end{slide}

\begin{slide}{}
A quadratic regression of \dcode{abrasion-loss} on
\dcode{tensile-strength} produces the fit equation
\begin{displaymath}
Y = -280.1 + 5.986 X + -0.01845 X^{2} + \epsilon
\end{displaymath}
To put this curve on our plot we can use
{\Large
\begin{verbatim}
> (send p :clear)
NIL
> (def x (rseq 100 250 50))
X
> (send p :add-lines x
                    (+ -280.1 
                       (* x 5.986)
                       (* (^ x 2) -0.01845)))
NIL
> (send p :add-points tensile-strength
                      abrasion-loss)
\end{verbatim}}
\end{slide}

\begin{slide}{}
\subsection{Dynamic Simulations}
We can also use plot modification to make a simple animation.

As an example, let's look at the variability in a histogram of 20
normal random variables.

Start by setting up a histogram:
{\Large
\begin{verbatim}
> (setf h (histogram (normal-rand 20)))
#<Object: 303107988, prototype = HISTOGRAM...>
\end{verbatim}}
Then use a simple loop to replace the data:
{\Large
\begin{verbatim}
(dotimes (i 50)
  (send h :clear :draw nil)
  (send h :add-points (normal-rand 20)))
\end{verbatim}}
The \dcode{:draw} \dcode{nil} insures that each histogram remains on
the screen until it is replaced by the next one.
\end{slide}

\begin{slide}{}
\dcode{dotimes} is one of several simple looping constructs available
in Lisp; another is \dcode{dolist}.

If you have a very fast workstation or micro, this animation may move
by too quickly.

One solution is to insert a pause of $10/60$ of a second using
\begin{verbatim}
(pause 10)
\end{verbatim}

So an alternate version of the loop is
{\Large
\begin{verbatim}
(dolist (i (iseq 50))
  (send h :clear :draw nil)
  (pause 10)
  (send h :add-points (normal-rand 20)))
\end{verbatim}}
\end{slide}

\begin{slide}{}
\section{Regression}
Regression models are implemented using objects and
message-sending.

We can fit \dcode{abrasion-loss} to \dcode{tensile-strength} and
\dcode{hardness}:
{\Large
\begin{verbatim}
> (setf m (regression-model
           (list tensile-strength hardness)
           abrasion-loss))

Least Squares Estimates:

Constant                  885.537   (61.801)
Variable 0               -1.37537   (0.194465)
Variable 1                 -6.573   (0.583655)

R Squared:               0.840129
Sigma hat:                36.5186
Number of cases:               30
Degrees of freedom:            27

#<Object: 303170820, prototype = REGRESS...>
\end{verbatim}}
\end{slide}

\begin{slide}{}
The \dcode{regression-model} function accepts a number of keyword
arguments, including
\begin{description}
\item[] \dcode{:print} -- print summary or not
\item[] \dcode{:intercept} -- include intercept term or not
\item[] \dcode{:weights} -- optional weight vector
\end{description}
A range of messages are available for examining and modifying the
model:
{\Large
\begin{verbatim}
> (send m :help)
REGRESSION-MODEL-PROTO
Normal Linear Regression Model
Help is available on the following:

:ADD-METHOD :ADD-SLOT :BASIS :CASE-LABELS
:COEF-ESTIMATES :COEF-STANDARD-ERRORS ...
\end{verbatim}}
\end{slide}

\begin{slide}{}
Some examples:
{\Large
\begin{verbatim}
> (send m :coef-estimates)
(885.537 -1.37537 -6.573)
> (send m :coef-standard-errors)
(61.801 0.194465 0.583655)
> (send m :residuals)
(5.05705 2.43809 9.50074 19.9904 ...)
> (send m :fit-values)
(366.943 203.562 165.499 ...)
> (send m :plot-residuals)
#<Object: 1568394, prototype = SCATTER...>
> (send m :cooks-distances)
(0.00247693 0.000255889 0.00300101 ...)
> (plot-lines (iseq 1 30) *)
\end{verbatim}}
\end{slide}

\begin{slide}{}
\section{Defining Functions and Methods}
No system can provide all tools any user might want.

Being able to define your own functions lets you
\begin{itemize}
\item
provide short names for functions you use often
\item
provide simple commands to execute a long series of operations on
different data sets
\item
develop new methods not provided in the system
\end{itemize}
\end{slide}

\begin{slide}{}
\subsection{Defining Functions}
Functions are defined with the special form \dcode{defun}.

The simplest form of the \dcode{defun} syntax is
\begin{flushleft}\Large\tt
(defun \param{name} \param{parameters} \param{body})
\end{flushleft}
For example, a function to delete an observation can be defined as
{\Large
\begin{verbatim}
> (defun delete-case (i x)
    (select x (remove i (iseq (length x)))))
DELETE-CASE
> (delete-case 2 (list 3 7 5 9 12 3 14 2))
(3 7 9 12 3 14 2)
\end{verbatim}}
None of the arguments to \dcode{defun} are quoted: \dcode{defun} is a
special form that does not evaluate its arguments.
\end{slide}

\begin{slide}{}
Functions can send messages to objects:

Suppose \dcode{m1} is a submodel of \dcode{m2}.

A function to compute the $F$ statistic is
{\Large
\begin{verbatim}
(defun f-statistic (m1 m2)
"Args: (m1 m2)
Computes the F statistic for testing model m1
within model m2."
  (let ((ss1 (send m1 :sum-of-squares))
        (df1 (send m1 :df))
        (ss2 (send m2 :sum-of-squares))
        (df2 (send m2 :df)))
    (/ (/ (- ss1 ss2) (- df1 df2))
       (/ ss2 df2))))
\end{verbatim}}
The \dcode{let} construct is used to create local variables that
simplify the expression.

The documentation string is used by \dcode{help}:
{\Large
\begin{verbatim}
> (help 'f-statistic)
F-STATISTIC                      [function-doc]
Args: (m1 m2)
Computes the F statistic for testing model m1
within model m2.
\end{verbatim}}
\end{slide}

\begin{slide}{}
\subsection{Functions as Arguments}
Earlier we used the function definition of \dcode{sin} as an argument
to \dcode{plot-function}.

We can use functions defined with \dcode{defun} as well:
{\Large
\begin{verbatim}
> (defun f (x) (+ (* 2 x) (^ x 2)))
F
> (plot-function #'f -2 3)
\end{verbatim}}
Recall that \dcode{\#'f} is short for \dcode{(function~f)}, and is
used for obtaining the function associated with the symbol \dcode{f}.

\dcode{spin-function} lets you construct a rotating plot of a
function of two variables:
{\Large
\begin{verbatim}
> (defun f (x y) (+ (^ x 2) (^ y 2)))
F
> (spin-function #'f -1 1 -1 1)
\end{verbatim}}
\dcode{contour-function} takes the same required arguments and
produces a contour plot.
\end{slide}

\begin{slide}{}
\subsection{Graphical Animation Control}
We have already seen how to construct some simple animations.

Using functions, we can provide graphical controls for animations.

As an example, let's look at the effect of the Box-Cox power
transformation
\begin{displaymath}
h(x) = \left\{
\begin{array}{cl}
\frac{\textstyle x^{\lambda} - 1}{\textstyle \lambda}
& \mbox{if $\lambda \not= 0$}\\
\\
\log(x)
& \mbox{otherwise}
\end{array}
\right.
\end{displaymath}
on a normal probability plot of our \dcode{precipitation} variable.
\end{slide}

\begin{slide}{}
A function to compute the transformation and normalize the result:
{\Large
\begin{verbatim}
(defun bc (x p)
  (let* ((bcx (if (< (abs p) .0001)
                  (log x)
                  (/ (^ x p) p)))
         (min (min bcx))
         (max (max bcx)))
    (/ (- bcx min) (- max min))))
\end{verbatim}}
\dcode{let*} is like \dcode{let}, but it establishes its bindings
sequentially.

Next, sort the observations and compute the approximate expected
normal order statistics:
{\Large
\begin{verbatim}
(def x (sort-data precipitation))
(def nq (normal-quant (/ (iseq 1 30) 31)))
\end{verbatim}}
\end{slide}

\begin{slide}{}
A probability plot of the untransformed data is constructed by
\begin{verbatim}
(setf p (plot-points nq (bc x 1)))
\end{verbatim}
Since the power is 1, \dcode{bc} just rescales the data.

To change the power used in the graph, define
\begin{verbatim}
(defun change-power (r)
  (send p :clear :draw nil)
  (send p :add-points nq (bc x r)))
\end{verbatim}
Evaluating 
\begin{verbatim}
(change-power .5)
\end{verbatim}
redraws the plot for a square root transformation.

We could use a loop to run through some powers, but it is nicer to use
a slider dialog:
{\Large
\begin{verbatim}
(sequence-slider-dialog (rseq -1 2 31)
                        :action #'change-power)
\end{verbatim}}
The action function \dcode{change-power} is called every time the
slider is adjusted.
\end{slide}

\begin{slide}{}
\subsection{Defining Methods}
Objects are arranged in a hierarchy.

When an object receives a message, it will use its own method
to respond, if it has one.

If it does not have its own method, it asks its parents, and so on.

New methods are defined using \dcode{defmeth}.
\end{slide}

\begin{slide}{}
For example, suppose \dcode{h} is a histogram created by
{\Large
\begin{verbatim}
(setf h (histogram (normal-rand 20)))
\end{verbatim}}
We can define a method for changing the sample by
{\Large
\begin{verbatim}
(defmeth h :new-sample ()
  (send self :clear :draw nil)
  (pause 10)
  (send self :add-points (normal-rand 20)))
\end{verbatim}}
The variable \dcode{self} refers to the object receiving the message.

This special variable is needed because methods can be inherited.

The loop we used earlier can now be written as
{\Large
\begin{verbatim}
(dotimes (i 50) (send h :new-sample))
\end{verbatim}}

Later on we will see how to override and augment standard methods.
\end{slide}

\begin{slide}{}
\section{More Models and Techniques}
\subsection{Nonlinear Regression}
Suppose we record reaction rates \dcode{y} of a chemical reaction at
various concentrations \dcode{x}:
{\Large
\begin{verbatim}
(def x (list 0.02 0.02 0.06 0.06 0.11 0.11
             0.22 0.22 0.56 0.56 1.10 1.10))
(def y (list 76 47 97 107 123 139 159 
               152 191 201 207 200))
\end{verbatim}}
A model often used for the mean reaction rate is the
Michaelis-Menten model:
\begin{displaymath}
\eta(\theta) = \frac{\theta_{0} x}{\theta_{1} + x}
\end{displaymath}
A lisp function to compute the mean response:
{\Large
\begin{verbatim}
(defun f (theta)
  (/ (* (select theta 0) x)
     (+ (select theta 1) x)))
\end{verbatim}}
Initial estimates can be obtained from a plot
{\Large
\begin{verbatim}
> (plot-points x y)
#<Object: ...>
\end{verbatim}}
\end{slide}

\begin{slide}{}
The function \dcode{nreg-model} fits a nonlinear regression model:
{\Large
\begin{verbatim}
> (setf m (nreg-model #'f y (list 200 .1)))
Residual sum of squares:   7964.185
Residual sum of squares:   1593.158
...
Residual sum of squares:   1195.449

Least Squares Estimates:

Parameter 0           212.6837 (6.947153)
Parameter 1         0.06412127 (0.0082809)

R Squared:           0.9612608
Sigma hat:            10.93366
Number of cases:            12
Degrees of freedom:         10
\end{verbatim}}
Several methods are available for examining the model:
{\Large
\begin{verbatim}
> (send m :residuals)
(25.434 -3.56598 -5.81094 4.1891 ...)
> (send m :leverages)
(0.124852 0.124852 0.193059 0.193059 ...)
> (send m :cooks-distances)
(0.44106 0.0086702 0.041873 0.021761 ...)
\end{verbatim}}
\end{slide}

\begin{slide}{}
\subsection{Maximization and ML Estimation}
The function \dcode{newtonmax} maximizes a function using
Newton's method with backtracking.

As an example, times between failures on aircraft air-conditioning
units are
{\Large
\begin{verbatim}
(def x (list 90 10 60 186 61 49 14 24 56 20 79
             84 44 59 29 118 25 156 310 76 26
             44 23 62 130 208 70 101 208))
\end{verbatim}}
A simple model for these data assumes that the times are independent
gamma variables with density
\begin{displaymath}
\frac{(\beta / \mu)(\beta x / \mu)^{\beta - 1} e^{- \beta x / \mu}}
     {\Gamma(\beta)}
\end{displaymath}
where $\mu$ is the mean time between failures and $\beta$ is the gamma
exponent.
\end{slide}

\begin{slide}{}
A function to evaluate the log likelihood is
{\Large
\begin{verbatim}
(defun gllik (theta)
  (let* ((mu (select theta 0))
         (beta (select theta 1))
         (n (length x))
         (xbm (* x (/ beta mu))))
    (+ (* n (- (log beta)
               (log mu)
               (log-gamma beta)))
       (sum (* (- beta 1) (log xbm)))
       (sum (- xbm)))))
\end{verbatim}}
This definition uses the function \dcode{log-gamma} to evaluate
$\log(\Gamma(\beta))$.

Initial estimates for $\mu$ and $\beta$ are
{\Large
\begin{verbatim}
> (mean x)
83.5172
> (^ (/ (mean x) (standard-deviation x)) 2)
1.39128
\end{verbatim}}
\end{slide}

\begin{slide}{}
Using these starting values, we can maximize the log likelihood
function:
{\Large
\begin{verbatim}
> (newtonmax #'gllik (list 83.5 1.4))
maximizing...
Iteration 0.
Criterion value = -155.603
Iteration 1.
Criterion value = -155.354
Iteration 2.
Criterion value = -155.347
Iteration 3.
Criterion value = -155.347
Reason for termination: gradient size is less
than gradient tolerance.
(83.5173 1.67099)
\end{verbatim}}

You can use \dcode{numgrad} to check that the gradient is close to
zero, and \dcode{numhess} to compute an approximate covariance matrix.

\dcode{newtonmax} will return the derivative information if it is
given the keyword argument \dcode{:return-derivs} with value
\dcode{t}.

If \dcode{newtonmax} does not converge, \dcode{nelmeadmax} may be able
to locate the maximum.
\end{slide}

\begin{slide}{}
\subsection{Approximate Bayesian Computations}
Suppose \dcode{times-pos} are survival times and \dcode{wbc-pos} white
blood cell counts for AG-positive leukemia patients (Feigl \& Zelen,
1965).

The log posterior density for an exponential regression model with
a flat prior density is
\begin{displaymath}
\sum_{i=1}^{n} \theta_{1} x_{i}
-n \log(\theta_{0}) 
-\frac{1}{\theta_{0}} \sum_{i=1}^{n} y_{i} e^{\theta_{1} x_{i}}
\end{displaymath}
It is computed by
{\Large
\begin{verbatim}
(def transformed-wbc-pos
     (- (log wbc-pos) (log 10000)))

(defun llik-pos (theta)
  (let* ((x transformed-wbc-pos)
         (y times-pos)
         (theta0 (select theta 0))
         (theta1 (select theta 1))
         (t1x (* theta1 x)))
    (- (sum t1x)
       (* (length x) (log theta0))
       (/ (sum (* y (exp t1x)))
          theta0))))
\end{verbatim}}
\end{slide}

\begin{slide}{}
\dcode{bayes-model} finds the posterior mode and prints a summary
of first approximations to posterior means and standard deviations:
{\Large
\begin{verbatim}
> (setf lk (bayes-model #'llik-pos '( 33 .8)))
maximizing...
......
Reason for termination: gradient size ...

First Order Approx. to Posterior Moments:

Parameter 0             56.8489 (13.9713)
Parameter 1            0.481829 (0.179694)
\end{verbatim}}
More accurate approximations are available:
{\Large
\begin{verbatim}
> (send lk :moments)
((65.3085 0.485295) (17.158 0.186587))
\end{verbatim}}
Moments of functions of the parameters can also be approximated with
\dcode{:moments}.

Marginal densities can be approximated using the \dcode{:margin1}
message.
\end{slide}

\begin{slide}{}
\heading{Generalized Linear Models}
The generalized linear model system includes functions for fitting
models with gamma, binomial, and poisson errors and various links.

New error structures and link functions can be added.

An Example:
{\Large
\begin{verbatim}
> (def months-before (iseq 1 18))
MONTHS-BEFORE
> (def event-counts '(15 11 14 17 5 11 10 4
                      8 10 7 9 11 3 6 1 1 4))
EVENTS-RECALLED
> (def m (poissonreg-model months-before
                           event-counts))
Iteration 1: deviance = 26.3164
Iteration 2: deviance = 24.5804
Iteration 3: deviance = 24.5704
Iteration 4: deviance = 24.5704
\end{verbatim}}
\end{slide}

\begin{slide}{}
{\Large
\begin{verbatim}
Weighted Least Squares Estimates:

Constant               2.80316  (0.148162)
Variable 0          -0.0837691  (0.0167996)

Scale taken as:              1
Deviance:              24.5704
Number of cases:            18
Degrees of freedom:         16
\end{verbatim}
{\tt :residuals} and other methods are inherited:
\begin{verbatim}
> (send m :residuals)
(-0.0439191 -0.790305 ...)
> (send m :plot-residuals)
\end{verbatim}}
\end{slide}

\begin{slide}{}
\chapter{Some Lisp Programming}
\end{slide}

\begin{slide}{}
\section{Conditional Evaluation and Predicates}
The basic Lisp conditional evaluation construct is \dcode{cond}.
{\Large
\begin{verbatim}
> (defun my-abs (x)
    (cond ((> x 0) x)
          ((= x 0) 0)
          ((< x 0) (- x))))
MY-ABS
> (my-abs -2)
2
\end{verbatim}}
Several simplified versions exist, including \dcode{if}, \dcode{unless} and
\dcode{when}
{\Large
\begin{verbatim}
(defun my-abs (x) (if (>= x 0) x (- x)))
\end{verbatim}}
Another simplified version of \dcode{cond} is \dcode{case}:
{\Large
\begin{verbatim}
> (defun f (i)
    (case i
      (1 'one)
      (2 'two)
      (t (error "out of range"))))
> (f 1)
ONE
\end{verbatim}}
\end{slide}

\begin{slide}{}
Logical expressions can be combined using \dcode{and}, \dcode{or}, and
\dcode{not}.
{\Large
\begin{verbatim}
> (defun in-range (x) (and (< 3 x) (< x 5)))
IN-RANGE
> (in-range 2)
NIL
> (in-range 4)
T

> (defun not-in-range (x)
    (or (>= 3 x) (>= x 5)))
NOT-IN-RANGE
> (not-in-range 2)
T

> (defun in-range (x) (< 3 x 5))
IN-RANGE
> (in-range 2)
NIL
> (in-range 4)
T
> (defun not-in-range (x) (not (in-range x)))
NOT-IN-RANGE
\end{verbatim}}
\end{slide}

\begin{slide}{}
\section{More on Functions}
\subsection{Function as Data}
Suppose we want a function \dcode{num-deriv} to compute a numerical
derivative.

If we define
{\Large
\begin{verbatim}
(defun f (x) (+ x (^ x 2)))
\end{verbatim}}
then we want to get
{\Large
\begin{verbatim}
> (num-deriv #'f 1)
3
\end{verbatim}}
Defining \dcode{num-deriv} as
{\Large
\begin{verbatim}
(defun num-deriv (fun x)
  (let ((h 0.00001))
    (/ (- (fun (+ x h))
          (fun (- x h)))
       (* 2 h))))
\end{verbatim}}
will not work -- our function is the {\em value} of \dcode{fun},
not its {\em function definition}:
{\Large
\begin{verbatim}
> (num-deriv #'f 1)
error: unbound function - FUN
\end{verbatim}}
\end{slide}

\begin{slide}{}
We need a function that calls the value of \dcode{fun} with an
argument:
{\Large
\begin{verbatim}
> (funcall #'+ 1 2)
3
\end{verbatim}}
A correct definition of \dcode{num-deriv} is
{\Large
\begin{verbatim}
(defun num-deriv (fun x)
  (let ((h 0.00001))
    (/ (- (funcall fun (+ x h))
          (funcall fun (- x h)))
       (* 2 h))))
\end{verbatim}}
Another useful function is \dcode{apply}:
{\Large
\begin{verbatim}
> (apply #'+ '(1 2 3))
6
> (apply #'+ 1 2 '(3 4))
10
> (apply #'+ 1 '(2 3))
6
\end{verbatim}}
\end{slide}

\begin{slide}{}
\subsection{Anonymous Functions}
Defining and naming throw-away functions like \dcode{f} is awkward.

The same problem exists in mathematics.

Logicians developed the {\em lambda calculus}:
\begin{displaymath}
\lambda(x)(x + x^{2})
\end{displaymath}
is ``the function that returns $x + x^{2}$ for the argument x.''

Lisp uses this idea:
{\Large
\begin{verbatim}
(lambda (x) (+ x (^ x 2)))
\end{verbatim}}
is a {\em lambda expression} for our function.
\end{slide}

\begin{slide}{}
Lambda expressions are not yet Lisp functions.

To make them into functions, you need to use \dcode{function} or
\verb+#'+:
{\Large
\begin{verbatim}
#'(lambda (x) (+ x (^ x 2)))
\end{verbatim}}
To take our derivative:
{\Large
\begin{verbatim}
> (num-deriv #'(lambda (x) (+ x (^ x 2)))
             1)
3
\end{verbatim}}
To plot $2x+x^{2}$ over the range $[-2,4]$,
{\Large
\begin{verbatim}
(plot-function #'(lambda (x)
                 (+ (* 2 x) (^ x 2)))
               -2
               3)
\end{verbatim}}

Functions can also use lambda expressions to make new functions and
return them as the value of the function.

We will see a few examples of this a bit later.
\end{slide}

\begin{slide}{}
\section{Local Variables and Environments}
\subsection{Variables and Scoping}
A pairing of a variable symbol with a value is called a {\em binding}

Collections of bindings are called an {\em environment}.

Bindings can be global or they can be local to a group of expressions.

\dcode{let} and \dcode{let*} expressions and function definitions
set up local bindings.
\end{slide}

\begin{slide}{}
Consider the lambda expression
{\Large
\begin{verbatim}
(lambda (x) (+ x a))
\end{verbatim}}
The meaning of \dcode{x} in the body is clear -- it is bound to the
calling argument.

The meaning of \dcode{a} is not so clear -- it is a {\em free variable}.
\end{slide}

\begin{slide}{}
We need a convention for determining the bindings of free variables.

This is the reason we need to use \dcode{function} on lambda expressions:

\begin{itemize}
\item[]
Free variables in a function are bound to their values in the
environment where the function is created
\end{itemize}
This is called the {\em lexical}\/ or {\em static}\/ scoping rule.

Other scoping rules are possible.
\end{slide}

\begin{slide}{}
An example: making a derivative function:
{\Large
\begin{verbatim}
> (defun make-num-deriv (fun)
    (let ((h 0.00001))
      #'(lambda (x)
        (/ (- (funcall fun (+ x h))
              (funcall fun (- x h)))
           (* 2 h)))))
MAKE-NUM-DERIV
> (setf f (make-num-deriv
           #'(lambda (x) (+ x (^ x 2)))))
#<Closure: #120cbe80>
> (funcall f 1)
3
> (funcall f 3)
7
\end{verbatim}}
\end{slide}

\begin{slide}{}
Another example: making a normal log likelihood:

The log likelihood of a sample from a normal distribution is
\begin{displaymath}
-\frac{n}{2}\left[\log \sigma^{2}
+ \frac{(\overline{x}-\mu)^{2}}{\sigma^{2}}
+ \frac{s^{2}}{\sigma^{2}}\right]
\end{displaymath}
A function for evaluating this expression as a function of $\mu$
and $\sigma^{2}$ is returned by
{\Large
\begin{verbatim}
(defun make-norm-log-lik (x)
  (let ((n (length x))
        (x-bar (mean x))
        (s-2 (^ (standard-deviation x) 2)))
    #'(lambda (mu sigma-2)
      (* -0.5
         n
         (+ (log sigma-2)
            (/ (^ (- x-bar mu) 2) sigma-2)
            (/ s-2 sigma-2))))))
\end{verbatim}}
The result returned by this function can be maximized, or it can be
plotted with \dcode{spin-function} or \dcode{contour-function}.
\end{slide}

\begin{slide}{}
\subsection{Local Functions}
It is also possible to set up local functions using \dcode{flet}:
{\Large
\begin{verbatim}
> (defun f (x)
    (flet ((add-1 (x) (+ x 1)))
      (add-1 x)))
F
> (f 2)
3
> (add-1 2)
error: unbound function - ADD-1
\end{verbatim}}
\dcode{flet} sets up bindings in parallel, like \dcode{let}

\dcode{flet} cannot be used to define local recursive functions.

\dcode{labels} is like \dcode{flet} but allows mutually recursive function
definitions.
\end{slide}

\begin{slide}{}
\subsection{Optional, Keyword and Rest Arguments}
A number of functions used so far take optional arguments, keyword
arguments, or variable numbers of arguments.

A function taking an optional argument is defined one of three ways:
{\Large
\begin{verbatim}
(defun f (x &optional y) ...)
(defun f (x &optional (y 1)) ...)
(defun f (x &optional (y 1 z)) ...)
\end{verbatim}}
In the second and third forms, the default value is 1; in the first
form it is \dcode{nil}

In the third form, \dcode{z} is \dcode{t} if the optional argument is
supplied; otherwise \dcode{z} is \dcode{nil}.

We can add an optional argument for the step size to
\dcode{num-deriv}:
{\Large
\begin{verbatim}
(defun num-deriv (fun x &optional (h 0.00001))
  (/ (- (funcall fun (+ x h))
        (funcall fun (- x h)))
     (* 2 h)))
\end{verbatim}}
\end{slide}

\begin{slide}{}
Keyword arguments are defined similarly to optional arguments:
{\Large
\begin{verbatim}
(defun f (x &key y) ...)
(defun f (x &key (y 1)) ...)
(defun f (x &key (y 1 z)) ...)
\end{verbatim}}
The function is then called as
{\Large
\begin{verbatim}
(f 1 :y 2)
\end{verbatim}}
Using a keyword argument in \dcode{num-deriv}:
{\Large
\begin{verbatim}
(defun num-deriv (fun x &key (h 0.00001))
  (/ (- (funcall fun (+ x h))
        (funcall fun (- x h)))
     (* 2 h)))
\end{verbatim}}
With a keyword argument, \dcode{num-deriv} is called as
{\Large
\begin{verbatim}
(num-deriv #'f 1 :h 0.001)
\end{verbatim}}
\end{slide}

\begin{slide}{}
A function with a variable number of arguments is defined as
{\Large
\begin{verbatim}
(defun f (x &rest y) ...)
\end{verbatim}}
All arguments beyond the first are made into a list and bound to
\dcode{y}.

For example:
{\Large
\begin{verbatim}
> (defun my-plus (&rest x) (apply #'+ x))
MY-PLUS
> (my-plus 1 2 3)
6
\end{verbatim}}
If more than one of these modifications is used, they must appear in
the order \verb+&optional+, \verb+&rest+, \verb+&key+.

There is an upper limit on the number of arguments a function can receive.
\end{slide}

\begin{slide}{}
\section{Mapping}
{\em Mapping}\/ is the process of applying a function elementwise to a
list.

The primary mapping function is \dcode{mapcar}:
{\Large
\begin{verbatim}
> (setf x (normal-rand '(2 3 2)))
((0.27397 3.5358) (-0.11065 1.2178 1.050) 
 (0.78268 0.95955))
> (mapcar #'mean x)
(1.904895 0.71913 0.8711149)
\end{verbatim}}
Mapcar can take several lists as arguments:
{\Large
\begin{verbatim}
> (mapcar #'+ '(1 2 3) '(4 5 6))
(5 7 9)
\end{verbatim}}
Using \dcode{mapcar}, we can define a simple numerical integrator
for functions on $[0,1]$:
{\Large
\begin{verbatim}
> (defun integrate (f &optional (n 100))
    (let* ((x (rseq 0 1 n))
           (fv (mapcar f x)))
      (mean fv)))
INTEGRATE
> (integrate #'(lambda (x) (^ x 2)))
0.335017
\end{verbatim}}
\end{slide}

\begin{slide}{}
\section{More on Compound Data}
\subsection{Lists}
Lists are the most important compound data type.

Lists can be empty:
{\Large
\begin{verbatim}
> (list)
NIL
> '()
NIL
> ()
NIL
\end{verbatim}}
They can be used to represent sets:
{\Large
\begin{verbatim}
> (union '(1 2 3) '(3 4 5))
(5 4 1 2 3)
> (intersection '(1 2 3) '(3 4 5))
(3)
> (set-difference '(1 2 3) '(3 4 5))
(2 1)
\end{verbatim}}
\end{slide}

\begin{slide}{}
In addition to using \dcode{select}, you can get pieces of a list with
{\Large
\begin{verbatim}
> (first '(1 2 3))
1
> (second '(1 2 3))
2
> (rest '(1 2 3))
(2 3)
\end{verbatim}}
Two other useful functions are \dcode{remove-duplicates}
{\Large
\begin{verbatim}
> (remove-duplicates '(1 1 2 3 3))
(1 2 3)
\end{verbatim}}
and \dcode{count}:
{\Large
\begin{verbatim}
> (count 2 '(1 2 3 4) :test #'=)
1
> (count 2 '(1 2 3 4) :test #'<=)
3
\end{verbatim}}
\dcode{remove-duplicates} also accepts a \dcode{:test} argument.
\end{slide}

\begin{slide}{}
\subsection{Vectors}
Vectors are a second form of compound data.

A vector is constructed with the \dcode{vector} function
{\Large
\begin{verbatim}
> (vector 1 2 3)
#(1 2 3)
\end{verbatim}}
or by typing its printed representation:
{\Large
\begin{verbatim}
> (setf x '#(1 2 3))
#(1 2 3)
> x
#(1 2 3)
\end{verbatim}}
Elements of vectors can be extracted and changed with \dcode{select}:
{\Large
\begin{verbatim}
> (select x 1)
2
> (setf (select x 1) 5)
5
> x
#(1 5 3)
\end{verbatim}}
\end{slide}

\begin{slide}{}
Vectors can be copied with \dcode{copy-vector}.

Vectors are usually stored more efficiently than lists, and their
elements can be accessed more rapidly.

But there are fewer functions for operating on vectors than on lists:
{\Large
\begin{verbatim}
> (first x)
error: bad argument type - #(1 5 3)
> (rest x)
error: bad argument type - #(1 5 3)
\end{verbatim}}
\end{slide}

\begin{slide}{}
\subsection{Sequences}
Lists, vectors, and strings are sequences.

Several functions operate on any sequence:
{\Large
\begin{verbatim}
> (length '(1 2 3))
3
> (length '#(1 2 3))
3
> (length "abc")
3
> (select '(1 2 3) 0)
1
> (select '#(1 2 3) 0)
1
> (select "abc" 0)
#\a
\end{verbatim}}
Sequences can be coerced to different types with \dcode{coerce}:
{\Large
\begin{verbatim}
> (coerce '(1 2 3) 'vector)
#(1 2 3)
> (coerce "abc" 'list)
(#\a #\b #\c)
\end{verbatim}}
\end{slide}

\begin{slide}{}
\subsection{Arrays}
The \dcode{matrix} function constructs a two-dimensional array:
{\Large
\begin{verbatim}
> (matrix '(2 3) '(1 2 3 4 5 6))
#2A((1 2 3) (4 5 6))
\end{verbatim}}
Again you can type the printed representation
{\Large
\begin{verbatim}
> (setf m '#2A((1 2 3) (4 5 6)))
#2A((1 2 3) (4 5 6))
\end{verbatim}}
and \dcode{select} extracts and modifies elements:
{\Large
\begin{verbatim}
> (select m 1 1)
5
> (select m 1 '(0 1))
#2A((4 5))
> (select m '(0 1) '(0 1))
#2A((1 2) (4 5))
> (setf (select m 1 1) 'a)
A
> m
#2A((1 2 3) (4 A 6))
\end{verbatim}}
\end{slide}

\begin{slide}{}
\section{Format}
\dcode{format} is a very flexible output function.

It prints to {\em output streams} or to strings.

The default output stream is \dcode{*standard-output*}; it can be
abbreviated to \dcode{t}:
{\Large
\begin{verbatim}
> (format *standard-output* "Hello~%")
Hello
NIL
> (format t "Hello~%")
Hello
NIL
> (format nil "Hello")
"Hello"
\end{verbatim}}
\verb+~%+ is the {\em format directive} for a new line.
\end{slide}

\begin{slide}{}
Other useful format directives are \verb+~a+ and \verb+~s+:
{\Large
\begin{verbatim}
> (format t "Examples: ~a ~s~%" '(1 2) '(3 4))
Examples: (1 2) (3 4)
NIL
> (format t "Examples: ~a ~s~%" "ab" "cd")
Examples: ab "cd"
\end{verbatim}}
These two directives differ in their handling of {\em escape characters}.

There are many other format directives.
\end{slide}

\begin{slide}{}
\section{Some Statistical Functions}
\subsection{Some Basic Functions}
\begin{verbatim}
> (difference '(1 3 6 10))
(2 3 4)
> (pmax '(1 2 3) 2)
(2 2 3)
> (split-list '(1 2 3 4 5 6) 3)
((1 2 3) (4 5 6))
> (cumsum '(1 2 3 4))
(1 3 6 10)
> (accumulate #'* '(1 2 3 4))
(1 2 6 24)
\end{verbatim}

\subsection{Sorting Functions}
\begin{verbatim}
> (sort-data '(14 10 12 11))
(10 11 12 14)
> (rank '(14 10 12 11))
(3 0 2 1)
> (order '(14 10 12 11))
(1 3 2 0)
\end{verbatim}
\end{slide}

\begin{slide}{}
\subsection{Interpolation and Smoothing}
\begin{verbatim}
(spline x y :xvals xv)
(lowess x y)
(kernel-smooth x y :width w)
(kernel-dens x)
\end{verbatim}

\subsection{Linear Algebra Functions}
\begin{verbatim}
(identity-matrix 4)
(diagonal '(1 2 3))
(diagonal '#2a((1 2)(3 4)))
(transpose '#2a((1 2)(3 4)))
(transpose '((1 2)(3 4)))
(matmult a b)
(make-rotation '(1 0 0) '(0 1 0) 0.05)
\end{verbatim}

\begin{verbatim}
(lu-decomp a)
(inverse a)
(determinant a)
(chol-decomp a)
(qr-decomp a)
(sv-decomp a)
\end{verbatim}
\end{slide}

\begin{slide}{}
\section{Odds and Ends}
\subsection{Errors}
The \dcode{error} functions signals an error:
{\Large
\begin{verbatim}
> (error "bad value")
error: bad value
> (error "bad value: ~s" "A")
error: bad value: "A"
\end{verbatim}}

\subsection{Debugging}
Several debugging functions are available:
\begin{description}
\item[]
\dcode{debug}/\dcode{nodebug} -- toggle debug mode; in debug mode,
an error puts you into a break loop.
\item[]
\dcode{break} -- called within a function to enter a break loop
\item[]
\dcode{baktrace} -- prints traceback in a beak loop.
\item[]
\dcode{step} -- single steps through an evaluation.
\end{description}
\end{slide}

\begin{slide}{}
\section{Example}
\subsection{Estimating a Survival Function}
Suppose the variable \dcode{times} contains survival times and
\dcode{status} contains status values, with 1 representing death and 0
censoring.

To compute a Kaplan-Meier or Fleming-Harrington estimator, we first
need the death times and the unique death times:
{\Large
\begin{verbatim}
(setf dt-list
      (coerce (select times
                      (which (= 1 status)))
              'list))
(setf udt
      (sort-data
       (remove-duplicates dt-list :test #'=)))
\end{verbatim}}
\end{slide}

\begin{slide}{}
Next, we need the number of deaths and the number at risk at each
death time:
{\Large
\begin{verbatim}
(setf d (mapcar #'(lambda (x)
                  (count x dt-list :test #'=))
                udt))
(setf r (mapcar #'(lambda (x)
                  (count x times :test #'<=))
                udt))
\end{verbatim}}
Using these values, we can compute the Kaplan-Meier estimator at the
death times as
{\Large
\begin{verbatim}
(setf km (accumulate #'* (/ (- r d) r)))
\end{verbatim}}
The Fleming-Harrington estimator is
{\Large
\begin{verbatim}
(setf fh (exp (- (cumsum (/ d r)))))
\end{verbatim}}
\end{slide}

\begin{slide}{}
Greenwood's formula for the variance is
{\Large
\begin{verbatim}
(* (^ km 2) (cumsum (/ d r (pmax (- r d) 1))))
\end{verbatim}}
The \dcode{pmax} expression prevents a division by zero.

Tsiatis' formula leads to
{\Large
\begin{verbatim}
(* (^ km 2) (cumsum (/ d (^ r 2))))
\end{verbatim}}

To construct a plot we need a function that builds the consecutive
corners of a step function:
{\Large
\begin{verbatim}
(defun make-steps (x y)
  (let* ((n (length x))
         (i (iseq (+ (* 2 n) 1))))
    (list (append '(0) (repeat x (repeat 2 n)))
          (select (repeat (append '(1) y)
                          (repeat 2 (+ n 1)))
                  i))))
\end{verbatim}}
Then
{\Large
\begin{verbatim}
(plot-lines (make-steps udt km))
\end{verbatim}}
produces a plot of the Kaplan-Meier estimator.
\end{slide}

\begin{slide}{}
\subsection{Weibull Regression}
Suppose \dcode{times} are survival times, \dcode{status} contains
death/censoring indicators, and \dcode{x} contains a matrix of
covariates, including a column of ones.

A Weibull model for these data has a log likelihood of the form
\begin{displaymath}
\sum s_{i} \log \alpha + \sum (s_{i}\log \mu_{i}-\mu_{i})
\end{displaymath}
where
\begin{eqnarray*}
\log \mu_{i} & = & \alpha \log t_{i} + \eta_{i}\\
\eta_{i} & = & x_{i} \beta
\end{eqnarray*}
and $\alpha$ is the Weibull exponent, $\beta$ is a vector of
parameters.

A function to compute this log likelihood is
{\Large
\begin{verbatim}
(defun llw (x y s log-a b)
  (let* ((a (exp log-a))
         (eta (matmult x b))
         (log-mu (+ (* a (log y)) eta)))
    (+ (* (sum s) log-a)
       (sum (- (* s log-mu) (exp log-mu))))))
\end{verbatim}}
\end{slide}

\begin{slide}{}
Reasonable initial estimates for the parameters might be
$\alpha=1$,
\begin{displaymath}
\beta_{0} =\frac{\sum s_{i}}{\sum t_{i}}
\end{displaymath}
for the constant term, and $\beta_{i}=0$ for all other $i$.

For a single covariate:
{\Large
\begin{verbatim}
> (newtonmax
   #'(lambda (theta)
     (llw x
          times
          status 
          (first theta)
          (rest theta)))
   (list 0 (/ (sum status) (sum times)) 0))
maximizing...
Iteration 0.
Criterion value = -510.668
...
Iteration 8.
Criterion value = -47.0641
Reason for termination: gradient size ...
(0.311709 -4.80158 1.73087)
\end{verbatim}}
\end{slide}

\begin{slide}{}
\chapter{Objects}
\end{slide}

\begin{slide}{}
\section{Some Background}
Some basic features of object oriented programming include
\begin{itemize}
\item mutable state information
\item messages and methods
\item inheritance
\end{itemize}
Some people add other features, or place different emphasis on
different features.
\end{slide}

\begin{slide}{}
\subsection{Origins of OOP}
\begin{itemize}
\item
Earliest example: {\em Simula}, a simulation language.
\item
First ``pure'' object oriented language: {\em Smalltalk}.
\item
Smalltalk was used for interacting with graphical objects, building
graphical user interfaces.
\item
The Macintosh user interface inherits many features from the
Smalltalk-based Xerox system.
\end{itemize}
\end{slide}

\begin{slide}{}
\subsection{Some Variations}
\begin{itemize}
\item
Languages/systems can be purely object oriented or just support object
oriented programming.
\item
Two different approaches are used:
\begin{itemize}
\item class/inheritance systems
\item prototype/delegation systems
\end{itemize}
\item
Most newer systems support some form of multiple inheritance.
\end{itemize}
\end{slide}

\begin{slide}{}
\subsection{Some OOP Systems}
\begin{itemize}
\item {\em C++}\/ from ATT
\item {\em Objective C}, used in NeXT computers.
\item {\em Smalltalk-80}
\item {\em Object Pascal}, for programming the Macintosh.
\item {\em Flavors}, on {\em Symbolics}\/ Lisp machines
\item {\em Lisp Object Oriented Programming System (LOOPS)}
\item {\em Object Lisp}
\item {\em Portable Common LOOPS}
\item {\em CLOS}, the new Common Lisp standard
\item {\em Common Objects}, from HP Labs.
\end{itemize}
\end{slide}

\begin{slide}{}
\subsection{References on OOP}
\begin{itemize}
\item[]
Abelson, H. and Sussman, G. J., (1985), {\em Structure and Interpretation
of Computer Programs}, Cambridge, Ma: MIT Press.

\item[]
Winston, P. H. and Horn, B. K. P., (1988), {\em Lisp}, 3rd Edition,
Reading, Ma: Addison Wesley.

\item[]
Keene, S. E., (1989), {\em Object-Oriented Programming in Common Lisp: A
Programmer's Guide to CLOS}, Reading, Ma: Addison Wesley.

\item[]
{\em Proceedings of the ACM Conference on Object-Oriented Systems, Languages
and Applications (OOPSLA)}, in {\em ACM SIGPLAN Notices}, (1986-91).

\item[]
Wegner, P., (1990), ``Concepts and paradigms of object-oriented
programming,'' {\em ACM SIGPLAN Messenger}, vol. 1.
\end{itemize}
\end{slide}

\begin{slide}{}
\section{Introduction}
Suppose we would like to build an ``intelligent'' function for describing
or plotting a data set.

Initially we are looking at three types of data:
\begin{itemize}
\item single samples
\item multiple samples
\item paired samples
\end{itemize}
A \dcode{describe-data} function might be written as
\begin{flushleft}\Large\tt
(defun describe-data (data)\\
~~(cond\\
~~~((single-sample-p data) ...)\\
~~~((multiple-sample-p data) ...)\\
~~~((paired-sample-p data) ...)\\
~~~(t (error "don't know how to \~\\
~~~~~~~~~~~~~~describe this data set"))))
\end{flushleft}
\end{slide}

\begin{slide}{}
Now suppose we come up with a fourth type, a simple time series.

We would have to edit this function, possibly breaking good code. 

An alternative is to arrange for the appropriate code to be looked up
in the data set itself -- to dispatch on the data set type.

We would need some way of defining new data types and actions.

This is the basis of object oriented programming.
\end{slide}

\begin{slide}{}
\subsection{Some Terms}
An {\em object}\/ is a data structure that contains information in
{\em slots}.

Objects respond to {\em messages}\/ asking them to take certain
actions.

A message is {\em sent} to an object using the function \dcode{send}
in an expression like
\begin{flushleft}\Large\tt
(send \param{object} \param{selector} \param{arg-1} ... \param{arg-n})
\end{flushleft}
The \param{selector} is a keyword symbol used to identify the message.
\end{slide}

\begin{slide}{}
As an example, the \dcode{histogram} function returns a histogram object:
{\Large
\begin{verbatim}
> (setf p (histogram (normal-rand 20)))
#<Object: 1750540, prototype = HISTOGRAM-PROTO>
\end{verbatim}}
The object can be sent a message to add an additional set of points:
{\Large
\begin{verbatim}
> (send p :add-points (normal-rand 20))
NIL
\end{verbatim}}
The {\em message selector}\/ for the message is \dcode{:add-points}.

The {\em message argument}\/ is the sample of normal variables
generated by the expression \dcode{(normal-rand 20)}.

The {\em message}\/ consists of the selector and the arguments.

The procedure called to respond to this message is the {\em method}\/
for the message.
\end{slide}

\begin{slide}{}
\subsection{Inheritance and the Root Object}
Objects are organized in an {\em inheritance hierarchy}.

At the top is the {\em root object}, the value of \dcode{*object*}.

The root object contains methods for standard messages, such as
\begin{itemize}
\item[] \dcode{:own-slots} -- list of slots in an object
\item[] \dcode{:add-slot} -- insert a new slot 
\item[] \dcode{:delete-slot} -- delete a slot
\item[] \dcode{:slot-value} -- access or modify a slot's value
\end{itemize}
\end{slide}

\begin{slide}{}
Some examples:
{\Large
\begin{verbatim}
> (send *object* :own-slots)
(DOCUMENTATION PROTO-NAME INSTANCE-SLOTS)

> (send *object* :slot-value 'proto-name)
*OBJECT*

> *object*
#<Object: 302253140, prototype = *OBJECT*>

> (send *object* :slot-value 'instance-slots)
NIL
\end{verbatim}}
\end{slide}

\begin{slide}{}
\subsection{Constructing New Objects}
The root object is a {\em prototype}\/ object.

Prototypes serve as templates for building new objects -- {\em instances}.

The \dcode{:new} message is used to ask a prototype for a new instance.

Let's use an object to represent a data set:
{\Large
\begin{verbatim}
> (setf x (send *object* :new))
#<Object: 1750672, prototype = *OBJECT*>
\end{verbatim}}
Initially, the object has no slots of its own:
{\Large
\begin{verbatim}
> (send x :own-slots)
NIL
\end{verbatim}}
\end{slide}

\begin{slide}{}
We can add a slot for some data and a slot for a title string:
{\Large
\begin{verbatim}
> (send x :add-slot 'data (normal-rand 50))
(1.39981 -0.0601746 ....)
> (send x :add-slot 'title)
NIL
> (send x :own-slots)
(TITLE DATA)
> (send x :slot-value 'title)
NIL
> (send x :slot-value 'title "a data set")
"a data set"
> (send x :slot-value 'title)
"a data set"
\end{verbatim}}
\end{slide}

\begin{slide}{}
\subsection{Defining New Methods}
The code used to respond to a message is called a {\em method}.
 
\dcode{send} searches the receiving object and its ancestors until it
finds a method.

All messages used so far used methods in the root object.

New methods can be defined using \dcode{defmeth}.

The general form of a \dcode{defmeth} expression is:
\begin{flushleft}\Large\tt
(defmeth \param{object} \param{selector} \param{parameters}\\
~~\param{body})
\end{flushleft}
The argument \param{object} is evaluated, the others are not.
\end{slide}

\begin{slide}{}
An example:
{\Large
\begin{verbatim}
(defmeth x :describe (&optional (stream t))
  (let ((title (slot-value 'title))
        (data (slot-value 'data)))
    (format stream "This is ~a~%" title)
    (format stream
            "The sample mean is ~a~%"
            (mean data))
    (format stream
            "The sample SD is ~a~%"
            (standard-deviation data))))
\end{verbatim}}
The definition looks like a \dcode{defun}, except for the object
expression appearing as the first argument.

The new \dcode{:describe} method can be used like any other method:
{\Large
\begin{verbatim}
> (send x :describe)
This is a data set
The sample mean is 0.127521
The sample SD is 1.0005
NIL
\end{verbatim}}
\end{slide}

\begin{slide}{}
A few notes:
\begin{itemize}
\item
Within a method, the variable \dcode{self} refers to the object receiving
the message.
\item
\dcode{self} is needed since methods can be inherited.
\item
\dcode{self} is added to the environment for the function body before
the other arguments.
\item
Within a method, the function \dcode{slot-value} can be used to access
a slot.

Using this function is more efficient than using the
\dcode{:slot-value} message.
\item
The \dcode{slot-value} function can {\em only}\/ be used in the body of
a method.
\item
Within a method, \dcode{slot-value} can be used as a place form with
\dcode{setf}.
\end{itemize}
\end{slide}

\begin{slide}{}
It is good programming practice not to assume too much about the slots
an object has.

For our data set object, we can write {\em accessor methods}\/ for the
title and data slots:
{\Large
\begin{verbatim}
(defmeth x :title (&optional (title nil set))
  (if set (setf (slot-value 'title) title))
  (slot-value 'title))

(defmeth x :data (&optional (data nil set))
  (if set (setf (slot-value 'data) data))
  (slot-value 'data))
\end{verbatim}}
Using these accessors we can rewrite the \dcode{:describe} method:
{\Large
\begin{verbatim}
(defmeth x :describe (&optional (stream t))
  (let ((title (send self :title))
        (data (send self :data)))
    (format stream "This is ~a~%" title)
    (format stream
            "The sample mean is ~a~%"
            (mean data))
    (format stream
            "The sample SD is ~a~%"
     (standard-deviation data))))
\end{verbatim}}
\end{slide}

\begin{slide}{}
\subsection{Printing Objects}
The printing system prints objects by sending them the \dcode{:print}
message.

A method for \dcode{:print} must take an optional {\em stream}\/ argument.

A simple \dcode{:print} method for our data set:
{\Large
\begin{verbatim}
(defmeth x :print (&optional (stream t))
  (format stream "#<~a>" (send self :title)))
\end{verbatim}}
The result is
{\Large
\begin{verbatim}
> x
#<a data set>
\end{verbatim}}
\end{slide}

\begin{slide}{}
\subsection{Help Information for Messages}
If we change the definition of the \dcode{:describe} method to
{\Large
\begin{verbatim}
(defmeth x :describe (&optional (stream t))
"Method args: (&optional (stream t))
Prints a simple description of the object
to STREAM."
  (let ((title (send self :title))
        (data (send self :data)))
    ...))
\end{verbatim}}
then the string is used as a documentation string: {\Large
\begin{verbatim}
> (send x :help :describe)
:DESCRIBE
Method args: (&optional (stream t))
Prints a simple description of the object
to STREAM.
NIL
\end{verbatim}}
\end{slide}

\begin{slide}{}
\section{Prototypes}
\subsection{Creating Simple Prototypes}
Instead of repeating the process used so far for each new data set, we
can construct a {\em dataset prototype}\/ using \dcode{defproto}.

The simplest form of a \dcode{defproto} expression is
\begin{flushleft}\Large\tt
(defproto \param{name} \param{instance slots})
\end{flushleft}
\dcode{defproto} evaluates \param{instance slots} and
\begin{itemize}
\item
constructs a new object, assigns it to the global variable \param{name}
\item
installs a slot \dcode{proto-name} with value \dcode{name}
\item
installs a slot \dcode{instance-slots} with the symbols from
\param{instance slots}
\item
installs a slot with value nil for each symbol in the
\dcode{instance-slots} list
\end{itemize}
Different instances created from a prototype usually differ only in
the values of their instance slots.
\end{slide}

\begin{slide}{}
We can define a prototype for our data set as
{\Large
\begin{verbatim}
(defproto data-set-proto '(data title))
\end{verbatim}}
Prototype objects are like any other objects.

We can place values in their slots
{\Large
\begin{verbatim}
(send data-set-proto :slot-value
      'title "a data set")
\end{verbatim}}
and we can define methods for them:
{\Large
\begin{verbatim}
(defmeth data-set-proto
         :title (&optional (title nil set))
  (if set (setf (slot-value 'title) title))
  (slot-value 'title))

(defmeth data-set-proto :data ...)

(defmeth data-set-proto :describe ...)

(defmeth data-set-proto :print ...)

(defmeth data-set-proto :plot ()
  (histogram (send self :data)
             :title (send self :title)))
\end{verbatim}}
\end{slide}

\begin{slide}{}
\subsection{Creating Instances from Prototypes}
An instance is created by sending a prototype the \dcode{:new}
message.

The method for this message does the following:
\begin{itemize}
\item
creates a new object that inherits from the prototype
\item
adds a slot for each symbol in the prototype's \dcode{instance-slots}
list
\item
sets the value of each new slot to its value in the prototype
\item
sends the new object the \dcode{:isnew} message with the arguments
given to \dcode{:new}
\item
returns the new object
\end{itemize}
\end{slide}

\begin{slide}{}
The root object \dcode{:isnew} initialization method allows slots to
be initialized by keyword arguments:
{\Large
\begin{verbatim}
> (setf x (send data-set-proto :new
                :data (chisq-rand 20 5)))
#<a data set>
> (send x :title)
"a data set"
> (send x :data)
(10.7254 10.4314 2.32346 ...)
> (send x :describe)
This is a data set
The sample mean is 5.17844
The sample SD is 3.2129
\end{verbatim}}
\end{slide}

\begin{slide}{}
Any data set needs its own data.

We can define an \dcode{:isnew} method that requires a data argument:
{\Large
\begin{verbatim}
(defmeth data-set-proto
         :isnew (data &key title)
  (send self :data data)
  (if title (send self :title title)))
\end{verbatim}}
It is often useful to define an appropriate \dcode{:isnew}
initialization method.

It is (almost) never necessary to define a new \dcode{:new}
method.
\end{slide}

\begin{slide}{}
\subsection{Prototypes and Inheritance}
Prototypes can inherit from other objects, usually other prototypes.

The full form of a \dcode{defproto} expression is
\begin{flushleft}\Large\tt
(defproto \param{name}\\
~~~~~~~~~~\param{instance slots}\\
~~~~~~~~~~\param{shared slots}\\
~~~~~~~~~~\param{parents}\\
~~~~~~~~~~\param{doc string})
\end{flushleft}
All arguments except the first are evaluated.

\param{instance slots} and \param{shared slots} should be \dcode{nil}
or lists of symbols.

We won't be using shared slots.

\param{parents} should be a single object or a list of objects -- the
default is the root object.

The instance slots of the new prototype are
the union of the \param{instance slots} in the call and the 
instance slots of all ancestors.

So you only need to specify {\em additional}\/ slots you want.
\end{slide}

\begin{slide}{}
An equally-spaced time series prototype might add \dcode{origin} and
\dcode{spacing} slots:
{\Large
\begin{verbatim}
(defproto time-series-proto
          '(origin spacing) () data-set-proto)
\end{verbatim}}
Accessor methods for the new slots:
{\Large
\begin{verbatim}
(defmeth time-series-proto
         :origin (&optional (origin nil set))
  (if set (setf (slot-value 'origin) origin))
  (slot-value 'origin))

(defmeth time-series-proto
         :spacing (&optional (sp nil set))
  (if set (setf (slot-value 'spacing) sp))
  (slot-value 'spacing))
\end{verbatim}}
and default values for the \dcode{title} slot and the new slots can be
set as
{\Large
\begin{verbatim}
(send time-series-proto :title "a time series")
(send time-series-proto :origin 0)
(send time-series-proto :spacing 1)
\end{verbatim}}
\end{slide}

\begin{slide}{}
An example -- a short moving average series:
{\Large
\begin{verbatim}
> (let* ((e (normal-rand 21))
         (i (iseq 1 20))
         (d (+ (select e i)
               (* 0.6 (select e (- i 1))))))
    (setf y (send time-series-proto :new d)))
#<a time series>
> (send y :describe)
This is a time series
The sample mean is 0.194134
The sample SD is 1.15485
\end{verbatim}}
\end{slide}

\begin{slide}{}
\subsection{Overriding and Modifying Inherited Methods}
The inherited \dcode{:plot} and \dcode{:describe} methods are not
optimal for a time series.

We can {\em override}\/ the \dcode{:plot} method with the definition
{\Large
\begin{verbatim}
(defmeth time-series-proto :plot ()
  (let* ((data (send self :data))
         (start (send self :origin))
         (step (send self :spacing))
         (n (length data)))
    (plot-points (+ start (* step (iseq n)))
                 data)))
\end{verbatim}}
\end{slide}

\begin{slide}{}
Instead of overriding the \dcode{:describe} method completely,
we can augment it:
{\Large
\begin{verbatim}
(defmeth time-series-proto
         :describe (&optional (stream t))
  (let ((ac (autocor (send self :data))))
    (call-next-method stream)
    (format stream
            "The autocorrelation is ~a~%"
            ac)))
\end{verbatim}}
The \dcode{call-next-method} function calls the next method for the
current selector
\begin{flushleft}\Large\tt
(call-next-method \param{arg-1} ... \param{arg-n})
\end{flushleft}
The \dcode{autocor} function can be defined as
{\Large
\begin{verbatim}
(defun autocor (x)
  (let ((n (length x))
        (x (- x (mean x))))
    (/ (mean (* (select x (iseq 0 (- n 2))) 
                (select x (iseq 1 (- n 1)))))
       (mean (* x x)))))
\end{verbatim}}
\end{slide}

\begin{slide}{}
\subsection{Some Cautions}
\begin{itemize}
\item
Unfortunately the object system does not allow you to hide slots and
messages intended for internal use from public access.
\item
To protect yourself against breaking inherited methods you need to
follow a few guidelines:
\begin{itemize}
\item
Don't use slot names you didn't add yourself.
\item
Don't modify slots unless you are sure it is safe.
\item
Don't define methods for messages unless you know this won't do any
harm.
\end{itemize}
\end{itemize}
\end{slide}

\begin{slide}{}
\section{Examples}
\subsection{Survival Function Estimator}
We can use an object to organize our tools for fitting a survival function.

Since we need the unique death times, death counts, and numbers at risk,
we can define a prototype that adds slots to hold these values:
{\Large
\begin{verbatim}
(defproto survival-proto
          '(death-times num-deaths num-at-risk)
          ()
          data-set-proto)

(send survival-proto :title
      "a survival data set")
\end{verbatim}}
Accessors for these three slots are defined as
{\Large
\begin{verbatim}
(defmeth survival-proto :death-times ()
  (slot-value 'death-times))
(defmeth survival-proto :num-deaths ()
  (slot-value 'num-deaths))
(defmeth survival-proto :num-at-risk ()
  (slot-value 'num-at-risk))
\end{verbatim}}
\end{slide}

\begin{slide}{}
The \dcode{data} slot can be used to hold both the times and the
death indicators.

The accessor method can be defined as
{\Large
\begin{verbatim}
(defmeth survival-proto :data
         (&optional times status)
  (when times
    (call-next-method (list times status))
    (let* ((i (which (= 1 status)))
           (dt (select times i))
           (dt-list (coerce dt 'list))
           (udt (sort-data (remove-duplicates
                            dt-list
                            :test #'=)))
           (d (mapcar
               #'(lambda (x)
                 (count x dt-list :test #'=))
               udt))
           (r (mapcar
               #'(lambda (x)
                 (count x times :test #'<=))
               udt)))
      (setf (slot-value 'death-times) udt)
      (setf (slot-value 'num-deaths) d)
      (setf (slot-value 'num-at-risk) r)))
  (slot-value 'data))
\end{verbatim}}
\end{slide}

\begin{slide}{}
The \dcode{:isnew} method fills in the data:
{\Large
\begin{verbatim}
(defmeth survival-proto :isnew
         (times status &optional title)
  (send self :data times status)
  (if title (send self :title title)))
\end{verbatim}}
A \dcode{:describe} method:
{\Large
\begin{verbatim}
(defmeth survival-proto :describe
         (&optional (stream t))
  (let ((title (send self :title))
        (data (send self :data)))
    (format stream "This is ~a~%" title)
    (format stream
            "The mean time is ~a~%"
            (mean (first data)))
    (format stream
            "The number of failures is ~a~%"
            (sum (second data)))))
\end{verbatim}}
A \dcode{:plot} method:
{\Large
\begin{verbatim}
(defmeth survival-proto :plot ()
  (let ((km (send self :km-estimator))
        (udt (send self :death-times)))
    (plot-lines (make-steps udt km))))
\end{verbatim}}
\end{slide}

\begin{slide}{}
A method for the Kaplan-Meier estimator:
{\Large
\begin{verbatim}
(defmeth survival-proto :km-estimator ()
  (let ((r (send self :num-at-risk))
        (d (send self :num-deaths)))
    (accumulate #'* (/ (- r d) r))))
\end{verbatim}}
Methods for the Fleming-Harrington estimator and standard errors based on
Greenwood's formula and Tsiatis' formula are
{\Large
\begin{verbatim}
(defmeth survival-proto :fh-estimator ()
  (let ((r (send self :num-at-risk))
        (d (send self :num-deaths)))
    (exp (- (cumsum (/ d r))))))

(defmeth survival-proto :greenwood-se ()
  (let* ((r (send self :num-at-risk))
         (d (send self :num-deaths))
         (km (send self :km-estimator))
         (rd1 (pmax (- r d) 1)))
    (* km (sqrt (cumsum (/ d r rd1))))))

(defmeth survival-proto :tsiatis-se ()
  (let ((r (send self :num-at-risk))
        (d (send self :num-deaths))
        (km (send self :km-estimator)))
    (* km (sqrt (cumsum (/ d (^ r 2)))))))
\end{verbatim}}
\end{slide}

\begin{slide}{}
\subsection{Model Prototypes}
Linear regression, nonlinear regression and GLIM model
prototypes are arranged as:
\begin{center}
\begin{picture}(400,380)
%\put(105,350){\protoimage{regression-model-proto}}
\put(200,360){\makebox(0,0){\tt regression-model-proto}}
\put(200,360){\oval(250,20)}
\put(0,280){\protoimage{nreg-model-proto}}
\put(105,210){\protoimage{glim-proto}}
\put(0,140){\protoimage{poissonreg-proto}}
\put(105,90){\protoimage{binomialreg-proto}}
\put(210,140){\protoimage{gammareg-proto}}
\put(0,20){\protoimage{logitreg-proto}}
\put(210,20){\protoimage{probitreg-proto}}
\put(200,350){\line(0,-1){130}}
\put(200,350){\line(-2,-1){100}}
\put(200,210){\line(0,-1){100}}
\put(200,210){\line(-2,-1){100}}
\put(200,210){\line(2,-1){100}}
\put(200,90){\line(-2,-1){100}}
\put(200,90){\line(2,-1){100}}
\end{picture}
\end{center}
\end{slide}

\begin{slide}{}
\section{OOP and User Interfaces}
Users of a graphics workstation can take a limited set of actions.

They can
\begin{itemize}
\item hit keys
\item move the pointer
\item click the pointer
\end{itemize}
These actions can be viewed as messages that are being sent to
objects on the screen.

Graphical user interfaces present users with a model based on a desk:
\begin{itemize}
\item
Overlapping windows act like sheets of paper
\item
Different windows represent different software objects on the screen
\item
Each object may respond differently to key stroke or locator click
messages
\end{itemize}
\end{slide}

\begin{slide}{}
The {\em user interface conventions} of the window system
determine what objects receive what messages.

Several variations are possible:
\begin{itemize}
\item
On the Macintosh the front window receives all input.

A window becomes the front window by clicking on it.
\item
Under {\em twm}\/ in an {\em X11}\/ system, the window containing the
locator cursor receives all input.
\end{itemize}
Most window management systems are object-oriented in design.

The window system is responsible for converting user actions into
messages.

The programmer is responsible for defining any methods objects
need in order to respond to the user's actions.
\end{slide}

\begin{slide}{}
Some other variations include:
\begin{itemize}
\item
size and location of windows may have to be determined by the user
\item
resizing or moving windows from within a program may not be possible
\item
menus may be popped up or pulled down
\item
mouse buttons may mean different things in different settings
\item
mice have different numbers of buttons
\end{itemize}
\end{slide}

\begin{slide}{}
\section{Menus}
Menus are created using the \dcode{menu-proto} prototype.

The initialization method requires a single argument, the menu title.
{\Large
\begin{verbatim}
> (setf data-menu
        (send menu-proto :new "Data"))
#<Object: 4055334, prototype = MENU-PROTO>
\end{verbatim}}
A menu bar is available for holding menus not associated with a graph.

The \dcode{:install} message installs a menu; \dcode{:remove} removes it:
{\Large
\begin{verbatim}
> (send data-menu :install)
NIL
> (send data-menu :remove)
NIL
\end{verbatim}}
\end{slide}

\begin{slide}{}
Initially a menu contains no items.

Items are created using the \dcode{menu-item-proto} prototype.

The initialization method requires one argument, the item's title.

It also accepts several keyword arguments, including
\begin{itemize}
\item
\dcode{:action} -- a function to be called when the item is selected
\item
\dcode{:enabled} -- true by default
\item
\dcode{:key} -- a character to serve as the keyboard equivalent
\item
\dcode{:mark} -- \dcode{nil} (the default) or \dcode{t} to indicate a
check mark.
\end{itemize}
Analogous messages are available for accessing and changing these
values in existing menu items.
\end{slide}

\begin{slide}{}
Suppose we assign a data set object as the value of some global
symbol:
{\Large
\begin{verbatim}
(setf *current-data-set* x)
\end{verbatim}}
Two menu items for dealing with the current data set are:
{\Large
\begin{verbatim}
> (setf desc
        (send menu-item-proto :new "Describe"
              :action
              #'(lambda ()
                (send *current-data-set*
                      :describe))))
#<Object: 4034406, prototype = MENU-ITEM-PROTO>
> (setf plot 
        (send menu-item-proto :new "Plot"
              :action 
              #'(lambda ()
                (send *current-data-set*
                      :plot))))
#<Object: 3868686, prototype = MENU-ITEM-PROTO>
\end{verbatim}}
\end{slide}

\begin{slide}{}
You can force an item's action to be invoked by sending it the
\dcode{:do-action} message:
{\Large
\begin{verbatim}
> (send desc :do-action)
This is a data set
The sample mean is 4.165463
The sample SD is 1.921366
\end{verbatim}}

The system sends the \dcode{:do-action} message when you select the item
in a menu.

The \dcode{:append-items} message adds the items to the menu:
{\Large
\begin{verbatim}
> (send data-menu :append-items desc plot)
NIL
\end{verbatim}}
\end{slide}

\begin{slide}{}
On the Macintosh, the \dcode{:key} message adds a keyboard equivalent
to an item:
{\Large
\begin{verbatim}
> (send desc :key #\D)
#\D
\end{verbatim}}
This message is ignored on most other systems.

You can also enable and disable an item with the \dcode{:enabled}
message:
{\Large
\begin{verbatim}
> (send desc :enabled)
T
> (send desc :enabled nil)
NIL
> (send desc :enabled t)
T
\end{verbatim}}
Before a menu is presented, the system sends each of its menu items
the \dcode{:update} message.

You can define a method for this message that enables or disables the
item, or adds a check mark, as appropriate.
\end{slide}

\begin{slide}{}
\section{Dialogs}
Dialogs are similar to menus in that they are based on a dialog
prototype and dialog item prototypes.

There are, however many more variations.

Fortunately most dialogs you need fall into one of several categories
and can be produced by custom dialog construction functions.
\end{slide}

\begin{slide}{}
\subsection{Modal Dialogs}
Modal dialogs are designed to ask specific questions and wait until
they receive a response.

All other interaction is disabled until the dialog is dismissed -- they
place the system in dialog mode.

Six functions are available for producing some standard modal dialogs:

\begin{itemize}
\item
\dcode{(message-dialog \param{string})} -- presents a message with an
\macbold{OK} button; returns \dcode{nil} when the button is pressed.
\item
\dcode{(ok-or-cancel-dialog \param{string})} -- presents a message with an
\macbold{OK} and a \macbold{Cancel} button; returns \dcode{t} or \dcode{nil}
according to the button pressed.
\end{itemize}
\end{slide}

\begin{slide}{}
\begin{itemize}
\item
\dcode{\Large\tt(choose-item-dialog \param{string} \param{string-list})} --
presents a heading and a set of radio buttons for choosing one of the
strings.

Returns the index of the selected string on \macbold{OK} or
\dcode{nil} on \macbold{Cancel}.
{\Large
\begin{verbatim}
> (choose-item-dialog
   "Dependent variable:"
   '("X" "Y" "Z"))
1
\end{verbatim}}
\item
\dcode{\Large\tt(choose-subset-dialog \param{string} \param{string-list})} --
presents a heading and a set of check boxes for indicating which items
to select.

Returns a list of the list of selected indices on \macbold{OK} or
\dcode{nil} on \macbold{Cancel}.
{\Large
\begin{verbatim}
> (choose-subset-dialog
   "Independent variables:"
   '("X" "Y" "Z"))
((0 2))
\end{verbatim}}
\end{itemize}
\end{slide}

\begin{slide}{}
\begin{itemize}
\item
\dcode{\Large\tt(get-string-dialog \param{prompt} [:initial \param{expr}])}\\
presents a dialog with a prompt, an editable text field, an
\macbold{OK} and a \macbold{Cancel} button.

The initial contents of the editable field is empty or the
\verb+~a+-formated version of \param{expr}.

The result is a string or \dcode{nil}.
{\Large
\begin{verbatim}
> (get-string-dialog
   "New variable label:"
   :initial "X")
"Tensile Strength"
\end{verbatim}}
\item
\dcode{\Large\tt(get-value-dialog \param{prompt} [:initial \param{expr}])}\\
like \dcode{get-string-dialog}, except
\begin{itemize}
\item
the initial value expression is converted to a string with \verb+~s+
formatting
\item
the text is interpreted as a lisp expression and is evaluated
\item
the result is a list of the value, or \dcode{nil}
\end{itemize}
\end{itemize}
\end{slide}

\begin{slide}{}
There are two dialogs for dealing with files:
\begin{itemize}
\item
\dcode{(open-file-dialog)} -- presents a standard \macbold{Open File}
dialog and returns a file name string or \dcode{nil}.

Resets the working folder on \macbold{OK}.
\item
\dcode{(set-file-dialog \dcode{prompt})} -- presents a standard
\macbold{Save File} dialog.

Returns a file name string or \dcode{nil}.

Resets the working folder on \macbold{OK}.
\end{itemize}
In the {\em X11}\/ versions of these are currently just
\dcode{get-string-dialogs}.

The MS Windows versions are also not yet fully developed.
\end{slide}

\begin{slide}{}
\subsection{Modeless Dialogs}
Two standard modeless slider dialogs are available:
\begin{itemize}
\item
\dcode{sequence-slider-dialog} -- a slider for scrolling
through a sequence.

An action function is called each time the slider is adjusted.

The current sequence value or a value from a display sequence is displayed.
\item
\dcode{interval-slider-dialog} -- similar to the sequence slider, except
that a range to be divided up into a reasonable number of points is given.

By default, the range and number of points are adjusted to produce nice
printed values.
\end{itemize}
\end{slide}

\begin{slide}{}
\subsection{Custom Dialogs}
If the standard dialogs are not sufficient, you can construct custom
modal or modeless dialogs using a variety of items:
\begin{itemize}
\item static and editable text fields
\item radio button groups
\item check boxes
\item sliders
\item scrollable lists
\item push  buttons
\end{itemize}
\end{slide}

\begin{slide}{}
\chapter{Outline of the Graphics System}
\end{slide}

\begin{slide}{}
\section{Overview}
The graphics window object tree:
\begin{center}
\begin{picture}(400,360)
\put(120,330){\framebox(160,30){\tt window-proto}}
\put(100,260){\framebox(200,30){\tt graph-window-proto}}
\put(120,190){\framebox(160,30){\tt graph-proto}}
\put(240,140){\framebox(160,30){\tt scatmat-proto}}
\put(0,140){\framebox(160,30){\tt spin-proto}}
\put(220,70){\framebox(180,30){\tt histogram-proto}}
\put(0,70){\framebox(180,30){\tt namelist-proto}}
\put(100,0){\framebox(200,30){\tt scatterplot-proto}}
\put(200,330){\line(0,-1){40}}
\put(200,260){\line(0,-1){40}}
\put(200,190){\line(0,-1){160}}
\put(200,190){\line(-6,-1){120}}
\put(200,190){\line(6,-1){120}}
\put(200,190){\line(-2,-3){60}}
\put(200,190){\line(2,-3){60}}
\end{picture}
\end{center}
\end{slide}

\begin{slide}{}
All top-level windows share certain common features:
\begin{itemize}
\item a title
\item a way to be moved
\item a way to be resized
\end{itemize}
These common features are incorporated in the window prototype
\dcode{window-proto}.

Both dialog and graphics windows inherit from the window prototype.
\end{slide}

\begin{slide}{}
\section{Graph Windows}
\subsection{Outline of the Drawing System}
A graphics window is a view onto a {\em drawing canvas}.

The dimensions of the canvas can be {\em fixed}\/ or {\em elastic}.
\begin{itemize}
\item
If a dimension is fixed, it has a scroll bar.
\item
If it is elastic, it fills the window.
\end{itemize}
The name list window uses one fixed dimension; all other standard plots
use elastic dimensions by default.
\end{slide}

\begin{slide}{}
%***** Picture *****
%\vspace{2.5in}
\begin{center}
\mbox{\psfig{figure=coords.ps}}
\end{center}

The canvas has a coordinate system.
\begin{itemize}
\item coordinate units are pixels
\item the origin is the top-left corner
\item the $x$ coordinate increases from left to right
\item the $y$ coordinate increases from top to bottom
\end{itemize}
\end{slide}

\begin{slide}{}
A number of drawing operations are available.

Drawable objects include
\begin{itemize}
\item rectangles
\item ovals
\item arcs
\item polygons
\end{itemize}
These can be
\begin{itemize}
\item framed
\item painted
\item erased
\end{itemize}
Other drawables are
\begin{itemize}
\item symbols
\item strings
\item bitmaps
\end{itemize}
\end{slide}

\begin{slide}{}
The precise effect of drawing operations depends on the {\em state} of the
drawing system.

Drawing system state components include
\begin{itemize}
\item colors -- foreground and background
\item drawing mode -- normal or XOR
\item line type -- dashed or solid
\item pen width -- an integer
\item use color -- on or off
\item buffering -- on or off
\end{itemize}
\end{slide}

\begin{slide}{}
\subsection{Animation Techniques}
There are two basic animation methods
\begin{itemize}
\item
{\em XOR drawing} -- drawing ``inverts'' the colors on the screen
\item
{\em double buffering} -- a picture is built up in a background
buffer and then copied to the screen
\end{itemize}
Some tradeoffs:
\begin{itemize}
\item
XOR drawing is usually faster
\item
XOR drawing automatically preserves the background
\item
XOR drawing distorts background and object during drawing
\item
XOR drawing inherently involves a certain amount of flicker
\item
Moving several objects by XOR causes distortion -- only one can
move at a time
\item
{\em Inverting}\/ is not well-defined on color displays
\end{itemize}
\end{slide}

\begin{slide}{}
Lisp-Stat uses
\begin{itemize}
\item XOR drawing for moving the brush rectangle
\item double buffering for rotation
\end{itemize}

As an example, let's move a highlighted symbol down the diagonal
of a window using both methods.

A new graphics window is constructed by
{\Large
\begin{verbatim}
(setf w (send graph-window-proto :new))
\end{verbatim}}
\end{slide}

\begin{slide}{}
Using XOR drawing:
{\Large
\begin{verbatim}
(let ((width (send w :canvas-width))
      (height (send w :canvas-height))
      (mode (send w :draw-mode)))
  (send w :draw-mode 'xor)
  (dotimes (i (min width height))
    (send w :draw-symbol 'disk t i i)
    (pause 2)
    (send w :draw-symbol 'disk t i i))
  (send w :draw-mode mode))
\end{verbatim}}
Using double Buffering:
{\Large
\begin{verbatim}
(let ((width (send w :canvas-width))
      (height (send w :canvas-height)))
  (dotimes (i (min width height))
    (send w :start-buffering)
    (send w :erase-window)
    (send w :draw-symbol 'disk t i i)
    (send w :buffer-to-screen)))
\end{verbatim}}
\end{slide}

\begin{slide}{}
\subsection{Handling Events}
User actions produce various {\em events}\/ that the system handles
by sending messages to the appropriate objects.

There are several types of events:
\begin{itemize}
\item resize events
\item exposure or redraw events
\item mouse events -- motion and click
\item key events
\item idle ``events''
\end{itemize}
Resize and redraw events cause \dcode{:resize} and \dcode{:redraw}
messages to be sent to the window object.

Mouse and key events produce \dcode{:do-click}, \dcode{:do-motion},
and \dcode{:do-key} messages.

In idle periods, the \dcode{:do-idle} message is sent.

The \dcode{:while-button-down} message can be used to follow the mouse
inside a click\\
(for dragging, etc.)
\end{slide}

\begin{slide}{}
\subsection{Graphics Window Menus}
Every graphics window can have a menu.

The user interface guidelines of the window system determine how
the menu is presented:
\begin{itemize}
\item
On the Macintosh, the menu is installed in the menu bar when the window
is the front window.
\item
In MS Windows, the menu is installed in the application's menu bar when
the window is the front window.
\item
In {\em SunView}, the menu is popped up when the right mouse button is
pressed in the window.
\item
Under {\em X11}, the menu is popped up when the mouse is clicked in a
\macbold{Menu} button at the top of the window.
\end{itemize}
The \dcode{:menu} message retrieves a graph window's menu
or installs a new menu.
\end{slide}

\begin{slide}{}
\subsection{An Example}
As a simple example to illustrate the handling of events, let's
construct a window that shows a single highlighted symbol at its
center.

The window is constructed by
{\Large
\begin{verbatim}
(setf w (send graph-window-proto :new))
\end{verbatim}}
We can add slots for holding the coordinates of our point:
{\Large
\begin{verbatim}
(send w :add-slot 'x
      (/ (send w :canvas-width) 2))
(send w :add-slot 'y
      (/ (send w :canvas-height) 2))

(defmeth w :x (&optional (val nil set))
  (if set (setf (slot-value 'x) (round val)))
  (slot-value 'x))

(defmeth w :y (&optional (val nil set))
  (if set (setf (slot-value 'y) (round val)))
  (slot-value 'y))
\end{verbatim}}
New coordinate values are rounded since drawing operations require
integer arguments.
\end{slide}

\begin{slide}{}
The \dcode{:resize} method positions the point at the center of
the canvas:
{\Large
\begin{verbatim}
(defmeth w :resize ()
  (let ((width (send self :canvas-width))
        (height (send self :canvas-height)))
    (send self :x (/ width 2))
    (send self :y (/ height 2))))
\end{verbatim}}
The \dcode{:redraw} method erases the window and redraws
the symbol at the location specified by the coordinate values:
{\Large
\begin{verbatim}
(defmeth w :redraw ()
  (let ((x (send self :x))
        (y (send self :y)))
    (send self :erase-window)
    (send self :draw-symbol 'disk t x y)))
\end{verbatim}}
\end{slide}

\begin{slide}{}
The \dcode{:do-click} message positions the symbol at the click and then
allows it to be dragged:
{\Large
\begin{verbatim}
(defmeth w :do-click (x y m1 m2)
  (flet ((set-sym (x y)
           (send self :x x)
           (send self :y y)
           (send self :redraw)))
    (set-symbol x y)
    (send self :while-button-down #'set-sym)))
\end{verbatim}}
The \dcode{:do-idle} method can be used to move the symbol
in a random walk:
{\Large
\begin{verbatim}
(defmeth w :do-idle ()
  (let ((x (send self :x))
        (y (send self :y)))
    (case (random 4)
      (0 (send self :x (- x 5)))
      (1 (send self :x (+ x 5)))
      (2 (send self :y (- y 5)))
      (3 (send self :y (+ y 5))))
    (send self :redraw)))
\end{verbatim}}
\end{slide}

\begin{slide}{}
We can turn the random walk on by typing
{\Large
\begin{verbatim}
(send w :idle-on t)
\end{verbatim}}
and we can turn it off with
{\Large
\begin{verbatim}
(send w :idle-on nil)
\end{verbatim}}
A better solution is to use a menu item:
{\Large
\begin{verbatim}
(setf run-item
      (send menu-item-proto :new "Run"
            :action 
            #'(lambda ()
              (send w :idle-on
                    (not (send w :idle-on))))))
\end{verbatim}}
To put a check mark on this item when the walk is running, define
an \dcode{:update} method:
{\Large
\begin{verbatim}
(defmeth run-item :update ()
   (send self :mark (send w :idle-on)))
\end{verbatim}}
\end{slide}

\begin{slide}{}
It would also be nice to have a menu item for restarting the walk:
{\Large
\begin{verbatim}
(setf restart-item
      (send menu-item-proto :new "Restart"
            :action
            #'(lambda () (send w :restart))))
\end{verbatim}}
The \dcode{:restart} method is defined as
{\Large
\begin{verbatim}
(defmeth w :restart ()
  (let ((width (send self :canvas-width))
        (height (send self :canvas-height)))
    (send self :x (/ width 2))
    (send self :y (/ height 2))
    (send self :redraw)))
\end{verbatim}}
and a menu with the two items is installed by
{\Large
\begin{verbatim}
(setf menu
      (send menu-proto :new "Random Walk"))
(send menu :append-items restart-item run-item)
(send w :menu menu)
\end{verbatim}}
\end{slide}

\begin{slide}{}
There may be some flickering when the random walk is running.

This flickering can be eliminated by modifying the \dcode{:redraw}
method to use double buffering:
{\Large
\begin{verbatim}
(defmeth w :redraw ()
  (let ((x (round (send self :x)))
        (y (round (send self :y))))
    (send self :start-buffering)
    (send self :erase-window)
    (send self :draw-symbol 'disk t x y)
    (send self :buffer-to-screen)))
\end{verbatim}}
\end{slide}

\begin{slide}{}
\section{Statistical Graphics Windows}
\dcode{graph-proto} is the statistical graphics prototype.

It inherits from \dcode{graph-window-proto}.

The graph prototype is responsible for managing the data used
by all statistical graphs.

Variations in how these data are displayed are implemented in
separate prototypes for the standard graphs.
\end{slide}

\begin{slide}{}
The graph prototype is a view into $m$ dimensional space.

It allows the display of both points and connected line segments

The default methods in this prototype implement a simple scatterplot
of two of the $m$ dimensions.

Many features of graphics windows are enhanced to simplify
adding new features to graphs.
\end{slide}

\begin{slide}{}
The graph prototype adds the following features:
\begin{itemize}
\item $m$-dimensional point and {\em line start}\/ data
\item affine transformations consisting of
\begin{itemize}
\item centering and scaling
\item a linear transformation
\end{itemize}
\item ranges for raw, scaled and canvas coordinates
\item mouse modes for controlling interaction
\item linking strategy
\item window layout management
\begin{itemize}
\item margin, content and aspect
\item background (axes)
\item overlays
\item content
\end{itemize}
\item standard menus and menu items
\end{itemize}
\end{slide}

\begin{slide}{}
\subsection{Data and Axes}
The \dcode{:isnew} method for the graph prototype
requires one argument, the number of variables:
{\Large
\begin{verbatim}
> (setf w (send graph-proto :new 4))
#<Object: 302823396, prototype = GRAPH-PROTO>
\end{verbatim}}
Using the stack loss data as an illustration, we can add data
{\Large
\begin{verbatim}
> (send w :add-points
        (list air temp conc loss))
NIL
\end{verbatim}}
and adjust scaling to make the data visible:
{\Large
\begin{verbatim}
> (send w :adjust-to-data)
NIL
\end{verbatim}}
We can also add line segments:
{\Large
\begin{verbatim}
> (send w :add-lines (list air temp conc loss))
NIL
\end{verbatim}}
\end{slide}

\begin{slide}{}
You can control whether axes are drawn with the \dcode{:x-axis} and
\dcode{:y-axis} messages:
{\Large
\begin{verbatim}
> (send w :x-axis t)
(T NIL 4)
\end{verbatim}}

The range shown can be accessed and changed:
{\Large
\begin{verbatim}
> (send w :range 0)
(50 80)
> (send w :range 1)
(17 27)
> (send w :range 1 15 30)
(15 30)
\end{verbatim}}
The function \dcode{get-nice-range} helps choosing a range and
the number of ticks:
{\Large
\begin{verbatim}
> (get-nice-range 17 27 4)
(16 28 7)
\end{verbatim}}
To remove the axis:
{\Large
\begin{verbatim}
> (send w :x-axis nil)
(NIL NIL 4)
\end{verbatim}}
\end{slide}

\begin{slide}{}
Initially, the plot shows the first two variables:
\begin{verbatim}
> (send w :current-variables)
(0 1)
\end{verbatim}
This can be changed:
\begin{verbatim}
> (send w :current-variables 2 3)
(2 3)
> (send w :current-variables 0 1)
(0 1)
\end{verbatim}
Plot data can be cleared by several messages:
\begin{verbatim}
(send w :clear-points)
(send w :clear-lines)
(send w :clear)
\end{verbatim}
If the \dcode{:draw} keyword argument is \dcode{nil} the plot
is not redrawn.

The default value is \dcode{t}.
\end{slide}

\begin{slide}{}
\subsection{Scaling and Transformations}
The scale type controls the action of the default
\dcode{:adjust-to-data} method.

The initial scale type is \dcode{nil}:
{\Large
\begin{verbatim}
> (send w :scale-type)
NIL
> (send w :range 0)
(50 80)
> (send w :scaled-range 0)
(50 80)
\end{verbatim}}
Two other scale types are \dcode{variable} and \dcode{fixed}.

For variable scaling:
{\Large
\begin{verbatim}
> (send w :scale-type 'variable)
VARIABLE
> (send w :range 0)
(35 95)
> (send w :scaled-range 0)
(-2 2)
\end{verbatim}}
The \dcode{:scale}, \dcode{:center}, and \dcode{:adjust-to-data}
messages let you build your own scale types.

\end{slide}

\begin{slide}{}
Initially there is no transformation:
{\Large
\begin{verbatim}
> (send w :transformation)
NIL
\end{verbatim}}
If the current variables are 0 and 1, a rotation can be applied to
replace \dcode{air} by \dcode{conc} and \dcode{temp} by \dcode{loss}:
{\Large
\begin{verbatim}
(send w :transformation '#2A((0  0 -1  0)
                             (0  0  0 -1)
                             (1  0  0  0)
                             (0  1  0  0)))
\end{verbatim}}
The transformation can be removed by
{\Large
\begin{verbatim}
(send w :transformation nil)
\end{verbatim}}
\end{slide}

\begin{slide}{}
A transformation can also be applied incrementally:
{\Large
\begin{verbatim}
(let* ((c (cos (/ pi 20)))
       (s (sin (/ pi 20)))
       (m (+ (* c (identity-matrix 4))
             (* s '#2A((0  0 -1  0)
                       (0  0  0 -1)
                       (1  0  0  0)
                       (0  1  0  0))))))
  (dotimes (i 10)
    (send w :apply-transformation m)))
\end{verbatim}}
A simpler message allows rotation within coordinate planes:
{\Large
\begin{verbatim}
(dotimes (i 10)
  (send w :rotate-2 0 2 (/ pi 20) :draw nil)
  (send w :rotate-2 1 3 (/ pi 20)))
\end{verbatim}}
Several messages are available for accessing data values in
raw, scaled, and screen coordinates.

Other messages are available for converting among coordinate systems.

\end{slide}

\begin{slide}{}
\subsection{Mouse Events and Mouse Modes}
The graph prototype organizes mouse interactions into {\em mouse modes}.

Each mouse mode includes
\begin{itemize}
\item a symbol for choosing the mode from a program
\item a title string used in the mode selection dialog
\item a cursor to visually identify the mode
\item mode-specific click and motion messages
\end{itemize}
It should not be necessary to override a graph's \dcode{:do-click} or
\dcode{:do-motion} methods.
\end{slide}

\begin{slide}{}
Initially there are two mouse modes, \dcode{selecting} and
\dcode{brushing}.

We can add a new mouse mode by
{\Large
\begin{verbatim}
(send w :add-mouse-mode 'identify
      :title "Identify"
      :click :do-identify
      :cursor 'finger)
\end{verbatim}}
The \dcode{:do-identify} method can be defined as
{\Large
\begin{verbatim}
(defmeth w :do-identify (x y m1 m2)
  (let* ((cr (send self :click-range))
         (p (first
             (send self :points-in-rect 
                   (- x 2) (- y 2) 4 4))))
    (if p
        (let ((mode (send self :draw-mode))
              (lbl (send self :point-label p)))
          (send self :draw-mode 'xor)
          (send self :draw-string lbl x y)
          (send self :while-button-down
                #'(lambda (x y) nil))
          (send self :draw-string lbl x y)
          (send self :draw-mode mode)))))
\end{verbatim}}
\end{slide}

\begin{slide}{}
The button down action does nothing; it just waits.

An alternative is to allow the label to be dragged, perhaps to make
it easier to read:
{\large
\begin{verbatim}
(defmeth w :do-identify (x y m1 m2)
  (let* ((cr (send self :click-range))
         (p (first
             (send self :points-in-rect 
                   (- x 2) (- y 2) 4 4))))
    (if p
        (let ((mode (send self :draw-mode))
              (lbl (send self :point-label p)))
          (send self :draw-mode 'xor)
          (send self :draw-string lbl x y)
          (send self :while-button-down
                #'(lambda (new-x new-y)
                    (send self :draw-string lbl x y)
                    (setf x new-x)
                    (setf y new-y)
                    (send self :draw-string lbl x y)))
          (send self :draw-string lbl x y)
          (send self :draw-mode mode)))))
\end{verbatim}}
\end{slide}

\begin{slide}{}
\subheading{Standard Mouse Modes and Linking}
The click and motion methods of the two standard modes
use a number of messages.

In \dcode{selecting} mode, a click
\begin{itemize}
\item
Sends \dcode{:unselect-all-points}, unless the extend modifier is used.
\item
Sends \dcode{:adjust-points-in-rect} with click $x$ and $y$
coordinates, width and height returned by \dcode{:click-range},
and the symbol \dcode{selected} as arguments.
\item
While the button is down, a dashed rectangle is stretched from the
click to the mouse.

When the button is released, \dcode{:adjust-points-in-rect} is sent
with the rectangle coordinates and \dcode{selected} as arguments.
\end{itemize}
\end{slide}

\begin{slide}{}
In \dcode{brushing} mode, a click
\begin{itemize}
\item
Sends \dcode{:unselect-all-points} unless the extend modifier is used.
\item
While the mouse is dragged, sends \dcode{:adjust-points-in-rect} with
the brush rectangle and \dcode{selected} as arguments.
\end{itemize}
In \dcode{brushing} mode, moving the mouse
\begin{itemize}
\item
Sends \dcode{:adjust-points-in-rect} with the brush rectangle and
\dcode{hilited} as arguments.
\end{itemize}
\end{slide}

\begin{slide}{}
Points can be in four states:
\begin{itemize}
\item[] \dcode{invisible}
\item[] \dcode{normal}
\item[] \dcode{hilited}
\item[] \dcode{selected}.
\end{itemize}
Linking is based on a {\em loose linking}\/ model:
\begin{itemize}
\item points are related by index number
\item only point states are adjusted
\end{itemize}
The system uses two messages to determine which plots are linked:
\begin{itemize}
\item
\dcode{:links} returns a list of plots linked to the plot (possibly
including the plot itself)
\item \dcode{:linked} determines if the plot is linked, and
turns linking on and off.
\end{itemize}
\end{slide}

\begin{slide}{}
When a point's state is changed in a plot,
\begin{itemize}
\item
Each linked plot (and the plot itself) is sent the
\dcode{:adjust-screen-point} message with the index as argument.
\item
The action taken by the method for \dcode{:adjust-screen-point} may
depend on both current and previous states.
\end{itemize}
Since it is not always feasible to redraw single points,
\begin{itemize}
\item
the \dcode{:needs-adjusting} method can be used to check or set a flag
\item
the \dcode{:adjusting-screen} method can redraw the entire plot if the
flag is set
\end{itemize}
The easiest, though not necessarily the most efficient, way to augment
standard mouse modes is to define a new \dcode{:adjust-screen} method
\end{slide}

\begin{slide}{}
Some useful messages:
\begin{itemize}
\item[] For points specified by index:
\begin{itemize}
\item[] \dcode{:point-showing}
\item[] \dcode{:point-hilited}
\item[] \dcode{:point-selected}
\end{itemize}
For sets of indices:
\begin{itemize}
\item[] \dcode{:selection} or \dcode{:points-selected}
\item[] \dcode{:points-hilited}
\item[] \dcode{:points-showing}
\end{itemize}
Other operations:
\begin{itemize}
\item[] \dcode{:erase-selection}
\item[] \dcode{:show-all-points}
\item[] \dcode{:focus-on-selection}
\item[] \dcode{:adjust-screen}
\end{itemize}
Some useful predicates:
\begin{itemize}
\item[] \dcode{:any-points-selected-p}
\item[] \dcode{:all-points-showing-p} 
\end{itemize}
\end{itemize}
\end{slide}

\begin{slide}{}
\heading{Window Layout and Redrawing}
The \dcode{:resize} method maintains a margin and a content rectangle
\begin{center}
\mbox{\psfig{figure=layout.ps,height=3.4in}}
\end{center}

\begin{itemize}
\item
The plot is surrounded by a margin, used by plot controls.
\item
The content rectangle can use a fixed or a variable aspect ratio.
\item
The size of the content depends on the aspect ratio and the axes.
\item
The plot can be covered by overlays, resized with \dcode{:resize-overlays}.
\end{itemize}
\end{slide}

\begin{slide}{}
The aspect type used can be changed by
\begin{verbatim}
(send w :fixed-aspect t)
\end{verbatim}
The \dcode{:redraw} method sends three messages:
\begin{itemize}
\item
\dcode{:redraw-background} -- erases the canvas and draws the axes
\item 
\dcode{:redraw-overlays} -- sends each overlay the \dcode{:redraw} message
\item
\dcode{:redraw-content} -- redraws points, lines, etc.
\end{itemize}
Many methods, like \dcode{:rotate-2}, also send \dcode{:redraw-content}.

\end{slide}

\begin{slide}{}
Plot overlays are useful for holding controls.
\begin{itemize}
\item Overlays inherit from \dcode{graph-overlay-proto}.
\item Overlays are like transparent sheets of plastic.
\item Overlays are drawn from the bottom up.
\item Overlays can intercept mouse clicks.
\item Clicks are processed from the top down:
\begin{itemize}
\item
Each overlay is sent the \dcode{:do-click} message until one returns a
non-\dcode{nil} result.
\item
Only if no overlay accepts a click is the click passed to the current
mouse mode.
\end{itemize}
\end{itemize}
The controls of a rotating plot are implemented as an overlay.

Other examples of overlays are in \dcode{plotcontrols.lsp} in the
\macbold{Examples} folder.
\end{slide}

\begin{slide}{}
\subsection{Menus and Menu Items}
To help construct standard menus
\begin{itemize}
\item \dcode{:menu-title} returns the title to use
\item \dcode{:menu-template} returns a list of items or symbols
\item \dcode{:new-menu} constructs and installs the new menu
\end{itemize}
The \dcode{:isnew} method sends the plot the \dcode{:new-menu}
message when it is created.
\end{slide}

\begin{slide}{}
Standard items can be specified as symbols in the template:
\begin{itemize}
\item \dcode{color}
\item \dcode{dash}
\item \dcode{focus-on-selection}
\item \dcode{link}
\item \dcode{mouse}
\item \dcode{options}
\item \dcode{redraw}
\item \dcode{erase-selection}
\item \dcode{rescale}
\item \dcode{save-image}
\item \dcode{selection}
\item \dcode{show-all}
\item \dcode{showing-labels}
\item \dcode{symbol}
\end{itemize}
\end{slide}

\begin{slide}{}
\section{Standard Statistical Graphs}
Each of the standard plot prototypes needs only a few new
methods.

The main additional or changes methods are:
\begin{description}\normalsize
\item[]
\dcode{scatterplot-proto}:

Overrides: \dcode{:add-points}, \dcode{:add-lines},
\dcode{:adjust-to-data}.

New messages: \dcode{:add-boxplot}, \dcode{:add-function-contours},
\dcode{:add-surface-contour}, \dcode{:add-surface-contours}.
\item[]
\dcode{scatmat-proto}:

Overrides: \dcode{:add-lines},
\dcode{:add-points}, \dcode{:adjust-points-in-rect},
\dcode{:adjust-screen-point}, \dcode{:do-click}, \dcode{:do-motion},
\dcode{:redraw-background}, \dcode{:redraw-content}, \dcode{:resize}.
\item[]
\dcode{spin-proto}:

Overrides: \dcode{:adjust-to-data},
\dcode{:current-variables}, \dcode{:do-idle}, \dcode{:isnew},
\dcode{:resize}, \dcode{:redraw-content}.

New methods: \dcode{:abcplane}, \dcode{:add-function},
\dcode{:add-surface}, \dcode{:angle}, \dcode{:content-variables},
\dcode{:depth-cuing}, \dcode{:draw-axes}, \dcode{:rotate},
\dcode{:rotation-type}, \dcode{:showing-axes} 
\item[]
\dcode{histogram-proto}:

Overrides: \dcode{:add-points},
\dcode{:adjust-points-in-rect}, \dcode{:adjust-screen},
\dcode{:adjust-screen-point}, \dcode{:adjust-to-data},
\dcode{:clear-points}, \dcode{:drag-point}, \dcode{:isnew},
\dcode{:redraw-content}, \dcode{:resize}.

New methods: \dcode{:num-bins}, \dcode{:bin-counts}.
\item[]
\dcode{name-list-proto}:

Overrides: \dcode{:add-points},
\dcode{:adjust-points-in-rect}, \dcode{:adjust-screen-point},
\dcode{:redraw-background}, \dcode{:redraw-content}.
\end{description}
\end{slide}

\begin{slide}{}
\chapter{Some Dynamic Graphics Examples}
\end{slide}

\begin{slide}{}
\section{Some Background}

Dynamic graphs usually have some of the following characteristics:
\begin{itemize}
\item high level of interaction
\item motion
\item more than two dimensional data
\item require high performance graphics display hardware
\end{itemize}
Some of the basic techniques that appear to be useful are
\begin{itemize}
\item animation
\begin{itemize}
\item controlled interactively
\item controlled by a program
\end{itemize}
\item direct interaction and manipulation
\begin{itemize}
\item linking and brushing
\end{itemize}
\item rotation
\end{itemize}
Many ideas are described in the papers in the book edited by
Cleveland and McGill (1988).
\end{slide}

\begin{slide}{}
Research on the effective use of dynamic graphics is just beginning.

Some of the questions to keep in mind are
\begin{itemize}
\item How should dynamic graphs be used?
\item What should you plot?
\item How do you interpret dynamic images?
\item What techniques are useful, and for which problems?
\end{itemize}
Even in simple two-dimensional graphics there are difficult open issues.

Answering these questions will take time.

We can make a start by learning how to try out some of the more
promising dynamic graphical ideas, and variations on these ideas. 
\end{slide}


\begin{slide}{}
\section{Some Examples of Animation}
The basic idea in animation is to display a sequence of views
rapidly enough and smoothly enough to
\begin{itemize}
\item create an illusion of motion
\item allow the change in related elements to be tracked visually
\end{itemize}
The particular image is determined by the values of one or more
parameters.
\begin{itemize}
\item
For a small number of parameters, (1 or 2), the values can
be controlled directly, e.g. using scroll bars.
\item
For more parameters, automated methods of exploring the parameter space
are useful.
\end{itemize}
\end{slide}

\begin{slide}{}
\subsection{Dynamic Box-Cox Plots}
Fowlkes's implementation of dynamic transformations in the late 1960's
was one of the earliest examples of dynamic statistical graphics.

Start by defining a function to compute the transformation and scale
the data to the unit interval:
{\Large
\begin{verbatim}
(defun bc (x p)
  (let* ((bcx (if (< (abs p) .0001)
                  (log x)
                  (/ (^ x p) p)))
         (min (min bcx))
         (max (max bcx)))
    (/ (- bcx min) (- max min))))
\end{verbatim}}
Next, construct an ordered sample and a set of approximate expected
normal order statistics:
{\Large
\begin{verbatim}
(setf x (sort-data precipitation))
(setf nq (normal-quant (/ (iseq 1 30) 31)))
\end{verbatim}}
\end{slide}

\begin{slide}{}
We can construct an initial plot as
{\Large
\begin{verbatim}
(setf p (plot-points nq (bc x 1)))
\end{verbatim}}
To change the plot content to reflect a square-root transformation, we
can use the \dcode{:clear} and \dcode{:add-points} messages:
{\Large
\begin{verbatim}
(send p :clear)
(send p :add-points nq (bc x 0.5))
\end{verbatim}}
To achieve a smooth transition, you can give the \dcode{:clear} message
a keyword argument \dcode{:draw} with value \dcode{nil}:
{\Large
\begin{verbatim}
(send p :clear :draw nil)
(send p :add-points nq (bc x 0.0))
\end{verbatim}}
These steps can be built into a \dcode{:change-power} method:
{\Large
\begin{verbatim}
(defmeth p :change-power (pow)
  (send self :clear :draw nil)
  (send self :add-points nq (bc x pow)))
\end{verbatim}}
\end{slide}

\begin{slide}{}
To view a range of powers, you can use a \dcode{dolist} loop
{\Large
\begin{verbatim}
(dolist (pow (rseq -1 2 20))
  (send p :change-power pow))
\end{verbatim}}
or you can use one of two {\em modeless}\/ dialogs to get finer control
over the animation:
{\Large
\begin{verbatim}
(sequence-slider-dialog (rseq -1 2 21) :action
  #'(lambda (pow) (send p :change-power pow)))
\end{verbatim}}
or
{\Large
\begin{verbatim}
(interval-slider-dialog '(-1 2)
  :points 20
  :action
  #'(lambda (pow) (send p :change-power pow)))
\end{verbatim}}
\end{slide}

\begin{slide}{}
To avoid losing color, symbol or state information in the points, we
can change the coordinates of the points in the plot instead of
replacing the points.

To preserve the original data ordering, we need to base the $x$ values
on ranks instead of ordering the data
{\Large
\begin{verbatim}
(let ((ranks (rank precipitation)))
  (setf nq (normal-quant (/ (+ ranks 1) 31))))
\end{verbatim}}
and use these to construct our plot:
{\Large
\begin{verbatim}
(setf p (plot-points nq (bc precipitation 1)))
\end{verbatim}}
\end{slide}

\begin{slide}{}
To change the coordinates, we need a list of the indices of all
points.

To avoid generating this list every time, we can store it in a slot:
{\Large
\begin{verbatim}
(send p :add-slot 'indices (iseq 30))
(defmeth p :indices () (slot-value 'indices))
\end{verbatim}}
The method for changing the power now becomes
{\Large
\begin{verbatim}
(defmeth p :change-power (pow)
  (send self :point-coordinate
        1
        (send self :indices)
        (bc precipitation pow))
  (send self  :redraw-content))
\end{verbatim}}
and the slider is again set up by
{\Large
\begin{verbatim}
(interval-slider-dialog '(-1 2) 
  :points 20
  :action
  #'(lambda (pow) (send p :change-power pow)))
\end{verbatim}}
\end{slide}

\begin{slide}{}
The same idea can be used with a regression model.

To transform the dependent variable and scale the residuals,
define the functions
{\Large
\begin{verbatim}
(defun bcr (x p)
  (if (< (abs p) 0.0001)
      (log x)
      (/ (^ x p) p)))
\end{verbatim}}
and
{\Large
\begin{verbatim}
(defun sc (x)
  (let ((min (min x))
        (max (max x)))
    (/ (- x min) (- max min))))
\end{verbatim}}
\end{slide}

\begin{slide}{}
Let's use the stack loss data as an illustration.

A set of approximate expected normal order statistics 
and a set of indices are set up as
{\Large
\begin{verbatim}
(setf nqr
      (let ((n (length loss)))
        (normal-quant (/ (iseq 1 n) (+ 1 n)))))

(setf idx (iseq (length loss)))
\end{verbatim}}
The regression model and initial plot are set up by
{\Large
\begin{verbatim}
(setf m (regression-model (list air conc temp)
                          loss))

(setf p (let ((r (send m :residuals)))
          (plot-points (select nqr (rank r))
                       (sc r))))
\end{verbatim}}
\end{slide}

\begin{slide}{}
The new \dcode{:change-power} method transforms the dependent variable
in the regression model and then changes the plot using the new residuals:
{\Large
\begin{verbatim}
(defmeth p :change-power (pow)
  (send m :y (bcr loss pow))
  (let* ((r (send m :residuals))
         (r-nqr (select nqr (rank r))))
    (send self :point-coordinate 0 idx r-nqr)
    (send self :point-coordinate 1 idx (sc r))
    (send self :redraw-content)))
\end{verbatim}}
Again, a slider for controlling the animation is set up by
{\Large
\begin{verbatim}
(interval-slider-dialog '(-1 2)
  :action
  #'(lambda (pow) (send p :change-power pow)))
\end{verbatim}}
Since the point indices in the plot correspond to the indices in the
regression, it is possible to track the positions of the residuals
for groups of observations as the power is changed.

\end{slide}

\begin{slide}{}
\subsection{Density Estimation}
Choosing a bandwidth for a kernel density estimator is a difficult
problem.

It may help to have an animation that shows the effect of changes in
the bandwidth on the estimate.

An initial plot can be set up as
{\Large
\begin{verbatim}
(setf w (plot-lines (kernel-dens precipitation
                                 :width 1)))
\end{verbatim}}
The initial bandwidth can be stored in a slot
{\Large
\begin{verbatim}
(send w :add-slot 'kernel-width 1)
\end{verbatim}}
and the accessor method for the slot can adjust the plot when the
width is changed:
{\Large
\begin{verbatim}
(defmeth w :kernel-width (&optional width)
  (when width
        (setf (slot-value 'kernel-width) width)
        (send self :set-lines))
  (slot-value 'kernel-width))
\end{verbatim}}
\end{slide}

\begin{slide}{}
It may also be useful to store the data in a slot and provide an accessor
method:
{\Large
\begin{verbatim}
(send w :add-slot 'kernel-data precipitation)

(defmeth w :kernel-data ()
  (slot-value 'kernel-data))
\end{verbatim}}

The method for changing the plot is
{\Large
\begin{verbatim}
(defmeth w :set-lines ()
  (let ((width (send self :kernel-width))
        (data (send self :kernel-data)))
    (send self :clear-lines :draw nil)
    (send self :add-lines
          (kernel-dens data :width width))))
\end{verbatim}}
and a slider for controlling the width is given by
{\Large
\begin{verbatim}
(interval-slider-dialog '(.25 1.5)
  :action 
  #'(lambda (s) (send w :kernel-width s)))
\end{verbatim}}
\end{slide}

\begin{slide}{}
\subsection{Assessing Variability by Animation}
One way to get a feeling for the uncertainty in a density estimate
is to re-sample the data with replacement, i.e. to look at bootstrap
samples, and see how the estimate changes.

We can do this by defining a \dcode{:do-idle} method:
{\Large
\begin{verbatim}
(defmeth w :do-idle ()
  (setf (slot-value 'kernel-data)
        (sample precipitation 30 :replace t))
  (send self :set-lines))
\end{verbatim}}
We can turn the animation on and off with the \dcode{:idle-on}
message, but it is better to use a menu item.

\end{slide}

\begin{slide}{}
We have already used such an item once; since we may need one again,
it is probably worth constructing a prototype:
{\Large
\begin{verbatim}
(defproto run-item-proto
          '(graph) () menu-item-proto)

(send run-item-proto :key #\R)

(defmeth run-item-proto :isnew (graph)
  (call-next-method "Run")
  (setf (slot-value 'graph) graph))

(defmeth run-item-proto :update ()
  (send self :mark
        (send (slot-value 'graph) :idle-on)))

(defmeth run-item-proto :do-action ()
  (let ((graph (slot-value 'graph)))
    (send graph :idle-on
          (not (send graph :idle-on)))))
\end{verbatim}}
We can now add a run item to the menu by
{\Large
\begin{verbatim} 
(send (send w :menu) :append-items
      (send run-item-proto :new w))
\end{verbatim}}
\end{slide}

\begin{slide}{}
A problem with this version is that the $x$ values at which the density
estimate is evaluated change with the sample.

To avoid this, we can add another slot and accessor method,
{\Large
\begin{verbatim}
(send w :add-slot
      'xvals 
      (rseq (min precipitation)
            (max precipitation)
            30))

(defmeth w :xvals () (slot-value 'xvals))
\end{verbatim}}
and change the \dcode{:set-lines} method to use these values:
{\Large
\begin{verbatim}
(defmeth w :set-lines ()
  (let ((width (send self :kernel-width))
        (xvals (send self :xvals)))
    (send self :clear-lines :draw nil)
    (send self :add-lines
          (kernel-dens (send self :kernel-data)
                       :width width
                       :xvals xvals))))
\end{verbatim}}
\end{slide}

\begin{slide}{}
Another possible problem is that the curve may jump around too
much to make it easy to watch.

We can reduce this jumping by changing one observation at a time
instead of changing all at once:
{\Large
\begin{verbatim}
(send w :slot-value
      'kernel-data (copy-list precipitation))

(defmeth w :do-idle ()
  (let ((d (slot-value 'kernel-data))
        (i (random 30))
        (j (random 30)))
    (setf (select d i)
          (select precipitation j))
    (send self :set-lines)))
\end{verbatim}}
\end{slide}

\begin{slide}{}
A similar idea may be useful for estimated survival curves.

Suppose \dcode{s} is a survival distribution object.

A plot of the Fleming-Harrington estimator is
{\Large
\begin{verbatim}
(setf p
      (let ((d (send s :num-deaths))
            (r (send s :num-at-risk))
            (udt (send s :death-times)))
        (plot-lines
         (make-steps
          udt
          (exp (- (cumsum (/ d r))))))))
\end{verbatim}}
\end{slide}

\begin{slide}{}
We can put a copy of the estimated hazard increments in a slot:
{\Large
\begin{verbatim}
(send p :add-slot 
     'del-hazard
     (copy-list (/ (send s :num-deaths)
                   (send s :num-at-risk))))

(defmeth p :del-hazard ()
  (slot-value 'del-hazard))
\end{verbatim}}
For certain prior distributions, a Bayesian analysis produces
approximately a gamma posterior distribution for the hazard increments.
\end{slide}

\begin{slide}{}
We can use this distribution to animate the plot:
{\Large
\begin{verbatim}
(defmeth p :do-idle ()
  (let* ((udt (send s :death-times))
         (d (send s :num-deaths))
         (r (send s :num-at-risk))
         (dh (send self :del-hazard))
         (n (length dh))
         (i (random n))
         (di (select d i))
         (ri (select r i)))
    (setf (select dh i)
          (/ (first (gamma-rand 1 di)) ri))
    (send self :clear-lines :draw nil)
    (send self :add-lines
          (make-steps udt
                      (exp (- (cumsum dh)))))))
\end{verbatim}}
A run item lets us control the animation:
{\Large
\begin{verbatim}
(send (send p :menu) :append-items
      (send run-item-proto :new p))
\end{verbatim}}
Another way to think of this animation is as a simulation
from an approximation to the sampling distribution.
\end{slide}

\begin{slide}{}
\section{Modifying Mouse Responses}

\subsection{Modifying Selection and Brushing}
We have already seen several examples of using the standard selecting
and brushing modes.

The series of messages used to implement these modes lets us modify
what happens when the brush is moved or points are selected.

The simplest way to build on the standard modes is to modify the
\dcode{:adjust-screen} method.

This message is sent each time the set of selected or highlighted
points is changed with the mouse.

As an example, if we are using selection or brushing to examine a
bivariate conditional distribution, it may be useful to augment the
bivariate plot with a smoother.
\end{slide}

\begin{slide}{}
Suppose we have a histogram of \dcode{hardness} and a scatterplot of
\dcode{abrasion-loss} against \dcode{tensile-strength} for the
abrasion loss data {\Large
\begin{verbatim}
(setf h (histogram hardness))

(setf p (plot-points tensile-strength
                     abrasion-loss))
\end{verbatim}}
The method
{\Large
\begin{verbatim}
(defmeth p :adjust-screen ()
  (call-next-method)
  (let ((i (union
            (send self :points-selected)
            (send self :points-hilited))))
    (send self :clear-lines :draw nil)
    (if (< 1 (length i))
        (let ((x (select tensile-strength i))
              (y (select abrasion-loss i)))
          (send self :add-lines
                (kernel-smooth x y)))
        (send self :redraw-content))))
\end{verbatim}}
adds a kernel smooth to the plot if at least two points are selected
or hilited.
\end{slide}

\begin{slide}{}
To have control over the smoother bandwidth, we can add a slot and
accessor method, modify the \dcode{:adjust-screen} method, and add
a slider:
{\large
\begin{verbatim}
(send p :add-slot 'kernel-width 50)

(defmeth p :kernel-width (&optional width)
  (when width
        (setf (slot-value 'kernel-width) width)
        (send self :adjust-screen))
  (slot-value 'kernel-width))

(defmeth p :adjust-screen ()
  (call-next-method)
  (let ((i (union
            (send self :points-selected)
            (send self :points-hilited))))
    (send self :clear-lines :draw nil)
    (if (< 1 (length i))
        (let ((x (select tensile-strength i))
              (y (select abrasion-loss i))
              (w (send self :kernel-width)))
          (send self :add-lines
                (kernel-smooth x y :width w)))
        (send self :redraw-content))))

(interval-slider-dialog '(20 100)
  :action #'(lambda (w) (send p :kernel-width w)))
\end{verbatim}}
\end{slide}

\begin{slide}{}
A similar idea is useful for multivariate response data (e.g
repeated measures or time series).

Suppose \dcode{p} is a plot of covariates and \dcode{q} is a
plot of $m$ responses, contained in the list \dcode{resp}, against
their indices.

Then the method
{\Large
\begin{verbatim}
(defmeth p :adjust-screen ()
  (call-next-method)
  (let ((i (union
            (send self :points-selected)
            (send self :points-hilited))))
      (send q :clear-lines :draw nil)
      (if i
          (flet ((ms (x) (mean (select x i))))
            (let ((y (mapcar #'ms resp))
                  (j (iseq 1 (length resp))))
              (send q :add-lines j y)))
          (send q :redraw-content))))
\end{verbatim}}
shows in \dcode{q} the average response profile for the highlighted
and selected observations.
\end{slide}

\begin{slide}{}
Another example was recently posted to \dcode{statlib} by Neely
Atkinson, M. D. Anderson Cancer Center, Houston.

Here is a simplified version:

Suppose we have some censored survival data and a number of covariates.

To examine the relationship between the covariates and the times
\begin{itemize}
\item
put the covariates in a plot, for example a scatterplot matrix
\item
in a second plot show the Kaplan-Meier estimate of the distribution of
the survival times for points selected in the covariate plot.
\end{itemize}

To speed up performance, first define a faster Kaplan-Meier estimator
that assumes the data are properly ordered:
{\Large
\begin{verbatim}
(defun km (x s)
  (let ((n (length x)))
    (accumulate #'* (- 1 (/ s (iseq n 1))))))
\end{verbatim}}
\end{slide}{}

\begin{slide}{}
Next, set up the covariate plot and a plot of the Kaplan-Meier
estimator for all survival times: {\Large
\begin{verbatim}
(setf p (scatterplot-matrix (list x y z)))
(setf q (plot-lines
         (make-steps times (km times status))))
\end{verbatim}}
Finally, write an \dcode{:adjust-screen} method for \dcode{p} that
changes the Kaplan-Meier estimator in \dcode{q} to use only the
currently selected and highlighted points in \dcode{p}:
{\Large
\begin{verbatim}
(defmeth p :adjust-screen ()
  (call-next-method)
  (let ((s (union
            (send self :points-selected)
            (send self :points-hilited))))
    (send q :clear-lines :draw nil)
    (if s
        (let* ((s (sort-data s))
               (tm (select times s))
               (st (select status s)))
          (send q :add-lines
                (make-steps tm (km tm st))))
        (send q :redraw-content))))
\end{verbatim}}
\end{slide}

\begin{slide}{}
\subsection{Using New Mouse Modes}
We have already seen a few examples of useful new mouse modes.

As another example, let's make a plot for illustrating the
sensitivity of least squares to points with high leverage.

Start with a plot of some simulated data:
{\Large
\begin{verbatim}
(setf x (append (iseq 1 18) (list 30 40)))
(setf y (+ x (* 2 (normal-rand 20))))
(setf p (plot-points x y))
\end{verbatim}}
Next, add a new mouse mode:
{\Large
\begin{verbatim}
(send p :add-mouse-mode 'point-moving
      :title "Point Moving"
      :cursor 'finger
      :click :do-point-moving)
\end{verbatim}}
\end{slide}

\begin{slide}{}
The point moving method can be broken down into several pieces:
{\Large
\begin{verbatim}
(defmeth p :do-point-moving (x y a b)
  (let ((p (send self :drag-point
                 x y :draw nil)))
    (if p (send self :set-regression-line))))

(defmeth p :set-regression-line ()
  (let ((coefs (send self :calculate-coefs)))
    (send self :clear-lines :draw nil)
    (send self :abline
          (select coefs 0)
          (select coefs 1))))

(defmeth p :calculate-coefs ()
  (let* ((i (iseq (send self :num-points)))
         (x (send self :point-coordinate 0 i))
         (y (send self :point-coordinate 1 i))
         (m (regression-model x y :print nil)))
    (send m :coef-estimates)))
\end{verbatim}}
\end{slide}

\begin{slide}{}
If we want to change the fitting method, we only need to change the
\dcode{:calculate-coefs} method.

Finally, add a regression line and put the plot into point
moving mode
{\Large
\begin{verbatim}
(send p :set-regression-line)

(send p :mouse-mode 'point-moving)
\end{verbatim}}

Several enhancements and variations are possible:
\begin{itemize}
\item
The plot can be modified to use only visible points
for fitting the line.
\item
Several fitting methods can be provided, with a dialog to choose
among them.
\item
The same idea can be used to study the effect of outliers on smoothers.
\end{itemize}
\end{slide}

\begin{slide}{}
As another example, we can set up a plot for obtaining graphical function
input.

Start by setting up the plot, turning off the $y$ axis, and adding
a new mouse mode:
{\Large
\begin{verbatim}
(setf p (plot-lines (rseq 0 1 50)
                    (repeat 0 50)))
(send p :y-axis nil)

(send p :add-mouse-mode 'drawing
      :title "Drawing"
      :cursor 'finger
      :click :mouse-drawing)
\end{verbatim}}
We can put the plot in our new mode with
{\Large
\begin{verbatim}
(send p :mouse-mode 'drawing)
\end{verbatim}}
\end{slide}

\begin{slide}{}
The drawing method is defined as
{\large
\begin{verbatim}
(defmeth p :mouse-drawing (x y m1 m2)
  (let* ((n (send self :num-lines))
         (rxy (send self :canvas-to-real x y))
         (rx (first rxy))
         (ry (second rxy))
         (old-i (max 0 (min (- n 1) (floor (* n rx)))))
         (old-y ry))
    (flet ((adjust (x y) 
             (let* ((rxy (send self :canvas-to-real x y))
                    (rx (first rxy))
                    (ry (second rxy))
                    (new-i (max 0
                                (min (- n 1) 
                                     (floor (* n rx)))))
                    (y ry))
               (dolist (i (iseq old-i new-i))
                 (let ((p (if (= old-i new-i) 
                              1
                              (abs
                               (/ (- i old-i) 
                                  (- new-i old-i))))))
                   (send self :linestart-coordinate 1 i 
                         (+ (* p y) (* (- 1 p) old-y)))))       
               (send self :redraw-content)
               (setf old-i new-i old-y y))))
          (adjust x y)
          (send self :while-button-down #'adjust))))
\end{verbatim}}
\end{slide}

\begin{slide}{}
Some notes:
\begin{itemize}
\item
\dcode{:canvas-to-real} converts the click coordinates to data
coordinates.
\item
The \dcode{dolist} loop is needed to smooth out the curve if the mouse
is moved rapidly.
\item
If the plot is to be normalized as a density, it can be adjusted before
the \dcode{:redraw-content} message is sent.
\end{itemize}
A method for retrieving the lines is given by
{\Large
\begin{verbatim}
(defmeth p :lines ()
  (let ((i (iseq 50)))
    (list
     (send self :linestart-coordinate 0 i)
     (send self :linestart-coordinate 1 i))))
\end{verbatim}}
\end{slide}

\begin{slide}{}
\section{Some Issues in Rotation}
\subsection{Statistical Issues}
Rotation is a method for viewing a three dimensional set of data.

The objective of rotation is to create an illusion of 3D structure on
a 2D screen.

Factors that can be used to produce or enhance such an illusion are
\begin{itemize}
\item stereo imaging
\item perspective
\item lighting
\item motion parallax
\end{itemize}
Motion parallax is the principle driving rotation.
\end{slide}

\begin{slide}{}
Scaling can have a significant effect on the shape of a point cloud.

Possible scaling strategies include
\begin{itemize}
\item fixed scaling --  comparable scales
\item variable scaling -- non-comparable scales
\item combinations
\end{itemize}

Several methods are available for enhancing the 3D illusion
produced by rotation
\begin{itemize}
\item depth cuing
\item framed box
\item Rocking
\item slicing
\end{itemize}
\end{slide}

\begin{slide}{}
The nature of the rotation controls may also affect the illusion.

Some possible control strategies are
\begin{itemize}
\item rotation around screen axes: horizontal (Pitch), vertical (Yaw), out-of-screen
(Roll)
\item rotation around data axes: X, Y, Z
\item direct manipulation (like a globe)
\end{itemize}

Rotation is still a fairly new technique; we are still learning
how to use it effectively.

Some questions to keep in mind as you use rotation to explore
your data sets:
\begin{itemize}
\item What types of phenomena are easy to detect?
\item What is hard to detect?
\item What enhancements aid in detection?
\end{itemize}
It may help to try rotation on some artificial structures and see if
you can identify these structures.
\end{slide}

\begin{slide}{}
\subheading{Implementation Issues}
The default scaling used by the rotating plot is \dcode{variable}.

You can change the scale type with the \macbold{Options} dialog or
the \dcode{:scale-type} message.

You can implement non-standard scaling strategies by overriding the
default \dcode{:adjust-to-data} method.

The standard rotating plot has controls for screen axis rotation.

You can use the \dcode{:transformation} and \dcode{:apply-transformation}
methods to implement alternate control methods.

Plot overlays can be used to hold controls for these strategies.
\end{slide}

\begin{slide}{}
\subsection{Some Examples}
By rocking the plot back and forth we can get the 3D illusion
while keeping the view close to fixed.

A method for rocking the plot can be defines as
{\Large
\begin{verbatim}
(defmeth spin-proto :rock-plot
         (&optional (a 0.15))
  (let* ((angle (send self :angle))
         (k (round (/ a angle))))
    (dotimes (i k)
      (send self :rotate-2 0 2 angle))
    (dotimes (i (* 2 k))
      (send self :rotate-2 0 2 (- angle)))
    (dotimes (i k)
      (send self :rotate-2 0 2 angle))))
\end{verbatim}}
\end{slide}

\begin{slide}{}
A method for rotating around a specified data axis is
{\Large
\begin{verbatim}
(defmeth spin-proto :data-rotate
         (axis &optional (angle pi))
  (let* ((alpha (send self :angle))
         (cols (column-list
                (send self :transformation)))
         (m (case axis
              (x (make-rotation (select cols 1) 
                                (select cols 2)
                                alpha))
              (y (make-rotation (select cols 0)
                                (select cols 2)
                                alpha))
              (z (make-rotation (select cols 0)
                                (select cols 1)
                                alpha)))))
    (dotimes (i (floor (/ angle alpha)))
      (send self :apply-transformation m))))
\end{verbatim}}
\end{slide}

\begin{slide}{}
To allow rotation to be done by direct manipulation, we can add
a new mouse mode
{\Large
\begin{verbatim}
(send spin-proto :add-mouse-mode 'hand-rotate
      :title "Hand Rotate"
      :cursor 'hand
      :click :do-hand-rotate)
\end{verbatim}}
and define the \dcode{:do-hand-rotate} method as
{\normalsize
\begin{verbatim}
(defmeth spin-proto :do-hand-rotate (x y m1 m2)
  (flet ((calcsphere (x y)
           (let* ((norm-2 (+ (* x x) (* y y)))
                  (rad-2 (^ 1.7 2))
                  (z (if (< norm-2 rad-2) (sqrt (- rad-2 norm-2)) 0)))
             (if (< norm-2 rad-2) 
                 (list x y z)
                 (let ((r (sqrt (max norm-2 rad-2))))
                   (list (/ x r) (/ y r) (/ z r)))))))
    (let* ((oldp (apply #'calcsphere 
                        (send self :canvas-to-scaled x y)))
           (p oldp)
           (vars (send self :content-variables))
           (trans (identity-matrix (send self :num-variables))))
      (send self :idle-on nil)
      (send self :while-button-down 
            #'(lambda (x y) 
                (setf oldp p)
                (setf p (apply #'calcsphere 
                               (send self :canvas-to-scaled x y)))
                (setf (select trans vars vars) (make-rotation oldp p))
                (when m1 
                      (send self :slot-value 'rotation-type trans)
                      (send self :idle-on t))
                (send self :apply-transformation trans))))))
\end{verbatim}}
\end{slide}

\begin{slide}{}
\section{4D and Beyond}
\subsection{Introduction and Background}
As with two and three dimensional plots, the objectives of
higher dimensional graphics are, among others, to
\begin{itemize}
\item detect groupings or clusters
\item detect lower dimensional structure
\item detect other patterns
\end{itemize}
Higher dimensions introduce new difficulties:
\begin{itemize}
\item sparseness
\item loss of intuition
\end{itemize}
\end{slide}

\begin{slide}{}
A rough ranking of types of structures that can be detected might
look like this:
\begin{itemize}
\item points (zero-dimensional structures) in 1D and 2D plots
\item curves (one-dimensional structures) in 2D and 3D plots
\item surfaces (two-dimensional structures) in 3D plots
\end{itemize}
Possible structures in four dimensions include
\begin{itemize}
\item separate point clusters
\item curves
\item 2D surfaces
\item 3D surfaces
\end{itemize}
\end{slide}

\begin{slide}{}
Useful general techniques include
\begin{itemize}
\item glyph augmentation
\begin{itemize}
\item colors
\item symbol types
\item symbol sizes
\end{itemize}
\item dimension reduction
\begin{itemize}
\item projection
\item slicing, masking, conditioning
\end{itemize}
\end{itemize}
Some issues to keep in mind:
\begin{itemize}
\item using symbols or colors loses ordering
\item projections to 2D will only become 1D if part of the structure
is linear
\item masking/conditioning requires large data sets
\end{itemize}
\end{slide}

\begin{slide}{}
Some specific implementations:
\begin{itemize}
\item scatterplot matrix
\item linked 2D plots
\item 3D plot linked with a histogram
\item plot interpolation
\item grand tours and variants
\item parallel coordinates
\end{itemize}
\end{slide}

\begin{slide}{}
\subheading{Plot Interpolation}
Plot interpolation provides a way of examining the relation among 4
variables.

The idea is to take two scatterplots and rotate from one to the other.

(Interpolate using trigonometric interpolation.)

Watching the interpolation allows you to 
\begin{itemize}
\item track groups of points from one plot to another
\item identify clusters based on both location and velocity
\end{itemize}
\end{slide}

\begin{slide}{}
A simple function to set up an interpolation plot for a data set with
four variables:
{\Large
\begin{verbatim}
(defun interp-plot (data &rest args)
  ;; Should check here that data is 4D
  (let ((w (apply #'plot-points data
                  :scale 'variable args))
        (m (matrix '(4 4) (repeat 0 16))))
    (flet ((interpolate (p)
             (let* ((a (* (/ pi 2) p))
                    (s (sin a))
                    (c (cos a)))
               (setf (select m 0 0) c)
               (setf (select m 0 2) s)
               (setf (select m 1 1) c)
               (setf (select m 1 3) s)
               (send w :transformation m))))
      (let ((s (interval-slider-dialog
                '(0 1) 
                :points 25 
                :action #'interpolate)))
        (send w :add-subordinate s)))
    w))
\end{verbatim}}
\end{slide}

\begin{slide}{}
Some points:
\begin{itemize}
\item
The definition allows keywords to be passed on to \dcode{plot-points}.
\item
The plot \dcode{w} and the matrix \dcode{m} are locked into the
environment of \dcode{interpolate}.
\item
The slider is registered as a subordinate.
\end{itemize}
Some examples to try:
\begin{itemize}
\item Stack loss data
\item Places rated data
\item Iris data
\end{itemize}
\end{slide}

\begin{slide}{}
\subsection{The Grand Tour}
The Grand Tour provides a way of examining ``all'' one- or two-dimensional
projections of a higher-dimensional data set.

The projections are moved smoothly in a way that eventually brings them
near every possible projection.

One way to construct a Grand Tour is to repeat the following steps:
\begin{itemize}
\item Choose two directions at random to define a rotation plane.
\item Rotate in this plane by a specified angle a random number of times.
\end{itemize}
\end{slide}

\begin{slide}{}
A simple Grand Tour function:
{\Large
\begin{verbatim}
(defun tour-plot (&rest args)
  (let ((p (apply #'spin-plot args)))
    (send p :add-slot 'tour-count -1)
    (send p :add-slot 'tour-trans nil)
    (defmeth p :tour-step ()
      (when (< (slot-value 'tour-count) 0)
        (let ((vars (send self :num-variables))
              (angle (send self :angle)))
          (setf (slot-value 'tour-count)
                (random 20))
          (setf (slot-value 'tour-trans) 
                (make-rotation
                 (sphere-rand vars) 
                 (sphere-rand vars)
                 angle))))
      (send self :apply-transformation
            (slot-value 'tour-trans))
      (setf (slot-value 'tour-count)
            (- (slot-value 'tour-count) 1)))
    (defmeth p :do-idle ()
      (send self :tour-step))
    (send (send p :menu) :append-items
          (send run-item-proto :new p))
    p))
\end{verbatim}}
\end{slide}

\begin{slide}{}
The \dcode{sphere-rand} function can be defined as
{\Large
\begin{verbatim}
(defun sphere-rand (n)
  (let* ((z (normal-rand n))
         (r (sqrt (sum (^ z 2)))))
    (if (< 0 r)
        (/ z r)
        (repeat (/ (sqrt n)) n))))
\end{verbatim}}
Some examples to try the tour on:
\begin{itemize}
\item Diabetes data
\item Iris data
\item Stack loss data
\item 4D structures
\end{itemize}
\end{slide}

\begin{slide}{}
Some variations and additions:
\begin{itemize}
\item Make new rotation orthogonal to current viewing plane
\item Controls for replaying parts of the tour
\item Touring on only some of the variables (independent variables only)
\item Constraints on the tour (Correlation Tour)
\item Integration with slicing/conditioning
\item Guided tours (Projection Pursuit -- XGobi)
\end{itemize}
\end{slide}

\begin{slide}{}
A similar idea can be used for examining functions.

As an example, suppose we want to determine if $f(x)$ looks like
a multivariate standard normal density.

This is true if and only if $g(z)=f(zu)$ looks like a univariate
standard normal density for any unit vector $u$.

By moving $u$ through space with a series of random rotations, we can look
at these univariate densities for a variety of different directions.
\end{slide}

\begin{slide}{}
The normality checking plot can be implemented as a prototype:
{\Large
\begin{verbatim}
(defproto ncheck-plot-proto
  '(function direction xvals
    tour-count tour-trans angle)
  ()
  scatterplot-proto)

(send ncheck-plot-proto :slot-value
      'tour-count -1)
(send ncheck-plot-proto :slot-value
      'angle 0.2)
\end{verbatim}}
The method for adjusting the image to the current state is
{\Large
\begin{verbatim}
(defmeth ncheck-plot-proto :set-image ()
  (let* ((x (slot-value 'xvals))
         (f (slot-value 'function))
         (d (slot-value 'direction))
         (y (mapcar #'(lambda (x)
                      (funcall f (* x d)))
                    x)))
    (send self :clear-lines :draw nil)
    (send self :add-lines (spline x y))))
\end{verbatim}}
\end{slide}

\begin{slide}{}
The method for moving the direction is similar to the tour method
used earlier:
{\Large
\begin{verbatim}
(defmeth ncheck-plot-proto :tour-step ()
  (when (< (slot-value 'tour-count) 0)
        (let* ((d (slot-value 'direction))
               (n (length d))
               (a (abs (slot-value 'angle))))
          (setf (slot-value 'tour-count)
                (random (floor (/ pi
                                  (* 2 a)))))
          (setf (slot-value 'tour-trans) 
                (make-rotation d
                               (normal-rand n)
                               a))))
  (setf (slot-value 'direction)
        (matmult (slot-value 'tour-trans)
                 (slot-value 'direction)))
  (send self :set-image)
  (setf (slot-value 'tour-count)
        (- (slot-value 'tour-count) 1)))
\end{verbatim}}
\end{slide}

\begin{slide}{}
Two additional methods:
{\Large
\begin{verbatim}
(defmeth ncheck-plot-proto :do-idle ()
  (send self :tour-step))

(defmeth ncheck-plot-proto :menu-template ()
  (append
   (call-next-method)
   (list (send run-item-proto :new self))))
\end{verbatim}}
Finally, the initialization method sets up the plot:
{\Large
\begin{verbatim}
(defmeth ncheck-plot-proto :isnew (f d)
  (setf (slot-value 'function) f)
  (setf (slot-value 'direction) d)
  (setf (slot-value 'xvals) (rseq -3 3 7))
  (call-next-method 2)
  (send self :range 0 -3 3)
  (send self :range 1 0 1.2)
  (send self :x-axis t nil 7)
  (send self :y-axis nil)
  (send self :set-image))
\end{verbatim}}
\end{slide}

\begin{slide}{}
Some examples to try:
\begin{itemize}
\item independent gamma variates
\item normal mixtures
\item likelihood function
\item posterior densities
\end{itemize}
Some possible enhancements:
\begin{itemize}
\item normal curve for comparison
\item replay tools
\item integration with transformations
\item guided tour
\end{itemize}
\end{slide}

\begin{slide}{}
\subsection{Some References}
\normalsize
\begin{itemize}
\item[]
{\sc Asimov, D.}, (1985), ``The grand tour: a tool for viewing
multidimensional data,'' {\em SIAM J. of Scient. and Statist.
Comp.} 6, 128-143.
\item[]
{\sc Becker, R. A. and Cleveland, W. S.}, (1987), ``Brushing scatterplots,''
{\em Technometrics} 29, 127-142, reprinted in {\em Dynamic Graphics
for Statistics}, W. S. Cleveland and M. E. McGill (eds.), Belmont,
Ca.: Wadsworth.
\item[]
{\sc Becker, R. A., Cleveland, W. S. and Weil, G.}, (1988), ``The use of
brushing and rotation for data analysis,'' in {\em Dynamic Graphics
for Statistics}, W. S. Cleveland and M. E. McGill (eds.), Belmont,
Ca.: Wadsworth.
\item[]
{\sc Bolorforush, M., and Wegman, E. J.} (1988), ``On some graphical
representations of multivariate data,'' {\em Computing Science and
Statistics: Proceedings of the 20th Symposium on the Interface}, E. J.
Wegman, D. T. Ganz, and J. J. Miller, editors, Alexandria, VA: ASA, 121-126.
\item[]
{\sc Buja, A., Asimov, D., Hurley, C., and McDonald, J. A.}, (1988),
``Elements of a viewing pipeline for data analysis,'' in {\em Dynamic
Graphics for Statistics}, W. S. Cleveland and M. E. McGill (eds.),
Belmont, Ca.: Wadsworth.
\item[]
{\sc Donoho, A. W., Donoho, D. L. and Gasko, M.}, (1988), ``MACSPIN:
Dynamic Graphics on a Desktop Computer,'' n {\em Dynamic
Graphics for Statistics}, W. S. Cleveland and M. E. McGill (eds.),
Belmont, Ca.: Wadsworth.
\item[]
{\sc Fisherkeller, M. A., Friedman, J. H. and Tukey, J. W.},  (1974),
``PRIM-9: An interactive multidimensional data display and analysis
system,'' in {\em Data: Its Use, Organization and Management},
140-145, New York: ACM, reprinted in {\em Dynamic
Graphics for Statistics}, W. S. Cleveland and M. E. McGill (eds.),
Belmont, Ca.: Wadsworth.
\item[]
{\sc Inselberg, A., and Dimsdale, B.} (1988), ``Visualizing
multi-di\-mensional geometry with parallel coordinates,'' {\em Computing
Science and Statistics: Proceedings of the 20th Symposium on the
Interface}, E. J. Wegman, D. T. Ganz, and J. J. Miller, editors,
Alexandria, VA: ASA, 115-120.
\end{itemize}
\end{slide}

\begin{slide}{}
\section{Comments}
Some statistical issues:
\begin{itemize}
\item
Many good ideas for viewing higher-dimensional data are available
\item
There is room for many more ideas
\item
Eventually, it will be useful to learn to quantify the effectiveness
of different viewing methods in different situations.
\item
It is also important to explore numerical enhancements as well as
combinations of different graphical strategies.
\end{itemize}
\end{slide}

\begin{slide}{}
Some computational issues:
\begin{itemize}
\item
For point clouds, image rendering is quite fast, even on stock
hardware.
\item
This is not true of lines and surfaces.
\item
The effective use of high performance 3D hardware is worth exploring.
\item
For data sets derived from models, computing speed can still be limiting
\item
Pre-computation can help, but limits interaction.
\end{itemize}
\end{slide}

\begin{slide}{}
\chapter{Final Notes}
\end{slide}

\begin{slide}{}
\section{Where can you go from here?}
Get a copy of the software for your computer and try it on your problems:
\begin{itemize}
\item try the standard tools
\item try building tools of your own
\item try graphics tailored to your problems
\end{itemize}
If you develop new tools and ideas that might be useful to others,
please share them (e.g. by submitting them to \dcode{statlib} or the
\dcode{stat-lisp-news} mailing list).
\end{slide}

\begin{slide}{}
\section{Electronic sources of information}
There is a mailing list of users who help to answer questions, share
ideas, etc.

To join the mailing list, send mail to:
\begin{center}
\Large
\dcode{stat-lisp-news-request@umnstat.stat.umn.edu}
\end{center}
Once you are on the list, you can send mail to all members of the list
by sending a message to
\begin{center}
\dcode{stat-lisp-news@umnstat.stat.umn.edu}
\end{center}
The \dcode{statlib} archive contains contributed Lisp-Stat code, as
well as the source code for XLISP-STAT.

To find out what is available from \dcode{statlib}, send mail to
\begin{center}
\dcode{statlib@lib.stat.cmu.edu}
\end{center}
containing the single line
\begin{center}
\dcode{send index}
\end{center}
\dcode{statlib} can also be accessed by {\em ftp}.
\end{slide}

\begin{slide}{}
\section{Where Lisp-Stat is headed}
Lisp-Stat is an evolving system.

Plans for the near term are to
\begin{itemize}
\item improve and update current versions
\item develop a Common Lisp version
\end{itemize}
Longer-term plans include
\begin{itemize}
\item improving the support for compound data objects
\item some support for missing data conventions
\item increased graphics capabilities
\item support for constraints
\end{itemize}
\end{slide}

\begin{slide}{}
Further development depends on the users of Lisp-Stat:
\begin{itemize}
\item
If people find the system useful, new methods for handling a variety
of problems will become available (e.g. through \dcode{statlib}).
\item
As new ideas are implemented, the need for new tools and primitives,
or for modifications to existing ones, will become evident.
\end{itemize}
\end{slide}

\begin{slide}{}
Some books on Lisp and Lisp programming:

\refitem{{\sc Abelson, H. and Sussman, G. J.} (1985), {\em Structure and 
Interpretation of Computer Programs}, New York: McGraw-Hill.}

\refitem{{\sc Franz Inc.} (1988), {\em Common Lisp: The Reference},
Reading, MA: Addison-Wesley.}

\refitem{{\sc Steele, Guy L.} (1990), {\em Common Lisp: The Language}, 
second edition, Bedford, MA: Digital Press.}

\refitem{{\sc Winston, Patrick H. and Berthold K. P. Horn}, (1988),
{\em LISP}, 3rd Ed., New York: Addison-Wesley.}
\end{slide}

\end{document}
